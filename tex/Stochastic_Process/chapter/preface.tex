
\frontmatter
\thispagestyle{empty}
\newpage
\begin{center}
	\textbf{\LARGE 前言}
\end{center}

    本文档涉及以下中科大2024春季课程:随机过程(张土生)\footnote{期末卷子也许不能在明面上发布,悄悄在这里说一下,回忆版本:\url{http://home.ustc.edu.cn/~fa1247/2024SPFinal.pdf}}、应用随机过程(翟建梁)。
	参考书:《Probability Theory and Examples (5th edition)》(作者:Durrett )、
	《Brownian Motion,Martingales,and Stochastic Calculus》(GTM275,作者:Le gall)等。
	阅读本文档可能需要的前置知识(按照重要程度排序):数学分析、线性代数、实分析(主要是测度论)、初等概率论、泛函分析(不是很重要)。

	这份文档以随机过程(以下简称“研随”)课程内容为主体,
	小节标题标注“*”代表此部分不属于研随课程范围。
	按照顺序主要可以分成两大部分:离散过程和连续过程。
	也可以分成前置知识(一些基本的测度论定理,以及条件数学期望)、鞅(离散鞅和连续鞅)与马氏过程(离散马氏链,Poisson过程、布朗运动则是特殊的连续马氏过程)。
	鞅是我认为的本课程最有趣的部分,先分析了各种情况下的收敛性,
	然后介绍了最重要的择停定理,在经过作业和考试的洗礼之后,“给鞅套一个停时再取极限”已经刻在了DNA里。
	马氏过程部分没有讲太应用的例子,大概就是把基本概念介绍了一遍,
	但也足够精彩,逻辑环环相扣叙述完备,总比会算东西但不明不白强很多。

	本文档的另一部分则为应用随机过程(以下简称“应随”),应随部分期中前重点研究离散时间马氏链的状态分类和极限分布,
	期中后重点研究跳跃过程(通过跳链分析性质,还是蛮有意思的),并对Poisson过程、布朗运动、随机积分做了简单介绍。
	此课程全程拿初等语言讲述,
	如果强行全部整理进来会造成笔记整体的不和谐,
	部分深刻内容一带而过(例如随机积分),令人不明觉厉。
	但应随内容并非完全没有价值,研随虽然足够有深度,却缺乏直观性,
	也没能提及很多实际研究经常利用的技巧和例子。
	因此推荐读者当做习题课内容来阅读。
	在笔记的正文部分,我尝试对应随的内容基于研随已经给出的理论进行重新排版,略去了很多证明(因为懒),
	与研随部分结合着一起看体验会比较好,也能给自己带来更多直观理解。

	现在(2024.06.22)的这个版本是笔记的最初版本,可能会有很多Typo,也会有叙述不完整的部分,
	还望读者不吝赐教、提出修改意见。
	最后,特别感谢yhb同学\footnote{完成了应随连续马氏链部分翟老师讲义的“中译中”工作,讲义手稿原文乱七八糟,我是真看不下去。}
	为本笔记的完善做出的贡献。

	\if{0}{
		2024.01.31:经历10天的低效啃书,今天终于是
		把Durrett第一章看完了。因为实分析的内容忘得太多,
		差不多有一半的时间在翻以前的实分析笔记,
		真感谢以前的我整理了这么详细的笔记,不然叙述如此粗糙的第一章我是真的看不下去,
		目前让我感到恼火的点有:
		\begin{enumerate}
			\item 第一节莫名其妙丢出个Stieltjes measure function,您这个时候
			就直接搬出来还不给进一步说明,
			我哪知道这东西是干啥的呀。我知道这个Stieltjes是很重要,因为后面用这东西
			证明了同分布就是同分布函数,但您放在第一小节也太不合适了吧,
			等分布函数那一小节再拿出来举例不是更好吗?
			\item 先讲分布函数,再讲随机变量,您不讲随机变量怎么定义的分布函数?
			为什么不干脆先讲随机变量,然后诱导出分布函数,
			难道这样不是更好理解也更严谨?
			\item 概率测度那一块儿的记号,实在是抽象至极,$\P( X\in A )$这种写法我能理解,不就是
			随机变量落在$A$的概率嘛,但是您这是什么书啊,您这是测度论下的概率论啊!
			测度论里$\P$括起来的东西怎么也得是个集合吧?
			您括个事件,且不说会不会在某种情况下造成歧义,
			一旦形式复杂起来(比如Chebyshev不等式的那个地方)就非常容易理解错了,
			咱好好写记号犯法吗?就算真的要搞简写,能不能提前说一声啊?
			提前说一下$\P$括个事件就是扣了个集合有这么难吗?
			\item 随机元(random element)是个什么东西?为什么不能明确定义一下?
			我还是上百度搜出来的这玩意儿的定义,其实就是概率空间上的可测映射呗,
			动动您的手多写一句话有这么难吗?
			\item 积分的那一块儿东西,您嫌弃这些东西是dirty work我能理解,因为测度论上的积分
			这块儿定义、性质和证明都繁琐至极,但是您既然选择了提一嘴,为什么不干脆说清楚呢?
			比如$\int_S f(y)\mu(\d y)$这个积分表示方法,前文什么时候提到过?
			难道您觉得来看书的都是有高等概率论基础的学生?
			那您干脆丢个详细的参考书让我们自己去看好了,
			既解决了学生学到的符号体系和您书上不一样的问题(比如我),
			而且还不用劳烦您特地把积分的定义又重新抄一遍啊。
		\end{enumerate}
		不过收获还是很大的,
		一方面我彻底理清楚了半代数、代数等等那几个集合族的关系,
		也搞清楚了测度空间是怎么来的;另一方面也是终于解决了我学实分析的时候就
		想不明白的一个疑问:
		$\R$上可测函数的定义是所有Borel集的逆像Lebesgue可测,为啥前后不用一样的测度呢?
		现在明白了,可测函数就是到$(\R,\mathcal{R})$这个测度空间的可测映射,
		实分析里选择了Lebesgue测度空间作为原空间,
		可能是因为Lebesgue是个完备的测度吧。
	}\fi

	\if{0}{
		2024.02.06:前一周睡眠质量奇差,遂不学习了调整作息。
		今天这第二章又看得我窝大火。唉,这b概率论怎么这么难,这b书怎么这么烂。
		我当初为什么不去看中文教材呢?
	
		对于一本书,尤其是数学领域教科书而言,我称其为“破书”的那些书都具备以下若干要素:
		\begin{enumerate}
			\item 作者尝试用自然语言解释抽象的概念但失败了,最后出现一些莫名其妙的叙述性文字,惹的读者一头雾水。
			\item 想要涉及更多理论让自己的书的内容更全面,但因篇幅所限或者作者水平不足,最后虎头蛇尾潦草结束。
			\item 记号没有定义就直接使用,或者前后文不一致。\footnote{Durrett书中测度积分的三种写法的后两种都没有作明确说明,Random element这个术语第一次出现时也没有解释什么意思。
			这样的疏漏实在太多了,尤其是在第一章。感觉作者是默认读者都懂测度论了,所以就潦草且随意地一笔带过了这部分内容。我个人认为直接删了第一章比较好。}
		\end{enumerate}
		那么,我认为相应的解决办法是:
		\begin{enumerate}
			\item 知识的诅咒:“懂得知识之后就会忘记不懂时是如何思考的”。如果我不知道怎么解释,干脆不要解释,平铺直叙列出所有的定理和证明,根据自己的理解划分小节、起好总结性的小标题。
			给读者的建议是,先坚持看下去,看完一小节之后梳理脉络,找到自己的理解。
			\item 找好自己写的这本书的定位,比如你在一本概率论的书中想要提及测度论,要么为读者列好推荐书籍(并且尽量标明使用了推荐书籍里的哪些记号和结论),要么自己完完整整
			写一章把你后文要用得到的所有结论都讲述清楚,不要虎头蛇尾!不要留下一句“读者可去自行了解”!
			\item 纯铸币行为,完全无法容忍,我想避免这类疏漏应当是任何一个科学工作者应当具备的基本的严谨态度。
		\end{enumerate}
		这也是本文档的诞生动机之一,我希望自己做一份风格统一、记号统一、所有理论细节尽量完整的文档,
		一方面是强迫症使然,另一方面方便我需要时随时查阅和复习。
	
		简写虽然书写时方便,但如果从文档的半山腰处开始阅读的话,也会面临上述问题。所以我在第一章之前放了一个“字典”,
		包含了所有文档里用到的记号、专业术语等的定义或者在文档里第一次出现的位置。
	}\fi

	\if{0}{
		2024.05.25:性质上本文档属于“课程笔记”,笔记都是带有个人风格的,所以阅读他人笔记的体验往往并不好。
		但笔者希望把这份笔记慢慢修缮成一本“参考文档”。
		按照个人经验,无论是书籍还是课程,如果你感觉到作者刻意地把某些概念说得很模糊,
		那可能的原因有两个:一是作者自身水平不行,试图蒙混过关;
		二是因为有些背景知识你不了解,但因篇幅所限又没办法展开讲,只好一笔带过。
		那么,作为一份“参考文档”,自然是不需要考虑篇幅的问题的,
		所以这份文档会尽可能详细地把笔者所了解到的所有(概率论的)知识都塞进去。
		
		笔者曾经习惯于“面向考试学习”,概念搞不清楚?没关系,背背作业题和往年题应付考试就足够了。
		所以实际上学的很粗糙,在整理知识的过程中也解决了不少以前没搞懂的问题,很让人有成就感,算是这份文档的初衷吧。
		但现在还远不是这份文档的终点,因为求学之路还很漫长,笔者很期待未来能够把学到的更深刻、更有趣的知识塞进文档里,
		如同在海边捡漂亮石头装进口袋的孩童。
	}\fi

	\begin{flushright}
		最后更新:\today
	\end{flushright}
\frontmatter