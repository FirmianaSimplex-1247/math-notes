\chapter{连续时间鞅}
    本章内容主要来自Le gall chapter3,有些英文术语我也找不到很合适的翻译,就按照自己顺口的习惯翻译了,并且标注了原文。
\section{滤流}
    开始这一章之前,我们先阐明滤流和随机过程的详细定义。
    我们以下默认在概率空间$(\Omega,\F,\P)$上讨论,并且取指标集$I=[0,+\infty]$.
    \begin{definition}[滤流]
        一族$\sigma$-域$(\F_t)_{t\geqslant 0}$满足:
        \begin{equation*}
            \forall t\in [0,+\infty],\F_t\subset \F
        \end{equation*}
        \begin{equation*}
            \forall 0\leqslant s\leqslant t,\F_0\subset \F_s\subset \F_t\subset \F_\infty
        \end{equation*}
        则称$(\F_t)_{t\geqslant 0}$是一个\textbf{滤流}(filtration),也称$(\Omega,\F,(\F_t),\P)$为滤流概率空间(filtered probability space).
    \end{definition}

    \begin{example}
        对于随机过程$X=(X_t)_{t\geqslant 0}$,我们取
        \begin{equation*}
            \F_t^X= \sigma(X_s,0\leqslant s\leqslant t)
        \end{equation*}
        且$\F_\infty^X=\sigma(X_s,s\geqslant 0)$,那么$(\F_t^X)_{t\geqslant 0}$就是一个滤流,被称为$X$的正规滤流(canonical filtration).
    \end{example}

    \begin{definition}[右连续的滤流]
        对于滤流$(\F_t)_{t\geqslant 0}$,定义:
        \begin{equation*}
            \F_{t+}=\bigcap_{s>t}\F_s,\ \F_{\infty+}=\F_\infty
        \end{equation*}
        如果$\forall t\geqslant 0$,$\F_{t+}=\F_t$,称滤流$(\F_t)_{t\geqslant 0}$是\textbf{右连续}的。
    \end{definition}
    $(\F_t+)_{t\geqslant 0}$就是一个右连续的滤流。

    \begin{definition}[完备的滤流]
        对于滤流$(\F_t)_{t\geqslant 0}$,记$N$为测度空间$(\Omega,\F_\infty,\P)$上的所有零测集的子集,即
        \begin{equation*}
            N=\{ A\subset \Omega:\exists A'\in \F_\infty{\rm\ s.t.\ }A\subset A',\P(A')=0 \}
        \end{equation*}
        如果$N\subset \F_0$,则称$(\F_t)_{t\geqslant 0}$是\textbf{完备}的。
    \end{definition}
    如果滤流$(\F_t)_{t\geqslant 0}$不完备,我们可以定义$\F_t'$为包含$\F_t$和$\sigma(N)$
    的最小$\sigma$-域,记作
    \begin{equation*}
        \F_t'=\F_t\vee \sigma(N),\forall t\geqslant 0
    \end{equation*}
    则$(\F_t')_{t\geqslant 0}$是完备的,称为
    滤流$(\F_t)_{t\geqslant 0}$的完备化。

    \begin{definition}[可测、适应、循序]
        对于随机过程$X=(X_t)_{t\geqslant 0}$,$X_t$在可测空间$(E,\mathcal{E})$上取值,如果映射
        \begin{align*}
            f:( \Omega\times \R_+,\F\otimes \mathcal{B}(\R_+) )&\rightarrow (E,\mathcal{E})\\
            (\omega,t)&\mapsto X_t(w)
        \end{align*}
        是可测的,则称随机过程$X$是\textbf{可测}的(measurable).

        对于随机过程$X=(X_t)_{t\geqslant 0}$,滤流$(\F_t)_{t\geqslant 0}$,如果$\forall t\geqslant 0$,
        $X_t$是$\F_t$-可测的,那么称$X$是(关于$(\F_t)_{t\geqslant 0}$的)\textbf{适应}的(adapted).

        如果映射
        \begin{align*}
            g:( \Omega\times [0,t],\F_t\otimes \mathcal{B}([0,t]) )&\rightarrow (E,\mathcal{E})\\
            (\omega,s)&\mapsto X_s(w)
        \end{align*}
        是可测的,则称$X$是\textbf{循序}的(progressive).
    \end{definition}
    
    \begin{proposition}\label{progressive from right-continous}
        随机过程$X=(X_t)_{t\geqslant 0}$在度量空间$(E,d)$
        (取Borel $\sigma$-域得到可测空间)上取值,
        $X$关于滤流$(\F_t)_{t\geqslant 0}$适应,
        并且$X$的样本轨道是右连续或者左连续的,则$X$是循序的。
    \end{proposition}
    \begin{proof}
        只说明一下右连续的情况,左连续类似。固定$t>0$,$\forall n\geqslant 1,s\in [0,t]$,取r.v.
        \begin{equation*}
            X_s^n\defeq X_{ \frac{kt}{n} },{\rm\ where\ }t{\rm\ s.t.\ }s\in \left[ \frac{(k-1)t}{n},\frac{kt}{n} \right),k\in \{1,2,\cdots,n\}
        \end{equation*}
        并且$X_t^n=X_t$,样本轨道右连续确保了:$\forall s\in [0,t],\omega\in\Omega$,
        \begin{equation*}
            X_s(\omega)\fun{lim}{n\rightarrow\infty} X_s^n(\omega)
        \end{equation*}
        那么任取Borel集$A\subset E$,
        \begin{align*}
             &\{ (\omega,s)\in \Omega\times[0,t]:X_s^n(\omega)\in A \}\\
            =&( \{X_t\in A\}\times\{t\} )\bigcup
            \left( \bigcup_{k=1}^n \left( \{ X_{\frac{kt}{n}}\in A \}\times \left[ \frac{(k-1)t}{n},\frac{kt}{n} \right) \right) \right)
        \end{align*}
        后者属于$\F_t\otimes \mathcal{B}( [0,t] )$,因此$\forall n\geqslant 1$,映射:
        \begin{align*}
            ( \Omega\times [0,t],\F_t\otimes \mathcal{B}( [0,t] ) )&\rightarrow (E,\mathcal{B}(E))\\
            (\omega,s)&\mapsto X_s^n(\omega)
        \end{align*}
        是可测的。由于可测函数的逐点极限也是可测的,这就说明$X$是渐进的。
    \end{proof}

    \begin{definition}[循序$\sigma$-域]
        固定$A\in \F\otimes \mathcal{B}(\R_+)$,并令$X_t^A(\omega)=I_A(\omega,t)$,
        所有使得随机过程$X=(X_t^A)_{t\geqslant 0}$是循序过程的集合$A$,
        组成了一个$\sigma$-域,记作$\mathcal{P}$,称之为循序$\sigma$-域。
    \end{definition}

\section{停时}
    先来回顾一下停时相关的定义。
    \begin{definition}[停时]
        对于滤流$(\F_t)_{t\geqslant 0}$,随机变量$T$在$[0,+\infty]$上取值,如果
        $\forall t\geqslant 0$,$\{ T\leqslant t \}\in \F_t$,则称$T$是一个(关于滤流$(\F_t)_{t\geqslant 0}$)的停时。
    \end{definition}
    在本章的剩余内容中,如无特殊说明,默认停时$T$都是关于滤流$(\F_t)_{t\geqslant 0}$的。
    \begin{corollary}
        $T$是关于滤流$(\F_t)_{t\geqslant 0}$的停时,则有
        \begin{equation*}
            \{ T<t \}=\bigcap_{q\in \Q,q<t}\{ T\leqslant q \}\in \F_t
        \end{equation*}
        \begin{equation*}
            \{ T<+\infty \}=\bigcup_{q\in \Q,q\geqslant 0} \{ T\leqslant q \}\in \F_\infty
        \end{equation*}
    \end{corollary}

    \begin{definition}[停时前$\sigma$-域]
        $T$是关于滤流$(\F_t)_{t\geqslant 0}$的停时,
        \begin{equation*}
            \F_T=\{ A\in \F_\infty:\forall t\geqslant 0,A\cap \{T\leqslant t\}\in \F_t \}
        \end{equation*}
        则$\F_T$是一个$\sigma$-域,称为$T$前$\sigma$-域。
    \end{definition}

    \begin{theorem}\label{le gall prop3.6(1)}
        $T:\Omega\rightarrow [0,+\infty]$是随机变量,则以下结论等价:
        \begin{enumerate}[(1).]
            \item $T$关于$(\F_{t+})_{t\geqslant 0}$是停时。
            \item $\forall t>0$,$\{T<t\}\in \F_t$.
            \item $\forall t>0$,$T\wedge t$是$\F_t$可测的。
        \end{enumerate}
    \end{theorem}
    \begin{proof}
        $(1)\Rightarrow (2)$:$\forall t\geqslant 0$,
        \begin{equation*}
            \{ T<t \}=\bigcup_{q\in Q,q>0} \{ T\leqslant t-q \}\in \F_t
        \end{equation*}

        $(2)\Rightarrow (1)$:$\forall t\geqslant 0$,$\forall s>t$,
        \begin{equation*}
            \{ T>t \}=\bigcap_{q\in \Q,q>0} \{ T\geqslant t+q \}\in \F_{t+}
        \end{equation*}

        $(2)\Rightarrow (3)$:$\forall s<t$,
        \begin{equation*}
            \{ T\wedge t\leqslant s \}=\{ T\leqslant s\}\cup\{ t\leqslant s \}
            =\{ T\leqslant s \}\in \F_s\subset \F_t
        \end{equation*}

        $(3)\Rightarrow (2)$:$\forall t\geqslant 0$,
        \begin{equation*}
            \{ T<t \}=\bigcap_{q\in [0,t]\cap \Q} \{ T\leqslant t-q \}
            =\bigcap_{q\in [0,t]\cap \Q} \{ T\wedge t\leqslant t-q \}\in \F_t
        \end{equation*}        
    \end{proof}

    \begin{corollary}
        $T$是关于滤流$(\F_t)_{t\geqslant 0}$的停时,则$T$也是关于滤流$(\F_{t+})_{t\geqslant 0}$的停时。
    \end{corollary}

    \begin{theorem}\label{le gall prop3.6(2)}
        $T$是关于滤流$(\F_t)_{t\geqslant 0}$的停时,则
        \begin{equation*}
            \{ A\in \F_\infty:\forall t\geqslant 0,A\cap \{ T\leqslant t \}\in \F_{t+} \}
            =
            \{ A\in \F_\infty:\forall t\geqslant 0,A\cap \{ T<t \}\in \F_{t} \}
        \end{equation*}
        我们将上述$\sigma$-域记作$\F_{T+}$.
    \end{theorem}
    \begin{proof}
        设$A\in \mathcal{G}_T$,则
        \begin{equation*}
            A\cap \{ T<t \}=A\cap \left(\bigcup_{q\in \Q,q>0} \{ T\leqslant t-q \}\right)
            =\bigcup_{q\in \Q,q>0}( A\cap \{ T\leqslant t-q \})\in \F_t
        \end{equation*}
        所以$\mathcal{G}_T\subset \F_{T+}$;反之,设$A\in \F_{T+}$,
        \begin{align*}
            A\cap \{ T\leqslant t \}
            =A\cap \left( \bigcap_{q\in \Q,q>0}\{ T<t+q \} \right)
            =\bigcap_{q\in \Q,q>0}\left( A\cap \{ T<t+q \} \right)
            \in \F_{t+}
        \end{align*}
        所以$\F_{T+}\subset \mathcal{G}_T$.
    \end{proof}

    \begin{proposition}[停时的性质总结]
        $T$是关于滤流$(\F_t)_{t\geqslant 0}$的停时,
        下文中出现的停时如无特殊说明,默认是关于滤流$(\F_t)_{t\geqslant 0}$的停时。
        \begin{enumerate}[(1).]
            \item $\F_T\subset \F_{T+}$,如果$(\F_{t})$右连续,则$\F_{T+}=\F_T$.
            \item 如果$T=t$为常数,则$\F_T=\F_t$,且$\F_{T+}=\F_{t+}$.
            \item $T$是$\F_T$-可测的。
            \item 如果$A\in \F_\infty$,令
                \begin{equation*}
                    T^A(\omega)=\left\{ \begin{array}{ll}
                        T(\omega)&,\omega\in A\\
                        +\infty&,\omega\notin A
                    \end{array} \right.
                \end{equation*}
                则$A$是$\F_T$可测的当且仅当$T^A$是停时。
            \item $S$也是停时,且$S\leqslant T$,那么$\F_S\subset \F_T$,且$\F_{S+}\subset \F_{T+}$.
            \item $S$也是停时,则$S\wedge T$和$S\vee T$都是停时,并且$\F_{S\wedge T}=\F_S\cap \F_T$,$\{ S\leqslant T \}\in \F_{S\wedge T}$,$\{S=T\}\in \F_{S\wedge T}$.
            \item $\{S_n\}_{n\geqslant 1}$是一列单调递增的停时,则$S=\fun{lim}{n\rightarrow\infty}S_n$也是停时。
            \item $\{S_n\}_{n\geqslant 1}$是一列单调递减的停时,则$S=\fun{lim}{n\rightarrow\infty}S_n$是关于$(\F_{t+})$的停时,并且
                \begin{equation*}
                    \F_{S+}=\bigcap_{n=1}^\infty \F_{S_n+}
                \end{equation*}
            \item $\{S_n\}_{n\geqslant 1}$是一列单调递减的停时,$S=\fun{lim}{n\rightarrow\infty}S_n$,而且$\forall \omega$,都存在$N(\omega)$使得$\forall n>N(\omega)$有$S_n(\omega)=S$,则
                \begin{equation*}
                    \F_{S}=\bigcap_{n=1}^\infty \F_{S_n}
                \end{equation*}
            \item $(E,\mathcal{E})$是一个可测空间,对于映射
                \begin{equation*}
                    Y:\{ T<+\infty \}\rightarrow E,\omega\mapsto Y(\omega)
                \end{equation*}
                定义
                \begin{equation*}
                    Y_t:\{ T\leqslant t \}\rightarrow E,\omega\mapsto Y(\omega)
                \end{equation*}
                为$Y$在$\{ T\leqslant t \}$上的限制。那么,
                $Y$是$\F_T$-可测的当且仅当$\forall t\geqslant 0$,$Y_t$是$\F_t$-可测的。
        \end{enumerate}
    \end{proposition}
    \begin{proof}
        基本上都是验证定义以及利用\autoref{le gall prop3.6(1)}和\autoref{le gall prop3.6(2)}得到的。
        \begin{enumerate}[(1).]
            \item 我们熟知$\F_{t}\subset \F_{t+}$,那么
                \begin{equation*}
                    \F_{T+}=\{ A\in \F_\infty:\forall t\geqslant 0,A\cap \{ T\leqslant t \}\in \F_{t+} \}
                    \supset 
                    \{ A\in \F_\infty:\forall t\geqslant 0,A\cap \{ T\leqslant t \}\in \F_{t} \}=\F_T
                \end{equation*}
                如果$\F_{t+}=\F_t$,则上式中两个集合相等。
            \item 如果$T=t$是常数,
                则$A\in \F_T\Leftrightarrow \forall s\geqslant t,A\in \F_s\Leftrightarrow A\in \F_t$,
                所以$\F_T=\F_t$;
                $A\in \F_T\Leftarrow \forall s>t,A\in \F_s\Leftrightarrow A\in \F_{t+}$.
            \item $\forall s\geqslant 0$,$\forall t\geqslant 0$,
                \begin{equation*}
                    \{ T\leqslant s \}\cap 
                    \{ T\leqslant t \}=\{ T\leqslant s\wedge t \}
                    \in \F_t
                \end{equation*}
                所以$\{ T\leqslant s \}\in \F_T$,故$T$是$\F_T$-可测的。
            \item 注意到$\{ T^A\leqslant t \}=A\cap \{ T\leqslant t \}$.
            \item 注意到$\{ T\leqslant t \}\subset \{ S\leqslant t \}$,所以
                \begin{equation*}
                    A\cap \{ T\leqslant t \}=A\cap \{ S\leqslant t \}\cap \{ T\leqslant t \}
                \end{equation*}
                如果$A\in \F_S$,则$A\cap \{ S\leqslant t \}\in \F_t$,$\{ T\leqslant t \}\in \F_t$;
                如果$A\in \F_{S+}$,则$A\cap \{ S\leqslant t \}\in \F_{t+}$,
                $\{T\leqslant t\}\in \F_t\subset \F_{t+}$.
            \item $\forall t\geqslant 0$,
            \begin{equation*}
                \{S\wedge T\leqslant t\}=\{ S\leqslant t \}\cup\{ T\leqslant t \}\in \F_t
            \end{equation*}
            \begin{equation*}
                \{S\vee T\leqslant t\}=\{ S\leqslant t \}\cap\{ T\leqslant t \}\in \F_t
            \end{equation*}
            所以这俩是停时;由(5)可知$\F_{S\wedge T}\subset \F_S\cap \F_T$,另一方面。如果$A\in \F_S\cap \F_T$,
            则
            \begin{equation*}
                A\cap \{ S\wedge T\leqslant t \}
                =(A\cap \{S\leqslant t\})\cap (A\cap \{T\leqslant t\})\in \F_t
            \end{equation*}
            所以$\F_S\cap \F_T\subset \F_{S\wedge T}$;(待补充)
            \item $S_n\nearrow S$,则\begin{equation*}
                \{S\leqslant t\}=\bigcap_{n=1}^\infty \{S_n\leqslant t\}\in \F_t
            \end{equation*}
            \item $S_n\searrow S$,则
                \begin{equation*}
                    \{S<t\}=\bigcup_{n=1}^\infty \{ S_n<t \}\in \F_t
                \end{equation*}
                所以$S\in \F_{t+}$;因为$\F_{S+}\subset \F_{S_n+}$,所以
                \begin{equation*}
                    \F_{S+}\subset \bigcap_{n=1}^\infty \F_{S_n+}
                \end{equation*}
                另一方面,如果$A\in \F_{S_n+},\forall n$,则$\forall t$,
                \begin{equation*}
                    A\cap{S<t}=\bigcup_{n=1}^\infty (A\cap \{S_n<t\})\in \F_t 
                \end{equation*}
            \item 如果$A\in \F_S$,
                \begin{equation*}
                    \{S\leqslant t\}=\bigcup_{n=1}^\infty \{ S_n\leqslant t \}
                    \in \F_t
                \end{equation*}
                则$S$关于$(\F_t)$是停时;$\F_S\subset \F_{S_n}$,所以
                \begin{equation*}
                    \F_{S}\subset \bigcap_{n=1}^\infty \F_{S_n}
                \end{equation*}
                反之,如果$A\in \F_{S_n}$,则
                \begin{equation*}
                    A\cap \{S\leqslant t\}=\bigcup_{n=1}^\infty (A\cap \{S_n\leqslant t\})\in \F_t
                \end{equation*}
            \item 假设$Y$限制在集合$\{ T\leqslant t \}$上是关于$\F_t$-可测的,则对于$A\in \mathcal{B}(E)$,
                \begin{equation*}
                    \{ Y\in A \}\cap \{ T\leqslant t \}\in \F_t,\ \forall t\geqslant 0
                \end{equation*}
                此即$\{ Y\in A \}\in \F_T$.
                反之类似。
        \end{enumerate}
    \end{proof}

    \if{0}{
    \begin{theorem}
        $X=(X_t)_{t\geqslant 0}$是渐进过程,在可测空间$(E,\mathcal{E})$上取值,$T$是停时,那么映射
        \begin{equation*}
            Y:\{ T<+\infty \}\rightarrow E,
            \omega\mapsto X_{T(\omega)}(\omega)
        \end{equation*}
        是$\F_T$-可测的。
    \end{theorem}
    \begin{proof}
        
    \end{proof}
    }\fi

    \begin{theorem}\label{le gall prop3.8}
        $T$是停时,随机变量$S:\Omega\rightarrow [0,+\infty]$是$\F_T$-可测的,
        且满足$S\geqslant T$,则$S$也是停时。
    \end{theorem}
    \begin{proof}
        利用
        \begin{equation*}
            \{ S\leqslant t \}=\{ S\leqslant t \}\cup \{ T\leqslant t \}
        \end{equation*}
        其中$\{ S\leqslant t \}\in \F_T$,所以右式$\in \F_t$.
    \end{proof}

    \begin{corollary}\label{cor of order S.T.}
        $T$是停时,那么$\forall n\in\N_+$,
        \begin{equation*}
            T_n=\sum_{k=0}^\infty \frac{k+1}{2^n}I_{ \{ k\cdot 2^{-n}<T\leqslant (k+1)\cdot 2^{-n} \} }+\infty \cdot I_{ \{ T=+\infty \} }
        \end{equation*}
        也是停时,而且$T_n\searrow T$.
    \end{corollary}

    \begin{example}
        随机过程$X=(X_t)_{t\geqslant 0}$适应$(\F_t)_{t\geqslant 0}$,
        在度量空间$(E,d)$上取值,$X$的轨道连续,$F\subset E$为闭集,则
        \begin{equation*}
            T_F=\fun{inf}{}\{ t\geqslant 0:X_t\in F \}
        \end{equation*}
        是一个停时。
    \end{example}
    \begin{proof}
        只需注意到
        \begin{align*}
            \{ T_F\leqslant t \}&=\bigcup_{s\in [0,t]}\{ X_s\in F \}\\
            &=\{ \fun{inf}{s\in [0,t]} d(X_s,F)=0 \}\\
            &=\{ \fun{inf}{s\in [0,t]\cap Q} d(X_s,F)=0 \}\in \F_t
        \end{align*}
        这里利用了轨道和度量函数$d$的连续性,将不可数并转化为了可数并。
    \end{proof}

    \begin{example}
        随机过程$X=(X_t)_{t\geqslant 0}$适应$(\F_t)_{t\geqslant 0}$,
        在度量空间$(E,d)$上取值,$X$的轨道\textbf{右}连续,$O\subset E$为开集,则
        \begin{equation*}
            T_O=\fun{inf}{}\{ t\geqslant 0:X_t\in O \}
        \end{equation*}
        是一个关于$(\F_{t+})$的停时。
    \end{example}
    \begin{proof}
        只需注意到
        \begin{equation*}
            \{ T_O<t \}=\bigcup_{s\in [0,t)\cap \Q} \{X_s\in O\}\in \F_t
        \end{equation*}
        再利用\autoref{le gall prop3.6(1)}即可。
    \end{proof}

\section{鞅的定义与基本性质}
    在本章的是剩余内容中,默认随机过程在$\R$上取值。
    \begin{definition}
        $X=(X_t)_{t\geqslant 0}$适应$(\F_t)_{t\geqslant 0}$,
        且$\forall t,X_t\in L^1$,如果
        \begin{equation*}
            \forall 0\leqslant s<t,\E[X_t|\F_s]=X_s
        \end{equation*}
        则称$X$是(连续时间)鞅。如果把上式中的等号替换为小于等于号、大于等于号,则称之为
        上鞅、下鞅。
    \end{definition}

    \begin{example}[][Martingale generated by Independent Increment]
        随机过程$Z=(Z_t)_{t\geqslant 0}$在$\R$上取值,适应$(\F_t)$,并且有独立增量,即:
        $\forall 0\leqslant s<t$,$Z_t-Z_s$独立于$\F_s$. 那么
        \begin{enumerate}[(1).]
            \item 如果$\forall t\geqslant 0,Z_t\in L^1$,则$\tilde{Z}_t=Z_t-\E[Z_t]$是鞅。
            \item 如果$\forall t\geqslant 0,Z_t\in L^2$,则$\tilde{Y}_t=\tilde{Z}_t^2-\E[\tilde{Z}_t^2]$是鞅。
            \item 如果存在$\theta\in \R$使得$\forall t\geqslant 0,\E[{\rm e}^{\theta Z_t}]<+\infty$,则
                \begin{equation*}
                    X_t=\frac{{\rm e}^{\theta Z_t}}{\E[{\rm e}^{\theta Z_t}]}
                \end{equation*}
                是鞅。
        \end{enumerate}
    \end{example}
    \begin{proof}
        可积性和适应性显然,不再赘述,我们直接验证鞅的最后一条定义。
        对于$\forall t\geqslant 0$,我们记$\mu_t=\E[Z_t]$,那么对于$\forall s>0$,
        \begin{align*}
            \E[ \tilde{Z}_{t+s}|\F_t ]
            &=\E[ Z_{t+s}|\F_t ]-\mu_{t+s}\\
            &=\E[ Z_{t+s}-Z_t|\F_t ]+\E[Z_t|\F_t]-\mu_{t+s}\\
            &=\E[ Z_{t+s}-Z_t ]+Z_t-\mu_{t+s}\\
            &=\mu_{t+s}-\mu_t+Z_t-\mu_{t+s}=Z_t-\mu_t=\tilde{Z}_t
        \end{align*}
        \begin{align*}
            \E[ \tilde{Y}_{t+s}|\F_t ]
            &=\E[ (Z_{t+s}-\mu_{t+s})^2|\F_t ]-\E[(Z_{t+s}-\mu_{t+s})^2]\\
            &=\E[ (Z_{t+s}-Z_t+Z_t-\mu_{t+s})^2|\F_t ]-\E[Z_{t+s}^2]+\mu_{t+s}^2\\
            &=\E[ (Z_{t+s}-Z_t)^2+(Z_t-\mu_{t+s})^2+2(Z_{t+s}-Z_t)(Z_t-\mu_{t+s})|\F_t ]-\E[Z_{t+s}^2]+\mu_{t+s}^2\\
            &=\E[(Z_{t+s}-Z_t)^2]+(Z_t-\mu_{t+s})^2+2(Z_t-\mu_{t+s})\E[(Z_{t+s}-Z_t)]-\E[Z_{t+s}^2]+\mu_{t+s}^2\\
            &=\E[-2(Z_{t+s}-Z_t)Z_t-Z_t^2+Z_{t+s}^2]+(Z_t-\mu_{t+s})^2+2(Z_t-\mu_{t+s})(\mu_{t+s}-\mu_t)-\E[Z_{t+s}^2]+\mu_{t+s}^2\\
            &=-2(\mu_{t+s}-\mu_t)\mu_t-\E[Z_t^2]+\E[Z_{t+s}^2]+(Z_t-\mu_{t+s})^2+2(Z_t-\mu_{t+s})(\mu_{t+s}-\mu_t)-\E[Z_{t+s}^2]+\mu_{t+s}^2\\
            &\text{(全拆开之后消了很多项)}\\
            &=Z_t^2-2\mu_tZ_t-\E[Z_t^2]\\
            &=(Z_t-\mu_t)^2+\E[Z_t^2]-\mu_t^2=\tilde{Y}_t
        \end{align*}
        \begin{align*}
            \E[X_{t+s}|\F_t]
            &=\frac{ \E[ {\rm e}^{\theta Z_{t+s}}|\F_t ] }{\E[ {\rm e}^{\theta Z_{t+s}} ]}\\
            &=\frac{ \E[ {\rm e}^{\theta (Z_{t+s}-Z_t)}{\rm e}^{\theta Z_t}|\F_t ] }{\E[ {\rm e}^{\theta Z_{t+s}} ]}\\
            &=\frac{ \E[ {\rm e}^{\theta (Z_{t+s}-Z_t)} ]{\rm e}^{\theta Z_t} }{\E[ {\rm e}^{\theta Z_{t+s}} ]}\\
            &=\frac{ \E[ {\rm e}^{\theta (Z_{t+s}-Z_t)} ]\E[{\rm e}^{\theta Z_t}]{\rm e}^{\theta Z_t} }{\E[ {\rm e}^{\theta Z_{t+s}} ]\E[{\rm e}^{\theta Z_t}]}\\
            &=\frac{ \E[ {\rm e}^{\theta Z_{t+s}} ]{\rm e}^{\theta Z_t} }{\E[ {\rm e}^{\theta Z_{t+s}} ]\E[{\rm e}^{\theta Z_t}]}\\
            &=\frac{ {\rm e}^{\theta Z_t} }{\E[{\rm e}^{\theta Z_t}]}=X_t
        \end{align*}
    \end{proof}

    我们之前介绍过的Poisson过程和布朗运动就具有独立增量,因此我们得到了如下例子。
    \begin{example}[由Poisson过程生成的鞅]
        $N=(N_t)_{t\geqslant 0}$是一个参数为$\lambda$的Poisson过程,并设
        \begin{equation*}
            \F_t=\sigma(N_s,0\leqslant s\leqslant t),\ t\geqslant 0
        \end{equation*}
        那么:
        \begin{enumerate}[(1).]
            \item $M_t=N_t-\lambda t$是一个鞅。
            \item $Z_t=(N_t-\lambda t)^2-\lambda t$是一个鞅。
            \item 对于$\alpha>0$,设$\beta=({\rm e}^\alpha-1)\lambda$,则
                \begin{equation*}
                    L_t={\rm exp}\{ \alpha N_t-\beta t \},t\geqslant 0
                \end{equation*}
                是一个鞅。
        \end{enumerate}
    \end{example}
    \begin{proof}
        \begin{equation*}
            \E[ {\rm e}^{\alpha N_t} ]
            =\sum_{k=0}^\infty {\rm e}^{-\lambda}{\rm e}^{\alpha k}\frac{\lambda^k}{k!}={\rm e}^{\lambda({\rm e}^\alpha-1)}
        \end{equation*}
    \end{proof}

    \begin{example}[由布朗运动生成的鞅][Martingale generated from B.M.]
        $B=(B_t)_{t\geqslant 0}$是一个布朗运动,并设
        \begin{equation*}
            \F_t=\sigma(B_s,0\leqslant s\leqslant t),\ t\geqslant 0
        \end{equation*}
        那么$\forall \theta\in \R$,
        \begin{equation*}
            X_t={\rm exp}\left\{ \theta B_t-\frac{1}{2}\theta^2 t \right\}
        \end{equation*}
        是一个鞅。
    \end{example}
    \begin{proof}
        \begin{equation*}
            \E[{\rm e}^{\theta B_t}]=
            \int_{-\infty}^{+\infty} \frac{1}{\sqrt{2\pi t}}{\rm e}^{\frac{1}{2}t\theta^2}{\rm e}^{ \frac{1}{2t}(x-t\theta)^2 }\d x={\rm e}^{\frac{1}{2}\theta^2t}
        \end{equation*}
    \end{proof}
    \begin{theorem}[关于凸函数]\label{Cont-Martingale about convex}
        $f:\R\rightarrow \R$是一个有界凸函数,则
        \begin{enumerate}[(1).]
            \item $X=(X_t)_{t\geqslant 0}$是一个鞅,则$f(X_t)$是一个下鞅。
            \item $X=(X_t)_{t\geqslant 0}$是一个下鞅,且$f$单调递增,则$f(X_t)$是一个下鞅。
        \end{enumerate}
    \end{theorem}
    \begin{proof}
        琴生不等式易证。
    \end{proof}
    我们经常取凸函数$f(x)=|x|^p$和$f(x)=x^+$.

    \begin{theorem}
        $X=(X_t)_{t\geqslant 0}$是一个下鞅,那么$\forall t\geqslant 0$,
        \begin{equation*}
            \fun{sup}{0\leqslant s\leqslant t} \E[ |X_s| ]<+\infty
        \end{equation*}
        上鞅也有同样的结论。
    \end{theorem}
    \begin{proof}
        我们分别考虑$X_s$的正部和负部,
        根据\autoref{Cont-Martingale about convex},$(X_s^+)_{s\geqslant 0}$
        是一个下鞅,所以$\forall s\leqslant t$,
        \begin{equation*}
            \E[X_s^+]\leqslant \E[X_t^+]
        \end{equation*}
        另一方面,
        \begin{equation*}
            \E[X_s^-]=\E[X_s^+]-\E[X_s]\leqslant \E[X_t^+]-\E[X_0]
        \end{equation*}
        于是
        \begin{equation*}
            \fun{sup}{0\leqslant s\leqslant t}\E[|X_s|]
            =\fun{sup}{0\leqslant s\leqslant t}( \E[X_s^+]+\E[X_s^-] )
            \leqslant 2\E[X_t^+]-\E[X_0]<+\infty
        \end{equation*}
    \end{proof}

    \begin{theorem}[两个不等式]
        $X=(X_t)_{t\geqslant 0}$为上鞅,且其轨道右连续,则我们有以下结论:
        \begin{enumerate}[(1).]
            \item 最大值不等式:对于$\forall \lambda>0$,有
            \begin{equation*}
                \P\left( \fun{sup}{0\leqslant s\leqslant t}|X_s|>\lambda \right)
                \leqslant 
                \frac{1}{\lambda} \left( 2\E[ |X_t| ]+\E[ |X_0| ] \right)
            \end{equation*}
            \item Doob不等式:对于$p>1$和$t>0$,有
            \begin{equation*}
                \E\left[ \fun{sup}{0\leqslant s\leqslant t}|X_s|^p \right]\leqslant 
                \left( \frac{p}{p-1} \right)^p\E[ |X_t|^p ]
            \end{equation*}    
        \end{enumerate}
    \end{theorem}
    \begin{proof}
        我们的思路是用离散鞅来逼近连续鞅。
        首先,固定$t>0$,取$D$是$[0,t]$的可数稠密子集,
        并且满足$0\in D,t\in D$,设有一列集合$D_m\nearrow D$,
        其中$D_m$具有以下形式:
        \begin{equation*}
            D_m=\{ 0=t_0^m<t_1^m<\cdots<t_m^m=t \}
        \end{equation*}
        固定$m$,我们考虑$\{Y_n=X_{t_{n\wedge m}},n\in\N_+\}$,
        这是一个关于$(\mathcal{G}_t=\F_{t_{n\wedge m}})$的上鞅,
        利用离散鞅的\autoref{Doob's Inquality},
        \begin{equation*}
            \lambda \P( \fun{sup}{s\in D_m}|X_s|>\lambda )\leqslant \E[ |X_0| ]+2\E[|X_t|]
        \end{equation*}
        由概率测度的上连续性,
        \begin{equation*}
            \P( \fun{sup}{s\in D}|X_s|>\lambda )
            =
            \fun{lim}{m\rightarrow\infty}
            \P( \fun{sup}{s\in D_m}|X_s|>\lambda )\leqslant \frac{1}{\lambda}(\E[|X_0|]+2\E[|X_t|])
        \end{equation*}
        $X_s$的轨道右连续,所以
        \begin{equation*}
            \P( \fun{sup}{s\in [0,t]}|X_s|>\lambda )=\P( \fun{sup}{s\in D}|X_s|>\lambda )\leqslant \frac{1}{\lambda}(\E[|X_0|]+2\E[|X_t|])
        \end{equation*}

        第二个结论类似,利用离散情形的\autoref{Lp Maximum Inquality}即可。
    \end{proof}    

\clearpage
\section{鞅的收敛性}
\subsection{轨道修正}
    在鞅的定义中,我们没有对轨道$t\mapsto X_t(\omega)$的连续性作任何要求,
    这导致我们无法从“离散”来逼近“连续”(在布朗运动那一章我们经常使用这个技巧)。
    本小节的目的在于介绍某些情况下可以对鞅的轨道进行修正,从而使得其具有良好的
    轨道性质。
    \begin{definition}[上穿次数]
        对于函数$f:I\rightarrow \R$,其中$I\subset \R_+$,$a<b$为常数,
        定义$f$在$I$上关于区间$(a,b)$的上穿次数:
        \begin{equation*}
            M_{a,b}^f(I)\defeq{\rm sup}\{ k\in \N_+: \exists \{s_1<t_1<s_2<t_2<\cdots<s_k<t_k\}\subset I{\rm\ s.t.\ }f(s_i)\leqslant a,f(t_i)\geqslant b,\forall i\in [k] \}
        \end{equation*}
        如果右侧集合为空集,就取$M_{a,b}^f(I)=0$,即没有穿越出区间$(a,b)$.
    \end{definition}

    下面是一个分析的结论。
    \begin{lemma}\label{analysis of continous}
        $D\mathop{\subset}\limits^{\rm dense} \R_+$,$f:D\rightarrow \R$,设对于任意的$t\in D$,都有:
        \begin{enumerate}[$1^\circ$]
            \item $f$在$D\cap [0,t]$上有界;
            \item 任取有理数$a<b$,都有:
                \begin{equation*}
                    M_{a,b}^f( D\cap [0,t] )<+\infty
                \end{equation*}
        \end{enumerate}
        那么如下左、右极限存在:
        \begin{equation*}
            f(t-)=\fun{lim}{D\ni s\nearrow t}f(s),\ \forall t>0
        \end{equation*}
        \begin{equation*}
            f(t+)=\fun{lim}{D\ni s\searrow t}f(s),\ \forall t\leqslant 0
        \end{equation*}
        进一步地,定义$g(t)=f(t+)$,则$g(t)$右连续且左极限存在,简称右连左极(càdlàg,RCLL)。
    \end{lemma}
    \begin{proof}
        (反证)假设对于某个$t>0$,左极限不存在,那么存在有理数$a<b$使得
        \begin{equation*}
            \fun{liminf}{D\ni s\nearrow f} f(s)<a<b<\fun{limsup}{D\ni s\nearrow f}f(s)
        \end{equation*}
        这表明$M_{a,b}^f(D\cap [0,t])=\infty$,矛盾。

        右极限的情形类似。
    \end{proof}
    \begin{theorem}
        $X=(X_t)_{t\geqslant 0}$为一个关于$(\F_t)_{t\geqslant 0}$的(上)鞅,$D\mathop{\subset}\limits^{\rm dense} \R_+$,那么
        \begin{enumerate}[(1).]
            \item 以下极限a.s.存在:
            \begin{equation*}
                f(t-)=\fun{lim}{D\ni s\nearrow t}f(s),\ \forall t>0
            \end{equation*}
            \begin{equation*}
                f(t+)=\fun{lim}{D\ni s\searrow t}f(s),\ \forall t\leqslant 0
            \end{equation*}
            \item 对于$t\geqslant 0$,$X_{t+}\in L^1$且
                \begin{equation*}
                    X_t\geqslant \E[ X_{t+}|\F_t ]
                \end{equation*}
                其中等号成立当且仅当$t\mapsto \E[X_t]$右连续。
        \end{enumerate}
        因此,$(X_{t+})_{t\geqslant 0}$是一个关于$(\F_{t+})_{t\geqslant 0}$的(上)鞅。
    \end{theorem}
    \begin{proof}
        (1).固定$t\in D$,由最大值不等式可得
        \begin{equation*}
            \P( \fun{sup}{s\in D\cap [0,t]} |X_s|>\lambda )\leqslant \frac{1}{\lambda}( 2\E[ |X_T| ]+\E[ |X_0| ] )
        \end{equation*}
        令$\lambda \rightarrow \infty$可得
        \begin{equation*}
            \fun{sup}{s\in D\cap [0,t]}|X_s|<+\infty
        \end{equation*}
        选择$D$的一列有限子集$D_m$,满足:
        \begin{equation*}
            0,t\in D_m,\ D_m\nearrow D,\ D\cap [0,t]=\bigcup_{m} D_m
        \end{equation*}
        而$(X_t)$限制在$D_m$上是离散时间鞅,由离散时间鞅的上穿不等式(\autoref{thm3.4}),有
        \begin{equation*}
            \E[ M_{a,b}^X(D_m) ]\leqslant \frac{1}{b-a}\E[ (X_T-a)^- ]
        \end{equation*}
        令$m\rightarrow\infty$,由Fatou引理可得
        \begin{equation*}
            \E[ M_{a,b}^X(D\cap [0,t]) ]\leqslant \fun{liminf}{m\rightarrow\infty}\E[ M_{a,b}^X(D_m) ]\leqslant \frac{1}{b-a}\E[ (X_T-a)^- ]<+\infty
        \end{equation*}
        于是由\autoref{analysis of continous}则得证。

        (2).定义:
        \begin{equation*}
            X_{t+}(\omega)=\left\{ \begin{array}{ll}
                \fun{lim}{D\ni s\searrow t}X_s(\omega)&,\text{如果此极限存在}\\
                0&,\text{其他情况}
            \end{array} \right.
        \end{equation*}
        那么,其他情况的集合(是个零测集)可以记作
        \begin{equation*}
            N=\bigcup_{t\in D}\left( \left\{ \fun{sup}{t\in D\cap [0,t]}|X_t|=+\infty \right\}\cup \left\{ \bigcup_{a<b,a,b\in \Q} \{ M_{a,b}^X(D\cap [0,t])=+\infty \} \right\} \right)
        \end{equation*}
        \begin{enumerate}[{\rm Step1.}]
            \item 选取$D\ni t_n\searrow t,t_n>t$,由构造可知
                \begin{equation*}
                    X_{t+}=\fun{lim}{n\rightarrow\infty} X_{t_n}
                \end{equation*}
                记$Y_k=X_{t_{-k}},k\leqslant 0$,那么$(Y_k)$是一个向后鞅,而且
                \begin{equation*}
                    \fun{sup}{k\leqslant 0}\E[ |Y_k| ]
                    =
                    \fun{sup}{k\geqslant 0}\E[ |X_{t_k}| ]<+\infty
                \end{equation*}
                那么
                \begin{equation*}
                    Y_k\ra{L^1}Y_{-\infty}=X_{t+}\in L^1
                \end{equation*}
            \item 由$t_n>t$,
                \begin{equation*}
                    X_t\geqslant \E[ X_{t_n}|\F_t ]
                \end{equation*}
                令$n\rightarrow\infty$,可得
                \begin{equation*}
                    X_t\geqslant \E[X_{t+}|\F+t]
                \end{equation*}
                我们希望证明等号成立,那么只需证明:
                \begin{equation*}
                    \E[X_t]=\E[ \E[X_{t+}|\F+t] ]=\E[X_{t+}]
                \end{equation*}
                而
                \begin{equation*}
                    \E[X_{t+}]=\E[ \fun{lim}{n\rightarrow\infty} X_{t_n} ]
                    =\fun{lim}{n\rightarrow \infty}\E[ X_{t_n} ]
                    =\E[X_t]
                \end{equation*}
            \item 最后我们说明$(X_{t+})$是$(\F_{t+})$的上鞅,对于$s<t$,
                取$D$中序列$s_n\searrow s,t_n\searrow t,s_n\leqslant t_n$,取
                \begin{equation*}
                    A\in \F_{s+}\bigcap_{s_n}\F_{s_n}
                \end{equation*}
                那么
                \begin{equation*}
                    A\in \F_{s_n},\ \forall n\geqslant 1
                \end{equation*}
                于是
                \begin{align*}
                    \E[X_{s+}I_A]=\E[ \fun{lim}{n\rightarrow\infty}X_{s_n}I_A ]
                    &=\fun{lim}{n\rightarrow\infty} \E[ X_{s_n}I_A ]\\
                    &\geqslant \fun{lim}{n\rightarrow\infty} \E[ X_{t_n}I_A ]\\
                    &=\E[ X_{t+}I_A ]\\
                    &=\E[ \E[X_{t+}I_A\F_{s+}] ]
                    &=\E[\E[ X_{t+}\F_{s+} ]I_A]
                \end{align*}
                由此可得:
                \begin{equation*}
                    \E[ X_{t+}|\F_{s+} ]\leqslant X_{s+}
                \end{equation*}
                鞅的情形类似。
        \end{enumerate}
        这里利用了一个结论:$X,Y$关于$\F$可测,
        那么$X\leqslant Y$当且仅当$\E[ XI_A ]\leqslant \E[ YI_A ],\forall A\in \F$.
    \end{proof}

    \begin{theorem}[轨道修正]
        $(\F_t)_{t\geqslant 0}$右连续且完备,$X=(X_t)_{t\geqslant 0}$是
        关于$(\F_t)_{t\geqslant 0}$的上鞅,且$t\mapsto \E[X_t]$右连续,那么
        存在一个$(X_t)$的修改$(Y_t)$,其也是一个上鞅,且具有RCLL轨道。
    \end{theorem}
    \begin{proof}
        只需定义$Y_t=X_{t+}(\omega)I_{ \{ \fun{lim}{D\ni s\searrow t}X_s(\omega)\text{存在} \} }$,
        则$(Y_t)$是关于$(\F_{t+})=(\F_t)$的上鞅,
        那么
        \begin{equation*}
            X_t=\E[ X_{t+}|\F_t ]=\E[ X_{t+}|\F_{t+} ]=X_{t+}=Y_t{\rm\ a.s.}
        \end{equation*}
    \end{proof}
\subsection{右连续鞅的收敛性}
    \begin{theorem}[鞅收敛定理]\label{Cont-Martingale Convergence Theorem}
        $X$是上鞅,且有右连续轨道,如果
        \begin{equation*}
            \fun{sup}{0\leqslant t<\infty} \E[ |X_t| ]<+\infty
        \end{equation*}
        那么存在一个随机变量$X_\infty\in L^1$,使得
        \begin{equation*}
            \fun{lim}{t\rightarrow\infty}X_t(\omega)=X_\infty(\omega){\rm\ a.s.}
        \end{equation*}
    \end{theorem}
    \begin{proof}
        设$D$是$\R_+$的一个可数稠密子集,对于任意的$t\in D$,有理数$a<b$,有
        \begin{equation*}
            \E[ M_{a,b}^X(D\cap [0,t]) ]
            \leqslant \frac{1}{b-a}\E[ (X_T-a)^- ]
            \leqslant \frac{1}{b-a}\left( \fun{sup}{0\leqslant t<\infty}\E[|X_t|]+a \right)<+\infty
        \end{equation*}
        令$t\rightarrow\infty$可得
        \begin{equation*}
            \E[ M_{a,b}^X(D) ]<+\infty
        \end{equation*}
        从而可得
        \begin{equation*}
            M_{a,b}^X(D)<+\infty\ a.s.
        \end{equation*}
        所以存在
        \begin{equation*}
            X_\infty=\fun{lim}{D\ni t\rightarrow\infty}{\rm\ a.s.}
        \end{equation*}
        由Fatou引理,
        \begin{equation*}
            \E[ |X_\infty| ]=\E[ | \fun{lim}{D\ni t\rightarrow\infty} X_t | ]
            =\E[ \fun{lim}{D\ni t\rightarrow\infty}|X_t| ]
            \leqslant 
            \fun{liminf}{D\ni t\rightarrow\infty} \E[|X_t|]<+\infty
        \end{equation*}
        因此$X_\infty\in L^1$.

        我们目前只说明了$D$上的收敛性,下面将其扩充到$\R_+$上。对于$\forall \varepsilon>0$,
        存在$N$,当$D\ni t\geqslant N$时,
        \begin{equation*}
            |X_t-X_\infty|\leqslant \varepsilon
        \end{equation*}
        那么对于任意的$s\geqslant N$,存在一系列$s_n\in D$使得$s_n\searrow s$,
        因此
        \begin{equation*}
            |X_{s_n}-X_\infty|\leqslant \varepsilon
        \end{equation*}
        令$n\rightarrow\infty$得到
        \begin{equation*}
            |X_{s}-X_\infty|\leqslant \varepsilon
        \end{equation*}
    \end{proof}
    注意,相比于离散时间鞅的收敛定理(\autoref{Martingale Convergence Theorem}),此处多了一个结论:$X_\infty\in L^1$.

    \begin{definition}
        称一个鞅$(X_t)_{t\geqslant 0}$是闭的(closed),如果存在一个随机变量$Z\in L^1$,使得
        \begin{equation*}
            X_t=\E[Z|\F_t],\forall t\geqslant 0
        \end{equation*}
    \end{definition}
    \begin{theorem}
        $X=(X_t)_{t\geqslant 0}$是一个鞅,且具有右连续轨道,那么以下结论等价:
        \begin{enumerate}[(1).]
            \item $X$闭。
            \item $\{X_t,t\geqslant 0\}$一致可积。
            \item $t\rightarrow\infty$时,$X_t$几乎处处收敛,且依$L^1$收敛。
        \end{enumerate}
    \end{theorem}
    \begin{proof}
        $(1)\Rightarrow (2)$:由于
        \begin{equation*}
            X_t=\E[Z|\F_t],\ t\geqslant 0
        \end{equation*}
        是$Z$生成的“条件期望列”,所以是一致可积的,见\autoref{Integrability of Conditional Expectation}.

        $(2)\Rightarrow (3)$:一致可积可得
        \begin{equation*}
            \fun{sup}{t\geqslant 0}\E[ |X_t| ]<+\infty0
        \end{equation*}
        由\autoref{Cont-Martingale Convergence Theorem}可知$X_t\ra{\rm a.s.} X_\infty$,
        结合一致可积性可知$X_t\ra{L^1} X_\infty$

        $(3)\Rightarrow (1)$:对于$\forall s>t$,
        \begin{equation*}
            X_t=\E[X_s|\F_t]
        \end{equation*}
        令$s\rightarrow\infty$,
        \begin{equation*}
            X_t=\E[X_\infty|\F_t]
        \end{equation*}
        这说明$X$闭。
    \end{proof}

\clearpage
\section{鞅的择停定理}
    这一小节中,我们承认以下命题成立。
    \begin{proposition}
        $X=(X_t)_{t\geqslant 0}$是循序过程,适应$(\F_{t})_{t\geqslant 0}$,
        $T$是关于$(\F_{t})_{t\geqslant 0}$的停时,
        则
        \begin{equation*}
            X_T\cdot I_{T<+\infty}
        \end{equation*}
        是$\F_T$-可测的。
        
        进一步地,如果$\fun{lim}{t\rightarrow\infty}X_t=X_\infty\in \F_\infty$ a.s.,
        定义
        \begin{equation*}
            X_T(\omega)=X_\infty(\omega)I_{ \{T(\omega)=+\infty\} }+X_{T(\omega)}(\omega)I_{ \{T<+\infty\} }
        \end{equation*}
        那么,$X_T$是$\F_T$-可测的。
    \end{proposition}
    本节内容中我们大部分情况下假设鞅是在实数(度量)空间上取值、轨道右连续的,
    而我们在\autoref{progressive from right-continous}说明过,
    具有右连续轨道的适应过程是循序过程,
    所以我们可以直接使用$X_T\in \F_T$这个结论,
    以下不再作特殊说明。    

    Doob择停定理首先给出了一致可积、右连续情形下的结论。
    \begin{theorem}[Doob择停定理]\label{Doob's Cont-Martingale Optional Stopping Time Theorem}
        $X=(X_t)_{t\geqslant 0}$是一致可积鞅,且具有右连续轨道,
        那么对于任意停时$S\leqslant T$,有$X_S,X_T\in L^1$,且
        \begin{equation*}
            X_S=\E[X_T|\F_S]
        \end{equation*}
        特别地,取$T=\infty$,那么
        \begin{equation*}
            X_S=\E[X_\infty|\F_S]
        \end{equation*}
        \begin{equation*}
            \E[X_0]=\E[X_S]=\E[X_\infty]
        \end{equation*}
    \end{theorem}
    \begin{proof}
        对于任意的$n\in\N$,记
        \begin{equation*}
            T_n=\sum_{k=0}^\infty \frac{k+1}{2^n}I_{ \{ \frac{k}{2^n}<T\leqslant \frac{k+1}{2^n} \} }+(+\infty)\cdot I_{ \{T=+\infty\} }
        \end{equation*}
        \begin{equation*}
            S_n=\sum_{k=0}^\infty \frac{k+1}{2^n}I_{ \{ \frac{k}{2^n}<S\leqslant \frac{k+1}{2^n} \} }+(+\infty)\cdot I_{ \{S=+\infty\} }
        \end{equation*}
        这是我们在\autoref{cor of order S.T.}提到的构造,$T_n,S_n$都是停时,
        并且$T_n\searrow T,S_n\searrow S$.

        记
        \begin{equation*}
            Y_k^{(n)}=X_{k\cdot 2^{-n}},\ k\geqslant 0
        \end{equation*}
        \begin{equation*}
            \mathcal{H}_k^{(n)}=\F_{k\cdot 2^{-n}}
        \end{equation*}
        那么$\{Y_k^{(n)}\}$是一致可积的,由离散鞅的择停定理:
        \begin{equation*}
            X_{S_n}=T_{2^n\cdot S_n}^{(n)}
            =\E[ Y_{2^n\cdot T_n}^{(n)}|\mathcal{H}_{2^n\cdot S_n}^{(n)} ]
            =\E[X_{T_n}|\F_{S_n}]
        \end{equation*}
        任取$A\in \F_S\subset \F_{S_n}$,由条件期望的定义,
        \begin{equation*}
            \E[X_{S_n}I_A]=\E[X_{T_n}I_A]
        \end{equation*}
        结合轨道的右连续性,我们知道
        \begin{equation*}
            X_S=\fun{lim}{n\rightarrow\infty}X_{S_n},X_T=\fun{lim}{n\rightarrow\infty}X_{T_n}{\rm\ a.s.}
        \end{equation*}
        再结合一致可积性,可知上述收敛在$L^1$也成立,因此当$n\rightarrow\infty$时可得
        \begin{equation*}
            \E[X_SI_A]=\E[X_TI_A],\ \forall A\in \F_S
        \end{equation*}
        所以$X_S=\E[X_T|\F_S]$,然后我们接下来证明$X_S,X_T\in L^1$,
        注意到
        \begin{equation*}
            X_{S_n}=\E[X_\infty|\F_{S_n}],\forall n\in\N_+
        \end{equation*}
        所以$X_{S_n}\in L^1$,因此$X_S\in L^1$.
    \end{proof}

    如果我们不假设一致可积性,而假设停时有界,也有一样的结论。
    \begin{theorem}\label{Cont-Martingale Optional Stopping Time Theorem with Bounded S.T.}
        $X=(X_t)_{t\geqslant 0}$是一个鞅,且具有右连续轨道,
        如果$S\leqslant T$是两个有界的停时,那么
        \begin{equation*}
            X_S,X_T\in L^1
        \end{equation*}
        \begin{equation*}
            X_S=\E[X_T|\F_S]
        \end{equation*}
    \end{theorem}
    \begin{proof}
        不妨设$S\leqslant T\leqslant a$,
        考虑如下的鞅:$Y_t=\E[X_a|\F_t]=X_{t\wedge a}$,
        所以$Y_t$是闭的,进而是一致可积的,可得
        \begin{equation*}
            Y_S=\E[Y_T|\F_S]
            \Leftrightarrow
            X_{S\wedge a}=\E[X_{T\wedge a}|\F_S]
            \Leftrightarrow
            X_S=\E[X_T|\F_S]
        \end{equation*}
        同时,由于$Y_S,Y_T\in L^1$,所以$X_S,X_T\in L^1$.
    \end{proof}

    下面这个定理给出了停止鞅$(X_{t\wedge T})$和原来的鞅之间的关系。
    \begin{theorem}\label{Relationship between Martingale and Stoped-Martingale}
        $(X_t)_{t\geqslant 0}$是一个鞅,且具有右连续轨道,$T$是停时,那么
        \begin{enumerate}[(1).]
            \item $(X_{t\wedge T})_{t\geqslant 0}$也是一个鞅。
            \item 如果$(X_t)$一致可积,那么$(X_{t\wedge T})$一致可积,因此对于$\forall t\geqslant 0$,
                \begin{equation*}
                    X_{t\wedge T}=\E[X_{\infty\wedge T}|\F_t]=\E[ X_T|\F_t ]
                \end{equation*}
        \end{enumerate}
    \end{theorem}
    \begin{proof}
        我们先说明(2),希望证明:
        \begin{equation*}
            X_{t\wedge T}=\E[ X_T|\F_t ]
        \end{equation*}
        对于停时$T$,任取常数$t\geqslant 0$,$T\wedge t$也是一个停时,因此$X_{t\wedge T}\in \F_{t\wedge T}\subset \F_t$,
        接下来只需证明对于$\forall A\in \F_t$都有
        \begin{equation*}
            \E[ X_{t\wedge T}I_A ]=\E[ X_TI_A ]
        \end{equation*}
        注意到
        \begin{align*}
            \E[ X_{t\wedge T}I_A ]
            &=\E[ X_{t\wedge T}I_{A\cap \{T\geqslant t\}} ]+\E[ X_{t\wedge T}I_{A\cap \{T< t\}} ]\\
            &=\E[ X_{t\wedge T}I_{A\cap \{T\geqslant t\}} ]+\E[ X_{T}I_{A\cap \{T<t\}} ]
        \end{align*}
        所以我们只需证明
        \begin{equation*}
            \E[ X_{t\wedge T}I_{A\cap \{T\geqslant t\}} ]=\E[ X_{T}I_{A\cap \{T\geqslant t\}} ]
        \end{equation*}
        因为$(X_t)_{t\geqslant 0}$是一致可积鞅,我们对$t\wedge T\leqslant T$这两个停时应用择停定理,可得
        \begin{equation*}
            X_{t\wedge T}=\E[ X_T|\F_{w\wedge T} ]\tag*{$(\star)$}
        \end{equation*}
        而对于$A\in \F_t$,
        因为$\{ T\geqslant t \}\in \F_t$,
        因此$A\cap \{ T\geqslant t \}\in \F_t$,同时$\forall s\geqslant 0$,
        \begin{equation*}
            A\cap \{ T\geqslant t \}\cap \{ T\leqslant s \}=\left\{ \begin{array}{ll}
                \varnothing&,t>s\\
                \mathop{A\cap \{ T\geqslant t \}}\limits_{\in \F_t\subset \F_s}\cap \{ T\leqslant s \}&,t\leqslant s
            \end{array} \right.\ \in \F_s
        \end{equation*}
        因此$A\cap \{ T\geqslant t \}\in \F_{T}$,
        从而$A\cap \{ T\geqslant t \}\in \F_{t\wedge T}$,
        于是由$(\star)$式可知
        \begin{equation*}
            \E[ X_{t\wedge T}I_{A\cap\{ T\geqslant t \} } ]=\E[X_TI_{A\cap\{ T\geqslant t \} }]
        \end{equation*}
        于是得证。

        再来说明(1),任取常数$a>0$,那么$X_{t\wedge a}=\E[X_a|\F_t]$是一致可积鞅,进而由(2)可知$X_{t\wedge a\wedge T}$是一致可积鞅,
        也就是说,在$t\in [0,a]$时,$X_{t\wedge T}$是鞅,令$a\rightarrow\infty$即可。
    \end{proof}

    最后,我们给出非负上鞅的择停定理,它也不需要一致可积性的条件。
    \begin{theorem}\label{Non-negative Supermartingale Optional S.T. Theorem}
        $Z=(Z_t)_{t\geqslant 0}$是非负上鞅,且具有右连续轨道,
        如果$U\leqslant V$是两个停时,那么$Z_U,Z_V\in L^1$,且
        \begin{equation*}
            Z_U\geqslant \E[Z_V|\F_U]
        \end{equation*}
    \end{theorem}
    \begin{proof}
        先假设$U,V$都是有界停时,即$U\leqslant V\leqslant P$,$P$为常数,对于$n\in\N$,定义
        \begin{equation*}
            U_n=\sum_{k=0}^{ \lfloor P\cdot 2^n-1 \rfloor }\frac{k+1}{2^n}I_{ \{ \frac{k}{2^n}<U\leqslant \frac{k+1}{2^n} \} }
        \end{equation*}
        \begin{equation*}
            V_n=\sum_{k=0}^{ \lfloor P\cdot 2^n-1 \rfloor }\frac{k+1}{2^n}I_{ \{ \frac{k}{2^n}<V\leqslant \frac{k+1}{2^n} \} }
        \end{equation*}
        那么$U_n\searrow U,V_n\searrow V$,结合$(Z_t)$的有连续性,有
        \begin{equation*}
            Z_{U_n}\rightarrow Z_U,\ Z_{V_n}\rightarrow Z_V
        \end{equation*}
        由离散时间上鞅的Doob择停定理(有界停时),
        \begin{equation*}
            Z_{U_{n+1}}\geqslant \E[ Z_{U_n}|\F_{U_{n+1}} ]
        \end{equation*}
        此时,令
        $Y_n=Z_{U_{-n}}$,以及$\mathcal{H}_n=\F_{U_{-n}}$,那么$(Y_n)$是一个向后上鞅,所以
        \begin{equation*}
            \E[Y_n]=\E[Z_{U_{-n}}]\leqslant \E[Z_0]<+\infty
        \end{equation*}
        所以$\fun{sup}{n}\E[Y_n]<+\infty$,
        结合鞅收敛定理,可得
        $Z_{U_n}\ra{L^1}Z_U$和$Z_{V_n}\ra{L^1}Z_V$,
        由离散时间上鞅的Doob择停定理,
        \begin{equation*}
            \E[Z_{U_n}]\geqslant \E[Z_{V_n}]
        \end{equation*}
        令$n\rightarrow\infty$可得$\E[Z_n]=\E[U_n]$.

        现在考虑$U,V$不有界的情况,对于$P\geqslant 1$,考虑$U\wedge p$和$V\wedge p$
        是两个有界的停时,那么
        \begin{equation*}
            \E[Z_{U\wedge P}]\leqslant \E[Z_0],\ 
            \E[Z_{V\wedge P}]\leqslant \E[Z_0]
        \end{equation*}
        令$P\rightarrow\infty$,由Fatou引理,可得
        \begin{equation*}
            \E[Z_U],\E[Z_V]\leqslant \E[Z_0]<+\infty,
        \end{equation*}
        由此可知$Z_U,Z_V\in L^1$.

        接下来我们证明$Z_U\geqslant \E[Z_V|\F_U]$,其等价于证明$\forall A\in \F_U$,
        \begin{equation*}
            \E[Z_UI_A]\geqslant \E[Z_VI_A]
        \end{equation*}
        对 $A\in\mathcal{F}_U\subset\mathcal{F}_V$, 定义:
        \begin{equation*}
            U^A=\begin{cases}U,&\omega\in A\\[2ex]\infty,&\omega\in A^c\end{cases},\quad V^A=\begin{cases}V,&\omega\in A\\[2ex]\infty,&\omega\in A^c\end{cases}
        \end{equation*}
        那么$U^A$和$V^A$都是停时且其满足$U^A\leqslant V^A$.

        对任意的$P\geqslant 1$,有
        \begin{equation*}
            U^A\wedge P\leqslant V^A\wedge P
        \end{equation*}
        因此
        \begin{equation*}
            E[Z_{U^A\wedge P}]\geqslant E[Z_{V^A\wedge P}]
        \end{equation*}
        事实上,
        \begin{align*}
            \mathrm{LHS}
            &=E[Z_{U^{A}\wedge P}I_{A^{c}}]+E[Z_{U^{A}\wedge P}I_{A}]\\
            &=E[Z_{P}I_{A^{c}}]+E[Z_{U}I_{A}I\{U\leqslant P\}]+E[Z_{P}I_{A}I\{U>P\}]
        \end{align*}
        同理,
        \begin{align*}
            \mathrm{RHS}&=E[Z_{P}I_{A^{c}}]+E[Z_{V\wedge P}I_{A}I\{U\leqslant P\}]\\
            &+E[Z_{P}I_{A}I\{U>P\}]
        \end{align*}
        所以有
        \begin{equation*}
            E[Z_{U}I_{A}I\{U\leqslant P\}]\geqslant E[Z_{V\wedge P}I_{A}I\{U\leqslant P\}]
        \end{equation*}
        令$p\rightarrow\infty$,由Fatou引理可得
        \begin{align*}
            E[Z_{U}I_{A}I\{U<\infty\}]
            &\geqslant\operatorname*{lim}_{P\to\infty}E[Z_{V\wedge P}I_{A}I\{U\leqslant P\}]\\
            &\geqslant E[\operatorname*{lim}_{P\to\infty}Z_{V\wedge P}I_{A}I\{U\leqslant P\}]\\
            &=E[Z_{V}I_{A}I\{U<\infty\}]
        \end{align*}
        另一方面,由于$U=\infty\implies V=\infty$,所以有
        \begin{align*}
            E[Z_{U}I_{A}I\{U=\infty\}]
            &=E[Z_{V}I_{A}I\{V=\infty\}]\\
            &=E[Z_{\infty}I_{A}I\{U=\infty\}]
        \end{align*}
        即$E[Z_UI_A]\geqslant E[Z_VI_A]$,其与之前的式子相加,即可得到结论。
    \end{proof}

\clearpage
\section{例子与习题}
这一章的习题大概分为两类,
一类是各种收敛性可积性的证明,
另一类则是利用鞅的择停定理(总结:一致可积鞅的停止鞅也一致可积;右连续的前提下,一致可积鞅/停时有界/非负(上)鞅,可以使用择停定理,注意非负(上)鞅结论是不等号)
的应用。后者往往技巧性较强,记一下\autoref{Martingale generated by Independent Increment}以及其延伸出的两个例子
里面鞅的构造就行了,更抽象的构造要是考了那确实没办法。

下面是一个上课讲的例子。
\begin{example}
    $B=(B_t)_{t\geqslant 0}$是布朗运动,对于常数$c$,记停时
    \begin{equation*}
        T_c=\fun{inf}{}\{ t\geqslant 0:B_t=c \}
    \end{equation*}
    即首次到达$c$的时刻。设$a<0<b$,$T=T_a\wedge T_b$,
    \begin{enumerate}[(1).]
        \item 求$\P(T_a<T_b)$.
        \item 求$\E[T]$.
        \item 求$\E[ {\rm e}^{-\lambda T_a} ]$,其中$\lambda>0$.
        \item 设$a+b=0$,即
            \begin{equation*}
                T=\fun{inf}{}\{t\geqslant 0:|B_t|=b\}
            \end{equation*}
            求$\E[ T^2 ]$.
    \end{enumerate}
\end{example}
\begin{solve}
    \begin{enumerate}[(1).]
        \item 我们注意到鞅$B_{t\wedge T}\in [a,b]$有界,从而一致可积,
        由择停定理可知
        \begin{equation*}
            0=\E[B_{0\wedge T}]=\E[B_{\infty\wedge T}]=\E[B_T]=a\cdot \P(T_a<T_b)+b\cdot \P(T_a>T_b)
        \end{equation*}
        而且$\P(T_a<T_b)+\P(T_a>T_b)=1$,解得
        \begin{equation*}
            \P(T_a<T_b)=\frac{b}{b-a},\ \P(T_a>T_b)=\frac{-a}{b-a}
        \end{equation*}
        \item 注意到$X_t=B_t^2-t$是一个鞅,那么$X_{t\wedge T}$是一个鞅,可知
        \begin{equation*}
            0=\E[X_{0\wedge T}]=\E[X_{t\wedge T}]
        \end{equation*}
        所以
        \begin{equation*}
            \E[t\wedge T]=\E[ B_{t\wedge T}^2 ]
        \end{equation*}
        令$t\rightarrow\infty$,因为$B_{t\wedge T}^2\in [0,a^2\vee b^2]$有界,$t\wedge T$单调,
        利用控制收敛定理和单调收敛定理可知
        \begin{equation*}
            \E[T]=\E[B_T^2]=a^2\P(T_a<T_b)+b^2\P(T_a>T_b)=-ab
        \end{equation*}
        \item 对于$\theta>0$,考虑如下的鞅:
        \begin{equation*}
            N_t={\rm exp}\left\{ \theta B_t-\frac{\theta^2}{2}t \right\}
        \end{equation*}
        那么
        \begin{equation*}
            N_{t\wedge T_a}\leqslant {\rm exp}\{ \theta B_{t\wedge T_a} \}\leqslant {\rm exp}\{ \theta a \}
        \end{equation*}
        一致有界$\Rightarrow (N_{t\wedge T_a})$一致可积,由Doob择停定理可知
        \begin{equation*}
            1=\E[ N_{0} ]=\E[ N_{T_a} ]=\E[{\rm exp}\{ \theta a-\frac{\theta^2}{2}T_a \}]
        \end{equation*}
        即
        \begin{equation*}
            \E[ {\rm exp}\{ -\frac{1}{2}\theta^2T_a \}]={\rm exp}\{ -\theta a \}
        \end{equation*}
        取$\theta=\sqrt{2\lambda}$就得到
        \begin{equation*}
            \E[ {\rm e}^{-\lambda T_a} ]={\rm e}^{-a\sqrt{2\lambda}}
        \end{equation*}
        \item 考虑如下的鞅:
            \begin{equation*}
                K_t=B_t^4-6B_t^2 t+3t^2
            \end{equation*}
            具体过程省略。
    \end{enumerate}
\end{solve}

\begin{ex}[le gall(Exercise3.26)][le gall(Exercise3.26)]
    \begin{enumerate}
        \item $M=(M_t)_{t\geqslant 0}$是非负鞅,具有右连续轨道,$M_0=x$ a.s.,
        并且
        \begin{equation*}
            \fun{lim}{t\rightarrow\infty}M_t=0{\rm\ a.s.}
        \end{equation*}
        证明:对于$\forall y>x$,
        \begin{equation*}
            \P( \fun{sup}{t\geqslant 0}M_t\geqslant y )=\frac{x}{y}
        \end{equation*}
        \item $B=(B_t)_{t\geqslant 0}$是初值为$x>0$的布朗运动,
        $T_0\fun{inf}{}\{ t\geqslant 0:B_t=0 \}$,
        求
        \begin{equation*}
            \fun{sup}{t\leqslant T_0} B_t
        \end{equation*}
        的分布。
        \item $B=(B_t)_{t\geqslant 0}$是初值为$0$的布朗运动,$\mu>0$,
        利用合适的指数鞅,证明
        \begin{equation*}
            \fun{sup}{t\geqslant 0} (B_t-\mu t)
        \end{equation*}
        满足参数为$2\mu$的指数分布。
    \end{enumerate}
\end{ex}
\begin{solve}
    \begin{enumerate}
        \item 设
            \begin{equation*}
                T_y=\fun{inf}{}\{ t\geqslant 0:M_t\geqslant y \}
            \end{equation*}
            因为$M_t$具有(右)连续轨道,所以$(M_{t\wedge T_y})$也是一个鞅,那么
            \begin{equation*}
                \E[ M_{t\wedge T_y} ]=\E[ M_{0\wedge T_y} ]=x
            \end{equation*}
            而$M_{t\wedge y}\in [0,y]$,所以由控制收敛定理可得
            \begin{equation*}
                \E[ M_{T_y}I_{\{T_y<+\infty\}}+M_{\infty}I_{\{T_y=\infty\}} ]
                =\E[yI_{\{T_y<+\infty\}}]=x
            \end{equation*}
            从而
            \begin{equation*}
                \P(\fun{sup}{t\geqslant 0}M_t\geqslant y)=\P(T_y<+\infty)=\frac{x}{y}
            \end{equation*}
        \item 注意到$(B_{t\wedge T_0})$满足1.中条件,所以$\forall y\geqslant 0$,
            \begin{equation*}
                \P( \fun{sup}{t\leqslant T_0}B_t\geqslant y )
                =
                \P( \fun{sup}{t\geqslant 0}B_{t\wedge T_0}\geqslant y )
                =\frac{x}{y}
            \end{equation*}
        \item 观察$\fun{sup}{t\geqslant 0}(B_t-\mu t)$,我们可以作以下变换:
            \begin{align*}
                \P( \fun{sup}{t\geqslant 0}(B_t-\mu t)\geqslant y )
                &=\P( \fun{sup}{t\geqslant 0}(B_{\frac{1}{4\mu^2}t}-\mu\cdot \frac{1}{4\mu^2}t)\geqslant y )\\
                &=\P( \fun{sup}{t\geqslant 0}(2\mu B_{\frac{1}{4\mu^2}t}-\frac{1}{2}t)\geqslant 2\mu y )\\
                &=\P( \fun{sup}{t\geqslant 0}(B_{t}-\frac{1}{2}t)\geqslant 2\mu y )\\
                &=\P( \fun{sup}{t\geqslant 0}{\rm e}^{B_{t}-\frac{1}{2}t}\geqslant {\rm e}^{2\mu y} )
            \end{align*}
            其中$N_t={\rm e}^{B_{t}-\frac{1}{2}t}$是一个非负连续鞅,并且$N_t\ra{\rm a.s.}0$,
            $N_0=1$,满足1.中条件,所以$\forall y>0$,
            \begin{equation*}
                \P( \fun{sup}{t\geqslant 0}(B_t-\mu t)\geqslant y )=
                \P( \fun{sup}{t\geqslant 0} N_t\geqslant {\rm e}^{2\mu y} )={\rm e}^{-2\mu y}
            \end{equation*}
            $y\leqslant 0$时显然概率为$1$,于是$\fun{sup}{t\geqslant 0}(B_t-\mu t)$满足参数为$2\mu$的指数分布。
    \end{enumerate}
\end{solve}

\begin{ex}[le gall(Exercise3.27)][le gall(Exercise3.27)]
    $B=(B_t)_{t\geqslant 0}$是初值为$0$的布朗运动,$(\F_t=\sigma( B_s,s\in [0,t] ))$,
    记
    \begin{equation*}
        T_x=\fun{inf}{}\{ t\geqslant 0:B_t=x \}
    \end{equation*}
    并设$T=T_a\wedge T_b$,其中$a<0<b$为常数,对于$\lambda>0$,求:
    \begin{enumerate}
        \item $\E[ {\rm e}^{-\lambda T} ]$.
        \item $\E[ {\rm e}^{-\lambda T}I_{ \{ T=T_a \} } ]$.
        \item $\P(T_a<T_b)$.
    \end{enumerate}
\end{ex}
\begin{solve}
    \begin{enumerate}
        \item 注意到
            \begin{equation*}
                U_t={\rm e}^{\sqrt{2\lambda}B_t-\lambda t},\ 
                V_t={\rm e}^{-\sqrt{2\lambda}B_t-\lambda t}
            \end{equation*}
            都是鞅,那么对于$\alpha>0$(待定),
            \begin{equation*}
                M_t={\rm e}^{-\alpha \sqrt{2\lambda}}U_t+{\rm e}^{\alpha \sqrt{2\lambda}}V_t
            \end{equation*}
            也是鞅。同时,
            \begin{equation*}
                0\leqslant U_{t\wedge T}\leqslant {\rm e}^{ b\sqrt{2\lambda} },\ 
                0\leqslant V_{t\wedge T}\leqslant {\rm e}^{ -a\sqrt{2\lambda} }
            \end{equation*}
            说明$(U_{t\wedge T})$和$(V_{t\wedge T})$一致可积,
            进而$(M_{t\wedge T})$也一致可积,因此由Doob择停定理,
            \begin{align*}
                {\rm e}^{-\alpha \sqrt{2\lambda}}+{\rm e}^{\alpha \sqrt{2\lambda}}=\E[M_0]
                =&\E[M_T]\\
                =&{\rm e}^{-\alpha \sqrt{2\lambda}}\E[ U_t]+{\rm e}^{\alpha \sqrt{2\lambda}}\E[V_t ]\\
                =&{\rm e}^{-\alpha \sqrt{2\lambda}}\E[ {\rm e}^{\sqrt{2\lambda}\cdot a-\lambda T}I_{ \{ T_a<T_b \} }+{\rm e}^{\sqrt{2\lambda}\cdot b-\lambda T}I_{ \{ T_a>T_b \} } ]\\
                +&{\rm e}^{\alpha \sqrt{2\lambda}}\E[ {\rm e}^{-\sqrt{2\lambda}\cdot a-\lambda T}I_{ \{ T_a<T_b \} }+{\rm e}^{-\sqrt{2\lambda}\cdot b-\lambda T}I_{ \{ T_a>T_b \} } ]
            \end{align*}
            我们这里令$\alpha=\frac{1}{2}(a+b)$,就得到
            \begin{align*}
                \E[M_T]
                =&\E[ {\rm e}^{\sqrt{2\lambda}\cdot \frac{a-b}{2}-\lambda T}I_{ \{ T_a<T_b \} }+{\rm e}^{\sqrt{2\lambda}\cdot \frac{b-a}{2}-\lambda T}I_{ \{ T_a>T_b \} } ]\\
                +&\E[ {\rm e}^{\sqrt{2\lambda}\cdot \frac{b-a}{2}-\lambda T}I_{ \{ T_a<T_b \} }+{\rm e}^{\sqrt{2\lambda}\cdot \frac{a-b}{2}-\lambda T}I_{ \{ T_a>T_b \} } ]\\
                =&\left({\rm e}^{\sqrt{2\lambda}\cdot \frac{a-b}{2}}+{\rm e}^{\sqrt{2\lambda}\cdot \frac{b-a}{2}}\right)\E[ {\rm e}^{-\lambda T} ]
            \end{align*}
            从而
            \begin{equation*}
                \E[ {\rm e}^{-\lambda T} ]=\frac{{\rm e}^{-\frac{a+b}{2} \sqrt{2\lambda}}+{\rm e}^{\frac{a+b}{2} \sqrt{2\lambda}}}{{\rm e}^{\sqrt{2\lambda}\cdot \frac{a-b}{2}}+{\rm e}^{\sqrt{2\lambda}\cdot \frac{b-a}{2}}}
                =\frac{{\rm cosh}(\frac{a+b}{2}\sqrt{2\lambda})}{{\rm cosh}( \frac{b-a}{2}\sqrt{2\lambda} )}
            \end{equation*}
        \item 令
            \begin{equation*}
                N_t={\rm e}^{-\alpha \sqrt{2\lambda}}U_t-{\rm e}^{\alpha \sqrt{2\lambda}}V_t
            \end{equation*}
            取$\alpha=\frac{1}{2}(a+b)$,类似于上一问,可求得
            \begin{equation*}
                \E[ {\rm e}^{-\lambda T} I_{ \{ T=T_a \} }]=\frac{ {\rm sinh}( b\sqrt{2}\lambda ) }{ {\rm sinh}( (b-a)\sqrt{2\lambda} ) }
            \end{equation*}
        \item 利用2.中的结论,令$\lambda \rightarrow 0^+$,由DCT可得
            \begin{equation*}
                \P( T_a<T_b )=\fun{lim}{\lambda \rightarrow 0^+}\frac{ {\rm sinh}( b\sqrt{2}\lambda ) }{ {\rm sinh}( (b-a)\sqrt{2\lambda} ) }
                =\frac{b}{b-a}
            \end{equation*}
    \end{enumerate}
\end{solve}
\begin{remark}
    这个构造方法多少有点逆天了,看看就好。
\end{remark}

\begin{ex}[le gall(Exercise3.28)][le gall(Exercise3.28)]
    $B=(B_t)_{t\geqslant 0}$是初值为$0$的布朗运动,$(\F_t=\sigma( B_s,s\in [0,t] ))$,
    $a>0$,设
    \begin{equation*}
        \sigma_a=\fun{inf}{}\{ t\geqslant 0:B_t\leqslant t-a \}
    \end{equation*}
    \begin{enumerate}
        \item 证明$\sigma_a$是停时,且$\sigma_a<+\infty$ a.s.
        \item 利用合适的指数鞅,证明:$\forall \lambda \geqslant 0$,
            \begin{equation*}
                \E[ {\rm e}^{-\lambda \sigma_a} ]={\rm exp}\{ -a( \sqrt{1+2\lambda}-1 ) \}
            \end{equation*}
            (实际上,对于$\lambda \in [-\frac{1}{2},0]$结论也成立,证明涉及到分析的知识所以较为复杂,此处不作要求,但下一问仍可使用此结论。)
        \item 对于$\mu\in \R$,令$M_t={\rm exp}\{ \mu B_t-\frac{1}{2}\mu^2 t \}$,证明停止鞅$M_{\sigma_a\wedge t}$是闭的当且仅当$\mu\leqslant 1$.
        (提示:仿照\autoref{Relationship between Martingale and Stoped-Martingale}的过程,证明$\E[ M_\sigma ]=1\Rightarrow M_{\sigma_a\wedge t}$闭)
    \end{enumerate}
\end{ex}
\begin{solve}
    \begin{enumerate}
        \item 注意到$\fun{liminf}{t\rightarrow\infty}(B_t-t)=-\infty$即可。
        \item 先取$\mu\in \R$(待定),考虑如下的鞅
            \begin{equation*}
                M_t={\rm e}^{ \mu B_t-\frac{1}{2}\mu^2 t }
            \end{equation*}
            那么
            \begin{equation*}
                M_{t\wedge \sigma_a}\leqslant 
                {\rm e}^{ \mu( (t\wedge \sigma_a)-a )-\frac{1}{2}\mu^2(t\wedge \sigma_a) }
                ={\rm e}^{ -\mu a }{\rm e}^{ (-\frac{1}{2}\mu^2+\mu)(t\wedge \sigma_a) }
            \end{equation*}
            于是我们后面取$\mu$时需要确保$-\frac{1}{2}\mu^2+\mu\leqslant 0$,这样的话就有
            \begin{equation*}
                M_{t\wedge \sigma_a}\leqslant {\rm e}^{ -\mu a }
            \end{equation*}
            所以$(M_{t\wedge \sigma_a})$一致可积, 由Doob择停定理可得
            \begin{equation*}
                1=\E[ M_0 ]=\E[M_{\sigma_a}]
                =\E[{\rm e}^{ \mu (\sigma-a)-\frac{1}{2}\mu^2 \sigma_a }]
            \end{equation*}
            这就说明
            \begin{equation*}
                \E[{\rm e}^{ (-\frac{1}{2}\mu^2+\mu) \sigma_a }]
                ={\rm e}^{\mu a}
            \end{equation*}
            我们取$\mu$使得
            \begin{equation*}
                -\frac{1}{2}\mu^2+\mu=-\lambda\leqslant 0
            \end{equation*}
            则得到结论。
        \item 考虑
            \begin{equation*}
                \E[M_{\sigma_a}]=\E[ {\rm e}^{ -(\frac{1}{2}\mu^2-\mu)\sigma_a-\mu_a } ]
                ={\rm e}^{-\mu a}\E[{\rm e}^{ -(\frac{1}{2}\mu^2-\mu)\sigma_a}]
            \end{equation*}
            注意到其中的$\frac{1}{2}\mu^2-\mu\geqslant -\frac{1}{2}$,套用上一问结论可得
            \begin{equation*}
                \E[M_{\sigma_a}]={\rm e}^{-\mu a}\cdot {\rm e}^{-a( |\mu-1|-1 )}={\rm e}^{-a( |\mu-1|-1+\mu )}
            \end{equation*}
            所以$\E[M_{\sigma_a}]=1\Leftrightarrow \mu\leqslant 1$. 如果$M_{t\wedge \sigma_a}$闭,那么
            \begin{equation*}
                1=\E[ M_{0\wedge \sigma_a} ]=\E[ M_{\infty\wedge \sigma_a} ]=\E[ M_{\sigma_a} ]
            \end{equation*}
            因此,我们只剩下证明$\E[ M_\sigma ]=1\Rightarrow M_{\sigma_a\wedge t}$闭,我们希望证明:$\forall t\geqslant 0$,
            \begin{equation*}
                M_{t\wedge \sigma_a}=\E[ M_{\sigma_a}|\F_t ]
            \end{equation*}
            首先,$M_{t\wedge \sigma_a}\in \F_{t\wedge \sigma_a}\subset \F_t$,故只需证$\forall A\in \F_t$,
            \begin{equation*}
                \E[ M_{t\wedge \sigma_a}I_A ]=\E[ M_{\sigma_a}I_A ]
            \end{equation*}
            也就是
            \begin{equation*}
                \E[ M_{t\wedge \sigma_a}I_{A\cap \{ \sigma_a\geqslant t \}} ]=\E[ M_{\sigma_a}I_{A\cap \{ \sigma_a\geqslant t \}} ]\tag*{$(\star)$}
            \end{equation*}
            然后我们考虑$M_t$,它是一个非负鞅(进而是非负上鞅),
            套用非负上鞅的择停定理(\autoref{Non-negative Supermartingale Optional S.T. Theorem})可得
            \begin{equation*}
                M_{t\wedge \sigma_a}\geqslant \E[ M_{\sigma_a}|\F_{t\wedge \sigma_a} ]
            \end{equation*}
            但因为$\E[ M_\sigma ]=1$,我们发现
            \begin{equation*}
                \E[ M_{t\wedge \sigma_a} ]=\E[M_{0\wedge \sigma_a}]=1=\E[ M_\sigma ]=\E[\E[ M_{\sigma_a}|\F_{t\wedge \sigma_a} ]]
            \end{equation*}
            因此
            \begin{equation*}
                M_{t\wedge \sigma_a}= \E[ M_{\sigma_a}|\F_{t\wedge \sigma_a} ]
            \end{equation*}
            容易验证$A\cap \{ \sigma_a\geqslant t \}\in \F_{t\wedge \sigma_a}$,根据条件期望的定义可得$(\star)$式成立。
    \end{enumerate}
\end{solve}

\begin{ex}[le gall(Exercise3.29.1-2)][le gall(Exercise3.29.1-2)]
    $(Y_t)$是一致可积鞅,且具有右连续轨道,$Y_0=0$ a.s.,设$Y_\infty=\fun{lim}{t\rightarrow \infty}Y_t$,
    固定$p\geqslant 1$,我们称鞅$Y$满足性质(P)是指:
    存在常数$C$使得对于任意停时$T$都有
    \begin{equation*}
        \E[ |Y_\infty-Y_T|^p |\F_T ]\leqslant C
    \end{equation*}
    \begin{enumerate}
        \item 证明如果$Y_\infty$有界,则$Y$满足性质(P).
        \item $B=(B_t)_{t\geqslant 0}$是初值为$0$的布朗运动,$(\F_t=\sigma( B_s,s\in [0,t] ))$,
            证明鞅$Y_t=B_{t\wedge 1}$满足性质(P).
    \end{enumerate}
\end{ex}
\begin{proof}
    \begin{enumerate}
        \item 不妨设$|Y_\infty|<C$,$(Y_t)$一致可积,于是$Y_t=\E[Y_\infty|\F_t]$,那么
            \begin{equation*}
                |Y_t|=|\E[Y_\infty|\F_t]|
                \leqslant \E[|Y_\infty||\F_t]<C,\ \forall t\geqslant 0
            \end{equation*}
            从而$|Y_T|<C$. 那么我们就得到
            \begin{equation*}
                \E[ |Y_\infty-Y_T|^p |\F_T ]
                \leqslant \E[ ( |Y_\infty|+|Y_T| )^p|\F_T ]\leqslant (2C)^p
            \end{equation*}
        \item 先考虑$p=1$的情形,任取$A\in \F_T$,
            \begin{equation*}
                \E[ \E[ |Y_\infty-Y_T|\ |F_T ]I_A ]
                =\E[ |Y_\infty-Y_T|I_A ]\leqslant \E[ |Y_\infty|I_A ]+\E[ |Y_T|I_A ]
            \end{equation*}
            同时由一致可积鞅的择停定理,$Y_T=\E[ Y_\infty|\F_T ]$,因此
            \begin{align*}
                \E[ |Y_T|I_A ]&=\E[ |\E[ |Y_\infty-Y_T|]|\ I_A ]\\
                &\leqslant \E[ \E[ |Y_\infty|\ |\F_T ]I_A ]\\
                &=\E[ |Y_\infty|I_A ]
            \end{align*}
            从而
            \begin{equation*}
                \E[ \E[ |Y_\infty-Y_T|\ |F_T ]I_A ]\leqslant 2\E[ |Y_\infty|I_A ]\leqslant 2\E[ |Y_\infty|]
            \end{equation*}
            取$C=2\E[ |Y_\infty|]$即可。然后我们考虑$p>1$,根据Lp最大值不等式,
            \begin{equation*}
                \E[ \fun{sup}{t\geqslant 0}|Y_t|^p ]
                \leqslant \E[ \fun{sup}{0\leqslant t\leqslant 1} |B_t|^p ]
                \leqslant ( \frac{p}{p-1} )^p \E[ |B_1|^p ]
            \end{equation*}
            因此$\fun{sup}{t\geqslant 0}|Y_t|^p\in L^p$.
            因此,对于任意的$A\in \F_T$,
            \begin{align*}
                \E[ \E[ |Y_\infty-Y_T|^p|\F_T ]I_A ]
                &=\E[ |Y_\infty-Y_T|^pI_A ]\\
                &\leqslant \E[ (|Y_\infty|+|Y_T|)^pI_A ]\\
                &=\E[ (2\fun{sup}{t\geqslant 0}|Y_t|)^pI_A ]\\
                &=2^p\E[ (\fun{sup}{t\geqslant 0}|Y_t|)^pI_A ]\\
                &\leqslant 2^p\E[ \fun{sup}{t\geqslant 0}|Y_t|^p ]\leqslant 2^p(\frac{p}{p-1})^p\E[ |B_1|^p ]<+\infty
            \end{align*}
            那么由$A$的任意性,
            \begin{equation*}
                \E[ |Y_\infty-Y_T|^p|\F_T ]\leqslant 2^p(\frac{p}{p-1})^p\E[ |B_1|^p ]
            \end{equation*}
            后者是一个固定的常数,结论得证。
    \end{enumerate}
\end{proof}

最后放两道去年期末题。第一道的解法应该是类比于\autoref{Cont-Martingale Convergence Theorem}
的证明,第二道则是对\autoref{le gall(Exercise3.28)}的改编。
\begin{ex}[2023SPFinal4.]
    $(X_t)$是非负上鞅,且具有右连续轨道,证明:
    存在$X_\infty\in L^1$使得$X_t\ra{\rm a.s.}X_\infty$,
    并且$\E[X_\infty|\F_t]\leqslant X_t$.
\end{ex}
\begin{solve}
    取$\R_+$的可数稠密子集$D$,对于$\forall t\in D$,有理数$a<b$,
    由离散鞅的上穿次数估计不等式:
    \begin{equation*}
        \E[M_{a,b}^X(D\cap [0,t])]
        \leqslant \frac{1}{b-a}\E[ (X_t-a)^- ]
    \end{equation*}
    由于$X_t$非负,$\E[ (X_t-a)^- ]\leqslant \E[ |a| ]=|a|$,所以
    \begin{equation*}
        \E[M_{a,b}^X(D\cap [0,t])]\leqslant \frac{|a|}{b-a}<+\infty
    \end{equation*}
    那么由Fatou引理,
    \begin{equation*}
        \E[M_{a,b}^X(D)]\leqslant \fun{liminf}{D\ni t\rightarrow\infty} \E[M_{a,b}^X(D\cap [0,t])]\leqslant \frac{|a|}{b-a}<+\infty
    \end{equation*}
    从而$\fun{lim}{D\ni t\rightarrow\infty}X_t$ a.s.存在。现在将收敛性扩展到
    $\R_+$上,对于$\forall \varepsilon>0$,存在$N$使得$D\ni t\geqslant N$时,
    \begin{equation*}
        |X_t-X_\infty|\leqslant \varepsilon
    \end{equation*}
    那么,对于任意的$s\geqslant N$,($D$稠密)取$D\ni s_n\searrow s$,由右连续性得到
    \begin{equation*}
        |X_s-X_\infty|=\fun{lim}{n\rightarrow\infty}|X_{s_n}-X_\infty|
        \leqslant \varepsilon
    \end{equation*}
    这就说明了
    \begin{equation*}
        X_\infty=\fun{lim}{t\rightarrow\infty}X_t{\rm\ a.s.}\text{存在}
    \end{equation*}

    由Fatou引理,
    \begin{equation*}
        \E[|X_\infty|]\leqslant \fun{liminf}{D\ni t\rightarrow\infty}\E[ |X_t| ]<+\infty
    \end{equation*}
    因此$X_\infty\in L^1$.

    最后,考虑常数$t<s$,根据$X$是上鞅
    以及条件期望的Fatou引理,
    \begin{equation*}
        \E[ X_\infty|\F_t ]
        \leqslant \fun{liminf}{s\rightarrow\infty}\E[ X_s|\F_t ]
        \leqslant X_t
    \end{equation*}
    (或者考虑常数$t$和$\infty$作为停时利用非负上鞅的择停定理。)
\end{solve}

\begin{ex}[2023SPFinal.5]
    $B=(B_t)_{t\geqslant 0}$是布朗运动,设停时$T=\fun{inf}{}\{ t\geqslant 0:B_t\geqslant 1-2t \}$,
    \begin{enumerate}[(1).]
        \item 证明$T<+\infty$ a.s.
        \item 求$\E[ {\rm e}^{-\lambda T} ]$.
    \end{enumerate}
\end{ex}
\begin{proof}
    (1).设$T_1=\fun{inf}{}\{ t\geqslant 0:B_t\geqslant 1\}$,则$\{T_1<+\infty\}\subset \{T<+\infty\}$,
    由于$T_1<+\infty$ a.s.,所以$T<+\infty$ a.s.

    (2).对于任意的$\theta\in \R$,令
    \begin{equation*}
        X_t={\rm exp}\left\{ \theta B_t-\frac{1}{2}\theta^2 t \right\}
    \end{equation*}
    容易验证$\{X_t,t\geqslant 0\}$关于$\F_t=\sigma(B_s,0\leqslant s\leqslant t)$是一个鞅,
    我们令$\theta>0$,这确保了
    \begin{equation*}
        0\leqslant X_{t\wedge T}
        ={\rm exp}\left\{ \theta(B_{t\wedge T})-\frac{1}{2}\theta^2 T \right\}
        \leqslant {\rm exp}\left\{ \theta(1-2T)-\frac{1}{2}\theta^2 T \right\}
        = {\rm exp}\left\{ \theta-(\frac{1}{2}\theta^2+2\theta) T \right\}\leqslant {\rm exp}\{ \theta \}
    \end{equation*}
    一致有界从而一致可积,由择停定理可知
    \begin{equation*}
        1=\E[X_0]=\E[X_T]={\rm e}^\theta \E[ {\rm e}^{-(\frac{1}{2}\theta^2+2\theta) T} ]
    \end{equation*}

    我们取$\theta=\sqrt{2\lambda +4}-2>0$,就得到
    \begin{equation*}
        \E[ {\rm exp}\{ -\lambda T \} ]={\rm e}^{-\theta}={\rm e}^{ 2-\sqrt{2\lambda +4} }
    \end{equation*}
\end{proof}

\if{0}{
本小节用到的一个结论:
\begin{lemma}
    $B=(B_t)_{t\geqslant 0}$是布朗运动,$b>0$,
    设停时
    \begin{equation*}
        T=\fun{inf}{}\{ t\geqslant 0:|B_t|=b \}
    \end{equation*}
    那么$B_T$与$T$相互独立。
\end{lemma}
\begin{proof}
    设$B_t'=-B_t$,是一个新的布朗运动,那么
    \begin{align*}
        \P(B_T=b|T>u)
        &=\P( B_T=b|\forall t\in [0,u],|B_t|<b )\\
        &=\P( -B_T=-b|\forall t\in [0,u],|-B_t|<b )\\
        &=\P( B_T'=-b|\forall t\in [0,u],|B_t'|<b )\\
        &=\P( B_T=-b|\forall t\in [0,u],|B_t|<b )\\
        &=\P( B_T=-b|T>u )=\frac{1}{2}=\P(B_T=b)
    \end{align*}
    所以$B_T$和$T$相互独立。
\end{proof}
\item 根据本小节给出的引理可知$B_T$与$T$独立,然后我们设
            \begin{equation*}
                U_t={\rm exp}\{ \sqrt{2\lambda}B_t-\lambda t \}
            \end{equation*}
            \begin{equation*}
                V_t={\rm exp}\{ -\sqrt{2\lambda}B_t-\lambda t \}
            \end{equation*}
            \begin{equation*}
                M_t=U_t+V_t
            \end{equation*}
            易证$M_{t\wedge T}$是一致可积鞅,因此
            \begin{equation*}
                \E[M_{T}]=\E[M_0]=2
            \end{equation*}
            同时
            \begin{align*}
                \E[M_T]
                &=\E[{\rm exp}\{ \sqrt{2\lambda}B_{T}-\lambda T \}]+\E[{\rm exp}\{ -\sqrt{2\lambda}B_{T}-\lambda T \}]\\
                &=\E[ {\rm exp}^{-\lambda T} ]\E[ {\rm exp}\{ \sqrt{2\lambda}B_T \}+{\rm exp}\{ -\sqrt{2\lambda}B_T \} ]\\
                &=\E[ {\rm exp}^{-\lambda T} ]({\rm exp}\{ \sqrt{2\lambda}b \}+{\rm exp}\{ -\sqrt{2\lambda}b \})
            \end{align*}
            所以
            \begin{equation*}
                \E[ {\rm exp}^{-\lambda T} ]=\frac{2}{{\rm exp}\{ \sqrt{2\lambda}b \}+{\rm exp}\{ -\sqrt{2\lambda}b \}}=\frac{1}{{\rm cosh}(b\sqrt{2\lambda})}
            \end{equation*}
}\fi