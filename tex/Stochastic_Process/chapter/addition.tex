\thispagestyle{empty}
\newpage
\begin{center}
	\textbf{\LARGE 关于本文档记号的解释}
\end{center}

	在不同的书籍和课程中,记号的默认含义可能并不相同,笔者会尽量统一整份文档的记号形式,
	但难免有疏漏。特在此部分罗列所有可能出现歧义或者解释不清的记号,以便读者随时查询。

	\textbf{基础记号}:
	\begin{enumerate}[(1).]
		\item $\N\defeq \{0,1,2,\cdots\}$,为全体自然数;$\Z=\{\cdots,-2,-1,0,1,2,\cdots\}$,为全体整数;$\Z_+=\N_+=\{ 1,2,\cdots \}$,为全体正整数。
		\item $n\in \N_+$时,用$[n]$代表$\{1,2,\cdots,n\}$,一般多用于指标集。
	\end{enumerate}

	\textbf{涉及到无穷的记号}:
	\begin{enumerate}[(1).]
		\item 
	\end{enumerate}

	\textbf{关于积分以及积分顺序}:
	设$\mu$是可测空间$(\Omega,\F)$上的一个测度,$\Omega$上的元素用$\omega$表示,
	映射$f:\Omega\rightarrow \R$是$\F$-可测的,
	那么以下都表示$f$在$\Omega$上的积分:
	\begin{equation*}
		\int f \d\mu =\int f(\omega)\mu(\d \omega)
	\end{equation*}
	
	\begin{flushright}
		最后更新:\today
	\end{flushright}
\frontmatter