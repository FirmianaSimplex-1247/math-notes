\chapter{附录}

附录内容原封不动搬自实分析笔记第四章,有部分润色修改。

\section{测度空间}
    \begin{definition}[代数与$\sigma$代数]
        \label{appendix:algebra}\label{appendix:sigma-algebra}
        如果集合族$\mathcal{A}\subset 2^X$满足:
        \begin{enumerate}[(1).]
            \item $\varnothing,X\in\mathcal{A}$.
            \item $E_1,E_2\in \mathcal{A}\Rightarrow E_1\cup E_2\in\mathcal{A}$.
            \item $E\in\mathcal{A}\Rightarrow E^c\mathcal{A}$.
        \end{enumerate}
        则称$\mathcal{A}$是$X$上的一个代数。

        把$(2)$换成可列并,就是$\sigma$-代数的定义。如果$m$是$X$上的一个
        $\sigma$-代数,则$(X,m)$称为一个可测空间,$m$中的元素称为可测集。
    \end{definition}
    \begin{definition}
        \label{appendix:sigma-algebra generated by}
        设$\mathcal{F}\subset 2^X$,定义
        \begin{equation*}
            \sigma(\mathcal{F})=
            \bigcap_{m\supset\mathcal{F}}m
        \end{equation*}
        即包含$\mathcal{F}$的最小的$\sigma$-代数,称为$\mathcal{F}$生成的$\sigma$-代数。
    \end{definition}
    \begin{example}
        {\rm Borel }$\sigma$-代数是$\mathbb{R}^n$上全体开集生成的$\sigma$-代数。
    \end{example}

    \begin{definition}
        $(X,m)$是一个可测空间,如果有函数$\mu:m\rightarrow [0,+\infty]$满足:
        \begin{enumerate}[(1).]
            \item $\mu(\varnothing)=0$.
            \item 可数可加性:$\{E_k\}_{k=1}^\infty\subset m$且互不相交,则
            \begin{equation*}
                \mu\left( \bigcup_{k=1}^\infty E_k \right)=\sum_{k=1}^\infty \mu(E_k)
            \end{equation*}
        \end{enumerate}
        则称$\mu$是$(X,m)$上的一个测度,$(X,m,\mu)$称为测度空间。
    \end{definition}
    \begin{example}
        \begin{enumerate}[(1).]
            \item $\mathbb{R}^n$上的{\rm Lebesgue}测度。
            \item {\rm Dirac}测度:\begin{equation*}
                \delta_a(E)=\left\{ \begin{array}{ll}
                    1&,E\ni a\\
                    0&,{\rm otherwise}
                \end{array} \right.
            \end{equation*}
            \item 计数测度:\begin{equation*}
                \mu(E)=\left\{ \begin{array}{ll}
                    \#E&,\#E<\infty\\
                    +\infty&,{\rm otherwise}
                \end{array} \right.
            \end{equation*}
        \end{enumerate}
    \end{example}
    \begin{theorem}[测度的性质]
        \label{appendix:properties of measure}
        设$\mu$是$(X,m)$上的一个测度,$\{ E_k \}_{k=1}^\infty\subset m$,则满足:
        \begin{enumerate}[(1).]
            \item 单调性:$E_1\subset E_2\Rightarrow \mu(E_1)\leqslant \mu(E_2)$.
            \item 次可加性:$\mu\left( \bigcup_{k=1}^\infty E_k \right)\leqslant
            \sum_{k=1}^\infty \mu(E_k)$.
            \item 连续性:向上:
            \begin{equation*}
                E_k\nearrow E\Rightarrow \mu(E)=\mathop{\rm lim}\limits_{k\rightarrow\infty}\mu(E_k)
            \end{equation*}
                向下:
            \begin{equation*}
                E_k\searrow E,\mu(E_1)<\infty\Rightarrow \mu(E)=\mathop{\rm lim}\limits_{k\rightarrow\infty}\mu(E_k)
            \end{equation*}       
        \end{enumerate}
    \end{theorem}
    \begin{definition}
        $(X,m,\mu)$上,如果$E\in m$,$\mu(E)=0$,则称$E$是一个$\mu$-零集。
        如果一个性质,对于除了一个$\mu$-零集以外的所有$x\in X$都成立,则称它几乎处处成立,记作$\mu$-{\rm a.e.}
        称$(X,m,\mu)$是完备的,如果$\mu$-零集的任何子集都可测。
        即$\mu$-零集的任何子集还在$X$的子集族$m$当中。
    \end{definition}
    \begin{example}
        $\mathbb{R}$上的{\rm Lebesgue}测度$m$完备,但
        {\rm Borel}集$\mathcal{B}$上的{\rm Lebesgue}测度$m$不完备。如果完备,由{\rm Lebesgue}
        可测集的构造可得$\mathcal{B}=\mathcal{L}$,而我们在之前的习题里得出的结论是
        $\mathcal{B}\subsetneq \mathcal{L}$.
    \end{example}

    \begin{definition}
        $X$是非空集合,如果函数$\mu^*:2^X\rightarrow [0,+\infty]$满足
        \begin{enumerate}[(1).]
            \item $\mu^*(\varnothing)=0$
            \item 单调性:$E_1\subset E_2\Rightarrow \mu^*(E_1)\leqslant \mu^*(E_2)$.
            \item 可数可加性:互不相交的一列集合$\{E_k\}_{k=1}^\infty$,有\begin{equation*}
                \mu^*\left(\bigsqcup_{k=1}^\infty E_k\right)=\sum_{k=1}^\infty \mu^*(E_k)
            \end{equation*}
        \end{enumerate}
        则称$\mu^*$是$X$上的一个外测度。
    \end{definition}
    \begin{definition}
        设$\mu^*$是$X$上的一个外测度,如果$E\subset X$满足:$\forall A\subset X$,
        \begin{equation*}
            \mu^*(A)=\mu^*(A\cap E)+\mu^*(A\cap E^c)
        \end{equation*}
        则称$E$是$\mu^*$-可测的。
    \end{definition}
    \begin{theorem}[Caratheodory]
        $X$上的全体$\mu^*$-可测集$m$是$X$上的$\sigma$-代数,而且
        $\mu\mathop{=}\limits^{\rm def} \mu^*|_m$是一个完备的测度。
    \end{theorem}
    \begin{proof}
        第一步:$m$是代数。只需证明$m$对有限并封闭,设$E_1,E_2\in m$,我们希望证明:$\forall A\subset X$,
        \begin{equation*}
            \mu^*(A)=\mu^*( A\cap (E_1\cup E_2) )+\mu^*(A\cap (E_1\cup E_2)^c)
        \end{equation*}
        左式小于等于右式是显然的,而
        \begin{align*}
            \mu^*(A)&=\mu^*(A\cap E_1)+\mu^*(A\cap E_1^c)\\
            &=\mu^*(A\cap E_1\cap E_2)+\mu^*(A\cap E_1\cap E_2^c)+\mu^*(A\cap E_1^c\cap E_2)+\mu^*(A\cap E_1^c\cap E_2^c)\\
            &\geqslant \mu^*( A\cap (E_1\cup E_2) )+\mu^*(A\cap (E_1\cup E_2)^c)
        \end{align*}

        第二步:$\mu^*|_m$满足有限可加性。设$E_1,E_2\in m$,$E_1\cap E_2=\varnothing$,
        \begin{equation*}
            \mu^*(E_1\cup E_2)
            =\mu^*((E_1\cup E_2)\cap E_1)+\mu^*( (E_1\cup E_2)\cap E_1^c )
            =\mu^*(E_1\cup E_2)=\mu^*(E_1)+\mu^*(E_2)
        \end{equation*}

        第三步:$m$是$\sigma$-代数。设互不相交的$E_k\in m,k=1,2,\cdots$不妨设互不相交,否则
        令
        \begin{equation*}
            \tilde{E_1}=E_1,\tilde{E_k}=E_k\backslash \bigcup_{j=1}^{k-1}E_j,k\geqslant 2
        \end{equation*}
        {\rm Claim:}
        \begin{equation*}
            \mu^*\left( A\cap \left( \bigcup_{k=1}^n E_k \right) \right)
            =\sum_{k=1}^n \mu^*(A\cap E_k),\forall A\subset X,\forall n\in \mathbb{N}
        \end{equation*}
        归纳,显然上式对$n=1$成立,如果对$n$成立,那么
        \begin{align*}
            \mu^*\left( A\cap \left( \bigcup_{k=1}^{n+1} E_k \right) \right)
            &=\mu^*\left( A\cap \left( \bigcup_{k=1}^n E_k \right)\cap E_{N+1} \right)
            +\mu^*\left( A\cap \left( \bigcup_{k=1}^n E_k \right)\cap E_{N+1}^c \right)\\
            &=\mu^*(A\cap E_{n+1})+\sum_{k=1}^n \mu^*(A\cap E_k)\\
            &=\sum_{k=1}^{n+1} \mu^*(A\cap E_k)
        \end{align*}
        令$E=\bigcup_{k=1}^\infty E_k$,对于$\forall A\subset X$,$\forall n$
        \begin{align*}
            \sum_{k=1}^n \mu^*(A\cap E_k)+\mu^*(A\cap E^c)
            &\leqslant \sum_{k=1}^n \mu^*(A\cap E_k)+
            \mu^*\left( A\cap \left( \bigcup_{k=1}^n E_k \right)^c \right)\\
            &=\mu^*\left( A\cap \left( \bigcup_{k=1}^n E_k \right) \right)
            +\mu^*\left( A\cap \left( \bigcup_{k=1}^n E_k \right)^c \right)\\
            &=\mu^*(A)
        \end{align*}
        \begin{equation*}
            \Rightarrow \sum_{k=1}^\infty \mu^*(A\cap E_k)+\mu^*(A\cap E^c)\leqslant \mu^*(A)\tag*{(1)}
        \end{equation*}
        另一方面:
        \begin{align*}
            \mu^*(A)&\leqslant \mu^*(A\cap E)+\mu^*(A\cap E^c)\\
            &\leqslant \sum_{k=1}^\infty \mu^*(A\cap E_k)_\mu^*(A\cap E^c)\tag*{(2)}\\
            (1)+(2)&\Rightarrow
            \mu^*(A)=\mu^*(A\cap E)+\mu^*(A\cap E^c)\\
            &\Rightarrow E\in m
        \end{align*}

        第四步:令$\mu=\mu^*|_m$,证明其满足可数可加性,从而是一个测度。
        设$E_k\in m,k=1,2,\cdots$互不相交,令
        \begin{equation*}
            E=\bigcup_{k=1}^\infty E_k
        \end{equation*}
        由第三步中的$(1)+(2)$,
        \begin{equation*}
            \mu^*(A)=\sum_{k=1}^\infty \mu^*(A\cap E_k)+\mu^*(A\cap E^c),\forall A\subset X
        \end{equation*}
        取$A=E$,
        \begin{equation*}
            \mu^*(E)=\sum_{k=1}^\infty \mu^*(E_k)
        \end{equation*}

        第五步:$\mu$完备。只需证:$\forall E{\rm\ with\ }\mu^*(E)=0\Rightarrow E\in m$.由于$\forall A\subset X$,
        \begin{equation*}
            \mu^*(A)\leqslant \mu^*(A\cap E)+\mu^*(A\cap E^c)\leqslant \mu^*(A)
        \end{equation*}
        $\Rightarrow E\in m$.
    \end{proof}
    \begin{definition}
        $\mathcal{A}$是$X$上的代数,如果$\mu_o:\mathcal{A}\rightarrow[0,+\infty]$满足:
        \begin{enumerate}[(1).]
            \item $\mu_o(\varnothing)=0$.
            \item $\mathcal{A}$上一列集合$\{A_k\}_{k=1}^\infty$互不相交的,且$\bigsqcup_{k=1}^\infty A_k \in\mathcal{A}$,则
                \begin{equation*}
                    \mu_o\left(\bigsqcup_{k=1}^\infty A_k\right)=\sum_{k=1}^\infty \mu_o(A_k)
                \end{equation*}
        \end{enumerate}
        则称$\mu_o$是$X$上的一个预测度。
    \end{definition}
    \begin{remark}
        $(2)\Rightarrow $单调性、有限可加性。
    \end{remark}
    \begin{definition}
        $(X,m,\mu)$上,如果
        \begin{equation*}
            X=\bigcup_{k=1}^\infty E_k{\rm\ with\ }\mu(E_k)>\infty,\forall k
        \end{equation*}
        则称$\mu$是$\sigma$-有限测度。
    \end{definition}
    \begin{theorem}\label{appendix:extension}
        $\mathcal{A}$是$X$上的代数,$\mu_o$是$X$上的一个预测度,对于$E\subset X$,令
        \begin{equation*}
            \mu^*(E)\mathop{=}\limits^{\rm def}
            {\rm inf}\left\{ \sum_{k=1}^\infty \mu_o(A_k):\{A_k\}_{k=1}^\infty
            \subset\mathcal{A},E\subset \bigcup_{k=1}^\infty A_k \right\}
        \end{equation*}
        则:
        \begin{enumerate}[(1).]
            \item $\mu^*$是$X$上的外测度。
            \item $\mu^*|_{\mathcal{A}}=\mu_o$.
            \item $\mathcal{A}\subset m\mathop{=}\limits^{\rm def}\{ \mu^*\mbox{-可测集} \}$.
            \item $\mu\mathop{=}\limits^{\rm def} \mu^*|_m$是一个测度,若$v$是$m$上另一测度使得$v|\mathcal{A}=\mu_o$,
            则$v<\mu$.而且若$\mu(E)<\infty$,则$v(E)=\mu(E)$.特别地,如果$\mu_o$是
            $\sigma$-有限测度,则$v=\mu$.(即$\mu$是$\mu_o$从$\mathcal{A}$到$m$上的唯一延拓。)
        \end{enumerate}
    \end{theorem}
    \begin{proof}
        \begin{enumerate}[(1).]
            \item 只验证可数可加性,设$\{ E_k \}_{k=1}^\infty \subset 2^X$,不妨设
                \begin{equation*}
                    \mu^*(E)<\infty,\forall k.
                \end{equation*}
                $\forall \varepsilon>0$,对于每个$E_k$,存在
                $A_j^{(k)}\in \mathcal{A},j=1,2,\cdots$使得
                $E_k\subset \bigcup_{j=1}^\infty A_j^{(k)}$且
                \begin{equation*}
                    \sum_{j=1}^\infty \mu_o(A_j^{(k)})
                    <\mu^*(E_k)+\frac{\varepsilon}{2^k}
                \end{equation*}
                \begin{align*}
                    \Rightarrow \mu^*
                    \left( \bigcup_{k=1}^\infty E_k \right)&\leqslant \sum_{k,j}
                    \mu_o(A_j^{(k)})\\
                    &\leqslant \sum_{k=1}^\infty \left[ \mu_o(E_k)+\frac{\varepsilon}{2^k} \right]
                    =\sum_{k=1}^\infty \mu^*(E_k)+\varepsilon\\
                    \Rightarrow \mu^*\left( \bigcup_{k=1}^\infty E_k \right)
                    &\leqslant \sum_{k=1}^\infty \mu^*(E_k)
                \end{align*}
            \item 设$E\in\mathcal{A}$,则
                \begin{equation*}
                    \mu^*(E)\leqslant \mu_o(E)
                \end{equation*}
                另一方面,$\forall A_k\in\mathcal{A},k=1,2,\cdots{\rm\ with\ }E\subset \bigcup_{k=1}^\infty A_k$,令
                \begin{equation*}
                    E_1=E\cap A_1
                \end{equation*}
                \begin{equation*}
                    E_k=E\cap\left( A_k\backslash \bigcup_{j=1}^{k-1}A_j \right),k\geqslant 2
                \end{equation*}
                $\Rightarrow E_k\in\mathcal{A},k=1,2,\cdots$互不相交,且
                \begin{align*}
                    &\bigcup_{k=1}^\infty E_k=E\in\mathcal{A}\\
                    \Rightarrow &\mu_o(E)=\sum_{k=1}^\infty \mu_o(E_k)\leqslant \sum_{k=1}^\infty \mu_o(A_k)\\
                    \Rightarrow &\mu_o(E)\leqslant \mu^*(E)
                \end{align*}
            \item 设$A\in\mathcal{A}$,希望证明$A\in m$,即
                \begin{equation*}
                    \forall E\subset X,\mu^*(E)=\mu^*(E\cap A)+\mu^*(E\cap A^c)
                \end{equation*}
                $\forall \varepsilon>0$,存在$A_k\in\mathcal{A},k=1,2,\cdots$使得
                $E\subset \bigcup_{k=1}^\infty A_k$,且
                \begin{align*}
                    \sum_{k=1}^\infty \mu_o(A_k)<&\mu^*(E)+\varepsilon\\
                    \Rightarrow
                    \mu^*(E)\leqslant& \mu^*(E\cap A)+\mu^*(E\cap A^c)\\
                    \leqslant& 
                    \mu^*\left( \bigcup_{k=1}^\infty (A_k\cap A) \right)
                    +\mu^*\left( \bigcup_{k=1}^\infty (A_k\cap A)^c \right)\\
                    =& \sum_{k=1}^\infty 
                    \left[ \mu^*(A_k\cap A)+\mu^*(A_k\cap A^c) \right]\\
                    \mathop{=}\limits^{\mu^*|_{\mathcal{A}}=\mu_o}
                    & \sum_{k=1}^\infty \left[ \mu_o(A_k\cap A)+\mu_o(A_k\cap A) \right]\\
                    \mathop{=}\limits^{\mu_o\mbox{有限可加}}
                    &\sum_{k=1}^\infty \mu_o(A_k)<\mu^*(E)+\varepsilon\\
                    \Rightarrow
                    \mu^*(E)=\mu^*(E\cap A)+\mu^*(E\cap A^c)
                \end{align*}
            \item 设$E\in m$,$\forall A_k\in\mathcal{A},k=1,2,\cdots{\rm\ with\ }E\subset \bigcup_{k=1}^\infty A_k$
                \begin{align*}
                    &v(E)\leqslant \sum_{k=1}^\infty v(A_k)=\sum_{k=1}^\infty \mu_o(A_k)\\
                    \Rightarrow&
                    v(E)\leqslant \mu^*(E)=\mu(E)
                \end{align*}
                设$\mu(E)<\infty$,来证明反向不等式:$\forall \varepsilon>0$,
                $\exists A_k\in\mathcal{A},k=1,2,\cdots{\rm\ s.t.\ }E\subset \bigcup_{k=1}^\infty A_k$,且
                \begin{equation*}
                    \sum_{k=1}^\infty \mu_o(A_k)<\mu(E)+\varepsilon
                \end{equation*}
                令$A=\bigcup_{k=1}^\infty A_k$
                \begin{align*}
                    \Rightarrow &\mu(A)\leqslant \sum_{k=1}^\infty \mu(A_k)=\sum_{k=1}^\infty \mu_o(A_k)<\mu(E)+\varepsilon\\
                    \Rightarrow &\mu(A\backslash E)=\mu(A)-\mu(E)<\varepsilon\\
                    \Rightarrow &
                    v(A)=\mathop{\rm lim}\limits_{N\rightarrow \infty}
                    v\left( \bigcup_{k=1}^N A_k \right)
                    =\mathop{\rm lim}\limits_{N\rightarrow \infty} 
                    \mu_o\left( \bigcup_{k=1}^N A_k \right)
                    =\mathop{\rm lim}\limits_{N\rightarrow \infty} 
                    \mu\left( \bigcup_{k=1}^N A_k \right)=\mu(A)\\
                    \Rightarrow & \mu(E)\leqslant \mu(A)=v(A)=v(E)+v(A\backslash E)
                    \leqslant v(E)+\mu(A\backslash E)\leqslant v(E)+\varepsilon\\
                    \Rightarrow &\mu(E)\leqslant v(E)
                \end{align*}
                若$\mu_o$是$\sigma$-有限的,
                \begin{equation*}
                    X=\bigsqcup_{k=1}^\infty A_k{\rm\ with\ }\mu_o(A_k)<\infty,\forall k
                \end{equation*}
                \begin{equation*}
                    \forall E\in m,v(E)=\sum_{k=1}^\infty v(E\cap A_k)
                    =\sum_{k=1}^\infty \mu(E\cap A_k)=\mu(E)
                \end{equation*}
        \end{enumerate}
    \end{proof}

    \begin{definition}
        如果$\mathcal{F}\subset 2^X$满足:
        \begin{enumerate}[(1).]
            \item $\varnothing\in \mathcal{F}$.
            \item 对有限交封闭。
            \item $\forall E\in \mathcal{F},\exists \{E_k\}_{k=1}^\infty\subset \mathcal{F}{\rm\ s.t.\ }E^c=\bigcup_{k=1}^\infty E_k.$
        \end{enumerate}
        则称$\mathcal{F}$是$X$上的一个半代数。
    \end{definition}
    \begin{example}
        $\mathbb{R}$上区间全体是一个半代数。
    \end{example}
    \begin{corollary}
        $\mathcal{F}$是$X$上的一个半代数,$\mathcal{A}=\{\mathcal{F}$中成员的有限不交并$\}$是$X$上的代数。
    \end{corollary}
    \begin{proof}
        设$A,B\in\mathcal{F}$,
        \begin{align*}
            \Rightarrow &B^c=\bigsqcup_{k=1}^N C_k{\rm\ with\ }C_k\in \mathcal{F}\\
            \Rightarrow &A\backslash B=\bigsqcup_{k=1}^N(A\cap C_k)\in\mathcal{A}\\
            \Rightarrow &A\cup B=(A\backslash B)\cup B
            =\bigsqcup_{k=1}^N (A\cap C_k)\cup B\in\mathcal{A}
        \end{align*}
        由此,结合$\mathcal{A}$的定义可见其对有限并封闭。
        \begin{align*}
            \forall E\in\mathcal{A}\Rightarrow& 
            E=\bigsqcup_{k=1}^N A_k{\rm\ with\ }A_k\in\mathcal{F}\\
            & A_k^c=\bigsqcup_{j=1}^{N_k}C_j^{(k)}{\rm\ with\ }C_j^{(k)}\in\mathcal{F}\\
            \Rightarrow
            & E^c=\bigcap_{k=1}^N A_k^c=\bigcap_{k=1}^N\left( \bigcup_{j=1}^{N_k}C_j^{(k)} \right)\\
            & =\bigcup_{1\leqslant j_k\leqslant N_k,1\leqslant k\leqslant N}
            C_{j_1}^{(1)}\cap\cdots\cap C_{j_N}^{(N)}\in\mathcal{A}
        \end{align*}
    \end{proof}

\section{可测函数与积分}
    \begin{definition}
        可测空间$(X,m)$上,如果$f:X\rightarrow [-\infty,+\infty]$满足
            \begin{equation*}
                \forall a\in\mathbb{R},\{ f>a \}\in m
            \end{equation*}
        则称$f$可测。对于复值函数,如果${\rm Re}f$和${\rm Im}f$都可测,则称$f$可测。
    \end{definition}
    \begin{corollary}
        $\{f_n\}_{n=1}^\infty$可测,则
        $\mathop{\rm sup}\limits_n f_n$,
        $\mathop{\rm inf}\limits_n f_n$,
        $\mathop{\rm limsup}\limits_{n\rightarrow\infty} f_n$,
        $\mathop{\rm liminf}\limits_{n\rightarrow\infty} f_n$都可测。

        进而可测函数全体对于极限运算封闭。
    \end{corollary}
    \begin{definition}
        $X$上全体非负可测函数记作$L^+(X)$,$X$上的
        简单函数定义为可测集示性函数的线性组合。
    \end{definition}
    \begin{theorem}
        $\forall f\in L^+(X)$,$\exists {\rm\ simple\ }\varphi_k\geqslant 0{\rm\ s.t.\ }\varphi_k\nearrow f$.
    \end{theorem}
    \begin{proof}
        对于
        \begin{equation*}
            k=0,1,2,\cdots
        \end{equation*}
        \begin{equation*}
            j=0,1,2,\cdots,2^{2k}-1
        \end{equation*}
        令
        \begin{align*}
            E_{k,j}&=\left\{ \frac{j}{2^k}<f\leqslant \frac{j+1}{2^k} \right\}\\
            F_k&=\{ f>2^k \}\\
            \varphi_k=\sum_{j=1}^{2^{2k}-1}\frac{j}{2^k}\chi_{E_{k,j}}+2^k\chi_{F_k}
        \end{align*}
    \end{proof}
    \begin{definition}
        $(X,m,\mu)$上,
        \begin{enumerate}[(1).]
            \item 对于非负简单函数有标准表示:$\varphi=\sum_{j=1}^N c_k\chi_{E_k}$,定义积分\begin{equation*}
                \int_X \varphi{\rm d}\mu=\sum_{k=1}^N c_k\mu(E_k)
            \end{equation*}
            \item 对于$f\in L^+(X)$,定义
            \begin{equation*}
                \int_X f{\rm d}\mu={\rm sup}\left\{ \int_X \varphi{\rm d}\mu:\varphi{\rm\ simple\ },0\leqslant \varphi\leqslant f \right\}
            \end{equation*}
            称为$f$在$X$上关于$\mu$的积分。
            \item 对于可测函数$f:X\rightarrow[-\infty,+\infty]$,如果
                $\int_X f^+{\rm d}\mu$和$\int_X f^-{\rm d}\mu$中至少有一个有限,则定义
                \begin{equation*}
                    \int_X f{\rm d}\mu=\int_X f^+{\rm d}\mu-\int_X f^-{\rm d}\mu
                \end{equation*}
                如果都有限,则称$f$在$X$上可积。$X$上全体可积函数记作$L^1(X,\mu)$.
            \item 设$E\in m$,$f$在$E$上可测,令\begin{equation*}
                \int_X f{\rm d}\mu=\int_X f\cdot\chi_E{\rm d}\mu
            \end{equation*}
        \end{enumerate}
    \end{definition}
    \begin{corollary}[线性]
        $\forall f,g\in L^1(X,\mu)$,$\forall \alpha,\beta\in\mathbb{R}$,
        \begin{equation*}
            \int_X (\alpha f+\beta g){\rm d}\mu=\alpha\int_X f{\rm d}\mu+\beta\int_X g{\rm d}\mu
        \end{equation*}
    \end{corollary}
    \begin{theorem}[MCT]\label{appendix:MCT}\label{1.5.7}
        $L^+(X)\ni f_k\nearrow f$,则
        \begin{equation*}
            \mathop{\rm lim}\limits_{k\rightarrow\infty}\int_X f_k{\rm d}\mu=\int_X f{\rm d}u
        \end{equation*}
    \end{theorem}
    \begin{theorem}[Fatou]\label{appendix:Fatou}\label{1.5.5}
        对于$\{ f_k \}_{k=1}^\infty\subset L^+(X)$,
        \begin{equation*}
            \int_X \mathop{\rm liminf}\limits_{k\rightarrow\infty}f_k{\rm d}\mu
            \leqslant \mathop{\rm liminf}\limits_{k\rightarrow\infty}
            \int_X f_k{\rm d}\mu
        \end{equation*}
    \end{theorem}
    \begin{theorem}[DCT]\label{appendix:DCT}\label{1.5.8}
        设$f_k$是一列可测函数,$f_k\rightarrow f{\rm\ a.e.}$如果
        $\exists g\in L^1(X,\mu){\rm\ s.t.}$
        \begin{equation*}
            |f_k|\leqslant g{\rm\ a.e.\ \forall k}
        \end{equation*}
        则
        \begin{equation*}
            \mathop{\rm lim}\limits_{k\rightarrow\infty}\int_X f_k{\rm d}\mu=\int_X f{\rm d}u
        \end{equation*}
    \end{theorem}

\section{乘积测度与Fubini定理}
    问:如果有两个测度空间$(X_1,m_1,\mu_1)$
    和$(X_2,m_2,\mu_2)$,那么是否有乘积测度空间:
    \begin{equation*}
        (X_1\times X_2,m_1\otimes m_2,\mu_1\times \mu_2)    
    \end{equation*}
    \begin{definition}
        一般来说$m_1\times m_2=\{A\times B:A\in m_1,B\in m_2\}$不是$\sigma$-代数。
        我们把这里的$A\times B$叫做可测矩形。

        定义
        \begin{equation*}
            m_1\otimes m_2=\sigma(m_1\times m_2)
        \end{equation*}
        即$m_1\times m_2$生成的$\sigma$-代数。
    \end{definition}
    问:如何定义$\mu_1\times \mu_2$使得{\rm Fubini Thm}成立?一个必要条件是:
    \begin{equation*}
        \mu_1\times \mu_2(A\times B)=\mu_1(A)\mu_2(B)
    \end{equation*}
    \begin{corollary}
        $m_1\times m_2$是半代数。
    \end{corollary}
    \begin{proof}
        \begin{equation*}
            (A_1\times B_2)\cap (A_2\times B_2)=
            (A_1\cap A_2)\times (B_1\cap B_2)
        \end{equation*}
        \begin{equation*}
            (A\times B)^c=(X_1\times B^c)\cup(A^c\times B)
        \end{equation*}
    \end{proof}
    \begin{definition}
        $\mathcal{A}=\{ \mbox{可测矩形的有限不交并} \}$,则是一个$X_1\times X_2$上的一个代数。

        令\begin{equation*}
            \mu_o\left( \bigsqcup_{k=1}^n(A_k\times B_k) \right)
            =\sum_{k=1}^n \mu_1(A_k)\mu_2(B_k)
        \end{equation*}
        则$\mu_o$是$\mathcal{A}$上的一个预测度。

        对于$E\subset X_1\times X_2$,定义
        \begin{equation*}
            \mu^*(E)={\rm inf}\left\{ \sum_{k=1}^\infty \mu_o(E_k):E_k\in\mathcal{A},E\subset \bigcup_{k=1}^\infty E_k \right\}
        \end{equation*}
        则$\mu^*$是$X_1\times X_2$上的一个外测度。

        定义
        \begin{equation*}
            m=\{ \mu^*\mbox{-可测集} \}
        \end{equation*}
        \begin{equation*}
            \mu=\mu^*|_m
        \end{equation*}
        于是$\mu$是一个完备测度,且$m_1\otimes m_2\subset m$.

        最后定义$\mu_1\times \mu_2=\mu^*|_{m_1\otimes m_2}$.
    \end{definition}
    \begin{remark}
        一般来说,$\mu_1\times \mu_2$不完备,除非$m_1\otimes m_2=m$.
    \end{remark}
    \begin{definition}
        乘积测度空间
        $(X_1\times X_2,m_1\otimes m_2,\mu_1\times \mu_2)$上,
        设$E\subset X_1\times X_2$,对于$x\in X_1$,
        \begin{equation*}
            E_x\mathop{=}\limits^{\rm def}\{ y\in X_2:(x,y)\in E \}
        \end{equation*}
        称为$E$的$x$切片;类似定义
        \begin{equation*}
            E^y\mathop{=}\limits^{\rm def}\{ x\in X_1:(x,y)\in E \}
        \end{equation*}
        对于$f:X_1\times X_2\rightarrow [-\infty,+\infty]$,
        \begin{equation*}
            f_x(y)\mathop{=}\limits^{\rm def}f(x,y)
            ,f^y(x)\mathop{=}\limits^{\rm def}f(x,y)
        \end{equation*}
    \end{definition}
    \begin{theorem}\ 
        \begin{enumerate}[(1).]
            \item 设$E\in m_1\otimes m_2$,则$\forall x\in X_1,y\in X_2$,都有$E_x\in m_2,E^y\in m_1$.
            \item $f\ m_1\otimes m_2$可测$\Rightarrow f_x\ m_2$可测,$f^y\ m_1$可测
        \end{enumerate}
    \end{theorem}
    \begin{theorem}[Tonelli]
        设$(X_1,m_1,\mu_1)$,$(X_2,m_2,\mu_2)$都是$\sigma$-有限的,$f\in L^+(X_1\times X_2)$.
        \begin{enumerate}[(1).]
            \item $x\mapsto \int_{X_2}f_x{\rm d}\mu_2\in L^+(X_1)$,$y\mapsto \int_{X_1}f^y{\rm d}\mu_1\in L^+(X_2)$.
            \item $\int_{X_1\times X_2}f{\rm d}(\mu_1\times \mu_2)
            =\int_{X_1}\left[ \int_{X_2}f(x,y){\rm d}\mu_2(y) \right]{\rm d}\mu_1(x)
            =\int_{X_2}\left[ \int_{X_1}f(x,y){\rm d}\mu_1(x) \right]{\rm d}\mu_2(y)$
        \end{enumerate}
    \end{theorem}
    \begin{theorem}[Fubini]
        设$(X_1,m_1,\mu_1)$,$(X_2,m_2,\mu_2)$都是$\sigma$-有限的,$f\in L^1(X_1\times X_2,\mu_1\times\mu_2)$.
        \begin{enumerate}[(1).]
            \item $f_x\in L^1(X_2,\mu_2){\rm\ for\ }\mu_1$-{\rm a.e.\ }$x\in X_1$,$f^y\in L^1(X_1,\mu_1){\rm\ for\ }\mu_2$-{\rm a.e.\ }$y\in X_2$.
            \item $x\mapsto \int_{X_2}f_x{\rm d}\mu_2\in L^1(X_1,\mu_1)$,
            $y\mapsto \int_{X_1}f^y{\rm d}\mu_1\in L^1(X_2,\mu_2)$
            \item 同{\rm Tonelli}第二条。
        \end{enumerate}
    \end{theorem}
    证明与{\rm Lebesgue}测度下的相关定理类似,不再赘述。

    \begin{definition}
        如果$\mathcal{F}\subset 2^X$满足:
        \begin{equation*}
            \mathcal{F}\ni E_k\nearrow E\Rightarrow E\in\mathcal{F}
        \end{equation*}
        \begin{equation*}
            \mathcal{F}\ni E_k\searrow E\Rightarrow E\in\mathcal{F}
        \end{equation*}
        则称$\mathcal{F}$是一个单调类。
    \end{definition}
    \begin{example}
        $\sigma$-代数就是一个单调类。
    \end{example}
    \begin{theorem}[单调类引理,Monotone Class Lemma,MCL]
        $\mathcal{A}$是$X$上的代数,
        $\mathcal{A}$生成的单调类,就是$\mathcal{A}$生成的$\sigma$-代数。
    \end{theorem}
    \begin{proof}
        令
        \begin{equation*}
            m\mathop{=}\limits^{\rm def}\sigma(\mathcal{A})
        \end{equation*}
        \begin{equation*}
            \mathcal{F}=\mathcal{A}\mbox{生成的单调类}
        \end{equation*}
        于是$\mathcal{F}\subset m$,因为$\sigma$-代数是单调类。所以只需
        证明$\mathcal{F}$是$\sigma$-代数。

        对于$E\in\mathcal{F}$,令
        \begin{equation*}
            \mathcal{F}_E
            \mathop{=}\limits^{\rm def}
            \{ F\in\mathcal{F}:E\backslash F,F\backslash E,E\cap F\in\mathcal{F} \}
        \end{equation*}
        \begin{enumerate}[(1).]
            \item $\varnothing,E\in\mathcal{F}_E$.
            \item $E\in\mathcal{F}_F\Leftrightarrow F\in\mathcal{F}_E$.
            \item $\mathcal{F}_E$是单调类。
            \item $E\in\mathcal{A}\Rightarrow \mathcal{A}\subset \mathcal{F}_E$,因为\begin{equation*}
                \forall F\in\mathcal{A},E\backslash F,F\backslash E,E\cap F\in\mathcal{A}\subset\mathcal{F}
            \end{equation*}
            \item $E\in\mathcal{A}\Rightarrow \mathcal{F}\subset\mathcal{F}_E$,因为\begin{equation*}
                4^\circ\mathcal{A}\subset\mathcal{F}_E
                \mathop{\Rightarrow }\limits^{3^\circ}
                \mathcal{F}\subset\mathcal{F}_E
            \end{equation*}
            \item \begin{align*}
                E\in\mathcal{F}&\Rightarrow \mathcal{F}\subset \mathcal{F}_E\\
                E\in\mathcal{F}&\mathop{\Rightarrow }\limits^{5^\circ} E\in\mathcal{F}_A,\forall A\in \mathcal{A}\\
                &\mathop{\Rightarrow }\limits^{2^\circ}A\in \mathcal{F}_E,\forall A\in\mathcal{A}\\
                &\Leftrightarrow \mathcal{A}\subset\mathcal{F}_E\\
                &\mathop{\Rightarrow }\limits^{3^\circ}
                \mathcal{F}\subset \mathcal{F}_E
            \end{align*}
            \item $\mathcal{F}$是代数,因为$\forall E,F\in \mathcal{F}
            \mathop{\Rightarrow }\limits^{6^\circ}
            E\in \mathcal{F}\subset \mathcal{F}_F\Rightarrow E\backslash F,E\cap F\in\mathcal{F}.$
            \item $\mathcal{F}$是$\sigma$-代数。设$A_k\in\mathcal{F},k=1,2,\cdots$,令\begin{equation*}
                A\mathop{=}\limits^{\rm def}\bigcup_{k=1}^\infty A_k,B_n\mathop{=}\limits^{\rm def}\bigcup_{k=1}^n A_k
            \end{equation*}
            \begin{equation*}
                \mathop{\Rightarrow }\limits^{7^\circ}
                B_n\in\mathcal{F}\mathop{\Rightarrow }\limits^{B_n\nearrow A}A\in\mathcal{F}
            \end{equation*}
        \end{enumerate}
    \end{proof}
    \begin{theorem}
        设$(X_1,m_1,\mu_1)$,$(X_2,m_2,\mu_2)$都是$\sigma$-有限的,$E\in m_1\otimes m_2$.
        \begin{enumerate}[(1).]
            \item $x\mapsto \mu_2(E_x)\ m_1$-可测,$y\mapsto \mu_1(E^y)\ m_2$-可测。
            \item $(\mu_1\times \mu_2)(E)=\int_{X_1}\mu_2(E_x){\rm d}\mu_1=\int_{X_2}\mu_1(E^y){\rm d}\mu_2$
        \end{enumerate}
    \end{theorem}
    \begin{proof}
        假设$\mu_1,\mu_2$都是有限测度,令
        \begin{equation*}
            \mathcal{F}\mathop{=}\limits^{\rm def}
            \{ E\in m_1\otimes m_2:E{\rm\ satisfy\ } 1^\circ,2^\circ\}
        \end{equation*}

        {\rm Claim}:$\mathcal{F}=m_1\otimes m_2$.
        
        先证明$m_1\times m_2\subset \mathcal{F}$,设
        $E=A\times B,A\in m_1,B\in m_2$,则
        \begin{equation*}
            E_x=\left\{ \begin{array}{ll}
                B&,x\in A\\
                \varnothing&,x\notin A
            \end{array} \right.,
            E^y=\left\{ \begin{array}{ll}
                A&,y\in B\\
                \varnothing&,y\notin B
            \end{array} \right.
        \end{equation*}
        于是
        \begin{equation*}
            \mu_2(E_x)=\mu_2(B)\chi_A(x),\mu_1(E^y)=\mu_1(A)\chi_B(y)
        \end{equation*}
        则$1^\circ$成立,而
        \begin{align*}
            \int_{X_1}\mu_2(E_x){\rm d}\mu_1(x)
            &=\mu_2(B)\int_{X_1}\chi_A{\rm d}\mu_1\\
            &=\mu_2(B)\mu_1(A)\\
            &=(\mu_1\times \mu_2)(E)
        \end{align*}
        同理,$(\mu_1\times \mu_2)(E)=\int_{X_2}\mu_1(E^y){\rm d}\mu_2(y)$,所以$2^\circ$成立,
        于是$m_1\times m_2\subset \mathcal{F}$.

        然后我们希望证明$\mathcal{F}$是一个单调类,然后由单调类引理即可得到
        $\mathcal{F}\subset m_1\times m_2$.
        令
        \begin{equation*}
            \mathcal{A}\mathop{=}\limits^{\rm def}
            \{ m_1\times m_2\mbox{中成员的有限不交并} \}
        \end{equation*}
        由$m_1\times m_2\subset \mathcal{F}$和可加性得到$\mathcal{A}\subset\mathcal{F}$,且
        $\mathcal{A}$是$X_1\times X_2$上的代数。设$\mathcal{F}\ni E_k\nearrow E$,令
        \begin{equation*}
            f_k(y)\mathop{=}\limits^{\rm def}\mu_1( (E_k)^y ),y\in X_2
        \end{equation*}
        \begin{align*}
            \mathop{\Rightarrow}\limits^{E_k\in\mathcal{F}}&
            f_k\ m_2\mbox{可测}\\
            E_k\nearrow E\Rightarrow&
            (E_k)^y\nearrow E^y\\
            \Rightarrow& f(y)\mathop{=}\limits^{\rm def}
            \mu_1(E^y)=\mathop{\rm lim}\limits_{k\rightarrow\infty}\mu_1( (E_k)^y )\\
            \Rightarrow &f\ m_2\mbox{可测且}f_k\nearrow f
        \end{align*}
        所以
        \begin{align*}
            \int_{X_2}\mu_1(E^y){\rm d}\mu_2(y)
            =&\int_{X_2}f(y){\rm d}\mu_2(y)\\
            \mathop{=}\limits^{\rm MCT}
            &\mathop{\rm lim}\limits_{k\rightarrow\infty}
            \int_{X_2}f_k(y){\rm d}\mu_2(y)\\
            =&\mathop{\rm lim}\limits_{k\rightarrow\infty}
            \int_{X_2}\mu_1( (E_k)^y ){\rm d}\mu_2(y)\\
            \mathop{=}\limits^{E_k\in\mathcal{F}}&
            \mathop{\rm lim}\limits_{k\rightarrow\infty}\mu_1\times \mu_2(E_k)
            =\mu_1\times \mu_2(E)\tag*{(测度的上连续性)}
        \end{align*}
        所以$E\in\mathcal{F}$,关于$\mathcal{F}\ni E_k\searrow E\Rightarrow E\in\mathcal{F}$
        的证明是完全类似的,只是在最后一步要利用测度的下连续性,$\mu_1,\mu_2$都是有限测度保证了这一点。

        最后我们来看一下一般情形,即$\mu_1$和$\mu_2$不都是有限测度。因为
        $X_1$和$X_2$是$\sigma$-有限的:
        \begin{equation*}
            X_1=\bigcup_{k=1}^\infty A_k{\rm\ with\ }\mu(A_k)<\infty,A_k\nearrow X_1
        \end{equation*}
        \begin{equation*}
            X_2=\bigcup_{k=1}^\infty B_k{\rm\ with\ }\mu(B_k)<\infty,B_k\nearrow X_2
        \end{equation*}
        所以$X_1\times X_2=\bigcup_{k=1}^\infty (A_k\times B_k)$.因为$m_1\cap A_k$是
        $A_k$上的$\sigma$-代数,定义
        \begin{equation*}
            \mu_1^k\mathop{=}\limits^{\rm def}
            \mu_1|_{m_1\cap A_k}
        \end{equation*}
        类似地定义
        \begin{equation*}
            \mu_2^k\mathop{=}\limits^{\rm def}
            \mu_2|_{m_2\cap B_k}
        \end{equation*}
        都是有限测度,对于$E\in m_1\otimes m_2$,
        \begin{equation*}
            ( E\cap (A_k\times B_k) )^y
            =\left\{ \begin{array}{ll}
                E^y\cap A_k&,y\in B_k\\
                \varnothing&,y\notin B_k
            \end{array} \right.
        \end{equation*}
        于是当$k\rightarrow\infty$,
        \begin{align*}
            \mu_1\times \mu_2(E\cap (A_k\times B_k))&\rightarrow (\mu_1\times \mu_2)(E)\tag*{(测度的上连续性)}\\
            =\int_{X_2}
            \mu_1(E^y\cap A_k)\chi_{B_k}(y){\rm d}\mu_2(y)
            &\rightarrow \int_{X_2}\mu_1(E^y){\rm d}\mu_2(y)\tag*{(\rm MCT)}
        \end{align*}
    \end{proof}