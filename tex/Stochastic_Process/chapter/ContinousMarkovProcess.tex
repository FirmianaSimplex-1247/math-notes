\chapter{连续时间马氏过程}
    本章节由两部分组成,不标注星号的内容来自研随课程,
    标注星号的内容来自应随课程。
    前者的内容标题应该叫做“\textbf{连续马氏过程的一般理论}”,
    用比较基础的语言叙述了连续情形的马氏过程,
    并以Levy过程等为例给出了详细介绍,
    能够帮助我们更好地引入后续内容。
    后者则着重分析了\textbf{离散状态空间连续时间马氏链(跳跃过程)}的性质。
\section{马氏过程的定义}

\subsection{转移核与转移半群}
    \begin{definition}[转移核]
        $(E,\mathcal{E})$是一个可测空间,映射$Q:E\times \mathcal{E}\rightarrow [0,1]$如果满足以下性质:
        \begin{enumerate}[(1).]
            \item $\forall x\in E$,$Q(x,\cdot)$作为$\mathcal{E}\rightarrow [0,1]$的映射是一个$(E,\mathcal{E})$上的概率测度。
            \item $\forall A\in \mathcal{E}$,$Q(\cdot,A)$作为$E\rightarrow [0,1]$的映射是$\mathcal{E}$-可测的。
        \end{enumerate}
        那么就称$Q$是一个从$E$到$E$的马氏转移核(Markovian transition kernel),以下简称转移核。
    \end{definition}
    转移核的概念类似于离散情形下的转移概率函数。

    \begin{definition}[转移半群]
        $(Q_t)_{t\geqslant 0}$是$E$上的转移核组成的集合,如果其满足:
        \begin{enumerate}[(1).]
            \item $\forall x\in E$,$Q_0(x,\d y)=\delta_x(\d y)$,这里$\delta_x$是$x$处的独点分布,即
                \begin{equation*}
                    \int I_A Q_0(x,\d y)=\left\{ \begin{array}{ll}
                        1&,x\in A\\
                        0&,x\notin A
                    \end{array} \right.
                \end{equation*}
            \item $\forall s,t\geqslant 0$,$A\in \mathcal{E}$,
                \begin{equation*}
                    Q_{t+s}(x,A)=\int_E Q_t(x,\d y)Q_s(y,A)\tag*{Chapman-Kolmogorov identity}
                \end{equation*}
            \item $\forall A\in \mathcal{E}$,映射$(t,x)\mapsto Q_t(x,A)$是关于$\mathcal{B}(\R_+)\otimes \mathcal{E}$可测的。
        \end{enumerate}
        则称$(Q_t)_{t\geqslant 0}$是一个转移半群(transition semigroup).
    \end{definition}
    转移半群的概念类似于离散情形下的转移概率矩阵。

    \begin{definition}[转移算子]
        $Q$是一个转移核,
        如果映射$f:E\rightarrow \R$有界可测(或者非负可测),
        定义映射$Qf: E\rightarrow \R$为
        \begin{equation*}
            Qf(x)=\int Q(x,\d y)f(y)
        \end{equation*}
        那么$Qf$依然是有界可测(或者非负可测)的。
        记$B(E)$为$E$上的有界可测实函数全体构成的向量空间,并给予范数
        \begin{equation*}
            ||f||=\fun{sup}{}\{ |f(x)|:x\in E \}
        \end{equation*}
        那么就可以把$Q$视为一个$B(E)\rightarrow B(E)$的线性算子。
    \end{definition}
    \begin{corollary}
        根据$0\leqslant Q(x,A)\leqslant 1$可知,$Q$是一个压缩映射,
        算子范数$\leqslant 1<+\infty$,从而是连续算子。
    \end{corollary}

    \begin{proposition}
        从线性算子的观点来看,由Chapman-Kolmogorov identity可以得到
        \begin{equation*}
            Q_{s+t}=Q_tQ_s,\forall s,t\geqslant 0
        \end{equation*}
        这里$Q_tQ_s$是指$Q_t\circ Q_s$,即算子的复合。
    \end{proposition}
    \begin{proof}
        $\forall f\in B(E),x\in E$,由Chapman-Kolmogorov identity,
        \begin{align*}
            Q_{s+t}f(x)
            &=\int_E Q_{s+t}(x,\d z)f(z)\\
            &=\int_E  \left(\int_E Q_t(x,\d y)Q_s(y,\d z)\right) f(z)\\
            &\text{(交换积分顺序)}\\
            &=\int_E Q_t(x,\d y)\int_E Q_s(y,\d z) f(z)\\
            &=\int_E Q_t(x,\d y)Q_s f(y)\\
            &=Q_tQ_sf(x)
        \end{align*}
    \end{proof}
    这也表明$Q_tQ_s=Q_sQ_t$,即复合是可交换的。

\subsection{马氏过程}
    \begin{definition}[马氏过程]\label{Def of Cont-MarkovProcess}
        $(Q_t)_{t\geqslant 0}$是$E$上的转移半群,
        考虑滤流概率空间$(\Omega,\F,(\F_t)_{t\geqslant \infty},\P)$
        上的一个在$E$上取值的$(\F_t)$-适应过程$X=(X_t)_{t\geqslant 0}$,
        如果其满足$\forall s,t\geqslant 0,f\in B(E)$,
        \begin{equation*}
            \E[f(X_{s+t})|\F_s]=Q_t f(X_s)
        \end{equation*}
        那么就称$X$是(关于滤流$(\F_t)$的、转移半群为$(Q_t)$的连续时间)马氏过程。
    \end{definition}
    在本章的剩余内容中,如无特殊说明,默认马氏过程$X$是关于滤流$(\F_t^X)$的。

    如果取$f=I_A$,我们得到
    \begin{equation*}
        \P( X_{s+t}\in A|\F_s )=Q_t(X_s,A)
    \end{equation*}
    特别地,
    \begin{equation*}
        \P( X_{s+t}\in A|X_r,0\leqslant r\leqslant s )=Q_t(X_s,A)
    \end{equation*}
    这表明,给定“过去”$(X_r,0\leqslant r\leqslant s)$的条件下,
    $X_{s+t}$的条件分布只取决于“现在”的$X_s$,并由$Q_t(X_s,\cdot)$给出,这就是\textbf{马氏性}。
    
    \begin{theorem}[马氏过程的有限维分布]\label{Law of Cont-MarkovProcess}
        设$X_0$的分布为$\gamma(\d y)$,对于任意的$0=t_0<t_1<t_2<\cdots<t_p$,我们有
        \begin{equation*}
            \P(X_{t_0}\in A_0,\cdots,X_{t_p}\in A_p)
            =\int_{A_0} \gamma(\d x)\int_{A_1}Q_{t_1}(x,\d x_1)\int_{A_2} Q_{t_2-t_1}(x_1,\d x_2)
            \cdots \int_{A_p}Q_{t_p-t_{p-1}}(x_{p-1},\d x_p)
        \end{equation*}
        更一般地,有:
        \begin{equation*}
            \E[ f_0(X_{t_0})\cdots f_p(X_{t_p}) ]
            =\int_E f_0(x)\gamma (\d x)\int_E f_1(x_1)Q_{t_1}(x,\d x_1)
            \cdots \int_E f_p(x_p)Q_{t_p-t_{p-1}}(x_{p-1},\d x_p)
        \end{equation*}
    \end{theorem}
    \begin{proof}
        归纳:先考虑$p=0$的情形,即
        \begin{equation*}
            \E[ f_0(X_{t_0}) ]=\int_E f_0(x)\gamma(\d x)
        \end{equation*}
        命题成立;再假设$p-1$时命题成立,考虑$p$时的情形:
        \begin{align*}
            &\E[ f_0(X_{t_p})\cdots f_{p-1}(X_{t_{p-1}})f_p(X_{t_p}) ]\\
            &=\E[ \E[ f_0(X_{t_p})\cdots f_{p-1}(X_{t_{p-1}})f_p(X_{t_p}) |\F_{t_{p-1}}] ]\\
            &=\E[ f_0(X_{t_p})\cdots f_{p-1}(X_{t_{p-1}})\E[ f_p(X_{t_p}) |\F_{t_{p-1}}] ]\\
            &=\E[ f_0(X_{t_p})\cdots f_{p-1}(X_{t_{p-1}})Q_{t_p-t_{p-1}}f_p(X_{t_{p-1}}) ]\\
            &=\int_E f_0(x)\gamma (\d x)\int_E f_1(x_1)Q_{t_1}(x,\d x_1)
            \cdots \int_E f_{p-1}(x_{p-1})Q_{t_p-t_{p-1}}f_p(x_{p-1})Q_{t_{p-1}-t_{p-2},\d x_{p-1}}\\
            &=\int_E f_0(x)\gamma (\d x)\int_E f_1(x_1)Q_{t_1}(x,\d x_1)
            \cdots \int_E f_{p-1}(x_{p-1})Q_{t_{p-1}-t_{p-2},\d x_{p-1}}\int_E Q_{t_p-t_{p-1}}(x_{p-1},\d x_p)f_p(x_p)
        \end{align*}
        成立。综上结论得证。
    \end{proof}
    这个定理表明,马氏过程的有限维分布完全由其初始$X_0$的分布和转移半群决定。
    那么,给定转移半群$(Q_t)_{t\geqslant 0}$之后,我们取
    \begin{equation*}
        X=\{ (X_t)_{t\geqslant 0}: (X_t)_{t\geqslant 0}\text{是关于转移半群$(Q_t)_{t\geqslant 0}$的马氏过程} \}
    \end{equation*}
    其中,记初值为固定值$X_0=x$ a.s.的过程为$(X_t^x)_{t\geqslant 0}$,可以发现:
    \begin{equation*}
        Q_t f(x)=Q_t f(X_0^x)
        =\E[ f(X_t^x)|\F_0 ]=\E[ f(X_t^x) ]\defeq \E_x[ f(X_t) ]
    \end{equation*}
    由此,我们得到了转移算子的一个非常好用的表示方法。

    注意,我们接下来考虑马氏过程$X=(X_t)_{t\geqslant 0}$的时候,
    实际上考虑的是所有与转移半群$(Q_t)_{t\geqslant 0}$有关的马氏过程,
    也就是\textbf{所有“给定初始分布和转移半群$(Q_t)_{t\geqslant 0}$之后唯一确定的马氏过程”}。
    如无特殊说明,$X^x=(X_t^x)$就是指初值为固定值$x$的马氏过程;
    而如果我们在某个结论中没有指明初始分布,而是表述为$X=(X_t)$,就代表
    初始分布不影响此结论。

    回忆离散马氏链时的情形,当时我们也并没有给出一个具体的随机过程,
    而是重点分析了转移概率和状态分类。这是因为,时间齐次的马氏过程
    的关键不在于某一时刻的具体数值,而是一个时间间隔内的行为。
    \begin{example}
        回顾布朗运动的定义,我们当时规定了初值$B_0=0$,但事实上,初值可以换成任何一个
        固定的数值,甚至是一个分布。因为布朗运动的核心并不在于初始,
        而是“给定过去的分布,一段时间后的变化”。

        初值固定的布朗运动我们记作$\{ (B_t^x)_{t\geqslant 0}:x\in \R \}$,
        那么我们可以推导出布朗运动的转移半群为:
        \begin{equation*}
            Q_tf(x)=\E[ f(B_t^x) ]
            =\E[ f(B_t^0+x) ]
            =\int_{\R} \frac{1}{\sqrt{2\pi t}}{\rm exp}\left\{ -\frac{ (y-x)^2 }{2t} \right\}f(y)\d y
        \end{equation*}
    \end{example}

\section{马氏过程的构造}
    事先给定一个马氏转移半群$(Q_t)_{t\geqslant0}$,
    记
    \begin{equation*}
        \Omega^*=E^{\mathbb{R}_+}=\{\omega:\omega(\cdot):[0,\infty)\to E\}
    \end{equation*}
    可以定义$\Omega^*$上的坐标过程(Coordinate process)$X=(X_t)$,定义为
    \begin{equation*}
        X_t(\omega)=\omega(t):\Omega^*\to E
    \end{equation*}
    记$\mathcal{F}^*=\sigma(X_s,s\geqslant0),\mathcal{F}_t^X=\sigma(X_s,s\leqslant t)$,
    我们需要找到一个可测空间$(\Omega^*,\mathcal{F}^*)$上面的概率测度$\P^*$.
    
    对任意的有限子集$U=\{0<t_1<t_2<\cdots<t_p\}\subset\mathbb{R}_+$,
    定义一个在$E^U=\underbrace E\times E\times E\times\cdots\times E$上的概率
    测度,如下:
    \begin{equation*}
        \mu^U(A_1\times A_2\times\cdots\times A_p)=\int_E\gamma(\mathrm dx)\int_{A_1}Q_{t_1}(x,\mathrm dx_1)\cdots\int_{A_p}Q_{t_p-t_{p-1}}(x_{p-1},\mathrm dx_p)
    \end{equation*}
    记为$\{\mu^U\}$,其中$U$是有限子集。

    事实上,$\{\mu^U\}$具有相容性,即:对$U\subset V$,定义
    \begin{equation*}
        \pi_U^V:E^V\to E^U
    \end{equation*}
    其是一个投影(或限制),那么有
    \begin{equation*}
        \mu^U=\mu^V\circ(\pi_U^V)^{-1}
    \end{equation*}
    结合相容性和Kolmogorov扩张定理,一定存在一个$(\Omega^*,\mathcal{F}^*)$上的概率测度$\P^*$,
    使得
    \begin{equation*}
        \P^*(X_{t_1}\in A_1,\cdots,X_{t_p}\in A_p)=\mu^U(A_1\times\cdots\times A_p)
    \end{equation*}
    在$\P^*$之下,$(X_t)_{t\geqslant 0}$
    是一个与$\left(Q_t\right)$相联系的马氏过程,
    因为\autoref{Law of Cont-MarkovProcess}中的等式成立,
    而其可以推出马氏性。

\section{马氏过程的其他要素}
    本小节我们主要介绍三个工具概念:预解式、Feller半群、生成元,以及它们的性质与关系。
    用于进一步表述与研究马氏过程的性质。
\subsection{预解式}
    \begin{definition}
        设$\lambda>0$,半群$(Q_t)_{t\geqslant 0}$的$\lambda$-预解式(resolvent)是指线性算子$R_\lambda:\mathcal{B}(E)\rightarrow \mathcal{B}(E)$,
        其满足
        \begin{equation*}
            R_\lambda f(x)=\int_0^\infty {\rm e}^{-\lambda t} Q_tf(x)\d t
        \end{equation*}
        预解式实际上是一个Laplace变换。
    \end{definition}

    \begin{theorem}[预解式的性质]
        \begin{enumerate}[(1).]
            \item $||R_\lambda f||\leqslant \lambda^{-1}||f||$,这里的$||f||$为
                \begin{equation*}
                    ||f||=\fun{sup}{x\in E}|f(x)|
                \end{equation*}
            \item 如果$0\leqslant f\leqslant 1$,则$0\leqslant \lambda R_\lambda f(x)\leqslant 1$.
            \item 预解恒等式:$\lambda ,\mu>0$,则
                \begin{equation*}
                    R_\lambda -R_\mu+(\lambda-\mu)R_\lambda R_\mu=0
                \end{equation*}
        \end{enumerate}
    \end{theorem}
    \begin{proof}
        前两个由定义显然,我们只说明(3):$\forall f\in B(E),x\in E$,不妨$\lambda\neq \mu$,
        \begin{align*}
            R_\lambda R_\mu f(x)
            &=\R_\lambda(R_\mu f)(x)\\
            &=\int_0^{+\infty} {\rm e}^{-\lambda t}Q_t(R_\mu f)(x)\d t\\
            &=\int_0^{+\infty} {\rm e}^{-\lambda t} \int_E Q_t(x,\d y)R_u f(y) \d t\\
            &=\int_0^{+\infty} {\rm e}^{-\lambda t} \int_E Q_t(x,\d y) \int_0^{+\infty} {\rm e}^{-\mu s} \int_E Q_s(y,\d z)f(z)\d s \d t\\
            &\text{现在的积分顺序是$\d z\rightarrow \d s \rightarrow \d y\rightarrow \d t$,我们交换$\d s$和$\d y$的顺序}\\
            &=\int_0^{+\infty} \d t \int_0^{+\infty} \d s \int_E Q_t(x,\d y)\int_E Q_s(y,\d z)f(z){\rm e}^{-\lambda t-\mu s}\\
            &=\int_0^{+\infty} \d t \int_0^{+\infty} \d s Q_{s+t}f(x)  {\rm e}^{-\lambda t-\mu s}\\
            &\text{现在是我们熟悉的实数上的积分了,换元$r=s+t$,转化为区域$t\in (0,r),r\in (0,+\infty)$上的积分}\\
            &=\int_0^{+\infty} \d r \int_0^{r} \d t {\rm e}^{-(\lambda-\mu)t}{\rm e}^{-\mu r}Q_r f(x)\\
            &=\int_0^{+\infty} \frac{{\rm e}^{-\mu r}-{\rm e}^{-\lambda r}}{\lambda -\mu}Q_r f(x)d r\\
            &=\frac{R_\mu-R_\lambda}{\lambda-\mu}f(x)
        \end{align*}
    \end{proof}


    \begin{lemma}[一个积分]\label{a integration-20240531}
        $\lambda,\mu>0$,
        \begin{equation*}
            \int_0^\infty {\rm exp}\left\{ -\lambda t-\frac{\mu}{t} \right\}\frac{1}{\sqrt{t}}  \d t={\rm e}^{-2\sqrt{\lambda\mu}}\sqrt{\frac{\pi}{\lambda}}
        \end{equation*}
    \end{lemma}
    \begin{proof}
        设原积分为$I$,
        换元:
        \begin{equation*}
            t=\frac{\sqrt{\mu}}{\sqrt{\lambda}} s,\ \d t=\frac{\sqrt{\mu}}{\sqrt{\lambda}} \d s
        \end{equation*}
        则
        \begin{equation*}
            I=\int_0^\infty {\rm exp}\left\{ -\sqrt{\lambda\mu}(s+s^{-1}) \right\}\frac{1}{\sqrt{s}}  \sqrt{\frac{\sqrt{\mu}}{\sqrt{\lambda}}}\d s
        \end{equation*}
        整理一下得到
        \begin{equation*}
            \left( \frac{\lambda}{\mu} \right)^{\frac{1}{4}} I=\int_0^\infty {\rm exp}\left\{ -\sqrt{\lambda\mu}(s+s^{-1}) \right\}\frac{1}{\sqrt{s}}\d s
        \end{equation*}
        换元:
        \begin{equation*}
            s=u^2,\ \d s= 2u \d u
        \end{equation*}
        则
        \begin{equation*}
            \frac{1}{2}\left( \frac{\lambda}{\mu} \right)^{\frac{1}{4}} I
            =\int_0^\infty {\rm exp}\left\{ -\sqrt{\lambda\mu}(u^2+u^{-2}) \right\}\d u
        \end{equation*}
        换元:
        \begin{equation*}
            u=\frac{1}{v}, \d u= -\frac{1}{v^2}\d v
        \end{equation*}
        则
        \begin{equation*}
            \frac{1}{2}\left( \frac{\lambda}{\mu} \right)^{\frac{1}{4}} I
            =\int_0^\infty {\rm exp}\left\{ -\sqrt{\lambda\mu}(v^2+v^{-2}) \right\}\frac{1}{v^2}\d v
        \end{equation*}
        相加可得
        \begin{align*}
            \left( \frac{\lambda}{\mu} \right)^{\frac{1}{4}} I
            &=
            \int_0^\infty {\rm exp}\left\{ -\sqrt{\lambda\mu}(u^2+u^{-2}) \right\}(1+\frac{1}{u^2})\d u\\
            &={\rm e}^{-2\sqrt{\lambda\mu}}
            \int_0^\infty {\rm exp}\left\{ -\sqrt{\lambda\mu}(u-u^{-1})^2 \right\}(1+\frac{1}{u^2})\d u
        \end{align*}
        换元:
        \begin{equation*}
            w=u-\frac{1}{u},\ \d w=(1+\frac{1}{u^2})\d u
        \end{equation*}
        则
        \begin{align*}
            {\rm e}^{2\sqrt{\lambda\mu}}\left( \frac{\lambda}{\mu} \right)^{\frac{1}{4}} I
            &=\int_{-\infty}^\infty {\rm exp}\left\{ -\sqrt{\lambda\mu}w^2 \right\}\d w\\
            &=(\lambda \mu)^{-\frac{1}{4}}\sqrt{\pi}
        \end{align*}
        得到
        \begin{equation*}
            I={\rm e}^{-2\sqrt{\lambda\mu}}\sqrt{\frac{\pi}{\lambda}}
        \end{equation*}
    \end{proof}
    \begin{example}
        考虑布朗运动的转移半群:
        \begin{equation*}
            Q_t(x,A)=\int_A \frac{1}{\sqrt{2\pi t}}{\rm exp}\left\{ -\frac{(y-x)^2}{2t} \right\}\d y
        \end{equation*}
        $\lambda>0$,求其预解式$\R_\lambda$.
    \end{example}
    \begin{proof}
        \begin{align*}
            R_\lambda f(x)
            &=\int_0^\infty {\rm e}^{-\lambda t}\int_E Q_t(x,\d y)f(y)\d t\\
            &=\int_0^\infty {\rm e}^{-\lambda t} \int_{\R} \frac{1}{\sqrt{2\pi t}}{\rm exp}\left\{ -\frac{(y-x)^2}{2t} \right\}f(y)\d y\d t\\
            &\text{(先对$\d t$积分)}\\
            &=\int_{\R} f(y)\d y
            \int_0^\infty {\rm e}^{-\lambda t}\frac{1}{\sqrt{2\pi t}}{\rm exp}\left\{ -\frac{(y-x)^2}{2t} \right\} \d t\\
            &=\int_{\R} f(y)r_\lambda(y-x)\d y
        \end{align*}
        由\autoref{a integration-20240531},其中
        \begin{equation*}
            r_\lambda(y-x)=\frac{1}{\sqrt{2\lambda}}{\rm exp}\{ -|y-x|\sqrt{2\lambda} \}
        \end{equation*}
    \end{proof}

    \begin{example}
        $(X_t)$是关于半群$(Q_t)$的马氏过程,如果$h\in \mathcal{B}(E),h\geqslant 0,\lambda \geqslant 0$,则
        \begin{equation*}
            M_t={\rm e}^{-\lambda t}R_\lambda h(X_t),\ t\geqslant 0
        \end{equation*}
        是一个上鞅。
    \end{example}
    \begin{proof}
        $M_t$是关于$X_t$的函数,所以$M_t\in \F_t$,并且有界(从而可积)。
        注意到$\forall h\in B(E),x\in E,s\geqslant 0$,
        \begin{align*}
            Q_s R_\lambda h(x)
            &=Q_s(R_\lambda h)(x)\\
            &=\int_E Q_s(x,\d y)R_\lambda h(y)\\
            &=\int_E Q_s(x,\d y)\int_0^\infty {\rm e}^{-\lambda t}Q_t h(y)\d t\\
            &=\int_0^\infty {\rm e}^{-\lambda t}Q_{s+t}h(x)\d t
        \end{align*}
        从而可得
        \begin{equation*}
            {\rm e}^{-\lambda s}Q_s R_\lambda h(x)
            =\int_0^\infty {\rm e}^{-\lambda (s+t)}Q_{s+t}h(x)\d t
            =\int_s^\infty {\rm e}^{-\lambda t}Q_t h(x) \d t
            \leqslant R_\lambda h(x)
        \end{equation*}
        那么
        \begin{equation*}
            \E[ M_{s+t}|\F_t ]
            =\E[ {\rm e}^{-\lambda (s+t)}R_\lambda h(X_{s+t})|\F_t ]
            ={\rm e}^{-\lambda (s+t)}R_\lambda h(X_t)
            \leqslant {\rm e}^{-\lambda t}R_\lambda h(X_t)=M_t
        \end{equation*}
        所以$M_t$是上鞅。
    \end{proof}

\subsection{Feller半群}
    现在我们假设$E$是一个Polish空间,如果映射$f:E\rightarrow \R$满足:
    $\forall \varepsilon>0$,存在紧集$K\subset E$使得
    $|f(x)|\leqslant \varepsilon,\forall x\in E\backslash K$,
    则称$f$在无穷远处趋于$0$,这样的$f$全体记作$C_0(E)$.

    取范数
    \begin{equation*}
        ||f||=\fun{sup}{x\in E}|f(x)|
    \end{equation*}
    那么$(C_0(E),||\cdot ||)$成为一个Banach空间。
    \begin{definition}\label{Feller semigroup}
        $(Q_t)_{t\geqslant 0}$是转移半群,称之为Feller半群,如果:
        \begin{enumerate}[(1).]
            \item $\forall f\in C_0(E)$,$Q_t f\in C_0(E)$.
            \item $\forall f\in C_0(E)$,$|| Q_tf-f ||\rightarrow 0$ as $t\rightarrow 0$.
        \end{enumerate}
        如果一个马氏过程$X$的转移半群是Feller半群,就称之为Feller过程。
    \end{definition}

    \begin{corollary}
        $(Q_t)_{t\geqslant 0}$是Feller半群,固定$f\in C_0(E)$,映射
        \begin{equation*}
            \R_+\rightarrow C_0(E),\ t\mapsto Q_t f
        \end{equation*}
        是一致连续的。
    \end{corollary}
    \begin{proof}
        由\autoref{Feller semigroup}(2),$\forall s\geqslant 0$,
        因为$Q_s$是压缩映射,
        \begin{equation*}
            \fun{lim}{t\rightarrow 0^+}|| Q_{s+t}f-Q_s f ||
            =\fun{lim}{t\rightarrow 0^+}|| Q_s( Q_tf-f ) ||=0
        \end{equation*}
        并注意到上述极限关于$s\in \R_+$一致收敛。
    \end{proof}

    \begin{corollary}
        $(Q_t)_{t\geqslant 0}$是Feller半群,
        其预解式满足:$\forall \lambda >0$,$f\in C_0(E)$,都有$R_\lambda f\in C_0(E)$.
    \end{corollary}
    \begin{proof}
        由\autoref{Feller semigroup}(1)和DCT即得。
    \end{proof}
    
    \begin{theorem}
        对于Feller半群$(Q_t)_{r\geqslant 0}$,任取$\lambda >0$,定义
        \begin{equation*}
            R=\{ R_\lambda f:f\in C_0(E) \}
        \end{equation*}
        则$R$不取决于$\lambda$的选取,并且是$C_0(E)$的稠密子集。
    \end{theorem}
    \begin{proof}
        对于$\lambda \neq \mu$,由预解恒等式,
        \begin{equation*}
            R_\lambda f=R_\mu( f+(\mu-\lambda)R_\lambda f )
        \end{equation*}
        这说明$R_\lambda f\in \{  R_\mu f:f\in C_0(E)  \}$,
        即$R$不取决于$\lambda$的选取。
        
        $R$是$C_0(E)$的线性子空间,注意到
        \begin{equation*}
            \lambda R_\lambda f=\lambda \int_0^\infty {\rm e}^{-\lambda t}Q_t f\d t
            =\int_0^\infty {\rm e}^{-\lambda t}Q_{t\lambda^{-1}}f\d t\rightarrow f
            {\rm\ as\ }\lambda\rightarrow\infty
        \end{equation*}
        所以稠密性得证。
    \end{proof}

\subsection{生成元}
    本小节我们默认在Feller半群$(Q_t)_{t\geqslant 0}$上讨论。
    \begin{definition}
        令
        \begin{equation*}
            D(L)=\{ f\in C_0(E): t\rightarrow 0^+\text{时,}\frac{Q_tf-f}{t}\text{在$C_0(E)$中收敛}  \}
        \end{equation*}
        对于$f\in D(L)$,则定义
        \begin{equation*}
            Lf=\fun{lim}{t\rightarrow 0^+}\frac{Q_tf-f}{t}
        \end{equation*}
        于是$L$成为一个线性算子,称之为生成元(generator),$D(L)$称为$L$的定义域。
    \end{definition}

    \begin{proposition}[生成元与转移算子可交换]
        $f\in D(L)$,$s>0$,则$Q_sf\in D(L)$,并且$L(Q_s f)=Q_s(Lf)$.
    \end{proposition}
    \begin{proof}
        由于$Q_tQ_s$可交换,
        \begin{equation*}
            \frac{Q_t( Q_sf )-Q_s f}{t}=Q_s\left(\frac{ Q_tf-f }{t}\right)
        \end{equation*}
        因为$Q_s$是一个连续算子,所以
        \begin{equation*}
            L(Q_s f)=
            \fun{lim}{t\rightarrow 0^+}
            \frac{Q_t( Q_sf )-Q_s f}{t}
            =Q_s\left(\fun{lim}{t\rightarrow 0^+} \frac{ Q_tf-f }{t}\right)
            =Q_s(L f)
        \end{equation*}
    \end{proof}

    \begin{proposition}\label{integration of generator}
        $f\in D(L)$,$\forall t\geqslant 0$,都有
        \begin{equation*}
            Q_t f=f+\int_0^t Q_s(Lf)\d s
            =f+\int_0^t L(Q_sf)\d s
        \end{equation*}
    \end{proposition}
    \begin{proof}
        \begin{align*}
            \fun{lim}{s\rightarrow 0}\frac{Q_{t+s}f-Q_tf}{s}
            &=\fun{lim}{s\rightarrow 0}Q_t\left( \frac{Q_sf-f}{s} \right)\\
            &=Q_t(Lf)
        \end{align*}
        积分可得。
    \end{proof}

    \begin{proposition}[生成元与预解式的关系]
        $\lambda >0$,
        \begin{enumerate}[(1).]
            \item $\forall g\in C_0(E)$,有$R_\lambda g\in D(L)$且$(\lambda -L)R_\lambda g=g$.
            \item $f\in D(L)$,则$R_\lambda (\lambda -L)f=f$.           
        \end{enumerate}
        结合上述结论,可知$R=D(L)$,$R_\lambda=(\lambda -L)^{-1}$.
    \end{proposition}
    \begin{proof}
        \begin{enumerate}[(1).]
            \item 对于$g\in C_0(E)$,考虑
                \begin{align*}
                    Q_s(R_\lambda g)-R_\lambda g
                    &=Q_s\left( \int_0^\infty {\rm e}^{-\lambda t}Q_t g\d t \right)-R_\lambda g \\
                    &=\int_0^\infty {\rm e}^{-\lambda t}Q_{s+t} g\d t -R_\lambda g \\
                    &={\rm e}^{\lambda s}\int_0^\infty {\rm e}^{-\lambda (s+t)}Q_{s+t} g\d t -R_\lambda g \\
                    &={\rm e}^{\lambda s}\int_s^\infty {\rm e}^{-\lambda t}Q_{t} g\d t -R_\lambda g \\
                    &={\rm e}^{\lambda s}\left(R_\lambda g-\int_0^s {\rm e}^{-\lambda t}Q_{t} g\d t\right)-R_\lambda g \\
                    &=({\rm e}^{\lambda s}-1)R_\lambda g-{\rm e}^{\lambda s}\int_0^s {\rm e}^{-\lambda t}Q_{t} g\d t
                \end{align*}
                从而可得
                \begin{align*}
                    L(R_\lambda g)=&\fun{lim}{s\rightarrow 0^+}\frac{1}{s}(Q_s(R_\lambda g)-R_\lambda g)\\
                    =&\fun{lim}{s\rightarrow 0^+}\left(\frac{{\rm e}^{\lambda s}-1}{s}R_\lambda g-{\rm e}^{\lambda s}\frac{1}{s}\int_0^s {\rm e}^{-\lambda t}Q_{t} g\d t\right)\\
                    =&\lambda R_\lambda g-g
                \end{align*}
            \item 对于$f\in D(L)$,考虑
                \begin{align*}
                    R_\lambda (Lf)
                    &=\int_0^\infty {\rm e}^{-\lambda t}Q_t(Lf)\d t\\
                    &=\int_0^\infty {\rm e}^{-\lambda t}L(Q_t f)\d t\\
                    &=\int_0^\infty {\rm e}^{-\lambda t} \fun{lim}{s\rightarrow 0^+}\frac{Q_{s+t}f-Q_tf}{s} \d t\\
                    &=\fun{lim}{s\rightarrow 0^+} \frac{1}{s}\int_0^\infty {\rm e}^{-\lambda t} (Q_{s+t}f-Q_tf) \d t\\
                    &=\fun{lim}{s\rightarrow 0^+} \frac{1}{s}\left(\int_0^\infty {\rm e}^{-\lambda t} Q_{s+t}f \d t-R_\lambda f\right)\\
                    &\text{(这里类似于(1)的证明)}\\
                    &=\fun{lim}{s\rightarrow 0^+} \frac{1}{s}\left( {\rm e}^{\lambda s}\left( R_\lambda f-\int_0^s {\rm e}^{-\lambda t} Q_{t}f \d t \right) -R_\lambda f\right)\\
                    &=\fun{lim}{s\rightarrow 0^+} \frac{1}{s}\left( ({\rm e}^{\lambda s}-1)R_\lambda f-{\rm e}^{\lambda s}\int_0^s {\rm e}^{-\lambda t} Q_{t}f \d t \right)\\
                    &=\lambda R_\lambda f-f
                \end{align*}
        \end{enumerate}
    \end{proof}

    \begin{example}
        布朗运动的转移半群由下式给出:
        \begin{equation*}
            Q_tf(x)=\int_{\R} f(y)\frac{1}{\sqrt{2\pi t}}{\rm exp}\left\{ -\frac{(y-x)^2}{2t} \right\}\d y
            =\E[ f(B_t+x) ]
        \end{equation*}
        其中$f\in C_0(E)$,并设$f$光滑,证明:
        \begin{equation*}
            Lf(x)=\frac{1}{2}f''(x),\ x\in\R
        \end{equation*}
    \end{example}
    \begin{proof}
        我们先计算:
        \begin{align*}
            Q_tf(x)-f(x)
            &=\E[ f( B_t^x )-f(x) ]\\
            &=\int_{\R} [f(y)-f(x)]\frac{1}{\sqrt{2\pi t}}{\rm exp}\left\{ -\frac{(y-x)^2}{2t} \right\}\d y\\
            &=\int_{\R} [ f'(x)(y-x)+f''(x)\frac{(y-x)^2}{2}+f'''(\theta_{x,y})\frac{(y-x)^3}{6} ]\frac{1}{\sqrt{2\pi t}}{\rm exp}\left\{ -\frac{(y-x)^2}{2t} \right\}\d y\\
            &=f'(x)\E[ B_t^0 ]+\frac{1}{2}f''(x)\E[ (B_t^0)^2 ]+\int_{\R} [f'''(\theta_{x,y})\frac{(y-x)^3}{6} ]\frac{1}{\sqrt{2\pi t}}{\rm exp}\left\{ -\frac{(y-x)^2}{2t} \right\}\d y\\
            &=\frac{1}{2}f''(x)t+\int_{\R} [f'''(\theta_{x,y})\frac{(y-x)^3}{6} ]\frac{1}{\sqrt{2\pi t}}{\rm exp}\left\{ -\frac{(y-x)^2}{2t} \right\}\d y
        \end{align*}
        因此,
        \begin{align*}
            Lf(x)-\frac{1}{2}f''(x)
            &=\fun{lim}{t\rightarrow 0^+}\frac{Q_tf(x)-f(x)-\frac{1}{2}f''(x)}{t}\\
            &=\fun{lim}{t\rightarrow 0^+}\frac{\int_{\R} [f'''(\theta_{x,y})\frac{(y-x)^3}{6} ]\frac{1}{\sqrt{2\pi t}}{\rm exp}\left\{ -\frac{(y-x)^2}{2t} \right\}\d y}{t}
        \end{align*}
        \begin{lemma}
            $f\in C_0(E)$且$f$光滑,则$f'\in C_0(E)$,进而$f$的任意阶导数都满足此性质。
            \begin{proof}
                只需证明$f'$在无穷远处趋于零,假设不然:
                $\exists \varepsilon_0>0$使得$\forall M>0$,存在$x_0$满足$|x_0|>M$且$f'(x_0)>\varepsilon_0$,
                根据$f'$的连续性,存在$a>0$使得
                \begin{equation*}
                    f'(x)>\varepsilon_0,\ x\in (x_0,x_0+a)
                \end{equation*}
                从而得到
                \begin{equation*}
                    f(x_0+a)-f(x_0)>\varepsilon_0 a
                \end{equation*}
                同时,我们取$\varepsilon<\frac{1}{2}a \varepsilon_0$,由于$f$在无穷远处趋于零,
                所以存在$M>0$,使得
                \begin{equation*}
                    \fun{sup}{|x|>M} |f(x)|<\varepsilon
                \end{equation*}
                从而
                \begin{equation*}
                    \fun{sup}{|x|,|y|>M} |f(x)-f(y)|<2\varepsilon<\varepsilon_0 a
                \end{equation*}
                矛盾。
            \end{proof}
        \end{lemma}
        由引理可见,$f'''\in C_0(E)$,从而是有界的,这表明
        \begin{align*}
            \left|\int_{\R} [f'''(\theta_{x,y})\frac{(y-x)^3}{6} ]\frac{1}{\sqrt{2\pi t}}{\rm exp}\left\{ -\frac{(y-x)^2}{2t} \right\}\d y\right|
            \leqslant 
            C\E[ |B_t^0|^3 ]=C\cdot t^{ \frac{3}{2} }
        \end{align*}
        这说明
        \begin{equation*}
            \fun{lim}{t\rightarrow 0^+}\left|\int_{\R} [f'''(\theta_{x,y})\frac{(y-x)^3}{6} ]\frac{1}{\sqrt{2\pi t}}{\rm exp}\left\{ -\frac{(y-x)^2}{2t} \right\}\d y\right|
            \leqslant \fun{lim}{t\rightarrow 0^+} Ct^{\frac{1}{2}}=0
        \end{equation*}
        结论得证。
    \end{proof}

    \begin{theorem}\label{Martingale about generator}
        设$h,g\in C_0(E)$,以下命题等价:
        \begin{enumerate}[(1).]
            \item $h\in D(L)$,$Lh=g$.
            \item $\forall x\in E$,
                \begin{equation*}
                    M_t^x=h(X_t^x)-\int_0^t g(X_s^x)\d s,\ t\geqslant 0
                \end{equation*}
                是一个鞅。
        \end{enumerate}
    \end{theorem}
    \begin{proof}
        $(1)\Rightarrow (2)$:适应性和可积性不再赘述,我们验证:
        \begin{equation*}
            \E[ M_{s+t}^x|\F_s ]=M_s^x
        \end{equation*}
        观察左式,
        \begin{align*}
            \E[ M_{s+t}^x|\F_s ]
            &=\E\left[ \left. h(X_{s+t}^x)-\int_0^{s+t} g(X_u^x)\d u \right| \F_s \right]\\
            &=\E[ h(X_{s+t}^x)|\F_s ]-\int_0^s g(X_u^x)\d u-\E\left[ \left. \int_s^{s+t} g(X_u^x)\d u \right| \F_s \right]\\
            &=Q_th(X_s^x)-\int_0^s g(X_u^x)\d u-\E\left[ \left. \int_0^{t} g(X_{u+s}^x)\d u \right| \F_s \right]\\
            &=Q_th(X_s^x)-\int_0^s g(X_u^x)\d u-\int_0^{t}\E\left[ \left. g(X_{u+s}^x) \right| \F_s \right]\d u\\
            &=Q_th(X_s^x)-\int_0^s g(X_u^x)\d u-\int_0^{t} Q_u(g(X_s^x)) \d u
        \end{align*}
        由\autoref{integration of generator}可得
        \begin{equation*}
            \int_0^{t} Q_u(g(X_s^x)) \d u=\int_0^{t} Q_u(Lh(X_s^x)) \d u
            =Q_th(X_s^x)-h(X_s^x)
        \end{equation*}
        这说明
        \begin{align*}
            \E[ M_{s+t}^x|\F_s ]&=Q_th(X_s^x)-\int_0^s g(X_u^x)\d u-Q_th(X_s^x)+h(X_s^x)\\
            &=h(X_s^x)-\int_0^s g(X_u^x)\d u=M_s^x
        \end{align*}

        $(2)\Rightarrow (1)$:$M_t^x$是鞅,则
        \begin{equation*}
            \E[ M_t^x ]=\E[ M_0^x ]=h(x)
        \end{equation*}
        另一方面,
        \begin{align*}
            \E[ M_t^x ]
            &=\E[ h(X_t^x) ]-\E\left[ \int_0^t g(X_s^x)\d s \right]\\
            &=Q_th(x)-\int_0^t \E[g(X_s^x)] \d s\\
            &=Q_th(x)-\int_0^t Q_tg(x) \d s
        \end{align*}
        所以
        \begin{equation*}
            Lh(x)=\fun{lim}{t\rightarrow 0^+}\frac{Q_th-h}{t}(x)=
            \fun{lim}{t\rightarrow 0^+}\frac{1}{t}\int_0^t Q_t g(x)\d s=g(x)
        \end{equation*}
    \end{proof}

\section{马氏性与强马氏性}
    回顾一下之前的内容:
    考虑过程$(X_t^x)$,半群:
    \begin{equation*}
        Q_tf(x)=\E[ f(X_t^x) ],\ t\geqslant 0
    \end{equation*}
    预解式:
    \begin{equation*}
        R_\lambda f(x)=\int_0^\infty {\rm e}^{-\lambda t}Q_tf(x)\d t,\ \lambda >0
    \end{equation*}
    生成元:
    \begin{equation*}
        Lf=\fun{lim}{t\rightarrow 0^+}\frac{Q_tf-f}{t}
    \end{equation*}

    \begin{theorem}
        $(X_t)_{t\geqslant 0}$是马氏过程,$(Q_t)_{t\geqslant 0}$是相应的Feller半群,
        那么一定存在$X$的一个修正$X'$,使得$(Q_t)_{t\geqslant 0}$也是$X'$的Feller半群,且具有RCLL轨道。
    \end{theorem}
    我们省略详细证明\footnote{大致思路为:考虑上鞅$\{ {\rm e}^{-pt}g(X_t) \}$,对其进行轨道修正。}
    ,以下均不妨假设马氏过程有RCLL轨道。记
    \begin{equation*}
        D(E)=\{ f(t):[0,+\infty)\rightarrow E\text{是RCLL函数} \}
    \end{equation*}
    固定$\omega\in\Omega$,
    $t\mapsto \omega(t)\defeq X_t(\omega)$是RCLL函数a.s.,
    所以不妨认为$D(E)$就是$\Omega$,其上的$\sigma$-域就是:
    \begin{equation*}
        \mathcal{B}( D(E) )=\sigma(X_t(\omega)=\omega(t)\in D(E),0\leqslant t<+\infty)
    \end{equation*}
    \begin{definition}[时间平移算子]
        对于随机过程$(Y_t)_{t\geqslant 0}$,定义算子:
        \begin{equation*}
            Y_t\circ \theta_s=Y_{t+s},\ s\geqslant 0
        \end{equation*}
    \end{definition}

    \begin{theorem}[马氏性的一般表述]
        $(Q_t)_{t\geqslant 0}$为转移半群,
        $X=(X_t)_{t\geqslant 0}$是相应的马氏过程,
        且具有RCLL轨道。
        设$\Phi:D(E)\rightarrow \R$有界可测,
        那么对于$\forall s\geqslant 0$,
        \begin{equation*}
            \E[ \Phi(X\circ \theta_s)|\F_s ]=\E_{X_s} [ \Phi(X) ]
        \end{equation*}
    \end{theorem}
    \begin{remark}
        其中,$X$视为一个把$\omega\in\Omega$映为样本轨道的映射,那么
        $\Phi(X)$就是一个随机变量,相当于:
        \begin{equation*}
            \omega\mapsto \Phi\left( u\mapsto X_u(\omega) \right)
        \end{equation*}

        不难看出,如果令
        \begin{equation*}
            \Phi(X(\omega))=\Phi\left( u\mapsto X_u(\omega) \right)=f(X_t(\omega))
        \end{equation*}
        其中$t\geqslant 0$是一个固定的常数,$f$是一个固定的有界可测实函数,
        带入到本定理的结论中就得到
        \begin{equation*}
            \E[ f(X_{t+s})|\F_s ]=\E_{X_s}[f(X_t)]
            =Q_t f(X_s)
        \end{equation*}
        这正是我们在\autoref{Def of Cont-MarkovProcess}中的表述,
        这说明本定理给出的表述是更一般化的马氏性表述。
    \end{remark}
    \begin{proof}
        根据单调类定理,我们只需证明对以下形式的$\Phi$成立即可:
        \begin{equation*}
            \Phi(X(\omega))=\varphi_1(X_{t_1})\varphi_2(X_{t_2})\cdots \varphi_p(X_{t_p})
        \end{equation*}
        其中$t_1<t_2<\cdots,t_p$,$\forall \varphi_i\in B(E)$,也就是证明:
        \begin{equation*}
            \E[ \varphi_1(X_{t_1+s})\varphi_2(X_{t_2+s})\cdots \varphi_p(X_{t_p+s})|\F_s ]
            =\E_{X_s}[ \varphi_1(X_{t_1})\varphi_2(X_{t_2})\cdots \varphi_p(X_{t_p}) ]
        \end{equation*}
        归纳:$p=1$时由定义即得,设$p-1$时成立,考虑$p$的情形:
        \begin{align*}
            {\rm LHS}
            &=\E[ \varphi_1(X_{t_1+s})\varphi_2(X_{t_2+s})\cdots \varphi_p(X_{t_p+s})|\F_s ]\\
            &=\E[ \E[ \varphi_1(X_{t_1+s})\varphi_2(X_{t_2+s})\cdots \varphi_p(X_{t_p+s})|\F_{t_{p-1}+s} ]  |\F_s ]\\
            &=\E[ \varphi_1(X_{t_1+s})\varphi_2(X_{t_2+s})\cdots \varphi_{p-1}(X_{t_{p-1+s}})\E[ \varphi_p(X_{t_p+s})|\F_{t_{p-1}+s} ]  |\F_s ]\\
            &=\E[ \varphi_1(X_{t_1+s})\varphi_2(X_{t_2+s})\cdots \varphi_{p-1}(X_{t_{p-1+s}})Q_{t_p-t_{p-1}}\varphi_p(X_{ t_{p-1}+s })  |\F_s ]\\
            &=\E_{X_s}[ \varphi_1(X_{t_1})\varphi_2(X_{t_2})\cdots \varphi_{p-1}(X_{t_{p-1}}) Q_{t_p-t_{p-1}}\varphi_p(X_{t_{p-1}})]\\
            {\rm RHS}
            &=\E_{X_s}[ \varphi_1(X_{t_1})\varphi_2(X_{t_2})\cdots \varphi_{p}(X_{t_{p}}) ]\\
            &=\E_{X_s}[ \E[\varphi_1(X_{t_1})\varphi_2(X_{t_2})\cdots \varphi_{p}(X_{t_{p}})|\F_{t_p-1}] ]\\
            &=\E_{X_s}[ \varphi_1(X_{t_1})\varphi_2(X_{t_2})\cdots \varphi_{p-1}(X_{t_{p-1}})Q_{t_p-t_{p-1}}\varphi_p(X_{t_{p-1}}) ]
        \end{align*}
        二者相等。综上结论得证。
    \end{proof}

    \begin{theorem}[强马氏性]
        $(Q_t)_{t\geqslant 0}$为Feller半群,
        $X=(X_t)_{t\geqslant 0}$是相应的马氏过程,
        且具有RCLL轨道。设$\Phi:D(E)\rightarrow \R$有界可测,$T$是停时,则
        \begin{equation*}
            \E[ \Phi(X\circ \theta_T)\cdot I_{ \{T<+\infty\}} |\F_T ]=I_{ \{T<+\infty\}}\cdot \E_{X_T}[ \Phi(X) ]
        \end{equation*}
    \end{theorem}
    \begin{proof}
        根据单调类定理,我们只需证明对以下形式的$\Phi$成立即可:
        \begin{equation*}
            \Phi(X(\omega))=\varphi_1(X_{t_1})\varphi_2(X_{t_2})\cdots \varphi_p(X_{t_p})
        \end{equation*}
        其中$t_1<t_2<\cdots,t_p$,$\forall \varphi_i\in C_0(E)$,也就是证明:
        \begin{equation*}
            \E[ \varphi_1(X_{t_1+T})\varphi_2(X_{t_2+T})\cdots \varphi_p(X_{t_p+T}) I_{ \{ T<+\infty \} } |\F_T ]
            =I_{ \{ T<+\infty \}}\E_{X_T}[ \varphi_1(X_{t_1})\varphi_2(X_{t_2})\cdots \varphi_p(X_{t_p}) ]
        \end{equation*}
        由于$I\{T<\infty\}E_{X_T}[\Phi(X)]$关于$\mathcal{F}_T$可测,
        由条件期望的定义,只需证明:对$A\in\mathcal{F}_T$,有
        \begin{equation*}
            E\begin{bmatrix}I_{A\cap\{T<\infty\}}\Phi(X\circ\theta_T)\end{bmatrix}=E\begin{bmatrix}I_{A\cap\{T<\infty\}}E_{X_T}[\Phi(X)]\end{bmatrix}
        \end{equation*}
        归纳:先证明其对$p=1$的情况成立,此时$\Phi(X(\omega))=\phi_1(X_{t_1}(\omega))$,
        \begin{equation*}
            \text{LHS}
            =E\left[I_{A\cap\{T<\infty\}}\Phi(X\circ\theta_T)\right]=E\left[I_{A\cap\{T<\infty\}}\phi_1(X_{t_1+T})\right]
        \end{equation*}
        定义
        \begin{equation*}
            [T]_n=\frac k{2^n},\text{ 如果 }\frac{k-1}{2^n}\leqslant T<\frac k{2^n},\:k=0,1,\cdots 
        \end{equation*}
        那么有$[T]_n\searrow T$。进而,
        \begin{align*}
            \text{LHS}& =\lim_{n\to\infty}E\left[I_{A\cap\{T<\infty\}}\phi_{1}(X_{t_{1}+[T]_{n}})\right]  \\
            &=\lim_{n\to\infty}\sum_{k=1}^nE\left[I_{A\cap\{\frac{k-1}{2^n}\leqslant T<\frac{k}{2^n}\}}\phi_1(X_{t_1+\frac{k}{2^n}})\right] \\
            &=\lim_{n\to\infty}\sum_{k=1}^nE\left[I_{A\cap\{\frac{k-1}{2^n}\leqslant T<\frac{k}{2^n}\}}E\left[\phi_1(X_{t_1+\frac{k}{2^n}})\Big|\mathcal{F}_{\frac{k}{2^n}}\right]\right]\quad(A\in\mathcal{F}_T) \\
            &=\lim_{n\to\infty}\sum_{k=1}^nE\left[I_{A\cap\{\frac{k-1}{2^n}\leqslant T<\frac{k}{2^n}\}}Q_{t_1}\phi_1(X_{\frac{k}{2^n}})\right]\quad\text{(马氏性)} \\
            &=\lim_{n\to\infty}E\begin{bmatrix}I_{A\cap\{T<\infty\}}Q_{t_1}\phi_1(X_{[T]_n})\end{bmatrix} \\
            &=E\begin{bmatrix}I_{A\cap\{T<\infty\}}Q_{t_1}\phi_1(X_T)\end{bmatrix}\quad\text{(Feller 性质)}
        \end{align*}
        另一方面,
        \begin{equation*}
            \mathrm{RHS}=E\left[I_{A\cap\{T<\infty\}}E_{X_{T}}[\phi_{1}(X_{t_{1}})]\right]
            =E\begin{bmatrix}I_{A\cap\{T<\infty\}}Q_{t_1}\phi_1(X_{t_1})\end{bmatrix}
        \end{equation*}
        二者相等,于是得证。$p-1$到$p$的递推的证明和之前类似,不再赘述。
    \end{proof}

\section{两个例子:跳跃过程与Levy过程}
\subsection{跳跃过程}
    如果状态空间$E$是至多可数集,我们采用离散度量诱导的拓扑:
    \begin{equation*}
        d(x,y)=\delta_{x,y},\ \forall x,y,\in E
    \end{equation*}
    $E$的子集全体作为其$\sigma$-域。
    容易验证任意$\varphi:E\rightarrow\R$都是连续的,
    所以$C_0(E)=B(E)$. 
    
    用$D(E)$代表$E$上的RCLL函数全体,
    注意到$\forall f\in D(E)$,右连续$\Rightarrow f(0+)=f(0)$,
    故存在$t_1>0$使得
    \begin{equation*}
        f(t)=f(0),\ \forall t\in [0,t_1)
    \end{equation*}
    那么,我们就令
    \begin{equation*}
        \gamma_1^f\defeq \fun{sup}{}\{ t_1\geqslant 0:f(t)=f(0),\ \forall t\in [0,t_1) \}
    \end{equation*}
    即$f$第一次改变初始状态的时刻(可以取$+\infty$),
    如果$\gamma_1^f<+\infty$,以此类推:
    \begin{align*}
        \gamma_2^f&\defeq \fun{sup}{}\{ t_2\geqslant \gamma_1^f:f(t_2)=f(\gamma_1^f),\ \forall t\in [\gamma_1^f,t_2) \}\\
        \gamma_3^f&\defeq \fun{sup}{}\{ t_3\geqslant \gamma_1^f:f(t_3)=f(\gamma_2^f),\ \forall t\in [\gamma_2^f,t_3) \}\\
        \vdots&
    \end{align*}
    一个简单的推论是:$\gamma_n^f\nearrow+\infty$,否则,
    假设$\gamma_n^f\nearrow \gamma^f<+\infty$,则
    \begin{equation*}
        \fun{lim}{n\rightarrow\infty} f(\gamma_n^f)=f(\gamma^f-) 
    \end{equation*}
    从而存在$N$使得$n>N$时$f(\gamma_n^f)=f(\gamma^f-)$,矛盾。

    现在我们考虑$E$上的Feller半群$(Q_t)_{t\geqslant 0}$,
    $X$是相应的(轨道RCLL的)马氏过程,我们记概率测度$\P_x$是满足$\P_x(X_0=x)=1$的测度,
    $\E_x$是$\P_x$下的期望。
    那么,固定$\omega\in \Omega$,得到RCLL的样本轨道
    $t\mapsto X_t(\omega)$,根据前文中的分析,存在一系列
    \begin{equation*}
        0=T_0(\omega)\leqslant T_1(\omega)\leqslant T_2(\omega)\leqslant \cdots \leqslant +\infty
    \end{equation*}
    满足:如果$T_i(\omega)<+\infty$,那么
    \begin{equation*}
        X_t(\omega)=X_{T_i}(\omega),\ \forall t\in [T_i(\omega),T_{i+1}(\omega))
    \end{equation*}
    不难验证$T_i$都是停时。

    \begin{theorem}[指数时间]\label{Exponential Times}
        对于固定的$x\in E$,存在常数$q(x)\geqslant 0$,使得$\P_x$测度下,
        $T_1$服从参数为$q(x)$的指数分布\footnote{参数为$0$的指数分布定义为$=+\infty $a.s.}。
        如果$q(x)>0$,则$T_1$与$X_{T_1}$在$\P_x$测度下独立。
    \end{theorem}
    \begin{proof}
        任取$s,t\geqslant 0$,
        \begin{align*}
            \P_x( T_1>s+t )
            &=\E_x[ I_{ \{ T_1>s+t \} } ]\\
            &=\E_x[ I_{ \{ \omega:X_u(\omega)=X_0(\omega),\forall u\in [0,s+t] \} } ]\\
            &=\E_x[ I_{ \{ \omega:X_u(\omega)=X_0(\omega),\forall u\in [0,s] \} }I_{ \{ \omega:X_u(\omega)=X_s(\omega),\forall u\in [s,s+t] \} } ]\\
            &\text{(我们定义:$\Phi(f)=I_{ \{ f(u)=f(0),\forall u\in [0,t] \} }$)}\\
            &=\E_x[ I_{ \{T_1>s\} }\cdot \Phi( X\circ \theta_s ) ]\\
            &=\E_x[ \E[ I_{ \{T_1>s\} }\cdot \Phi( X\circ \theta_s )|\F_s ] ]\\
            &=\E_x[ I_{ \{T_1>s\} }\cdot\E[\Phi( X\circ \theta_s )|\F_s ] ]\\
            &=\E_x[ I_{ \{T_1>s\} }\cdot\E_{X_s}[\Phi(X)] ]\\
            &=\E_x[ I_{ \{T_1>s\} }\cdot\E_{X_s}[I_{ \{T_1>t\} }] ]\\
            &\text{(注意$T_1>s$时$X_s=x$)}\\
            &=\E_x[ I_{ \{T_1>s\} }\cdot\E_{x}[I_{ \{T_1>t\} }] ]\\
            &=\P_x(T_1>s)\P_x(T_1>t)
        \end{align*}
        此即无记忆性,可得$T_1$服从$\P_x$下的指数分布。

        如果$q(x)>0$,即$T_1<+\infty$ $\P_x$-a.s.,于是$\forall t\geqslant 0,y\in E$有
        \begin{align*}
            \P_x(T_1>t,X_{T_1}=y)
            &=\E_x[ I_{ \{T_1>t\} }I_{ \{X_{T_1}=y\} } ]\\
            &\text{(我们定义:$\Psi(f)=I_{ \{ \gamma_1^f<+\infty,f( \gamma_1^f)=y \} }$)}\\
            &=\E_x[ I_{ \{T_1>t\} }\Psi(X\circ \theta_{t}) ]\\
            &=\E_x[ \E[I_{ \{T_1>t\} }\Psi(X\circ \theta_{t})|\F_t] ]\\
            &=\E_x[ I_{ \{T_1>t\} }\cdot \E[\Psi(X\circ \theta_{t})|\F_t] ]\\
            &=\E_x[ I_{ \{T_1>t\} }\cdot \E_{X_t}[\Psi(X)] ]\\
            &=\E_x[ I_{ \{T_1>t\} }\cdot \E_{x}[ I_{ \{ X_{T_1}=y \} } ] ]\\
            &=\P_x( T_1>t )\P_x( X_{T_1}=y )
        \end{align*}
        独立性得证。
    \end{proof}

    接下来,如果$q(x)>0$,记$\pi(x,y)=\P_x(X_{T_1}=y)$,即从$x$出发第一次跳跃到$y$,
    不难验证:
    \begin{equation*}
        \sum_{y\in E}\pi(x,y)=1,\ \pi(x,x)=0
    \end{equation*}
    \begin{theorem}[生成元的表述]\label{Generator of Jumping Process}
        $(Q_t)_{t\geqslant 0}$的生成元为$L$,定义域
        $D(L)=C_0(E)=B(E)$,那么对于$\varphi\in B(E)$,
        \begin{equation*}
            L\varphi(x)=q(x)\sum_{y\neq x}\pi(x,y)( \varphi(y)-\varphi(x) )
        \end{equation*}
    \end{theorem}
    \begin{proof}
        如果$q(x)=0$,则
        \begin{equation*}
            Q_t\varphi(x)=\E_x[ \varphi(X_t) ]
            =\E_x[\varphi(x)]=\varphi(x)
        \end{equation*}
        从而$L\varphi(x)=0$.

        如果$q(x)>0$,
        \begin{lemma}
            $\P_x(T_2\leqslant t)=O(t^2)$,即存在$C>0$使得$\P_x(T_2\leqslant t)\leqslant Ct^2$.
            \begin{proof}
                \begin{align*}
                    \P_{x}(T_{2}\leqslant t)& \leqslant \P_{x}(T_{1}\leqslant t,\:T_{2}\leqslant T_{1}+t)  \\
                    &=\P_{x}(T_{1}\leqslant t,\:T_{1}+T_{1}\circ\theta_{T_{1}}\leqslant T_{1}+t) \\
                    &=\P_{x}(T_{1}\leqslant t,\:T_{1}\circ\theta_{T_{1}}\leqslant t) \\
                    &=\E_{x}[\E_{x}[I_{\{T_{1}\leqslant t\}}I_{\{T_{1}\circ\theta_{T_{1}}\leqslant t\}}|\mathcal{F}_{T_{1}}]] \\
                    &=\E_{x}[I_{\{T_{1}\leqslant t\}}\E_{X_{T_{1}}}[I_{\{T_{1}\leqslant t\}}]]\quad\text{(强马氏性)} \\
                    &\leqslant \E_x[I_{\{T_1\leqslant t\}}\sup_{y\in E}\P_y(T_1\leqslant t)] \\
                    &\leqslant\sup_{y\in E}\P_y(T_1\leqslant t)\P_x(T_1\leqslant t) \\
                    &=\sup_{y\in E}(1-e^{-q(y)t})(1-e^{-q(x)t}) \\
                    &\leqslant Ct^{2},\quad\text{当}t\to0
                \end{align*}
            \end{proof}
        \end{lemma}
        那么
        \begin{align*}
            Q_{t}\varphi(x)
            &=\E_{x}[\varphi(X_{t})]  \\
            &=\E_x[\varphi(X_t);\:t<T_1]+\E_x[\varphi(X_t);\:t\geqslant T_1] \\
            &=\E_x[\varphi(X_t);\:t<T_1]+\E_x[\varphi(X_t);\:T_1\leqslant t<T_2]+\E_x[\varphi(X_t);\:t>T_2] \\
            &=\E_x[\varphi(x);\:t<T_1]+\E_x[\varphi(X_{T_1});\:T_1\leqslant t<T_2]+\E_x[\varphi(X_t);\:t>T_2] \\
            &=\varphi(x)\P_x(t<T_1)+\E_x[\varphi(X_{T_1});\:t\geqslant T_1]+O(t^2) \\
            &=\varphi(x)e^{-q(x)t}+\E_x[\varphi(X_{T_1})]\P_x(t\geqslant T_1)+O(t^2) \\
            &=\varphi(x)e^{-q(x)t}+\sum_{y\neq x}\pi(x,y)\varphi(y)(1-e^{-q(x)t})+O(t^{2}) \\
            &=\varphi(x)e^{-q(x)t}+(1-e^{-q(x)t})\sum_{y\neq x}\pi(x,y)\varphi(y)+O(t^{2})
        \end{align*}
        因此
        \begin{equation*}
            \frac{Q_t\varphi(x)-\varphi(x)}{t}=\frac{\varphi(x)(e^{-q(x)t}-1)+(1-e^{-q(x)t})\sum_{y\ne x}\pi(x,y)\varphi(y)+O(t^2)}{t}
            \rightarrow q(x)\sum_{y\neq x}\pi(x,y)(\varphi(y)-\varphi(x))
        \end{equation*}
    \end{proof}

    \begin{theorem}[跳链]\label{Jumping Chain}
        如果$\forall q(y)>0$,
        那么固定$x\in E$,$T_1(\omega)<T_2(\omega)<\cdots$都是$\P_x$-a.s.有限的,
        此时,$\{ X_{T_n},n\in \N_+ \}$构成了一个(离散时间、离散状态)马氏链,
        其一步转移概率$p(x,y)$就是$\pi(x,y)$.
    \end{theorem}
    \begin{proof}
        \begin{align*}
            \P_{x}(X_{T_{1}}=z_{1},\:X_{T_{2}}=z_{2})
            &=\P_{x}(X_{T_{1}}=z_{1},\:X_{T_{1}+T_{1}\circ\theta_{T_{1}}}=z_{2})  \\
            &=\P_{x}(X_{T_{1}}=z_{1},\:X_{T_{1}}\circ\theta_{T_{1}}=z_{2}) \\
            &=\P_x(X_{T_1}=z_1,E_{X_{T_1}}[X_{T_1}=z_2])& \text{(强马氏性)}  \\
            &=\P_{x}(X_{T_{1}}=z_{1},\:E_{z_{1}}[X_{T_{1}}=z_{2}]) \\
            &=\P_{x}(X_{T_{1}}=z_{1})P_{z_{1}}(X_{T_{1}}=z_{2}) \\
            &=\pi(x,z_1)\pi(z_1,z_2)
        \end{align*}
        归纳可证
        \begin{equation*}
            P_x(X_{T_1}=z_1,\:X_{T_2}=z_2,\cdots,\:X_{T_n}=z_n)=\pi(x,z_1)\pi(z_1,z_2)\cdots\pi(z_{n-1},z_n)
        \end{equation*}
    \end{proof}

\subsection{Levy过程}
    \begin{definition}
        $Y=(Y_t)_{t\geqslant 0}$是一个在$\R$上取值的随机过程,
        如果:
        \begin{enumerate}[(1).]
            \item $Y_0=0$ a.s.
            \item $Y$有独立平稳增量:对于$\forall s\leqslant t$,$Y_t-Y_s$与$(Y_r,r\leqslant s)$独立,并且与$Y_{t-s}$同分布。
            \item 当$t\rightarrow 0$时,$Y_t$依概率收敛到$0$.
        \end{enumerate}
    \end{definition}

    \begin{theorem}
        定义$B(\R)$上的算子如下:固定$t\geqslant 0$,
        \begin{equation*}
            Q_tf(x)=\E[ f(Y_t+x) ]
        \end{equation*}
        那么,$(Q_t)_{t\geqslant 0}$是$C_0(\R)$上的一个Feller半群,
        并且$Y$是关于$(Q_t)_{t\geqslant 0}$的马氏过程。
    \end{theorem}

\section{习题}
    \begin{ex}[le gall(Exercise6.26)][le gall(Exercise6.26)]
        (Feynman-Kac formula)设$v\in C_0(E)$非负,
        对于$\varphi\in B(E)$,我们定义
        \begin{equation*}
            Q_t^* \varphi(x)=\E_x\left[ \varphi(X_t){\rm exp}\left\{ -\int_0^t v(X_s)\d s \right\} \right],\ 
            \forall x\in E,\forall t\geqslant 0
        \end{equation*}
        \begin{enumerate}
            \item 证明:对于$\forall \varphi\in B(E),s,t\geqslant 0$有$Q_{t+s}^* \varphi=Q_t^*(Q_s^*\varphi)$.
            \item 根据下面这个恒等式:
                \begin{equation*}
                    1-{\rm exp}\left\{ -\int_0^t v(X_s)\d s \right\}=\int_0^t v(X_s){\rm exp}\left\{ -\int_s^t v(X_r)\d r \right\}\d s
                \end{equation*}
                证明:$\forall \varphi\in B(E)$,都有
                \begin{equation*}
                    Q_t\varphi-Q_t^*\varphi=\int_0^t Q_s( vQ_{t-s}^* \varphi )\d s
                \end{equation*}
            \item 设$\varphi\in D(L)$,证明:
                \begin{equation*}
                    \frac{\d}{\d t}Q_t^* \varphi|_{t=0}=L\varphi-v\varphi
                \end{equation*}
        \end{enumerate}
    \end{ex}
    \begin{solve}
        本题没什么难度,用的都是很简单的马氏性变换技巧。
        \begin{enumerate}
            \item 
                \begin{align*}
                    Q_{t+s}^* \varphi(x)
                    &=\E_x\left[ \varphi(X_{t+s}){\rm exp}\left\{ -\int_0^{t+s} v(X_u)\d u \right\} \right]\\
                    &=\E_x\left[ \E\left[ \left. \varphi(X_{t+s}){\rm exp}\left\{ -\int_0^{t+s} v(X_u)\d u \right\} \right| \F_t\right] \right]\\
                    &=\E_x\left[ {\rm exp}\left\{ -\int_0^{t} v(X_u)\d u \right\}\cdot \E\left[ \left. \varphi(X_{t+s}){\rm exp}\left\{ -\int_t^{t+s} v(X_u)\d u \right\} \right| \F_t\right] \right]\\
                    &=\E_x\left[ {\rm exp}\left\{ -\int_0^{t} v(X_u)\d u \right\}\cdot \E_{X_t}\left[ \varphi(X_{s}){\rm exp}\left\{ -\int_0^{s} v(X_u)\d u \right\} \right] \right]\tag*{(由马氏性)}\\
                    &=\E_x\left[ {\rm exp}\left\{ -\int_0^{t} v(X_u)\d u \right\}\cdot Q_s^*(X_t) \right]\\
                    &=Q_t^*(Q_s^*)(x)
                \end{align*}
            \item
                \begin{align*}
                    Q_t\varphi(x)-Q_t^*\varphi(x)
                    &=\E_x\left[ \varphi(X_t)\cdot \left(1-{\rm exp}\left\{ -\int_0^t v(X_s)\d s \right\}\right) \right]\\
                    &=\E_x\left[ \varphi(X_t)\cdot \int_0^t v(X_s){\rm exp}\left\{ -\int_s^t v(X_r)\d r \right\}\d s \right]\\
                    &=\int_0^t \E_x\left[ \varphi(X_t)\cdot v(X_s){\rm exp}\left\{ -\int_s^t v(X_r)\d r \right\} \right]\d s\\
                    &=\int_0^t \E_x\left[ \E\left[ \left.  \varphi(X_t)\cdot v(X_s){\rm exp}\left\{ -\int_s^t v(X_r)\d r \right\} \right|\F_s \right] \right]\d s\\
                    &=\int_0^t \E_x\left[ v(X_s)\E\left[ \left.  \varphi(X_t)\cdot {\rm exp}\left\{ -\int_s^t v(X_r)\d r \right\} \right|\F_s \right] \right]\d s\\
                    &=\int_0^t \E_x\left[ v(X_s)\E_{X_s}\left[ \varphi(X_{t-s})\cdot {\rm exp}\left\{ -\int_0^{t-s} v(X_r)\d r \right\} \right] \right]\d s\\
                    &=\int_0^t \E_x\left[ v(X_s) Q_{t-s}^*\varphi(X_s) \right]\d s\\
                    &=Q_s( v(Q_{t-s}^*\varphi) )(x)
                \end{align*}
            \item 易证$Q_0^*\varphi=\varphi$,所以
                \begin{equation*}
                    \left.\frac{\d}{\d t}Q_t^*\varphi\right|_{t=0}
                    =\fun{lim}{t\rightarrow 0^+}\frac{ Q_t^*\varphi-\varphi }{ t }
                    =\fun{lim}{t\rightarrow 0^+}\frac{ Q_t^*\varphi-Q_t\varphi+Q_t\varphi-\varphi }{ t }
                    =L\varphi-\fun{lim}{t\rightarrow 0^+}\frac{ \int_0^t Q_s(vQ_{t-s}^*\varphi)\d s }{t}
                    =L\varphi-v\varphi
                \end{equation*}
        \end{enumerate}
    \end{solve}
    \if{0}{
    \begin{ex}[le gall(Exercise6.28)][le gall(Exercise6.28)]
        (Killing operation)以下假设$X$有连续轨道。设$A\subset E$为紧集,
        \begin{equation*}
            T_A=\fun{inf}{}\{ t\geqslant 0:X_t\in A \}
        \end{equation*}
        \begin{enumerate}
            \item 对于$\forall t\geqslant 0$,任意$E$上的有界可测函数$\varphi$,定义:
                \begin{equation*}
                    Q_t^* \varphi(x)=\E_x[ \varphi(X_t)I_{ \{t<T_A\} } ],\ \forall x\in E
                \end{equation*}
                证明:对于$\forall s,t\geqslant 0$有$Q_{t+s}^* \varphi=Q_t^*(Q_s^*\varphi)$.
            \item 令$\overline{E}=(E\backslash A)\cup\{\Delta\}$,其中$\Delta$是相对于$E\backslash A$的一个孤立点。
            对于$\forall t\geqslant 0$,任意$E$上的有界可测函数$\varphi$,定义:
            \begin{equation*}
                \overline{Q}_t \varphi(x)=\E_x[ \varphi(X_t)I_{ \{ t<T_A \} } ]
                +\P_x[ T_A\leqslant t ]\varphi(\Delta),\ {\rm if\ }x\in E\backslash A
            \end{equation*}
            并且$\overline{Q}_t \varphi(\Delta)=\varphi(\Delta)$. 证明:
            $(\overline{Q}_t)_{t\geqslant 0}$是$\overline{E}$上的一个转移半群(不必证明映射$(t,x)\mapsto \overline{Q}_t\varphi(x)$的可测性。)
            \item 概率测度$\P_x$下,定义过程$\overline{X}=(\overline{X}_t)_{t\geqslant 0}$:
                \begin{equation*}
                    \overline{X}_t=\left\{ \begin{array}{ll}
                        X_t&,t<T_A\\
                        \Delta&,t\geqslant T_A
                    \end{array} \right.
                \end{equation*}
                证明$\overline{X}$是(关于$X$的正规滤流的)马氏过程,且半群为$(\overline{Q}_t)$.
            \item 我们承认$(\overline{Q}_t)$是一个Feller半群,并记其生成元为$\overline{L}$,
                设$f\in D(L)$满足$f$和$Lf$在一个包含$A$的开集上vanish(应该是取值为零的意思),
                记$\overline{f}$是$f$在$E\backslash A$上的限制,并定义$\overline{f}(\Delta)=0$,
                那么$\overline{f}$是$\overline{E}$上的函数,证明:
                $\overline{f}\in D(\overline{L})$,$\overline{L}\overline{f}(x)=Lf(x)$,$\forall x\in E\backslash A$.
        \end{enumerate}
    \end{ex}
    \begin{solve}
        这个题先放着吧。
    \end{solve}
    }\fi
    
    \begin{ex}[le gall(Exercise6.29.1-4)][le gall(Exercise6.29.1-4)]
        \begin{enumerate}
            \item $g\in C_0(E)$,$x\in E$,$T$为停时,证明:
                \begin{equation*}
                    \E_x\left[ I_{ \{ T<+\infty \} }{\rm e}^{-\lambda T}\int_{\R} {\rm e}^{-\lambda t}g(X_{T+t})\d t \right]
                    =\E_x[ I_{ \{ T<+\infty \} }{\rm e}^{-\lambda T}R_\lambda g(X_T) ]
                \end{equation*}
            \item 证明:
                \begin{equation*}
                    R_\lambda g(x)=\E_x\left[ \int_0^T {\rm e}^{-\lambda t}g(X_t)\d t \right]
                    +\E_x[ I_{ \{ T<+\infty \} }{\rm e}^{-\lambda T}R_\lambda g(X_T) ]
                \end{equation*}
            \item 如果$f\in D(L)$,证明:
                \begin{equation*}
                    f(x)=\E_x\left[ \int_0^T {\rm e}^{-\lambda t}(\lambda f-Lf)(X_t)\d t \right]
                    +\E_x[ I_{\{ T<+\infty \}}{\rm e}^{-\lambda T}f(X_T) ]
                \end{equation*}
            \item 设$\E_x[T]<+\infty$,利用3.中结论证明:
                \begin{equation*}
                    \E_x\left[ \int_0^T Lf(X_t)\d t \right]
                    =\E_x[ f(X_T) ]-f(x)
                \end{equation*}
        \end{enumerate}
    \end{ex}
    \begin{solve}
        \begin{enumerate}
            \item \begin{align*}
                {\rm LHS}
                &=\E_x\left[ I_{ \{ T<+\infty \} }{\rm e}^{-\lambda T}\int_{\R} {\rm e}^{-\lambda t}g(X_{T+t})\d t \right]\\
                &=\E_x\left[ \E\left[ \left. I_{ \{ T<+\infty \} }{\rm e}^{-\lambda T}\int_{\R} {\rm e}^{-\lambda t}g(X_{T+t})\d t \right|\F_T \right] \right]\\
                &=\E_x\left[ {\rm e}^{-\lambda T}\E\left[ \left. I_{ \{ T<+\infty \} }\int_{\R} {\rm e}^{-\lambda t}g(X_{T+t})\d t \right|\F_T \right] \right]\\
                &=\E_x\left[ {\rm e}^{-\lambda T}I_{ \{ T<+\infty \} }\E_{X_T}\left[ \int_{\R} {\rm e}^{-\lambda t}g(X_{t})\d t \right] \right]\\
                &=\E_x\left[ {\rm e}^{-\lambda T}I_{ \{ T<+\infty \} }\int_{\R} \E_{X_T}\left[ {\rm e}^{-\lambda t}g(X_{t}) \right]\d t \right]\\
                &=\E_x\left[ {\rm e}^{-\lambda T}I_{ \{ T<+\infty \} }\int_{\R} {\rm e}^{-\lambda t}Q_tg(X_T)\d t \right]\\
                &=\E_x\left[ {\rm e}^{-\lambda T}I_{ \{ T<+\infty \} }R_\lambda g(X_T) \right]\\
                &={\rm RHS}
            \end{align*}
            \item \begin{align*}
                R_\lambda g(x)
                &=\int_0^\infty {\rm e}^{-\lambda t}Q_tg(x)\d t\\
                &=\int_0^\infty {\rm e}^{-\lambda t}\E_x[ g(X_t) ]\d t\\
                &=\E_x\left[ \int_0^\infty {\rm e}^{-\lambda t}g(X_t)\d t \right]\\
                &=\E_x\left[ \int_0^T {\rm e}^{-\lambda t}g(X_t)\d t \right]+\E_x\left[ \int_T^\infty {\rm e}^{-\lambda t}g(X_t)\d t \right]\\
                &=\E_x\left[ \int_0^T {\rm e}^{-\lambda t}g(X_t)\d t \right]+\E_x\left[ I_{ \{ T<+\infty \} }\int_T^\infty {\rm e}^{-\lambda t}g(X_t)\d t \right]\\
                &=\E_x\left[ \int_0^T {\rm e}^{-\lambda t}g(X_t)\d t \right]+\E_x\left[ {\rm e}^{-\lambda T}I_{ \{ T<+\infty \} }\int_0^\infty {\rm e}^{-\lambda t}g(X_{t+T})\d t \right]\\
                &=\E_x\left[ \int_0^T {\rm e}^{-\lambda t}g(X_t)\d t \right]+\E_x[ I_{ \{ T<+\infty \} }{\rm e}^{-\lambda T}R_\lambda g(X_T) ]\tag*{(由1.中结论)}
            \end{align*}
            \item 注意到$R_\lambda=(\lambda -L)^{-1}$,考虑$g=(\lambda-L)f$,代入2.中结论即可。
            \item $\E_x[T]<+\infty$说明$T<+\infty$ $\P_x$-a.s.,以下不再特殊说明。
            由于$f,L(f)$有界,$\E_x[T]<+\infty$,令3.中的$\lambda\rightarrow 0$,根据控制收敛定理,
            \begin{align*}
                f(x)
                &=\fun{lim}{\lambda\rightarrow 0}\E_x\left[ \int_0^T {\rm e}^{-\lambda t}(\lambda f-Lf)(X_t)\d t \right]+\fun{lim}{\lambda\rightarrow 0}\E_x[ I_{\{ T<+\infty \}}{\rm e}^{-\lambda T}f(X_T) ]\\
                &=\E_x\left[ \int_0^T \fun{lim}{\lambda\rightarrow 0}{\rm e}^{-\lambda t}(\lambda f-Lf)(X_t)\d t \right]+\E_x[ \fun{lim}{\lambda\rightarrow 0}{\rm e}^{-\lambda T}f(X_T) ]\\
                &=-\E_x\left[ \int_0^T Lf(X_t)\d t \right]+\E_x[ f(X_T) ]
            \end{align*}
        \end{enumerate}
        关于本题4.的结论,有一种更直接的证法:
        回顾\autoref{Martingale about generator},可知
        \begin{equation*}
            M_t=f(X_t)-\int_0^t Lf(X_s)\d s
        \end{equation*}
        是一个鞅,固定$K>0$,$M_{t\wedge K}$为一致可积鞅,
        由择停定理可知
        \begin{equation*}
            f(x)=\E_x[M_0]=\E_x[ f(X_{T\wedge K})-\int_0^{T\wedge K} Lf(X_s)\d s ]
        \end{equation*}
        由$\E_x[T]<+\infty$可知
        \begin{equation*}
            \fun{lim}{K\rightarrow\infty}f(X_{T\wedge K})=f(X_T){\rm\ }\P_x{\rm -a.s.}
        \end{equation*}
        由控制收敛定理即得
        \begin{equation*}
            f(x)=\E_x[ f(X_T) ]-\E_x\left[ \int_0^T Lf(X_t)\d t \right]
        \end{equation*}
    \end{solve}
    
    最后放一道去年期末题,比较简单,甚至都不需要用到马氏性。
    \begin{ex}[2023SPFinal.6]
        $(X_t^x)$为马氏过程,$(Q_t)_{t\geqslant 0}$为Feller半群,
        生成元为$L$,$X_0^x=x$,$v,h\in C_0(E)$,求证:如果
        \begin{equation*}
            M_t^x={\rm e}^{ -\int_0^t v(X_s^x)\d s }h(X_t^x)
        \end{equation*}
        是鞅,则$h\in D(L)$,并且$Lh=vh$.
    \end{ex}
    \begin{solve}
        任取$t,u\geqslant 0$,
        $\E[ M_{t+u}^x|\F_t ]=M_t^x$,左边等于
        \begin{equation*}
            \E\left[ \left.{\rm e}^{ -\int_0^{t+u} v(X_s^x)\d s }h(X_{t+u}^x)\right|\F_t \right]
            =
            {\rm e}^{ -\int_0^{t} v(X_s^x)\d s }\E\left[ \left.{\rm e}^{ -\int_t^{t+u} v(X_s^x)\d s }h(X_{t+u}^x)\right|\F_t \right]
        \end{equation*}
        因此
        \begin{equation*}
            \E\left[ \left.{\rm e}^{ -\int_t^{t+u} v(X_s^x)\d s }h(X_{t+u}^x)\right|\F_t \right]=h(M_t^x)
        \end{equation*}
        令$t=0$,得到
        \begin{equation*}
            h(x)=\E_x[ {\rm e}^{ -\int_0^{u} v(X_s)\d s }h(X_{u}) ]
        \end{equation*}
        所以
        \begin{align*}
            Lh(x)
            &=\fun{lim}{u\rightarrow 0^+}\frac{Q_uh(x)-h(x)}{u}\\
            &=\fun{lim}{u\rightarrow 0^+}\frac{\E_x[h(X_u)]-h(x)}{u}\\
            &=\fun{lim}{u\rightarrow 0^+}\frac{\E_x\left[  \left( 1-{\rm e}^{-\int_0^u v(X_s)\d s} \right)h(X_u)  \right]}{u}\\
            &=\E_x\left[ \fun{lim}{u\rightarrow 0^+}\frac{1-{\rm e}^{-\int_0^u v(X_s)\d s}}{u}h(X_u) \right]\\
            &=\E_x\left[ h(x)\fun{lim}{u\rightarrow 0^+}\frac{\int_0^u v(X_s){\rm exp}\left\{ -\int_s^u v(X_r)\d r \right\}\d s}{u} \right]\\
            &=\E_x\left[ h(x)\fun{lim}{u\rightarrow 0^+}v(X_0){\rm exp}\left\{ -\int_0^u v(X_r)\d r \right\} \right]\tag*{(洛必达)}\\
            &=\E_x\left[ h(x)v(x) \right]\\
            &=h(x)v(x)
        \end{align*}
        中间用到了\autoref{le gall(Exercise6.26)}.2里面的那个恒等式。
    \end{solve}
    
\clearpage
\section{关于跳跃过程的更多内容*}
    此部分来源于应随课程,
    主要内容为离散状态、连续时间马氏过程(即跳跃过程,也称之为连续时间马氏链)
    的更多性质和应用。

\subsection{基本概念}
    \begin{definition}[连续时间马氏链]
        概率空间$(\Omega,\F,\P)$上的随机过程$X=\{ X(t),t\geqslant 0 \}$的状态空间
        $S$为至多可数集。如果$X$满足马氏性:
        $\forall j,i_{1},\cdots,i_{n-1}\in S,0\leqslant t_1<\cdots<t_n$,都有
        \begin{equation*}
            \P( X(t_n)=j|X(t_{n-1})=i_{n-1},\cdots,X(t_1)=i_1 )=\P( X(t_n)=j|X(t_{n-1})=i_{n-1} )
        \end{equation*}
        并且$X$是右连续的,则称$X$为连续时间马氏链。

        其转移概率定义为
        \begin{equation*}
            p_{ij}(s,t)=\P( X(t)=j|X(s)=i )
        \end{equation*}
        如果$p_{ij}(s,t)=p_{ij}(0,t-s)$,则称$X$为时间齐次的,以下我们总假设时间齐次,并
        把$p_{ij}(0,t)$简记为$p_{ij}(t)$.

        记矩阵$\mathbf{P}_t=( p_{ij}(t) )_{ |S|\times |S| }$,
        $\{ \mathbf{P}_t,t\geqslant 0 \}$称为$X$的转移半群。
    \end{definition}

    \begin{theorem}
        $\{ \mathbf{P}_t,t\geqslant 0 \}$的性质:
        \begin{enumerate}[(1).]
            \item $\mathbf{P}_0=I$.
            \item $\forall p_{ij}(t)\geqslant 0$,$\sum_{j} p_{ij}(t)=1$.
            \item CK方程:$\mathbf{P}_{s+t}=\mathbf{P}_{s}+\mathbf{P}_{t}$.
        \end{enumerate}
    \end{theorem}
    所以,$X$的分布由$X(0)$和$\{ \mathbf{P}_t,t\geqslant 0 \}$决定。

    \begin{definition}[马氏性更一般化的定义]
        对于$A\in \F$,如果$A\in \sigma( \{ X(s),s<t \} )$,
        则称$A$ is $t$-historical;
        如果如果$A\in \sigma( \{ X(s),s>t \} )$,
        则称$A$ is $t$-future.

        马氏性:$\forall t>0$,设$H$ is $t$-historical,$F$ is $t$-future,则
        \begin{equation*}
            \P( F| X(t)=j,H  )=\P(F|X(t)=j),\ \forall j\in S
        \end{equation*}
    \end{definition}

    \begin{definition}[停时]
        称随机变量$T$为关于$X$的停时,如果$\forall t\geqslant 0$,
        $\{ T\leqslant t \}\in \sigma( \{ X(s),s\leqslant t \} )$.
    \end{definition}

    定义$T_0=0$,
    \begin{equation*}
        T_n=\fun{inf}{}\{ t>T_{n-1}:X(t)\neq X(T_{n-1}) \},\ n\geqslant 1
    \end{equation*}
    代表$X$第$n$次跳跃的时刻,
    记$U_m=T_{m+1}-T_m$为$X$在第$m$个状态下的停留时间。这些都是停时。
    令
    \begin{equation*}
        T_\infty\defeq \fun{lim}{n\rightarrow\infty}T_{n}
    \end{equation*}

    \begin{theorem}[强马氏性]
        $T$为关于$X$的停时,给定$I=\{ T<T_\infty \}\cap \{ X(T)=i \}$的条件下,
        $X^*=\{ X^*(u)=X(T+u),u\geqslant 0 \}$与给定$X(0)=i$条件下的$X$同分布,且与
        $\{ X(s),s<T \}$独立。
    \end{theorem}

    \begin{theorem}[指数时间]
        $U_0$服从参数为$g_i$的指数分布,$g_i$只与初始状态$X(0)=i$有关。
        给定$X(0)=i$的条件下,若$g_i>0$,则$X(U_0)$与$U_0$独立。
    \end{theorem}
    这是我们已经证明过的结论,参见\autoref{Exponential Times}.

    \begin{definition}[跳链与生成元]
        对于连续时间马氏链$X$,取离散时间马氏链$Y=\{ Y_n,n\in \N \}$,状态空间为$S$,转移概率为:
        \begin{equation*}
            y_{ij}=\left\{ \begin{array}{ll}
                \delta_{ij}&,g_i=0\\
                \P(X(U_0)=j|X(0)=i)&,g_i>0
            \end{array} \right.
        \end{equation*}
        $Y$称为$X$的跳链。

        令
        \begin{equation*}
            g_{ij}=g_i(y_{ij}-\delta_{ij})=\left\{ \begin{array}{ll}
                g_i y_{ij}&,i\neq j\\
                -g_i&,i=j
            \end{array} \right.
        \end{equation*}
        矩阵$G=(g_{ij})_{|S|\times |S|}$,称为$X$的生成元。
        生成元的行之和为$0$,这点与转移矩阵略有不同。
    \end{definition}

    一个连续时间马氏链$X=\{ X_t(\omega),t\geqslant 0 \}$
    可以由跳链$Y$和生成元$G$来描述:注意到
    \begin{equation*}
        \sum_{j\neq i} g_{ij}=g_i
    \end{equation*}
    我们取与$Y$独立的一系列随机变量$U_0,U_1,\cdots$,其中$U_i$服从参数为$g_i$的指数分布,
    并设$T_n=U_0+\cdots+U_{n-1}$,那么
    \begin{equation*}
        X_t(\omega)\defeq Y_n(\omega),{\rm\ where\ }t\in [ T_n(\omega),T_{n+1}(\omega) )
    \end{equation*}

    有一个问题:如果$\P(T_{\infty}<+\infty)>0$,对于$\omega\in \{T_\infty<+\infty\}$,
    $t>T_\infty$无定义。解决办法是:令$S'=S\cup \{+\infty\}$,且令$X(t)=+\infty,t\geqslant T_\infty$,
    此时我们称$X$是minimal markov process.

    跳链是很重要的工具概念,我们接下来会利用跳链对连续时间马氏链进行详细的分析。
\subsection{状态分类、状态之间的关系}
    我们先考虑$X$的跳链$Y$.
    \begin{theorem}
        $X$是连续时间马氏链,$Y$是$X$的跳链,给定$X(0)=i$条件下,$X$不爆炸的充分条件为以下任意之一:
        \begin{enumerate}[(1).]
            \item $S$有限。
            \item $\fun{sup}{j}g_j<+\infty$.
            \item $i$是$Y$的常返态。
        \end{enumerate}
    \end{theorem}
    \begin{proof}
        $(1)\Rightarrow (2)$.

        $(2)$:设$r=\fun{sup}{j}g_j<+\infty$,$U_n\sim {\rm exp}\{ g_{Y_n} \}$,
        如果存在某个$g_{Y_n}=0$,则$U_n=+\infty\Rightarrow T_\infty=\infty$ a.s.
        如果$g_{Y_n}>0$,则令$V_n=g_{Y_n}\cdot U_n\sim {\rm exp}\{1\}$,
        于是
        \begin{equation*}
            r\cdot T_\infty
            =\sum_{n=1}^\infty r U_n
            \geqslant \sum_{n=1}^\infty V_n=\infty{\rm\ a.s.}
        \end{equation*}

        $(3)$:$g_i=0$则不爆炸;$g_i>0$时,$i$是$Y$的常返态,$Y_0=i$,说明
        存在$0=N_0<N_1<\cdots$使得$Y_{N_j}=i$,从而
        \begin{equation*}
            g_iT_\infty=\sum_{j=0}^\infty g_iU_j
            \geqslant \sum_{j=0}^\infty g_iU_{N_j}=\infty{\rm\ a.s.}
        \end{equation*}
    \end{proof}

    \begin{example}
        设离散时间马氏链$Y=\{  T_n,n\in\N\}$,状态空间为$S$,转移矩阵
        $\mathbf{P}_Y=(y_{ij})_{|S|\times |S|}$,并且$\forall y_{ii}=0$,
        设$N$是参数为$\lambda$的Poisson过程,并设
        \begin{equation*}
            T_0=0,\ T_n=\fun{inf}{}\{ t\geqslant 0:N(t)=n \}
        \end{equation*}

        然后,我们定义$X(t)=Y_n$,其中$T_n\leqslant t<T_{n+1}$,则$X$是连续时间马氏链,转移概率为:
        \begin{align*}
            p_{ij}(t)
            &=\P(X(t)=j|X(0)=i)\\
            &=\sum_{n=0}^\infty \P(X(t)=j,N(t)=n|X(0)=i)\\
            &=\sum_{n=0}^\infty \P(Y_n=j,N(t)=n|Y(0)=i)\\
            &=\sum_{n=0}^\infty \P(N(t)=n)\cdot \P(Y_n=j|Y(0)=i)\\
            &=\sum_{n=0}^\infty {\rm e}^{-\lambda t}\frac{(\lambda t)^n (\mathbf{P}_Y)^n_{ij}}{n!}\\
            \Rightarrow\mathbf{P_X(t)}&=(p_{ij}(t))_{|S|\times |S|}={\rm e}^{\lambda t(\mathbf{P}_Y-I)}
        \end{align*}
        这里矩阵指数是指:
        \begin{equation*}
            {\rm e}^A=\sum_{n=0}^\infty \frac{A^n}{n!}
        \end{equation*}
    \end{example}

    \begin{definition}
        对于连续时间马氏链$X$,如果$\forall i,j\in S$,都存在$t>0$使得$p_{ij}(t)>0$,则称$X$不可约。
    \end{definition}
    \begin{corollary}
        $|S|=1$则显然$X$不可约,$|S|>1$时$X$不可约$\Rightarrow \forall g_i>0$.
    \end{corollary}

    \begin{theorem}
        连续时间马氏链$X$的跳链为$Y$,以下命题等价:
        \begin{enumerate}[(1).]
            \item $X$不可约。
            \item $\forall i,j\in S,\forall t>0,p_{ij}(t)>0$.
            \item $Y$不可约,且$\forall g_i>0$.
        \end{enumerate}
    \end{theorem}

    类似于离散时间马氏链,我们可以定义连续情形下的常返态:
    \begin{definition}[常返与瞬时]
        对于连续时间马氏链$X$,如果状态$i$满足
        \begin{equation*}
            \P( \{t\geqslant 0:X(t)=i\}\text{无界}|X(0)=i )=1
        \end{equation*}
        则称$i$是$X$的常返态;如果状态$i$满足
        \begin{equation*}
            \P( \{t\geqslant 0:X(t)=i\}\text{无界}|X(0)=i )=0
        \end{equation*}
        则称$i$是$X$的瞬时态。
    \end{definition}

    \begin{definition}[正常返与零常返]
        对于连续时间马氏链$X$的常返态$i$,定义最早返回时间:
        \begin{equation*}
            R_i=\fun{inf}{}\{ t>U_0:X(t)=i \}
        \end{equation*}
        如果$m_i=\E[ R_i|X(0)=i ]<+\infty$,则称状态$i$正常返,否则称其零常返。
    \end{definition}

    \begin{theorem}
        连续时间马氏链$X$的跳链为$Y$,
        \begin{enumerate}[(1).]
            \item 若$g_i=0$,则$i$为$X$的常返态。
            \item 若$g_i>0$,则$i$在$X$、$Y$中的常返/瞬时状态一致。
            \item 若$X$不可约,则所有状态要么常返,要么瞬时。
        \end{enumerate}
    \end{theorem}

\subsection{Kolmogorov方程}
    回顾连续时间马氏链$X$和其跳链$Y$的关系,
    得到转移概率的估计:
    \begin{theorem}
        连续时间马氏链$X$的生成元为$G=(g_{ij})$,那么
        \begin{equation*}
            p_{ij}(t+h)\defeq \P( X(t+h)=j|X(t)=i )
            =\delta_{ij}+g_{ij}h+o(h) \tag*{$(\star)$}
        \end{equation*}
    \end{theorem}
    \begin{proof}
        作业。
    \end{proof}

    我们借此尝试推导Kolmogorov方程。
    \begin{theorem}[Kolmogorov]
        连续时间马氏链$X$的状态空间$S$有限,转移半群为$\{\mathbf{P}_t,t\geqslant 0\}$,
        生成元为$G=(g_{ij})$,则:
        \begin{equation*}
            \frac{\d}{\d t}\mathbf{P}_t=\mathbf{P}_t\cdot G\tag*{向前方程}
        \end{equation*}
        \begin{equation*}
            \frac{\d}{\d t}\mathbf{P}_t=G\cdot \mathbf{P}_t\tag*{向后方程}
        \end{equation*}
        边界条件:$\mathbf{P}_0=I$.
    \end{theorem}
    \begin{proof}
        对于$p_{ij}(t+h)$,考虑用$i\ra{t}k\ra{h}j$拆分:
        \begin{align*}
            p_{ij}(t+h)
            &=\P( X(t+h)=j|X(0)=i )\\
            &=\sum_{k\in S}\P( X(t+h)=j|X(t)=k )\cdot \P( X(t)=k|X(0)=i )\\
            &=\sum_{k\in S}p_{kj}(h)p_{ik}(t)\\
            &=\sum_{k\in S}( \delta_{kj}+g_{kj}h+o(h) )p_{ik}(t)\\
            &=p_{ij}(t)+\sum_{k\in S}p_{ik}(t)g_{kj}h+o(h)\\ 
        \end{align*}
        整理一下得到:
        \begin{equation*}
            \frac{p_{ij}(t+h)-p_{ij}(t)}{h}=\sum_{k\in S}p_{ik}(t)g_{kj} +\frac{o(h)}{h}
        \end{equation*}
        令$h\rightarrow 0$则得到向前方程。类似的,
        考虑用$i\ra{h}k\ra{t}j$,就得到向后方程。
    \end{proof}

    \begin{corollary}
        连续时间马氏链$X$的状态空间$S$有限,转移半群为$\{\mathbf{P}_t,t\geqslant 0\}$,
        生成元为$G=(g_{ij})$,则求解Kolmogorov方程,得到唯一解:
        \begin{equation*}
            \mathbf{P}_t={\rm e}^{ t\cdot G }
            =\sum_{n=0}^\infty \frac{G^n}{n!}t^n
        \end{equation*}
    \end{corollary}

    现在的问题是:如果$|S|=\infty$,$\sum_{k\in S}$变为无穷求和,
    能否直接与$\fun{lim}{h\rightarrow 0}$交换?这会导致向后方程
    \begin{theorem}
        $|S|=\infty$,$G$为$S$上的生成元(即满足非负、行和为$0$的矩阵),
        令$X$为以$G$为生成元的minimal markov process,那么:
        \begin{enumerate}[(1).]
            \item $X$的转移半群$\{\mathbf{P}_t\}$为向后方程的最小非负解。即任何一个满足向后方程的其他解$\pi_{ij}(t)$,
                都有$p_{ij}(t)\leqslant \pi_{ij}(t)$.
            \item $X$的转移半群$\{\mathbf{P}_t\}$也是向前方程的最小非负解。
        \end{enumerate}
    \end{theorem}
    \begin{proof}
        我们只说明(1),(2)见(Norris 1997.p.100).

        设$T_1$是$X$第一次改变初始状态的时刻,即
        \begin{equation*}
            T_1=\fun{inf}{}\{ t>0:X(t)\neq X(0) \}
        \end{equation*}
        那么,
        \begin{align*}
            p_{ij}(t)
            &=\P(T_1>t,X(t)=j|X(0)=i)+\P(T_1\leqslant t,X(t)=j|X(0)=i)\\
            &=\P(T_1>t,X(t)=j|X(0)=i)+\sum_{k\neq i}\P(T_1\leqslant t,X(t)=j,X(T_1)=k|X(0)=i)\\
            &=\delta_{ij}{\rm e}^{-g_it}+\sum_{k\neq i}\int_0^t g_i{\rm e}^{-g_i s}y_{ik}p_{kj}(t-s)\d s \tag*{$(\star)$}
        \end{align*}
        这里利用Fubini定理交换求和与积分顺序,并换元$u=t-s$,得到:
        \begin{equation*}
            {\rm e}^{g_i t}p_{ij}(t)=\delta_{ij}+\int_0^t \sum_{k\neq i}
            g_i{\rm e}^{g_i u}y_{ik}p_{kj}(u)\d u
        \end{equation*}
        两边对$t$求导,就得到
        \begin{equation*}
            \frac{\d}{\d t}p_{ij}(t)=\sum_{k\in S}g_{ik}p_{kj}(t),\ i\in S
        \end{equation*}
        这就验证了$\mathbf{P}_t$符合向后方程。下面证明它是最小非负解,假设另一个非负解$\pi_{ij}(t)$满足向后方程,
        它同样满足$(\star)$式。如果设$T_n$为第$n$次改变初始状态的时刻,接下来我们利用归纳法证明:
        \begin{equation*}
            \pi_{ij}(t)\geqslant \P( X(t)=j,T_n>t|X(0)=i ),\ \forall i,j\in S,t>0,n\geqslant 1
        \end{equation*}
        当$n=1$时,
        \begin{equation*}
            \pi_{ij}(t)\geqslant \delta_{ij}{\rm e}^{-g_i t}=\P( X(t)=j,T_1>t|X(0)=i )
        \end{equation*}
        假设$1\leqslant n\leqslant N$都成立,考虑$n=N+1$的情形,
        \begin{align*}
            &\P( X(t)=j,T_{N+1}>t|X(0)=i )\\
            &=\P( X(t)=j,T_{N+1}>t,T_1>t|X(0)=i )+\P( X(t)=j,T_{N+1}>t,T_1\leqslant t|X(0)=i )\\
            &=\P( X(t)=j,T_1>t|X(0)=i )+\sum_{k\neq i}\P( X(t)=j,T_{N+1}>t,T_1\leqslant t,X(T_1)=k|X(0)=i )\\
            &=\delta_{ij}{\rm e}^{-g_i t}
            +\sum_{k\neq i}\P( X(t)=j,T_{N+1}>t,T_1\leqslant t|X(T_1)=k )\P(X(T_1)=k|X(0)=i)\\
            &=\delta_{ij}{\rm e}^{-g_i t}
            +\sum_{k\neq i}\int_0^t g_i{\rm e}^{-g_i s}\P( X(t-s)=j,T_{N}>t|X(0)=k )y_{ik}\d s
        \end{align*}
        从而由递推可得$n=N+1$的情形成立。

        那么,令$n\rightarrow +\infty$,$T_n\rightarrow T_\infty$,
        \begin{equation*}
            \pi_{ij}(t)\geqslant \fun{lim}{n\rightarrow \infty}
            \P( X(t)=j,T_n>t|X(0)=i )
            =\P( X(t)=j,T_\infty>t|X(0)=i )=p_{ij}(t)
        \end{equation*}
    \end{proof}
\subsection{平稳分布}
    \begin{definition}
        类似于离散情形,对于不可约、不爆炸的连续时间马氏链$X$,
        其转移半群为$\{ \mathbf{P}_t \}$,若$S$上的测度$\pi$满足
        \begin{equation*}
            \pi \mathbf{P}_t=\pi ,\ \forall t\geqslant 0
        \end{equation*}
        则称$\pi$为$X$的平稳测度。
        上述$\pi$如果是概率测度,则称其为平稳分布。
    \end{definition}

    回忆:关于常返态$i$,记
    \begin{equation*}
        R_i=\fun{inf}{}\{ t>T_1:X(t)=i \}
    \end{equation*}
    那么$R_i|X(0)=i$就是首次回到$i$的时刻,其期望记作
    \begin{equation*}
        m_i=\E[R_i|X(0)=i]
    \end{equation*}
    如果有限,则称正常返。

    \begin{theorem}
        不可约的连续时间马氏链$X$的状态空间$S$满足$|S|\geqslant 2$,若存在$k\in S$正常返,
        则存在唯一平稳分布$\pi$,并且$\pi$还是唯一满足$\pi G=0$的概率测度。

        若$X$不爆炸,且存在概率测度$\pi$满足$\pi G=0$,则:
        \begin{enumerate}[(1).]
            \item 所有状态都正常返;
            \item $\pi$为平稳分布;
            \item $\pi_k^{-1}=m_k g_k$.
        \end{enumerate}
    \end{theorem}

    \begin{lemma}
        不可约的连续时间马氏链$X$的跳链为$Y$,
        \begin{enumerate}[(1).]
            \item 测度$a$满足$aG=0$当且仅当测度$v=(v_i=a_ig_i,i\in S)$满足$v\mathbf{P}^Y=v$.进一步地若$X$常返,则$a$唯一。
            \item 若测度$a$满足$aG=0$,则$\forall a_j>0$.
            \item 假设$X$常返,任取$k\in S$,则令
                \begin{equation*}
                    \mu_j(k)=\E\left[ \left.\int_0^{R_k} I_{ \{ X(s)=j \} }\d s \right|X(0)=k  \right]
                \end{equation*}
                则$\mu(k)=(\mu_j(k),j\in S)$是$X$的平稳测度并且$\mu(k)G=0$.
        \end{enumerate}
    \end{lemma}

    \begin{example}
        连续时间马氏链$X$的状态空间为$\Z$,生成元$G=(g_{ij})$为:
        \begin{equation*}
            g_{m,m+1}=\left\{ \begin{array}{ll}
                4^m&,m\geqslant 0\\
                \frac{1}{2}4^{-m}&,m\leqslant -1
            \end{array} \right.
            ,\ 
            g_{m,m-1}=\left\{ \begin{array}{ll}
                \frac{1}{2}4^m&,m\geqslant 1\\
                4^{-m}&,m\leqslant 0
            \end{array} \right.
        \end{equation*}
        \begin{equation*}
            g_{m,m}=-g_{m,m+1}-g_{m,m-1}=
            \left\{ \begin{array}{ll}
                -\frac{3}{2}4^{|m|}&,m\neq 0\\
                -1&,m=0
            \end{array} \right.
        \end{equation*}
        其余为$0$.

        注意到$\forall g_{m,m}<0\Rightarrow \forall g_m>0$,$X$不可约。
        方程$\pi G=0$有两组解:
        \begin{equation*}
            \pi^{(1)}(m)=\frac{1}{3}2^{-|m|}
        \end{equation*}
        \begin{equation*}
            \pi^{(2)}(m)=\left\{ \begin{array}{ll}
                \frac{1}{3}4^{-m}&,m\geqslant 0\\
                \frac{1}{3}\frac{2^{-m+1}-1}{4^m}&,m\leqslant -1
            \end{array} \right.
        \end{equation*}
        这说明$X$爆炸。(如果$X$不爆炸,$\pi G=0$的解应当存在唯一。)
    \end{example}

    \begin{theorem}
        $X$是连续时间马氏链,不可约、不爆炸,
        \begin{enumerate}[(1).]
            \item 若存在平稳分布$\pi$,则唯一。且$\forall i,j\in S$,有$\fun{lim}{t\rightarrow\infty}p_{ij}(t)=\pi_j$.
            \item 若不存在平稳分布,则$\forall i,j\in S$,有$\fun{lim}{t\rightarrow\infty}p_{ij}(t)=0$.
        \end{enumerate}
    \end{theorem}

    \begin{example}
        生灭过程:状态空间为$\N$,生成元为:
        \begin{equation*}
            G=\begin{pmatrix}
                -\lambda_0&\lambda_0& & & \\
                \mu_1&-(\lambda_1+\mu_1)&\lambda_1& & \\
                 &\mu_2&-(\lambda_2+\mu_2)&\lambda_2& \\
                 & &\ddots&\ddots&\ddots
            \end{pmatrix}
        \end{equation*}
        其中$\lambda_{\cdot},\mu_{\cdot}>0$.
    \end{example}

    \begin{theorem}[Detailed Balance Condition]
        分布$\pi$满足$\forall j\neq k$有$\pi_j g_{jk}=\pi_k g_{kj}$,
        则$\pi$是平稳分布。
    \end{theorem}
\subsection{终止分布问题(Exit Distributions)}
\begin{theorem}[Exit Distributions, ED]
    $X$是连续时间马氏链,$Z$是其跳链,状态空间为$S$,
    $A,B\subset S$,$C=S-(A\cup B)$,
    令:
    \begin{equation*}
        V_A=\fun{inf}{}\{ t\geqslant 0:X_t\in A \}
    \end{equation*}
    \begin{equation*}
        V_A^Z=\fun{min}{}\{n\geqslant 0:Z_n\in A\}
    \end{equation*}
    $V_B,V_B^Z$类似。令$T=V_A\wedge V_B$,且$\forall i\in C$,$\P(T<+\infty|X_0=i)>0$.

    如果函数$h:S\rightarrow [0,1]$满足:$h(A)=1$,$h(B)=0$,
    \begin{equation*}
        \sum_{j\in S}g_{ij}h(j)=0,\ \forall i\in C
    \end{equation*}
    则$h(i)=\P(V_A<V_B|X_0=i)$.
\end{theorem}

\begin{theorem}[Exit Times, ET]
    $X$是连续时间马氏链,$Z$是其跳链,状态空间为$S$,$A\subset S$,
    $C=S-A$是有限集,并且$\forall i\in C$,$\P(V_A<+\infty|X_0=i)>0$,
    如果函数$h:S\rightarrow [0,1]$满足:$h(A)=0$,
    \begin{equation*}
        \sum_{j\in S}g_{ij}g(j)=-1,\ \forall i\in C
    \end{equation*}
    则$h(i)=\E[ V_A|X_0=i ]$.
\end{theorem}















