\chapter{作业汇总}
%\begin{center}
%    “泛函分析心泛寒”,说明泛函是一门很难的课,但是泛函其实并没有比数分、线代、实分析更难。
%    这是一门很抽象的课,由于前面的课没学好,很多同学学到这里学崩了。泛函是压死骆驼的最后一根稻草。
%\end{center}
%\rightline{——刘聪文老师,2023.09.04}
%\vspace{-5pt}
%\begin{center}
%    \pgfornament[width=0.36\linewidth,color=lsp]{88}
%\end{center}

    部分作业是课上某个定理、推论、命题或其中一个步骤的证明,
    就直接抄录在笔记里了。本章内容$=$课后作业$\cap $课本习题。
\section{第一章}
    \begin{exer}{\rm (1.1.1)}
        完备度量空间的闭子空间也是完备的;任一度量空间的完备子空间一定是闭子空间。
    \end{exer}
\begin{solve}
        \begin{enumerate}[$(1)$]
            \item 设$E$是完备度量空间$X$的闭子集,由于$X$是完备的,所以$E$上任意柯西列都在$X$上收敛,且由$E$是闭集
            $\Rightarrow \forall \{x_n\}_{n=1}^\infty \subset E$,如果
            $x_n\rightarrow x_0$,则$x_0\in E$,所以$E$上任意柯西列都在$E$上收敛,故$E$是完备子空间。
            \item 设$E$是度量空间$X$的完备子空间,设$\{x_n\}_{n=1}^\infty$是$X$上的收敛列,且$\{x_n\}_{n=1}^\infty\subset E$,
            那也一定是柯西列,故在$E$上收敛,即$x_n\rightarrow x_0\in E$,故$E$是闭集。
        \end{enumerate}
\end{solve}

    \begin{exer}{\rm (1.2.1)}
        $S$为复数列全体构成的集合,定义距离为
        \begin{equation*}
            \rho(x,y)=\sum_{k=1}^n \frac{1}{2^k}\cdot \frac{|\xi_k-\eta_k|}{1+|\xi_k-\eta_k|}
        \end{equation*}
        其中$x=(\xi_1,\cdots,\xi_k,\cdots)$,$y=(\eta_1,\cdots,\eta_k,\cdots)$.
        求证:$(S,\rho)$为完备度量空间。
    \end{exer}
\begin{solve}
    先验证$\rho$是距离函数:
    \begin{enumerate}[$1^\circ$]
        \item 非负性:$\forall x,y,\rho(x,y)\geqslant 0$成立,因为每一项都非负。
        \item 唯一性:$\rho(x,y)=0\Rightarrow \forall k,|\xi_k-\eta_k|=0\Rightarrow x=y$.
        \item 对称性:由$|\xi_k-\eta_k|=|\eta_k-\xi_k|$可得。
        \item 三角不等式:设$z=(\zeta_1,\zeta_2,\cdots)$,
            \begin{align*}
                \frac{t}{1+t}\nearrow {\rm\ on\ }(0,\infty)\Rightarrow\frac{|a+b|}{1+|a+b|}&\leqslant \frac{ |a|+|b| }{1+|a|+|b|}\\
                &=\frac{ |a| }{1+|a|+|b|}+\frac{ |b| }{1+|a|+|b|}\leqslant \frac{ |a| }{1+|a|}+\frac{ |b| }{1+|b|}
            \end{align*}
            所以
            \begin{align*}
                \rho(x,y)&=\sum_{k=1}^\infty \frac{1}{2^k}\frac{|\xi_k-\eta_k|}{1+|\xi_k-\eta_k|}
                =\sum_{k=1}^\infty \frac{1}{2^k}\frac{|(\xi_k-\zeta_k)+(\zeta_k-\eta_k)|}{1+|(\xi_k-\zeta_k)+(\zeta_k-\eta_k)|}\\
                &\leqslant
                \sum_{k=1}^\infty \frac{1}{2^k}\frac{|\xi_k-\zeta_k|}{1+|\xi_k-\zeta_k|}+\sum_{k=1}^\infty \frac{1}{2^k}\frac{|\zeta_k-\eta_k|}{1+|\zeta_k-\eta_k|}\\
                &=\rho(x,z)+\rho(y,z)
            \end{align*}
    \end{enumerate}

    下面证明完备性:设$\{x^{(n)}\}$是$S$中的柯西列,其中$x^{(n)}=(x_1^{(n)},x_2^{(n)},\cdots)$,则
    \begin{equation*}
        \rho(x^{(n+p)}-x^{(n)})=\sum_{k=1}^\infty \frac{1}{2^k}\frac{ |x^{(n+p)}_k-x^{(n)}_k| }{1+|x^{(n+p)}_k-x^{(n)}_k|}\rightarrow 0{\rm\ as\ }n\rightarrow\infty,\forall p\in\mathbb{N}
    \end{equation*}
    对于每个$k$,取$N_k$使得$n>N_k$时有
    \begin{equation*}
        \sum_{i=1}^\infty \frac{1}{2^i}\frac{|x^{(n+p)}_i-x^{(n)}_i|}{1+|x^{(n+p)}_i-x^{(n)}_i|}<\frac{1}{2^{k+1}}\varepsilon
    \end{equation*}
    左侧取级数的第$k$项,
    \begin{equation*}
        \frac{1}{2^k}\frac{|x^{(n+p)}_k-x^{(n)}_k|}{1+|x^{(n+p)}_k-x^{(n)}_k|}
        <\frac{1}{2^{k+1}}\varepsilon
    \end{equation*}
    $\Rightarrow |x^{(n+p)}_k-x^{(n)}_k|<\frac{\varepsilon}{2-\varepsilon}\rightarrow 0{\rm\ as\ }\varepsilon\rightarrow0$,可知
    \begin{equation*}
        |x^{(n+p)}_k-x^{(n)}_k|\rightarrow 0{\rm\ as\ }n\rightarrow 0,\forall p,k\in\mathbb{N}
    \end{equation*}
    这意味着每个$x^{(n)}$的第$k$个坐标组成$\mathbb{C}$上的柯西列,由$\mathbb{C}$的完备性可知其收敛,并设$x_k^{(n)}\rightarrow x_k^*$,令
    \begin{equation*}
        x^*=(x_1^*,x_2^*,\cdots)\in S
    \end{equation*}
    下面证明$x^{(n)}\rightarrow x^*$,
    \begin{align*}
        \rho(x^{(n)},x^*)&=\sum_{k=1}^\infty \frac{1}{2^k}\frac{ |x_k^{(n)}-x_k^*| }{1+|x_k^{(n)}-x_k^*|}\\
        &=\sum_{k=1}^{k_0} \frac{1}{2^k}\frac{ |x_k^{(n)}-x_k^*| }{1+|x_k^{(n)}-x_k^*|}+\sum_{k=n_0+1}^\infty \frac{1}{2^k}\frac{ |x_k^{(n)}-x_k^*| }{1+|x_k^{(n)}-x_k^*|}\\
        &<\sum_{k=1}^{k_0} \frac{1}{2^k}|x_k^{(n)}-x_k^*|+\sum_{k=n_0+1}^\infty \frac{1}{2^k}\\
        &=\sum_{k=1}^{k_0} \frac{1}{2^k}|x_k^{(n)}-x_k^*|+2^{-n_0}
    \end{align*}
    对每个$k$,存在$N_k$使得$n>N_k$时$|x_k^{(n)}-x_k^*|< \frac{\varepsilon}{2}$,
    于是取$N={\rm max}\{N_1,\cdots,N_{k_0}\}$,$n>N$时
    \begin{equation*}
        \sum_{k=1}^{k_0} \frac{1}{2^k}|x_k^{(n)}-x_k^*|<\sum_{k=1}^{k_0} \frac{1}{2^k}\frac{\varepsilon}{2}<\frac{\varepsilon}{2}
    \end{equation*}
    再取充分大的$n_0$使得$2^{-n_0}<\frac{\varepsilon}{2}$,得到$\rho(x^{(n)},x^*)<\varepsilon$,由$\varepsilon$的任意性可知$\rho(x^{(n)},x^*)\rightarrow 0$,完备性得证。
\end{solve}

    \begin{exer}{\rm (1.2.2)}
        度量空间上的基本列是收敛列当且仅当它存在一列收敛子列。
    \end{exer}
    \begin{solve}
        必要性显然,只说明充分性:设$\{x_n\}$是基本列,且存在收敛子列$\{x_{n_k}\}$,并设$x_{n_k}\rightarrow x$.任意$\varepsilon>0$,存在
        $N$使得$n,m>N$时
        \begin{equation*}
            \rho(x_n,x_m)<\varepsilon
        \end{equation*}
        存在$K$使得$k>K$时$n_k>N$,故
        \begin{equation*}
            \rho(x_n,x_{n_k})<\varepsilon,\forall n>N,k>K
        \end{equation*}
        上式中$k\rightarrow \infty$可得$\rho(x_n,x)\leqslant \varepsilon$,由$\varepsilon$任意性可知$x_n\rightarrow x$,故
        $\{x_n\}$为收敛列。
    \end{solve}

    \begin{exer}{\rm (1.2.3)}
        $F$是只有有限项不为0的实数列全体构成的集合,定义距离为
        \begin{equation*}
            \rho(x,y)=\fun{sup}{k\geqslant 1}|\xi_k-\eta_k|
        \end{equation*}
        其中$x=(\xi_1,\cdots,\xi_k,\cdots)$,$y=(\eta_1,\cdots,\eta_k,\cdots)$.
        求证:$(F,\rho)$不是完备度量空间,并指出其完备化。
    \end{exer}
    \begin{solve}
        令$x^{(n)}=(1,\frac{1}{2},\cdots,\frac{1}{n},0,0,\cdots)\in F$,
    $m>n$时,
    \begin{equation*}
        x^{(n)}-x^{(m)}=(0,0,\cdots,0,\frac{1}{n+1},\frac{1}{n+2},\cdots,\frac{1}{m},0,0,\cdots)
    \end{equation*}
    所以
    \begin{equation*}
        \rho(x^{(n)},x^{(m)})=\frac{1}{n+1}\rightarrow 0{\rm\ as\ }n\rightarrow\infty
    \end{equation*}
    于是$\{x^{(n)}\}$是柯西列,假设它收敛到$x\in F$,$x$只有有限项不为零,假设其从第$N$项开始全部是$0$,
    则$\rho(x^{(n)},x)\geqslant \frac{1}{N},\forall n>N$,矛盾。所以$F$不完备。

    收敛到$0$的实数列全体为$F$的完备化,记作$l^\infty_0$,证明如下:
    
    先证明完备性,$l^\infty_0$上的距离也定义为
    \begin{equation*}
        \rho(x,y)=\mathop{\rm sup}\limits_{k\geqslant 1}|\xi_k-\eta_k|
    \end{equation*}
    设$\{x^{(n)}\}$是柯西列,其中$x^{(n)}=(\xi_1^{(n)},\xi_2^{(n)},\cdots)$,$\xi_k^{(n)}\rightarrow 0{\rm\ as\ }k\rightarrow 0$.
    注意到
    \begin{equation*}
        \rho(x^{(n)},x^{(m)})=\mathop{\rm sup}\limits_{k\geqslant 1}|\xi_k^{(n)}-\xi_k^{(m)}|<\varepsilon
        \Rightarrow\forall k\geqslant 1,|\xi_k^{(n)}-\xi_k^{(m)}|<\varepsilon
    \end{equation*}
    因此对于每个$k$,$\{\xi_k^{(n)}\}_{n=1}^\infty$是$\mathbb{R}$上的柯西列,由$\mathbb{R}$的完备性可知其为收敛列,设
    $\xi_k^{(n)}\rightarrow \xi_k$,并令$x=(\xi_1,\xi_2,\cdots,\xi_k,\cdots)$.
    \begin{equation*}
        |\xi_k|\leqslant |\xi_k-\xi_k^{(n)}|+|\xi_k^{(n)}-\xi_j^{(n)}|+|\xi_j^{(N)}|
    \end{equation*}
    由于$\xi_k^{(n)}\rightarrow \xi_k$,
    可以取充分大的$n$使得
    右边第一项$<\varepsilon/3$;
    $\{\xi_{k}^{(n)}\}$是柯西列,可以取充分大的$k,j$使得
    右边第二项$<\varepsilon/3$;
    有定义$\xi_j^{(N)}\rightarrow 0{\rm\ as\ }j\rightarrow 0$,可以取充分大的$j$
    使得右边第三项$<\varepsilon/3$.因此得到$|\xi_k|<\varepsilon$,由$\varepsilon$的任意性可得
    $\xi_k\rightarrow0\Rightarrow x\in F$,现只需说明$x_n\rightarrow x$.
    \begin{equation*}
        \rho(x_n,x)=\mathop{\rm sup}\limits_{k\geqslant 1}|\xi_k^{(n)}-\xi_k|
    \end{equation*}
    由于$\xi_k^{(n)}\rightarrow 0$,$\xi_k\rightarrow 0$,任意$\varepsilon>0$,可以取充分大的$K$,使得$k>K$时
    $|\xi_k^{(n)}-\xi_k|\leqslant |\xi_k^{(n)}|+|\xi_k|\leqslant 2\varepsilon$,不妨设$|\xi_1^{(n)}-\xi_1|=\varepsilon>0$,
    则$\rho(x_n,x)=\mathop{\rm max}\limits_{1\leqslant k\leqslant K}|\xi_k^{(n)}-\xi_k|$,而这有限的$K$项每一项都趋于$0$,$\forall \varepsilon_0>0$,
    可以取充分大的$N$使得$n>N$时
    \begin{equation*}
        \rho(x_n,x)=\mathop{\rm max}\limits_{1\leqslant k\leqslant K}|\xi_k^{(n)}-\xi_k|<\varepsilon_0
    \end{equation*}
    因此$x_n\rightarrow x$.

    最后说明$F$是稠密子空间,实际上$\forall x=(\xi_1,\cdots,\xi_k,\cdots)\in l_\infty^0$,令
    $x_n\in F$的前$n$项与$x$相同,其余为$0$,则
    \begin{equation*}
        \mathop{\rm lim}\limits_{n\rightarrow\infty}\rho(x_n,x)=
        \mathop{\rm lim}\limits_{n\rightarrow\infty}\mathop{\rm sup}\limits_{k\geqslant n+1}|\xi_k|=0
    \end{equation*}
    \end{solve}

    \begin{exer}{\rm (1.2.4)}
        $[0,1]$上的多项式全体记作$P[0,1]$,定义距离
        \begin{equation*}
            \rho(p,q)=\int_0^1 p(x)q(x)\d x
        \end{equation*}
        求证:$(P[0,1],\rho)$不是完备度量空间,并指出其完备化。
    \end{exer}
    \begin{solve}
        设
    \begin{equation*}
        p_n(x)=1+x+\frac{x^2}{2}+\cdots+\frac{x^{n}}{n!},n\in\mathbb{N}_+
    \end{equation*}
    则
    \begin{equation*}
        \rho(p_n(x),p_{n+k}(x))
        =\int_0^1 \sum_{i=n+1}^{n+k}\frac{x^i}{i!}{\rm d}x
        =\sum_{i=n+1}^{n+k}\frac{1}{(i+1)!}\leqslant \sum_{i=n+1}^\infty \frac{1}{i(i+1)}=\frac{1}{n+1}\rightarrow 0
    \end{equation*}
    所以$p_n(x)$是基本列,而
    \begin{equation*}
        \rho(p_n(x),{\rm e^x})
        =\int_0^1 \sum_{i=n+1}^{\infty}\frac{x^i}{i!}{\rm d}x
        =\sum_{i=n+1}^{\infty}\frac{1}{(i+1)!}\leqslant \sum_{i=n+1}^\infty \frac{1}{i(i+1)}=\frac{1}{n+1}\rightarrow 0
    \end{equation*}
    说明$p_n(x)\mathop{\rightarrow}\limits^{\rho}{\rm e}^x$,这意味着如果$P[0,1]$完备,则${\rm e}^x$是有限次多项式,矛盾。

    $L^1[0,1]$是$P[0,1]$的完备化。
    \end{solve}

    \begin{exer}{\rm (1.3.1)}
        在完备度量空间中,子集$A$列紧的充要条件是:$\forall \varepsilon>0$,存在$A$的列紧的$\varepsilon$网。
    \end{exer}
    \begin{solve}
        必要性:列紧$\Rightarrow $完全有界,所以$\forall \varepsilon>0$,存在$A$的有限$\varepsilon$网,有限集是列紧的(因为有限集的无穷点列中一定能取出全由某个特定元素组成的点列,进而收敛),所以也是列紧$\varepsilon$网。

    充分性:设$N$为$A$的列紧的$\frac{\varepsilon}{2}$网,
    \begin{equation*}
        \forall x\in A,\exists \xi\in N{\rm\ s.t.\ }\rho(x,\xi)<\frac{\varepsilon}{2}
    \end{equation*}
    $N$列紧$\Rightarrow$完全有界,设$N_0$是$N$的有限$\frac{\varepsilon}{2}$网,则对于$\xi$,
    $\exists x_0\in N_0{\rm\ s.t.\ }\rho(\xi,x_0)<\frac{\varepsilon}{2}$,于是
    \begin{equation*}
        \rho(x,x_0)\leqslant \rho(x,\xi)+\rho(\xi,x_0)<\varepsilon
    \end{equation*}
    所以$N_0$是$A$的有穷$\varepsilon$网,所以$A$完全有界,完备度量空间所以$A$列紧。
    \end{solve}

    \begin{exer}{\rm (1.3.2)}
        度量空间中,紧集上的连续函数一定有界,且能够达到上、下确界。
    \end{exer}
    \begin{solve}
        假设$f(x)$无上界,则存在$\{x_n\}\subset M$使得$f(x_n)>n$,由于$M$是紧集,所以自列紧,
        所以存在子列$x_{n_k}\rightarrow x_0\in M$,$f$连续所以
        $f(x_{n_k})\rightarrow f(x_0)$,这与$f(x_n)\rightarrow\infty$矛盾。
        所以$f(x)$有上界,
        同理可证$f(x)$有下界,故$f$有界。
    
        设$\beta=\mathop{\rm sup}\limits_{x\in M}f(x)$,$\forall \varepsilon>0$,存在$x_\varepsilon\in M$,$f(x_\varepsilon)>\beta-\varepsilon$,
        取$\varepsilon_n=\frac{1}{n}$,得到点列$\{x_n\}$,
        有收敛子列$\{x_{n_k}\}$,设其收敛到$x_0\in M$,于是
        \begin{equation*}
            \beta-\frac{1}{n_k}<f(x_{n_k})\leqslant \beta
        \end{equation*}
        上式$k\rightarrow\infty$可得
        \begin{equation*}
            \mathop{\rm lim}\limits_{k\rightarrow\infty}f(x_{n_k})=\beta
        \end{equation*}
        同时,因为$f$连续,
        \begin{equation*}
            \mathop{\rm lim}\limits_{k\rightarrow\infty}f(x_{n_k})=f(x_0)
        \end{equation*}
        因此$f(x_0)=\beta$,能达到上确界。下确界同理。
    \end{solve}

    \begin{exer}{\rm (1.3.4)}
        $(X,\rho)$是度量空间,$F_1,F_2$是$X$的两个紧子集,求证:
        $\exists x_i\in F_i$,使得$\rho(F_1,F_2)=\rho(x_1,x_2)$,其中
        \begin{equation*}
            \rho(F_1,F_2):={\rm inf}\{ \rho(x,y)|x\in F_1,y\in F_2 \}
        \end{equation*}
    \end{exer}
    \begin{solve}
        设$\rho(F_1,F_2)=d$,由inf的定义,
        \begin{equation*}
            \forall n\in\mathbb{N},\exists x_n\in F_1,y_n\in F_2{\rm\ s.t.\ }
            d\leqslant \rho(x_n,y_n)<d+\frac{1}{n}
        \end{equation*}
        $F_1$紧则自列紧,$\{x_n\}$存在子列$\{x_{n_k}\}$收敛到$x'\in F_1$,
        相应的$\{ y_{n_k} \}$有子列$\{ y_{n_{k_j}} \}$收敛到$y'\in F_2$,于是
        \begin{equation*}
            d\leqslant \rho( x_{n_{k_j}},y_{n_{k_j}} )<d+\frac{1}{n_{k_j}}
        \end{equation*}
        令$j\rightarrow 0$,得$d=\rho(x',y')$.  
    \end{solve}

    \begin{exer}{\rm (1.3.6)}
        $E=\{ \sin{nt} \}_{n=1}^\infty$,证明$E\subset C[0,\pi]$不是列紧的。
    \end{exer}
    \begin{solve}
        只需证明$\{ \sin{nt} \}_{n=1}^\infty$不是等度连续的。对$\varepsilon_0=1$,
        $\forall \delta>0$,取$k\in\N$使得$\frac{1}{k}<\delta$,设$n_k=2k$,$t_k=\frac{\pi}{4k}\in[0,\pi]$,于是
        \begin{equation*}
            |t_k-t_0|=|t_k|=\frac{\pi}{4k}<\frac{1}{k}<\delta
        \end{equation*}
        \begin{equation*}
            |\sin{n_kt_k}-\sin{n_kt_0}|=\sin{\frac{\pi}{2}}=1=\varepsilon_0
        \end{equation*}
        所以$\{ \sin{nt} \}_{n=1}^\infty$不是等度连续的。
    \end{solve}
    
    \begin{exer}{\rm (1.3.7)}
        $S$空间的子集$A$列紧的充要条件是$\forall n\in\N,\exists C_n>0{\rm\ s.t.\ }\forall x=\{\xi_n\}_{n=1}^\infty\in A,|\xi_n|\leqslant C_n $.
    \end{exer}
    \begin{solve}
        必要性:$A$在$S$中列紧,
        任取无穷点列$\{\xi^{(m)}\}_{m=1}^\infty\subset A$
        有收敛子列$ \{\xi^{(m_k)}\}_{k=1}^\infty $,
        而$S$中的收敛与按坐标收敛等价,
        所以固定$m$,
        点列$\{\xi^{(m)}\}_{m=1}^\infty$中的每一个点的坐标序列
        $\{\xi_n^{(m)}\}_{n=1}^\infty$
        也可以从其任意无穷子集中取出收敛子列。
        固定$n$,$A$中所有点的第$n$个坐标构成$\C$上的集合,
        要从其任意无穷子集中取出收敛子序列要求该集合有界,
        此即为要求所证。
    
        充分性:只需构造$A$的列紧$\varepsilon$网。
        对$\forall \varepsilon>0$,
        选取充分大的$n$使得$\frac{1}{2^n}<\varepsilon$,考虑
        \begin{equation*}
            H=\{h_n=(\xi_1,\xi_2,\cdots,\xi_n,0,\cdots):(\xi_1,\xi_2,\cdots,\xi_n,\xi_{n+1},\cdots)\in A\}
        \end{equation*}
        因为
        \begin{equation*}
            \rho(x,h_n)=\sum_{k=n+1}^\infty \frac{1}{2^k}\frac{|\xi_k|}{1+|\xi_k|}
            \leqslant \sum_{k=n+1}^\infty \frac{1}{2^k}=\frac{1}{2^n}<\varepsilon
        \end{equation*}
        所以$H$是$A$的$\varepsilon$网。由假设
        \begin{equation*}
            |\xi_k|\leqslant C_k,\ k=1,2,\cdots,n
        \end{equation*}
        即每个坐标都是有界的,
        所以$H$可看做是$n$维空间中的有界集,从而是列紧的.
    \end{solve}
    
    \begin{exer}{\rm (1.3.9)}
        $(M,\rho)$是紧度量空间,$E\subset C(M)$,$E$中的函数一致有界并且满足:
        \begin{equation*}
            |x(t_1)-x(t_2)|\leqslant C\rho(t_1,t_2)^\alpha\ (\forall x\in E,\forall t_1,t_2\in M)
        \end{equation*}
        其中$0<\alpha\leqslant 1,C>0$,求证$E\subset C(M)$是列紧集。
    \end{exer}
    \begin{solve}
        $\forall \varepsilon>0$,取$\delta=\left( \frac{\varepsilon}{C} \right)^{\frac{1}{\alpha}}$,
        当$\rho(t_1,t_2)<\delta$时,
        $|x(t_1)-x(t_2)|\leqslant C\rho(t_1,t_2)^\alpha<\varepsilon$,
        所以$E$是等度连续的,再由{\rm Argela‑Ascoli}定理可得列紧。
    \end{solve}
    
    \begin{exer}{\rm (1.4.2)}
        $\forall x\in C(0,1]$,令$||x||=\fun{sup}{0<t\leqslant 1}|x(t)|$,求证:
        \begin{enumerate}
            \item $||\cdot||$是$C(0,1]$上的范数;
            \item $l^\infty$与$C(0,1]$的一个子空间是等距同构的。
        \end{enumerate}
    \end{exer}
    \begin{solve}
        $||\cdots ||$是$C(0,1]$上的范数:
        \begin{enumerate}[$(1)$]
            \item 正定性:绝对值非负所以$||x||\geqslant 0$,等号成立当且仅当$x(t)=0{\rm\ on\ }(0,1]$.
            \item 齐次性:$||Cx||=\fun{sup}{0<t\leqslant 1}|Cx(t)|=C\fun{sup}{0<t\leqslant 1}|x(t)|=C||x||$.
            \item 三角不等式:
                \begin{align*}
                    ||x+y||=&\fun{sup}{0<t\leqslant 1} |x(t)+y(t)|\\
                    \leqslant& \fun{sup}{0<t\leqslant 1} (|x(t)|+|y(t)|)\\
                    \leqslant& \fun{sup}{0<t\leqslant 1}|x(t)|+\fun{sup}{0<t\leqslant 1}|y(t)|=||x||+||y||
                \end{align*}
        \end{enumerate}
        以$1,\frac{1}{2},\frac{1}{3},\cdots,\frac{1}{n},\cdots$为节点的全体折线函数,构成
        $C(0,1]$的子空间,记作$C'(0,1]$.
        \begin{equation*}
            \forall f\in C'(0,1],x_f\defeq\{ f(\frac{1}{n}) \}_{n=1}^\infty\in l^\infty,
            ||x_f||_\infty=\fun{max}{n\geqslant 1}||f(\frac{1}{n})||\leqslant ||f||
        \end{equation*}
        反之,设$x=(\xi_1,\xi_2,\cdots,\xi_n,\cdots)\in l^\infty$,取一个以
        $1,\frac{1}{2},\frac{1}{3},\cdots,\frac{1}{n},\cdots$
        为节点的折线函数$f_x\in C(0,1]$,并令
        $f_x(\frac{1}{n})=\xi_n$,于是
        \begin{equation*}
            ||f_x||\leqslant \fun{max}{n\geqslant 1}||\xi_n||=||x||_\infty
        \end{equation*}
    \end{solve}
    
    \begin{exer}{\rm (1.4.3)}
        $\forall f\in C^1[a,b]$,令
        \begin{equation*}
            ||f||_1=\left( \int_a^b ( |f|^2+|f'|^2 )\d x \right)^\frac{1}{2}
        \end{equation*}
        证明$||\cdot||_1$是$C^1[a,b]$上的范数,在该范数下是否完备?
    \end{exer}
    \begin{solve}
        $||\cdot||_1$是$C^1[a,b]$上的范数:
        \begin{enumerate}[$(1)$]
            \item 正定性:$||f||_1\geqslant 0$,等号成立当且仅当
                $|f(x)|^2=0,\forall x\in [a,b]$当且仅当
                $f(x)=0,\forall x\in [a,b]$.
            \item 齐次性:
                \begin{equation*}
                    ||Cf||_1=\left( \int_a^b ( |Cf|^2+|Cf'|^2 ) \right)^{\frac{1}{2}}=C||f||_1
                \end{equation*}
            \item 三角不等式:
                \begin{align*}
                    ||f+g||_1^2&=\int_a^b ( |f|^2+|f'|^2 )+\int_a^b ( |g|^2+|g'|^2 )\\
                    &\leqslant \int_a^b ( |f|^2+|f'|^2 )+\int_a^b ( |g|^2+|g'|^2 )+2\left(\int_a^b ( |f|^2+|f'|^2 )\right)^{\frac{1}{2}}\left(\int_a^b ( |g|^2+|g'|^2 )\right)^{\frac{1}{2}}\\
                    &=(||f||_1+||g||_1)^2
                \end{align*}
            $(C^1[a,b],||\cdot ||_1)$不完备,取
            \begin{equation*}
                f_n(x)=\sqrt{x^2+\frac{1}{n^2}}\in C^1[-1,1]
            \end{equation*}
            设$f_n\mathop{\rightarrow }\limits^{||\cdot||_1}f\in C^1[a,b]$,则
            $f'n\mathop{\rightarrow }\limits^{L_2}f'$,又有:$||f_n-|x|||_1\rightarrow 0$,故
            \begin{equation*}
                f_n'\mathop{\rightarrow }\limits^{L_2}-\chi_{[-1,0)}+\chi_{[0,1]}
            \end{equation*}
            $L_p$收敛则存在子列{\rm a.e.}收敛,于是$f'\eq{a.e.}-\chi_{[-1,0)}+\chi_[0,1]$,
            这与$f'$连续矛盾。
        \end{enumerate}
    \end{solve}
    
    \begin{exer}{\rm (1.4.4)}
        $\forall f\in C[0,1]$,令:
        \begin{equation*}
            ||f||_1=\left( \int_0^1 |f(x)|^2\d x \right)^\frac{1}{2}
        \end{equation*}
        \begin{equation*}
            ||f||_2=\left( \int_0^1 (1+x)|f(x)|^2\d x \right)^\frac{1}{2}
        \end{equation*}
        证明两个范数等价。
    \end{exer}
    \begin{solve}
        $1\leqslant 1+x\leqslant 2\Rightarrow ||f||_1\leqslant ||f||_2\leqslant \sqrt{2}||f||_1$,故为等价范数。
    \end{solve}
    
    \begin{exer}{\rm (1.4.5)}
        $BC[0,\infty)$代表$[0,\infty)$上连续有界函数全体,$\forall x\in BC[0,\infty),a>0$,
        定义
        \begin{equation*}
            ||f||_a=\left( \int_0^\infty {\rm e}^{-ax}|f(x)|^2\d x \right)^\frac{1}{2}
        \end{equation*}
        求证:
        \begin{enumerate}
            \item $||\cdot ||_a$是$BC[0,\infty)$上的范数;
            \item 若$a,b>0,a\neq b$,求证两个范数不等价。
        \end{enumerate}
    \end{exer}
    \begin{solve}
        $||\cdot ||_a$是$BC[0,+\infty)$上的范数:
        \begin{enumerate}[$(1)$]
            \item 正定性:$||f||_a\geqslant 0$,等号成立当且仅当(因为${\rm e}^{-ax}|f(x)|^2$非负连续)
                ${\rm e}^{-ax}|f(x)|^2=0\Leftrightarrow f(x)=0$.
            \item 齐次性:
                \begin{align*}
                    ||Cf||_a=\left( C^2\int_0^{\infty}{\rm e}^{-ax}|f(x)|^2\d x \right)^{\frac{1}{2}}
                    =C||f||_a
                \end{align*}
            \item 三角不等式:
                \begin{align*}
                    ||f+g||_a=||{\rm e}^{-ax}(f+g)||_{L_2}
                    \leqslant ||{\rm e}^{-ax}f||_{L_2}+||{\rm e}^{-ax}g||_{L_2}
                    \leqslant ||f||_a+||g||_a
                \end{align*}
        \end{enumerate}
        不妨设$b>a>0$,设
        \begin{equation*}
            f_n(x)=\left\{ \begin{array}{ll}
                {\rm e}^{-\frac{ax}{2}}&,x\in [0,n]\\
                \sqrt{-x+n+{\rm e}^{-an}}&,x\in (n,n+{\rm e}^{-an}]\\
                0&, x\in (n+{\rm e}^{-\frac{an}{2}},+\infty)
            \end{array} \right.
        \end{equation*}
        则$||f_n(x)||_a\geqslant n$发散,而
        \begin{align*}
            ||f_n(x)||_b&=\int_0^n {\rm e}^{-(a+b)x}\d x
            +\int_n^{n+{\rm e}^{-an}} {\rm e}^{-bx}(-x+b+{\rm e}^{-an})\d x\\
            &=-\frac{1}{a+b}( {\rm e}^{-(a+b)n}-1 )+\mathop{\int_n^{n+{\rm e}^{-an}} {\rm e}^{-bx}(-x+b+{\rm e}^{-an})\d x}\limits_{\leqslant {\rm e}^{-an}\fun{max}{n\leqslant x\leqslant n+{\rm e}^{-an}}
            {\rm e}^{-bx}(-x+b+{\rm e}^{-an})\rightarrow 0}\\
            &\rightarrow \frac{1}{a+b}
        \end{align*}
        所以两个范数不等价。
    \end{solve}
    
    \begin{exer}{\rm (1.4.6)}
        两个Banach空间$(X_1,X_2)$的乘积空间上赋予范数
        \begin{equation*}
            ||(x_1,x_2)||={\rm max}( ||x_1||_1,||x_2||_2 )
        \end{equation*}
        证明乘积空间仍然是Banach空间。
    \end{exer}
    \begin{solve}
        任取$\mathfrak{X}$中的柯西列$\{x^{(n)}\}_{n=1}^\infty$,
        则$\{x_i^{(n)}\}_{n=1}^\infty$是$\mathfrak{X}_i$中的柯西列$(i=1,2)$,
        存在唯一$x_i\in \mathfrak{X}_i$使得$x_i^{(n)}\mathop{\rightarrow}\limits^{||\cdot||_i}x_i$,于是
        \begin{equation*}
            ||x^{(n)}-(x_1,x_2)||={\rm max}( ||x_1^{(n)}-x_1||_1,||x_2^{(n)}-x_2||_2 )\rightarrow 0
        \end{equation*}
        即$x^{(n)}\mathop{\rightarrow}\limits^{||\cdot||}(x_1,x_2)$.
    \end{solve}

    \begin{exer}{\rm (1.4.7)}
        $X$是$B^*$空间,求证:$X$是Banach空间的充要条件为:
        \begin{equation*}
            \forall \{x_n\}_{n=1}^\infty \subset X,\sum_{n=1}^\infty ||x_n||<\infty\Rightarrow \sum_{n=1}^\infty x_n\mbox{收敛}
        \end{equation*}
    \end{exer}
    \begin{solve}
        必要性:设$a_n=\sum_{k=1}^n x_k$,如果$\sum_{n=1}^\infty||x_n||<\infty$,
        则$\forall \varepsilon>0$,存在$N$,使得
        \begin{equation*}
            \sum_{n=N+1}^\infty ||x_n||<\varepsilon
        \end{equation*}
        则$n>N$时,任意正整数$p$,
        \begin{equation*}
            ||a_{n+p}-a_{n}||
            =\left|\left| \sum_{n+1\leqslant k\leqslant n+p}x_k \right|\right|
            \leqslant \sum_{n+1\leqslant k\leqslant n+p}||x_k||
            \leqslant \sum_{n=N+1}^\infty ||x_n||<\varepsilon
        \end{equation*}
        所以$\{a_n\}$是柯西列,进而收敛。
    
        充分性:任取柯西列$\{x_n\}$,由习题$1.2.2$,只需证明其存在收敛子列。由柯西列的定义,
        \begin{equation*}
            \forall k\in\N_+,\exists n_k{\rm\ s.t.\ }
            ||x_{n_k+1}-x_{n_k}||<\frac{1}{2^k}
        \end{equation*}
        令$y_k=x_{n_k}$,则
        \begin{equation*}
            \sum_{i=1}^\infty ||y_{i+1}-y_i||<\sum_{i=1}^\infty \frac{1}{2^k}=1<\infty
        \end{equation*}
        故$\sum_{i=1}^\infty y_{i+1}-y_i$收敛,于是
        $y_k=y_1+\sum_{i=1}^k y_{i+1}-y_i$在$k\rightarrow\infty$时收敛,
        此即为$\{x_n\}$的收敛子列。
    \end{solve}
    
    \begin{exer}{\rm (1.4.14)}
        $C_0$为以0为极限的实数全体,赋予范数:
        \begin{equation*}
            ||x||=\fun{max}{n\geqslant 1}|\xi_n|,\ \forall x=\{\xi_n\}_{n=1}^\infty\in C_0 
        \end{equation*}
        设
        \begin{equation*}
            M=\left\{ x=\{\xi_n\}_{n=1}^\infty\in C_0\left|  \sum_{n=1}^\infty \frac{\xi_n}{2^n}=0 \right.\right\}
        \end{equation*}
        证明:
        \begin{enumerate}
            \item $M$是$C_0$的闭线性子空间;
            \item $x_0=(2,0,\cdots,0,\cdots)$,求证:
                \begin{equation*}
                    \fun{inf}{z\in M}||x_0-z||=1
                \end{equation*}
                但是$\forall y\in M$有$||x_0-y||>1$.
        \end{enumerate}
    \end{exer}
    \begin{solve}
        \begin{enumerate}[$(1)$]
            \item 先证明$M$是线性子空间,只需说明$M$中的元素关于$\R$上的加法和数乘封闭:
            $\forall x=\{x_n\}_{n=1}^\infty,y=\{y_n\}_{n=1}^\infty\in M,\lambda\in\R$,
            \begin{equation*}
                    \sum_{n=1}^\infty \frac{x_n+y_n}{2^n}=\sum_{n=1}^\infty \frac{x_n}{2^n}+\sum_{n=1}^\infty \frac{y_n}{2^n}=0\Rightarrow x+y\in M
            \end{equation*}
            \begin{equation*}
                    \sum_{n=1}^\infty \frac{\lambda x_n}{2^n}
                    =\lambda \sum_{n=1}^\infty \frac{x_n}{2^n}=0\Rightarrow \lambda x\in M
            \end{equation*}
    
            再证明$M$是闭子空间:设$x^{(n)}=(\xi_1^{(n)},\cdots,\xi_k^{(n)},\cdots)\in M$,$x=(\xi_1,\cdots,\xi_k,\cdots)$,
            且$x^{(n)}\rightarrow x$,则
            \begin{align*}
                    &\fun{lim}{n\rightarrow\infty}\fun{sup}{k\geqslant 1}
                    |\xi_k^{(n)}-\xi_k|=0\\
                    \Rightarrow&\forall\varepsilon>0,\exists N{\rm\ s.t.\ }\forall n>N,
                    \fun{sup}{k\geqslant 1}|\xi_k^{(n)}-\xi_k|<\varepsilon
            \end{align*}
            则
            \begin{align*}
                    \ms{\sum_{k=1}^\infty \frac{\xi_k}{2^k}}
                    &=\ms{ \sum_{k=1}^\infty \frac{\xi_k-\xi_k^{(N)}}{2^k}+\sum_{k=1}^\infty \frac{\xi_k^{(N)}}{2^k} }\\
                    &=\ms{ \sum_{k=1}^\infty \frac{\xi_k-\xi_k^{(N)}}{2^k} }<\varepsilon
            \end{align*}
            令$\varepsilon\rightarrow 0$,$\sum_{k=1}^\infty \frac{\xi_k}{2^k}=0\Rightarrow x\in M$,从而$M$是$C_0$的闭线性子空间。
            \item $\forall y=(y_1,\cdots,y_k,\cdots)\in M$,
            假设$||x_0-y||\geqslant 1$,即
            \begin{equation*}
                    |2-y_1|\leqslant 1,|y_2|\leqslant 1,\cdots,|y_n|\leqslant 1,\cdots
            \end{equation*}
            于是$y_1\geqslant 1$,
            \begin{align*}
                    \sum_{k=1}^\infty \frac{y_k}{2^k}
                    = \frac{y_1}{2}+\sum_{k=2}^\infty \frac{y_k}{2^k} \geqslant \frac{1}{2}-\sum_{k=2}^\infty \frac{1}{2^k}=0
            \end{align*}
            $y\in M$则上述不等式等号成立,意味着$y_1=1,y_k=-1,k\geqslant 2$,则$y\notin C_0$,矛盾。所以
            $\forall y\in M$,$||x_0-y||<1$.
            取$x^{(m)}=(\mathop{1-\frac{1}{2^{m-1}}}\limits_{1},\mathop{-1}\limits_{2},\cdots,\mathop{-1}\limits_{m},0,0,\cdots)\in M$,
            则$\rho(x_0,x^{(m)})=1+\frac{1}{2^{m-1}}$,于是
            \begin{equation*}
                    1\leqslant\fun{inf}{z\in M}||x_0-z||\leqslant 1+\frac{1}{2^{m-1}}
            \end{equation*}
            令$m\rightarrow\infty$得到
            \begin{equation*}
                    \fun{inf}{z\in M}||x_0-z||=1
            \end{equation*}
        \end{enumerate}
    \end{solve}
    
    \begin{exer}{\rm (1.4.15)}
        设$X$是$B^*$空间,$M$是$X$的有限维真子空间,求证:
        $\exists y\in X,||y||=1$,使得$\forall x\in M,||y-x||\geqslant 1$.
    \end{exer}
    \begin{solve}
        $M$是有限维子空间,所以闭,进而$\forall y_0\in X\backslash M$,
        $d\defeq \fun{inf}{x\in M}\ms{y_0-x}>0$,于是
        \begin{equation*}
            \forall n\in\N_+,\exists x_n\in M{\rm\ s.t.\ }d\leqslant \ms{y_0-x_n}<d+\frac{1}{n}
        \end{equation*}
        那么$\ms{x_n}\leqslant \ms{y_0-x_n}+\ms{y_0}\leqslant \ms{y_0}+d+1$,即$\{x_n\}$有界,$M$有限维所以有收敛子列
        ${x_{n_k}}$,设$x_{n_k}\rightarrow x_0\in M$,
        \begin{equation*}
            d\leqslant ||y_0-x_{n_k}||<d+\frac{1}{n_k}\Rightarrow ||y_0-x_0||=d
        \end{equation*}
        令$y=\frac{y_0-x_0}{d}$,则$||y||=1$,对于$\forall x\in M$,
        \begin{equation*}
            \ms{y-x}=\ms{\frac{y_0-x_0}{d}-x}=\frac{1}{d}
            \ms{y_0-\mathop{(x_0+dx)}\limits_{\in M}}\geqslant \frac{d}{d}=1
        \end{equation*}
    \end{solve}
    
    \begin{exer}{\rm (1.6.2)}
        求证$C[a,b]$中的范数:
        \begin{equation*}
            ||f||=\fun{max}{a\leqslant x\leqslant b}|f(x)|
        \end{equation*}
        不可能由内积诱导。
    \end{exer}
    \begin{solve}
        只需给出不满足平行四边形法则的例子,如下:取
        $[a,b]=[0,1]$,$f(x)=\frac{1}{2}$,$g(x)=x$,则
        \begin{equation*}
            ||f+g||^2+||f-g||^2=\frac{5}{2}
        \end{equation*}
        \begin{equation*}
            2(||f||^2+||g||^2)=3
        \end{equation*}
    \end{solve}
    
    \begin{exer}{\rm (1.6.4)}
        $M,N$是内积空间中的两个子集,求证:
        \begin{equation*}
            M\subset N\Rightarrow N^\perp \subset M^\perp
        \end{equation*}
    \end{exer}
    \begin{solve}
        $\forall x\in N^\perp,m\in M\subset N\Rightarrow\agl{x,m}=0\Rightarrow x\in M^\perp$,故$N^\perp\subset M^\perp$.
    \end{solve}
    
    \begin{exer}{\rm (1.6.5)}
        $M$是Hilbert空间的子集,求证:
        \begin{equation*}
            (M^\perp)^\perp=\overline{ {\rm span}(M) }
        \end{equation*}
    \end{exer}
    \begin{solve}
        $x\in M^\perp\Leftrightarrow x\perp {\rm span}(M)\Leftrightarrow x\perp \overline{ {\rm span}(M) }\Leftrightarrow x\in ( {\rm span}(M) )^\perp$,
        所以$M^\perp=( {\rm span}(M) )^\perp$,即证
        \begin{equation*}
            [( {\rm span}(M) )^\perp]^\perp=\overline{ {\rm span}(M) }
        \end{equation*}
        只需证明,对于闭子空间$M$,有$(M^\perp)^\perp=M$.假设$M\subsetneqq (M^\perp)^\perp$,
        对于$x\in (M^\perp)^\perp\backslash M$,由正交分解
        \begin{equation*}
            x=y+z,\ y\in M,\ z\in M^\perp
        \end{equation*}
        对于$m\in M^\perp$,$y\in M\Rightarrow \agl{y}{m}=0$,$x\in (M^\perp)^\perp\Rightarrow\agl{x}{m}=0$,
        所以$\agl{z}{m}=0\Rightarrow z\in (M^\perp)^\perp$,
        于是$z\in M^\perp\cap (M^\perp)^\perp\Rightarrow z=0\Rightarrow x\in M$,矛盾。
    \end{solve}
    
    \begin{exer}{\rm (1.6.6)}
        $L^2[-1,1]$中,偶函数集的正交补是什么?
    \end{exer}
    \begin{solve}
        偶函数集记作$X$,奇函数集记作$Y$.
    
        所有的奇函数$g\in Z$都垂直于$X$,因为
        \begin{equation*}
            \forall f\in X,f(x)g(x)\mbox{为奇函数}\Rightarrow \agl{f}{g}=\int_{-1}^{1} f(x)g(x)\d x=0\Rightarrow f\perp g
        \end{equation*}
        所以$Y\subset X^\perp$
    
        对于$h\in X^\perp$,可以分解为$h=h_1+h_2$,其中
        \begin{equation*}
            h_1(x)=\frac{h(x)+h(-x)}{2}\in X,\ h_2(x)=\frac{h(x)-h(-x)}{2}\in Y
        \end{equation*}
        则有
        \begin{equation*}
            \agl{h}{h_1}=0\Rightarrow \agl{h_1}{h_1}=0\Rightarrow h_1\eq{a.e.} 0 
        \end{equation*}
        所以$h\in Y\Rightarrow X^\perp\subset Y$,于是$Y=X^\perp$得证。
    \end{solve}
    
    \begin{exer}{\rm (1.6.9)}
        Hilbert空间$X$中的两个正交规范集$\{e_n\}_{n=1}^\infty,\{f_n\}_{n=1}^\infty$满足:
        \begin{equation*}
            \sum_{n=1}^\infty ||e_n-f_n||^2<1
        \end{equation*}
        求证:两者其中一个完备蕴含另一个完备。
    \end{exer}
    \begin{solve}
        假设$\{e_n\}$完备,$\{f_n\}$不完备,则存在$u\in X,u\neq 0$,使得
        $\agl{u}{f_n}=0,\forall n$,则
        \begin{align*}
            ||u||^2=\sum_{n=1}^\infty |\agl{u}{e_n}|^2=\sum_{n=1}^\infty 
            | \agl{u}{e_n-f_n} |^2\leqslant 
            \sum_{b=1}^\infty ||u||^2||e_n-f_n||^2<||u||^2
        \end{align*}
        矛盾。
    \end{solve}
    
    \begin{exer}{\rm (1.6.10)}
        $X_0$为Hilbert空间$X$的闭线性子空间,
        $\{e_n\},\{f_n\}$分别是$X_0,X_0^\perp$的正交规范基,求证:
        $\{e_n\}\cup \{f_n\}$是$X$的正交规范基。
    \end{exer}
    \begin{solve}
        $X$是{\rm Hilbert}空间,$X_0$是其闭线性子空间,
        所以$X=X_0\oplus X_0^\perp$,即$\forall x\in X$,存在唯一的
        $y\in X_0,\zeta\in X_0^\perp$使得$x=y+\zeta$,而$y$和$\zeta$能被唯一地表示为:
        \begin{equation*}
            y=\sum_{n=1}^\infty \agl{y}{e_n}e_n,\ 
            \zeta=\sum_{n=1}^\infty \agl{\zeta}{f_n}f_n
        \end{equation*}
        同时因为$\agl{y}{f_n}=\agl{\zeta}{e_n}=0$,设$\{e_n\}\cup\{f_n\}=\{\alpha_n\}$,
        则
        \begin{equation*}
            y+\zeta=\sum_{n=1}^\infty \agl{y}{\alpha_n}\alpha_n
        \end{equation*}
        所以$\{\alpha_n\}$是{\rm O.N.B}.
    \end{solve}
    
    \begin{exer}{\rm (1.6.11)}
        这道题要用到课本例1.6.28相关内容。
        \begin{enumerate}[(1).]
            \item 如果$u(z)$的泰勒展开式为
                \begin{equation*}
                    u(z)=\sum_{k=0}^\infty b_k z^k
                \end{equation*}
                求证:
                \begin{equation*}
                    \sum_{k=0}^\infty \frac{|b_k|^2}{1+k}<\infty
                \end{equation*}
            \item 设$u(z),v(z)\in H^2(D)$,并且
                \begin{equation*}
                    u(z)=\sum_{k=0}^\infty a_kz^k,\ 
                    v(z)=\sum_{k=0}^\infty b_kz^k
                \end{equation*}
                求证:
                \begin{equation*}
                    (u,v)=\pi \sum_{k=0}^\infty \frac{a_k\overline{b_k}}{k+1}
                \end{equation*}
            \item 设$u(z)\in H^2(D)$,求证:
                \begin{equation*}
                    |u(z)|\leqslant \frac{||u||}{\sqrt{\pi}(1-|z|)},\ \forall |z|<1
                \end{equation*}
            \item 验证$H^2(D)$是Hilbert空间。
        \end{enumerate}
    \end{exer}
    \begin{solve}
        \begin{enumerate}[(1).]
            \item 由例{\rm 1.6.28},$\varphi_n(z)=\sqrt{\frac{n}{\pi}}z^{n-1}$是一组{\rm O.N.B.}那么
            \begin{equation*}
                u(z)=\sum_{n=1}^\infty (u,\varphi_n)\varphi_n(z)
                =\sum_{n=1}^\infty b_{n-1}z^{n-1}
            \end{equation*}
            可得
            \begin{equation*}
                \frac{b_k}{\sqrt{k+1}}=(u,\varphi_{k+1})\cdot \sqrt{\frac{1}{\pi}}
            \end{equation*}
            于是
            \begin{equation*}
                \sum_{k=0}^\infty \frac{|b_k|^2}{k+1}
                =\frac{1}{\pi}\sum_{k=0}^\infty |(u,\varphi_{k+1})|^2\mathop{=}\limits^{\rm Parseval} \frac{1}{\pi}||u||^2<+\infty
            \end{equation*}
            \item \begin{align*}
                    (u,v)&=\left( \sum_{n=1}^\infty (u,\varphi_n)\varphi_n,\sum_{m=1}^\infty (v,\varphi_m)\varphi_m \right)\\
                    &=\sum_{n=1}^\infty (u,\varphi_n)\overline{ (v,\varphi_n) }\\
                    &=\sum_{k=0}^\infty \frac{\sqrt{\pi}a_k}{\sqrt{k+1}}\frac{\sqrt{\pi}\overline{b_k}}{\sqrt{k+1}}
                    =\pi\sum_{k=0}^\infty \frac{a_k\overline{b_k}}{k+1}
            \end{align*}
            \item 由
            \begin{equation*}
                    \sum_{k=0}^\infty (k+1)|z|^k=\frac{1}{(1-|z|)^2},|z|<1
            \end{equation*}
            可得
            \begin{align*}
                    RHS^2=\frac{||u||^2}{\pi(1-|z|)^2}
                    &=\sum_{k=0}^\infty \frac{|b_k|^2}{k+1}
                    \cdot \sum_{k=0}^\infty (k+1)|z|^k\\
                    &\geqslant \sum_{k=0}^\infty \frac{|b_k|^2|z|^k}{k+1}
                    \cdot \sum_{k=0}^\infty (k+1)|z|^k\\
                    &\geqslant \sum_{k=0}^\infty |b_k z^k|^2\geqslant |u(z)|^2=LHS^2
            \end{align*}
            \item 设$u_n$为$H^2(D)$上的基本列,由$(3)$可知$u_n$内闭一致收敛
            至$u(z)$,故$u(z)$全纯,且
            \begin{equation*}
                    |u_n(z)|-|u_m(z)|\leqslant | u_n(z)-u_m(z) |
            \end{equation*}
            故$|u_n(z)|$是$L^2(D)$中的基本列,由$L^2(D)$的完备性,
            可知$|u(z)|\in L^2(D)$,即$u(z)\in H^2(D)$.
            \begin{equation*}
                    ||u_n(z)-u_m(z)||_{H^2(D)}\rightarrow 0
            \end{equation*}
            令$m\rightarrow\infty$,有$u_n(z)\mathop{\rightarrow}\limits^{H^2(D)}u(z)$.
        \end{enumerate}
    \end{solve}
    
    \begin{exer}{\rm (1.6.12)}
        $X$是内积空间,$\{e_n\}$是$X$中的正交规范集,求证:
        \begin{equation*}
            \left| \sum_{n=1}^\infty (x,e_n)\overline{(y,e_n)} \right|
            \leqslant ||x||\cdot ||y||,\ \forall x,y\in X
        \end{equation*}
    \end{exer}
    \begin{solve}
        \begin{align*}
            LHS&=\sum_{n=1}^\infty (x,e_n)\overline{(y,e_n)}\\
            &=\agl{ \sum_{n=1}^\infty (x,e_n)e_n }{ \sum_{n=1}^\infty (y,e_n)e_n }\\
            &\mathop{\leqslant}\limits^{\rm C-S} 
            \ms{ \sum_{n=1}^\infty (x,e_n)e_n }\cdot\ms{\sum_{n=1}^\infty (y,e_n)e_n}\\
            &=\sum_{n=1}^\infty |(x,e_n)|\cdot\sum_{n=1}^\infty |(y,e_n)|\\
            &\mathop{\leqslant}\limits^{\rm Bessel}||x||\cdot ||y||=RHS
        \end{align*}
    \end{solve}
    
    \begin{exer}{\rm (1.6.13)}
        $X$是内积空间,令
        \begin{equation*}
            C=\{ x\in X:||x-x_0||\leqslant r \}
        \end{equation*}
        其中$x_0\in X,r>0$,求证:
        \begin{enumerate}
            \item $C$是$X$中的闭凸集;
            \item \begin{equation*}
                y=\left\{ \begin{array}{ll}
                    x_0+r(x-x_0)/||x-x_0||&,x\notin C\\
                    x&,x\in C
                \end{array} \right.
            \end{equation*}
            是$x$在$C$中的最佳逼近元。
        \end{enumerate}
    \end{exer}
    \begin{solve}
        \begin{enumerate}[$(1)$]
            \item $C$闭显然,只需证明为凸集:$\forall x_1,x_2\in C,\forall \theta\in(0,1)$,
            \begin{align*}
                    ||\theta x_1+(1-\theta)x_2-x_0||&=||\theta (x_1-x_0)+(1-\theta)(x_2-x_0)||\\
                    &\leqslant \theta||x_1-x_0||+(1-\theta)||x_2-x_0||\\
                    &\leqslant \theta r+(1-\theta)r=r\\
                    \Rightarrow \theta x_1+(1-\theta)x_2&\in C
            \end{align*}
            \item 当$x\notin C$,对于$\forall z\in C$,
            \begin{align*}
                    ||x-y||&=||x-y||+||y-x_0||-r\\
                    &= ||x-x_0||-r\\
                    &\leqslant ||x-z||+||z-x_0||-r\\
                    &\leqslant ||x-z||
            \end{align*}
            等号成立当且仅当$z=y$,则$||x-y||=\fun{inf}{z\in M}||x-z||$;当$x\in C$则自身就是最佳逼近元。证毕。
        \end{enumerate}
    \end{solve}



\section{第二章}
\subsection{线性算子}
\begin{exer}{\rm (2.1.1)}
    证明:线性映射$T:X\rightarrow Y$有界当且仅当$T$将有界集映为有界集。
\end{exer}
\begin{solve}
    必要性:任取有界集$A\subset X$,存在$B(0,r)\supset A$,则
    \begin{equation*}
        ||Tx||\leqslant C||x||\leqslant Cr\Rightarrow Tx\in B(0,Cr)\Rightarrow T(A)\subset B(0,Cr)
    \end{equation*}
    所以$T$把有界集映为有界集。

    充分性:$B=\{x\in X:||x||\leqslant 1\}$是有界集,
    则存在$M>0$使得$||Tx||\leqslant M,\forall x\in B$,
    对于$\forall x\in X-\{0\}$,
    \begin{equation*}
        \left|\left| T\left( \frac{x}{||x||} \right) \right|\right|\leqslant M\Rightarrow ||Tx||\leqslant M||x||
    \end{equation*}
    因此$T$有界。
\end{solve}

\begin{exer}{\rm (2.1.2)}
    设线性映射$A:X\rightarrow Y$有界,证明:
    \begin{enumerate}[(1).]
        \item $\ms{A}=\fun{sup}{\ms{x}\leqslant 1}\ms{Ax}$
        \item $\ms{A}=\fun{sup}{\ms{x}<1}\ms{Ax}$
    \end{enumerate}
\end{exer}
\begin{solve}
    \begin{enumerate}[$(1)$]
        \item 一方面
            \begin{equation*}
                \fun{sup}{||x||\leqslant 1}||Ax||\geqslant \fun{sup}{||x||=1}||Ax||=||A||
            \end{equation*}
            另一方面
            \begin{equation*}
                ||A||=\fun{sup}{x\neq 0}\frac{||Ax||}{||x||}\geqslant 
                \fun{sup}{\mathop{x\neq 0}\limits_{||x||\leqslant 1}}\frac{||Ax||}{||x||}
                \geqslant \fun{sup}{||x||\leqslant 1} ||Ax||
            \end{equation*}
            所以二者相等。
        \item 由上一问,
            \begin{equation*}
                ||A||=\fun{sup}{||x||\leqslant 1}||Ax||\geqslant 
                \fun{sup}{||x||<1}||Ax||
            \end{equation*}
            另一方面,对于$\forall ||x||=1$,$\forall \varepsilon>0$,
            \begin{equation*}
                Ax=(1+\varepsilon)A( \frac{x}{1+\varepsilon} )
                \leqslant (1+\varepsilon)\fun{sup}{||x||<1}||Ax||
            \end{equation*}
            即
            \begin{equation*}
                ||A||\leqslant (1+\varepsilon)\fun{sup}{||x||<1}||Ax||
            \end{equation*}
            令$\varepsilon\rightarrow 0^+$得证。
    \end{enumerate}
\end{solve}

\begin{exer}{\rm (2.1.5)}
    $f$是$X$上的非零有界线性泛函,令:
    \begin{equation*}
        d={\rm inf}\{ \ms{x}:f(x)=1,x\in X \}
    \end{equation*}
    证明$\ms{f}=\frac{1}{d}$.
\end{exer}
\begin{solve}
    \begin{equation*}
        d={\rm inf}\{ ||x||:f(x)=1,x\in X \}
    \end{equation*}
    一方面,若$x\in X$满足$f(x)=1$,
    \begin{align*}
        &||f||\cdot ||x||=||x||\cdot \fun{sup}{0\neq z\in X}\frac{ ||f(z)|| }{||z||}\geqslant ||x||\cdot \frac{||f(x)||}{||x||}=||f(x)||=1\\
        \Rightarrow&||x||\geqslant \frac{1}{||f||}\\
        \Rightarrow&d\geqslant \frac{1}{||f||}  
    \end{align*}
    另一方面,由$||f||$的定义,$\forall \varepsilon>0$,存在$0\neq x_0\in X$使得
    \begin{equation*}
        \frac{||f(x_0)||}{||x_0||}>||f||-\varepsilon\Rightarrow
        \frac{1}{||f||-\varepsilon}>\left|\left| \frac{x_0}{||f(x_0)||} \right|\right|
    \end{equation*}
    而$f\left( \frac{x_0}{||f(x_0)||} \right)=1$,所以
    \begin{equation*}
        \frac{1}{||f||-\varepsilon}>\left|\left| \frac{x_0}{||f(x_0)||} \right|\right|\geqslant d
    \end{equation*}
    综上,
    \begin{equation*}
        \frac{1}{||f||}\leqslant d<\frac{1}{||f||-\varepsilon}
    \end{equation*}
    令$\varepsilon\rightarrow0^+$得证。
\end{solve}

\begin{exer}{\rm (2.1.7)}
    线性映射$T:X\rightarrow Y$,令:
    \begin{equation*}
        N(T)\defeq \{ x\in X:Tx=0 \}
    \end{equation*}
    \begin{enumerate}
        \item 若$T$有界,证明$N(T)$是$X$的闭线性子空间;
        \item $N(T)$是$X$的闭线性子空间能否推出$T$有界?
        \item 若$f$是线性泛函,求证:\begin{equation*}
            f\in X^*\Leftrightarrow N(f)\mbox{是闭线性子空间}
        \end{equation*}
    \end{enumerate}
\end{exer}
\begin{solve}
    \begin{enumerate}[$(1)$]
        \item $\{x_n\}\subset N(T)$,$x_n\rightarrow x$,$Tx_n=0$,$T$有界所以连续,令$n\rightarrow\infty$得到$Tx=0\Rightarrow x\in N(T)$,所以$N(T)$闭。
            只需验证$N(T)$关于加法和数乘封闭,
            \begin{equation*}
                x_1,x_2\in N(T)\Rightarrow T(x_1+x_2)=T(x_1)+T(x_2)=0\Rightarrow x_1+x_2\in N(T)
            \end{equation*}
            \begin{equation*}
                \lambda\in\K,x\in N(T)\Rightarrow T(\lambda x)=\lambda T(x)=0\Rightarrow \lambda x\in N(T)
            \end{equation*}
            所以$N(T)$是$X$的闭线性子空间。
        \item 不能,反例如下:在$l^\infty$中,取$a=(1,-1,0,0,\cdots)$,设
            \begin{equation*}
                f:l^\infty\rightarrow \R,\{x_n\}_{n=1}^\infty \mapsto \sum_{n=1}^\infty x_n
            \end{equation*}
            \begin{equation*}
                T:l^\infty\rightarrow l^\infty,\xi\mapsto \xi-f(\xi)a
            \end{equation*}
            \begin{enumerate}[$1^\circ$]
                \item $T$是线性映射:显然$f$是线性映射,\begin{align*}
                    T(\alpha \xi_1+\beta \xi_2)&=\alpha \xi_1+\beta \xi_2-f(\alpha \xi_1+\beta \xi_2)a\\
                    &=\alpha \xi_1+\beta \xi_2-[\alpha f(\xi_1)+\beta f(\xi_2)]a\\
                    &=\alpha (\xi_1-f(\xi_1)a)+\beta (\xi_2-f(\xi_2)a)\\
                    &=\alpha T(\xi_1)+\beta T(\xi_2),\forall \alpha,\beta\in\K,\xi_1,\xi_2\in l^\infty
                \end{align*}
                \item $N(T)=\{0\}$,为闭线性子空间:
                \begin{align*}
                    \xi=\{x_n\}_{n=1}^\infty
                    \in N(T)\Leftrightarrow& \xi-(f(\xi),-f(\xi),0,0,\cdots)=0\\
                    \Rightarrow& x_1=f(\xi),x_2=-f(\xi),x_3=0,x_4=0,\cdots\\
                    \Rightarrow& f(\xi)=x_1+x_2=0\Rightarrow \xi=0
                \end{align*}
                \item $f$无界:只需取$\xi_n=(1,\cdots,\mathop{1}\limits_{n},0,0,\cdots)$,$||\xi_n||=1$,$f(\xi_n)/||\xi_n||=n\rightarrow\infty$.
                \item $T$无界:假设有界,则
                \begin{align*}
                    \xi\in l^\infty\Rightarrow& ||\xi-f(\xi)a||=||T\xi||\leqslant ||T||\cdot ||\xi||\\
                    \Rightarrow& |f(\xi)|=|f(\xi)|\cdot ||a||\leqslant ||\xi-f(\xi)a||+||\xi||\leqslant (1+||T||)||\xi||
                \end{align*}
                这意味着$f$有界,矛盾。
            \end{enumerate}
        \item $(1)$已经证明必要性,只需说明充分性:假设$f$无界,即
            \begin{equation*}
                \forall n\in \N_+,\exists x_n\in X{\rm\ s.t.\ }||x_n||=1,f(x_n)\geqslant n
            \end{equation*}
            令
            \begin{equation*}
                y_n\defeq \frac{x_n}{f(x_n)}-\frac{x_1}{f(x_1)}
            \end{equation*}
            则$f(y_n)=1-1=0\Rightarrow y_n\in N(f)$,但是$y_n\rightarrow -\frac{x_1}{f(x_1)}\notin N(f)$,这与$N(f)$闭矛盾。
    \end{enumerate}
\end{solve}

\begin{exer}{\rm (2.1.8)}
    $f$是$X$上的线性泛函,记
    \begin{equation*}
        H_f^\lambda \defeq \{ x\in X:f(x)=\lambda \}
    \end{equation*}
    其中$\lambda\in\K$,如果$f\in X^*$,且$\ms{f}=1$,求证:
    \begin{enumerate}[(1).]
        \item $|f(x)|={\rm inf}\{ \ms{x-z}:\forall z\in H_f^0 \}$,$\forall x\in X$.
        \item $\forall \lambda\in\K$,$H_f^\lambda$上的任一点$x$到$H_f^0$的距离都等于$|\lambda|$.
    \end{enumerate}
    并对$X=\R^2,\K=\R$的情形解释上述命题的几何意义。
\end{exer}
\begin{solve}
    \begin{enumerate}[$(1)$]
        \item 记$d={\rm inf}\{ ||x-z||:\forall z\in H_f^0 \}$,
        一方面,对于$\forall z\in H_f^0$,
            \begin{equation*}
                |f(x)|=|f(x-z)|\leqslant ||f||\cdot ||x-z||=||x-z||
            \end{equation*}
            所以$|f(x)|\leqslant d$;
            另一方面,$\forall y\in X$满足$||y||=1$且$f(y)\neq 0$,有
            \begin{equation*}
                z=x-\frac{f(x)}{f(y)}y\in N(f)
            \end{equation*}
            而
            \begin{equation*}
                |f(x)|=|f(y)|\cdot ||x-z||\geqslant |f(y)|d
            \end{equation*}
            于是
            \begin{equation*}
                |f(x)|\geqslant ||f|| d=d
            \end{equation*}
        \item 对于$\forall x\in H_f^\lambda$,有$f(x)=\lambda$,由上一问可知
            \begin{equation*}
                |\lambda|=\rho(x,H_f^0)
            \end{equation*}
            在$\R^2$中,由$||f||=1$知$f((x,y))=\alpha x+\beta y$,其中$\alpha=f((1,0)),\beta=f((0,1)),\alpha^2+\beta^2=1$,
            \begin{equation*}
                \rho(x,H_f^0)=\rho(0,H_f^\lambda)=\left.\frac{ |\alpha x+\beta y-\lambda| }{\sqrt{\alpha^2+\beta^2}}\right|_{(0,0)}=|\lambda|
            \end{equation*}
    \end{enumerate}
\end{solve}

\subsection{Riesz表示定理}
\begin{exer}{\rm (2.2.1)}%done
    $H$是Hilbert空间,设$f_1,\cdots,f_n$是$H$上的一组有界线性泛函,对于$k=1,2,\cdots,n$,定义
    \begin{equation*}
        M\defeq \bigcap_{k=1}^n N(f_k),\ N(f_k)\defeq \{ x\in H:f_k(x)=0 \}
    \end{equation*}
    任取$x_0\in H$,记$y_0$为$x_0$在$M$上的正交投影,求证:$\exists y_1,y_2,\cdots,y_n\in N(f_k)^\perp$
    以及$\alpha_1,\cdots,\alpha_n\in\K$使得:
    \begin{equation*}
        y_0=x_0-\sum_{k=1}^n \alpha_k y_k
    \end{equation*}
\end{exer}
\begin{solve}
    由{\rm Riesz}表示定理,
    \begin{equation*}
        \forall 1\leqslant k\leqslant n,\exists y_k\in H{\rm\ s.t.\ }x\in H\Rightarrow f_k(x)=\ag{x,y_k}
    \end{equation*}
    则
    \begin{equation*}
        x\in M\Leftrightarrow \ag{x,y_k}=0,k=1,2,\cdots,n
    \end{equation*}
    所以$M=({\rm span}\{ y_k \}_{k=1}^n)^\perp$,由习题1.6.5可知
    \begin{equation*}
        M^\perp =\overline{ {\rm span}\{ y_k \}_{k=1}^n }
        ={\rm span}\{ y_k \}_{k=1}^n
    \end{equation*}
    因此$x_0-y_0\in M^\perp$可以表示成$\{y_k\}$的线性组合:
    \begin{equation*}
        x_0-y_0=\sum_{k=1}^n \overline{\ag{z_k,z_0}}z_k\in{\rm span}\{z_k\}_{k=1}^n={\rm span}\{y_k\}_{k=1}^n
    \end{equation*}
\end{solve}

\begin{exer}{\rm (2.2.3)}%done
    $H$是Hilbert空间,$H$的元素是定义在集合$S$上的复值函数。$\forall x\in S$
    ,由
    \begin{equation*}
        J_x(f)=f(x)
    \end{equation*}
    定义的映射$J_x:H\rightarrow \C$是$H$上的连续线性泛函,求证:存在$S\times S$上的复值函数$K(x,y)$,适合条件:
    \begin{enumerate}[(1).]
        \item 对任意固定的$y\in S$,作为$x$的函数有$K(x,y)\in H$.
        \item $f(y)=(f,K(\cdot,y))$,$\forall f\in H$,$\forall y\in S$.
    \end{enumerate}
\end{exer}
\begin{solve}
    由{\rm Riesz}表示定理,
    \begin{equation*}
        \forall x\in S,\exists f_x\in H{\rm\ s.t.\ }J_x(f)=\ag{f,f_x},\forall f\in H
    \end{equation*}
    令$K(x,y)\defeq \ag{f_x,f_y}$,则:
    对于任一固定的$y\in S$,
    \begin{equation*}
        K(x,y)=\ag{f_x,f_y}=J_x(f_y)=f_y(x),\forall x\in S
    \end{equation*}
    可得$K(\cdot,y)=f_y\in H$;另一方面,
    \begin{equation*}
        f(y)=\ag{f,f_y}=\ag{f,K(\cdot,y)},\forall f\in H,\forall y\in S
    \end{equation*}
    所以$K(x,y)$满足题意。
\end{solve}

\begin{exer}{\rm (2.2.5)}
    $H$是Hilbert空间,$L,M$是$H$上的闭线性子空间,求证:
    \begin{enumerate}[(1).]
        \item $L\perp M\Leftrightarrow P_LP_M=0$.
        \item $L=M^\perp\Leftrightarrow P_L+P_M=I$.
        \item $P_LP_M=P_{L\cap M}\Leftrightarrow P_LP_M=P_MP_L$.
    \end{enumerate}
\end{exer}
\begin{solve}
    \begin{enumerate}[$(1)$]
        \item 必要性:
            \begin{equation*}
                \ag{P_LP_Mx,y}=\ag{P_Mx,P_Ly}=0,\forall x,y\in H
            \end{equation*}
            充分性:
            \begin{equation*}
                \ag{x,y}=\ag{P_Lx,P_My}=\ag{x,P_LP_My}=\ag{x,0}=0,\forall x\in L,y\in M
            \end{equation*}
        \item 必要性:
            \begin{equation*}
                x=P_Lx+P_{L^\perp}x=P_Lx+P_Mx,\forall x\in H
            \end{equation*}
            充分性:
            \begin{align*}
                x\in L&\Rightarrow x=P_Lx+P_Mx\Rightarrow P_Mx=0\Rightarrow x\in M^\perp\\
                x\in M^\perp&\Rightarrow P_Mx=0\Rightarrow P_Lx=x-P_Mx=x\Rightarrow x\in L
            \end{align*}
            于是$L=M^\perp$.
        \item 必要性:
            \begin{equation*}
                P_LP_M=P_{L\cap M}=P_{M\cap L}=P_MP_L
            \end{equation*}
            充分性:对于$\forall y\in H$,
            \begin{equation*}
                M\ni P_MP_Lx=P_LP_Mx\in L
            \end{equation*}
            而$P_LP_Mx\in L\cap M$,则由变分引理,为证$P_LP_M=P_{L\cap M}$只需验证
            \begin{equation*}
                (x-P_LP_Mx)\perp (L\cap M),\forall x\in H
            \end{equation*}
            实际上
            \begin{align*}
                \ag{x-P_LP_Mx,y}&=\ag{x,y}-\ag{P_LP_Mx,y}\\
                &=\ag{x,y}-\ag{x,y}=0,\forall y\in L\cap M
            \end{align*}
            所以得证。
    \end{enumerate}
\end{solve}

\subsection{Baire纲定理}
    本小节没有布置课本上的习题。

\subsection{共鸣定理}
\begin{exer}{\rm (2.3.7)}%done
    设$X,Y$是Banach空间,$\{A_n\}\subset \mathcal{L}(X,Y)$,
    若对于$\forall x\in X$,$\{A_n x\}$在$Y$中收敛,求证:存在$A\in\mathcal{L}(X,Y)$使得
    \begin{equation*}
        A_n x\rightarrow Ax,\ ||A||\leqslant \fun{liminf}{n\rightarrow\infty}||A_n||
    \end{equation*}
\end{exer}
\begin{solve}
    对于$\forall x\in X$,因为$\{A_n x\}$收敛,可定义
    $A:x\mapsto\fun{lim}{n\rightarrow\infty}A_n x$,不难验证$A$是线性算子。
    收敛列必有界,所以$\forall x\in X,\fun{sup}{n\geqslant 1}||A_n x||<\infty$.
    由{\rm UBP},
    \begin{equation*}
        \exists M>0{\rm\ s.t.\ }\fun{sup}{n\geqslant 1}||A_n||\leqslant M
    \end{equation*}
    于是
    \begin{equation*}
        ||Ax||=\fun{lim}{n\rightarrow\infty}||A_n x||\leqslant \fun{liminf}{n\rightarrow\infty}||A_n||\cdot ||x||\leqslant M||x||,\forall x\in X
    \end{equation*}
    (稍微解释一下这里的小于等于号是怎么来的:$||A_nx||\leqslant ||A_n||\cdot ||x||$,
    尽管$\{||A_n||\cdot ||x||\}$不一定是收敛列,
    但我们知道它的收敛子列的极限一定大于等于$\{||A_nx||\}$对应收敛子列的极限,也就是$\{||A_nx||\}$的极限,
    因此由{\rm liminf}的定义可得。)
    这就证明了$A$有界且$||A||\leqslant\fun{liminf}{n\rightarrow\infty}||A_n||$.
\end{solve}

\begin{exer}{\rm (2.3.8)}%done
    设$1<p<\infty$,$p^{-1}+q^{-1}=1$,如果序列$\{\alpha_k\}$
    使得对于$\forall x\in \{\xi_k\}\in \ell^p$保证
    $\sum_{k=1}^\infty \alpha_k\xi_k$收敛,求证:$\{\alpha_k\}\in \ell^q$.

    若定义
    \begin{equation*}
        f:x\mapsto \sum_{k=1}^\infty \alpha_k\xi_k
    \end{equation*}
    求证:$f$是$\ell^p$上的线性泛函,而且
    \begin{equation*}
        ||f||=\left(  \sum_{k=1}^\infty |\alpha_k|^q  \right)
        ^\frac{1}{q}
    \end{equation*}
\end{exer}
\begin{solve}
    考虑
    \begin{equation*}
        f_n:\ell^p\rightarrow\K,x=\{\xi_k\}\mapsto \sum_{k=1}^n \alpha_k\xi_k
    \end{equation*}
    则由{\rm Holder}不等式,
    \begin{align*}
        |f_n(x)|=\left| \sum_{k=1}^n \alpha_k\xi_k \right|&\leqslant \left( \sum_{k=1}^n|\alpha_k|^q \right)^\frac{1}{q}\cdot \left( \sum_{k=1}^n|\xi_k|^p \right)^\frac{1}{p}\\
        &\leqslant \left( \sum_{k=1}^n|\alpha_k|^q \right)^\frac{1}{q}\cdot ||x||_p,\forall x\in \ell^p
    \end{align*}
    从而$f_n\in \mathcal{L}(\ell^p,\K)$,而且
    \begin{equation*}
        ||f_n||\leqslant \left( \sum_{k=1}^n|\alpha_k|^q \right)^\frac{1}{q}
    \end{equation*}
    于是由习题$2.3.7$可得,若令
    \begin{equation*}
        f:l^p\rightarrow\K,x=\{\xi_k\}\mapsto \sum_{k=1}^\infty \alpha_k\xi_k
    \end{equation*}
    则$f\in \mathcal{L}(\ell^p,\K)$,
    \begin{equation*}
        ||f||\leqslant \fun{liminf}{n\rightarrow\infty}||f_n||
        \leqslant \fun{liminf}{n\rightarrow\infty}\left( \sum_{k=1}^n|\alpha_k|^q \right)^\frac{1}{q}
    \end{equation*}

    另一方面,取
    \begin{equation*}
        x_k^{(n)}=\left\{ \begin{array}{ll}
            |\alpha_k|^{q-1}{\rm e}^{-{\rm i}\cdot {\rm arg}(\alpha_k)}&,1\leqslant k\leqslant n\\
            0&,k>n
        \end{array} \right.
    \end{equation*}
    则$x^{(n)}=(x_1^{(n)},\cdots,x_k^{(n)},\cdots)\in l^p$,且
    \begin{equation*}
        ||x^{(n)}||_p=\left( \sum_{k=1}^n|\alpha_k|^{(q-1)p} \right)^\frac{1}{p}=\left( \sum_{k=1}^n|\alpha_k|^q \right)^\frac{1}{p}
    \end{equation*}
    注意$\alpha_k=|\alpha_k|\e^{i\cdot{\rm arg}(\alpha_k)}$,
    \begin{align*}
        |f(x^{(n)})|&=\sum_{k=1}^n \alpha_k|\alpha_k|^{q-1}{\rm e}^{-{\rm i}\cdot {\rm arg}(\alpha_k)}\\
        &=\sum_{k=1}^n |\alpha_k|{\rm e}^{{\rm i}\cdot {\rm arg}(\alpha_k)}\cdot |\alpha_k|^{q-1}{\rm e}^{-{\rm i}{\rm arg}(\alpha_k)}\\
        &=\sum_{k=1}^n |\alpha_k|^q=\left( \sum_{k=1}^n|\alpha_k|^q \right)^{\frac{1}{p}+\frac{1}{q}}\\
        &=\left( \sum_{k=1}^n|\alpha_k|^q \right)^\frac{1}{q}\cdot ||x^{(n)}||_p
    \end{align*}
    于是
    \begin{equation*}
        ||f||\geqslant \frac{|f(x^{(n)})|}{||x^{(n)}||_p}
        =\left( \sum_{k=1}^n|\alpha_k|^q \right)^\frac{1}{q},\ \forall n\in \N
    \end{equation*}
    令$n\rightarrow\infty$则
    \begin{equation*}
        ||f||\geqslant \fun{limsup}{n\rightarrow \infty}\left( \sum_{k=1}^n|\alpha_k|^q \right)^\frac{1}{q}\tag{2}
    \end{equation*}
    结合$(1)(2)$式则可知等号成立,
    同时这也说明$||\{\alpha_k\}||_q<\infty\Rightarrow \{\alpha_k\}\in\ell^q$.
\end{solve}

\begin{exer}{\rm (2.3.9)}%done
    如果序列$\{\alpha_k\}$使得对于$\forall x=\{\xi_k\}\in\ell^1$,
    保证$\sum_{k=1}^\infty \alpha_k\xi_k$收敛,求证:$\{\alpha_k\}\in\ell^\infty$.

    若定义
    \begin{equation*}
        f:x\mapsto \sum_{k=1}^\infty \alpha_k\xi_k
    \end{equation*}
    作为$\ell^1$上的线性泛函,求证:
    \begin{equation*}
        ||f||=\fun{sup}{k\geqslant 1}|\alpha_k|
    \end{equation*}
\end{exer}
\begin{solve}
    思路和上一题基本相同。设
    \begin{equation*}
        f_n:l^1\rightarrow\K,x=\{\xi_k\}\mapsto \sum_{k=1}^n \alpha_k\xi_k
    \end{equation*}
    则
    \begin{equation*}
        |f_n(x)|=\left| \sum_{k=1}^n \alpha_k\xi_k \right|
        \leqslant \fun{max}{1\leqslant k\leqslant n}|\alpha_k|\cdot \sum_{k=1}^n |\xi_k|
        \leqslant \fun{max}{1\leqslant k\leqslant n}|\alpha_k|\cdot ||x||_1
    \end{equation*}
    所以$f_n$有界,且
    \begin{equation*}
        ||f_n||\leqslant \fun{max}{1\leqslant k\leqslant n}|\alpha_k|
    \end{equation*}
    于是由习题$2.3.7$可得,若令
    \begin{equation*}
        f:l^1\rightarrow\K,x=\{\xi_k\}\mapsto \sum_{k=1}^\infty \alpha_k\xi_k
    \end{equation*}
    则$f$有界,而且
    \begin{equation*}
        ||f||\leqslant\fun{liminf}{n\rightarrow\infty}||f_n||
        \leqslant \fun{max}{1\leqslant k\leqslant n}|\alpha_k|\leqslant \fun{sup}{k\geqslant 1}|\alpha_k|\tag{1}
    \end{equation*}

    另一方面,取$x^{(n)}=(0,0,\cdots,0,\mathop{1}\limits_{n},0,0,\cdots)$,则
    $x^{(n)}\in l^\infty$,$||x^{(n)}||=1$,且
    \begin{equation*}
        ||f||\geqslant |f(x^{(n)})|=\alpha_n
    \end{equation*}
    于是$||f||\geqslant \fun{sup}{n\geqslant 1}|\alpha_n|$,结合$(1)$式可得$||f||=\fun{sup}{k\geqslant 1}|\alpha_k|$.
\end{solve}

\subsection{开映射定理}
\begin{exer}{\rm (2.3.1)}%done
    $X$是Banach空间,$X_0$是其闭子空间,定义映射:
    \begin{equation*}
        \varphi:X\rightarrow X\backslash X_0,x\mapsto [x]
    \end{equation*}
    $[x]$表示含$x$的商类。求证$\varphi$是开映射。
\end{exer}
\begin{solve}
    \begin{equation*}
        ||[x]||=\fun{inf}{y\in [x]}||y||\leqslant ||x||
    \end{equation*}
    所以$\varphi$有界。
    由定义可知$\varphi$是满射,则
    由{\rm OMT}可知$\varphi$是开映射。
\end{solve}

\begin{exer}{\rm (2.3.2)}%done
    设$X,Y$是Banach空间,方程$Ux=y$对于$\forall y\in Y$有解,
    其中$U\in\mathcal{L}(X,Y)$,并且存在$m>0$使得
    \begin{equation*}
        ||Ux||\geqslant m||x||,\ \forall x\in X
    \end{equation*}
    求证:$U$有连续逆$U^{-1}$,并且$||U^{-1}||\leqslant \frac{1}{m}$.
\end{exer}
\begin{solve}
    方程$Ux=y$对于$\forall y\in Y$有解,说明$U$是满射;
    设$Ux_1=Ux_2=y$,则
    \begin{equation*}
        m||x_1-x_2|| \leqslant 0=||U(x_1-x_2)|| \Rightarrow x_1-x_2=0
    \end{equation*}
    所以$U$是单射,进而是双射,则由{\rm OMT},
    $U^{-1}$存在、有界且连续,且
    \begin{equation*}
        ||U^{-1}y||=||U^{-1}Ux||=||x||\leqslant \frac{1}{m}||Ux||=\frac{1}{m}||y||\Rightarrow ||U^{-1}||\leqslant \frac{1}{m}
    \end{equation*}
\end{solve}

\begin{exer}{\rm (2.3.3)}%done
    设$H$是Hilbert空间,$A\in\mathcal{L}(H)$,并且$\exists m>0$使得
    \begin{equation*}
        |\ag{Ax,x}|\geqslant m||x||^2,\ \forall x\in H
    \end{equation*}
    求证:存在$A^{-1}\in\mathcal{H}$.
\end{exer}
\begin{solve}
    由题意可知
    \begin{align*}
        x\in H\Rightarrow& m||x||^2\leqslant |\ag{Ax,x}|\leqslant ||Ax||\cdot||x||\\
        \Rightarrow& m||x||\leqslant ||Ax||
    \end{align*}
    于是$A$是单射:
    \begin{equation*}
        0=||Ax_1-Ax_2||\geqslant m||x_1-x_2||\Rightarrow x_1=x_2
    \end{equation*}
    然后证明$A$是满射,从而$A^{-1}\in\mathcal{L}(H)$.
    \begin{enumerate}[$1^\circ$]
        \item ${\rm Ran}(A)=\overline{{\rm Ran}(A)}$:
            \begin{align*}
                y\in \overline{{\rm Im}(A)}\Rightarrow& \exists x_n\in H{\rm\ s.t.\ }Ax_n\rightarrow y\\
                \Rightarrow& m||x_n-x_m||\leqslant ||Ax_n-Ax_m||\rightarrow 0{\rm\ as\ }n,m\rightarrow\infty\\
                \Rightarrow& \exists x\in H{\rm\ s.t.\ }x_n\rightarrow x,Ax_n\rightarrow Ax\\
                \Rightarrow& Ax=y
            \end{align*}
        \item ${\rm Ran}(A)^\perp=\{0\}$,从而${\rm Im}(A)=\overline{{\rm Im}(A)}=H$:
            \begin{align*}
                y\in {\rm Im}(A)^\perp\Rightarrow& \ag{y,Ax}=0,\forall x\in H\\
                \Rightarrow& 0=|\ag{y,Ay}|\geqslant m||y||^2\\
                \Rightarrow& y=0
            \end{align*}
    \end{enumerate}
    由变分引理,$1^\circ$表明$H={\rm Ran}(A)\oplus {\rm Ran}(A)^\perp$,
    所以$2^\circ\Rightarrow H={\rm Ran}(A)$.这个证明满射的思路在Lax-Milgram定理的证明中也用到了。
\end{solve}

\begin{exer}{\rm (2.3.5)}%done
    用等价范数定理证明:$(C[0,1],||\cdot||_1)$
    不是Banach空间,其中
    \begin{equation*}
        ||f||_1=\int_0^1 |f(t)|\d t,\forall f\in C[0,1]
    \end{equation*}
\end{exer}
\begin{solve}
    (反证)假设$(C[0,1],||\cdot||_1)$是{\rm Banach}空间,则
    \begin{equation*}
        ||f||_1=\int_0^1 |f(t)|\d t\leqslant \fun{max}{t\in [0,1]}|f(t)|
    \end{equation*}
    由等价范数定理可知,
    \begin{equation*}
        \exists C>0{\rm\ s.t.\ }||f||_1\geqslant C \fun{max}{t\in [0,1]}|f(t)|\ ,\forall f\in C[0,1] \tag{$*$}
    \end{equation*}
    若令
    \begin{equation*}
        f_n(x)=\left\{ \begin{array}{ll}
            2n(1-nx)&,x\in [0,\frac{1}{n}]\\
            0&,x\in [\frac{1}{n},1]
        \end{array} \right.
    \end{equation*}
    则$||f_n||_1=1$,$\fun{max}{t\in [0,1]}|f_n(t)|=2n$,
    当$n$充分大时与$(*)$式矛盾。
\end{solve}

\begin{exer}{\rm (2.3.11)}
    设$X,Y$是Banach空间,$A\in\mathcal{L}(X,Y)$是满射,求证:
    如果在$Y$中$y_n\rightarrow y_0$,则存在$C>0$和$x_n\rightarrow x_0$使得$Ax_n=y_n$,
    且$||x_n||\leqslant C||y_n||$.
\end{exer}
\begin{solve}
    记$N=\{x\in X:Ax=0\}$,则$N$是$X$的闭线性子空间,且商空间$X/N$是{\rm Banach}空间。定义映射:
    \begin{equation*}
        \tilde{A}:X/N\rightarrow Y,\ [x]\mapsto Ax
    \end{equation*}
    则
    \begin{enumerate}[(1).]
        \item $\tilde{A}$是单射:$\tilde{A}[x]=0\Rightarrow Ax=0\Rightarrow x\in N\Rightarrow [x]=0$.
        \item $\tilde{A}$是满射:$y\in Y\Rightarrow \exists x\in X,Ax=y\Rightarrow \tilde{A}[x]=Ax=y$.
        \item $\tilde{A}$是有界线性映射:
            $A$是线性映射易知$\tilde{A}$也线性.$\forall x'\in [x]$,
            \begin{equation*}
                ||\tilde{A}[x]||=||Ax'||\leqslant ||A||\cdot ||x'||
            \end{equation*}
            取$x'\in [x]$满足$||x'||\leqslant 2|| [x]||$,则
            \begin{equation*}
                ||\tilde{A}[x]||\leqslant 2||A|| \cdot ||[x]||
            \end{equation*}
            所以$\tilde{A}$有界。
    \end{enumerate}
    由{\rm IMT}可知$\tilde{A}^{-1}$存在且有界,
    不妨假设$y_0=0,y_n\rightarrow 0$,记$[x_n]=\tilde{A}^{-1}y_n$,则
    \begin{equation*}
        ||[x_n]||=||\tilde{A}^{-1}y_n||\leqslant ||\tilde{A}^{-1}y_n||\leqslant ||\tilde{A}^{-1}||\cdot ||y_n||
    \end{equation*}
    于是取$x_n\in [x_n]$,使得$||x_n||\leqslant 2||[x_n]||$,
    便有$||x_n||\leqslant C||y_n||$,其中$C=2||\tilde{A}^{-1}||$.
\end{solve}

\subsection{闭图像定理}
\begin{exer}{\rm (2.3.4)}
    设$X,Y$是赋范线性空间,$D$是$X$
    的线性子空间,并且$A:D\rightarrow Y$是线性映射,求证:
    \begin{enumerate}[(1).]
        \item $A$连续且$D$闭$\Rightarrow A$闭;
        \item $A$连续且闭,则$Y$完备$\Rightarrow D$闭;
        \item $A$单且闭$\Rightarrow A^{-1}$闭;
        \item $X$完备,$A$单且闭,${\rm Ran}(A)$在$Y$中稠密,并且$A^{-1}$连续,那么${\rm Ran}(A)=Y$.
    \end{enumerate}
\end{exer}
\begin{solve}
    \begin{enumerate}[(1).]
        \item 设$\{x_n\}\subset D$收敛到$x$,$Ax_n\rightarrow y$,则
            $D$闭$\Rightarrow x\in D$,$A$连续$\Rightarrow Ax_n\rightarrow Ax$,所以$Ax=y$,$A$是闭算子。
        \item 如果$Y$完备,设$\{x_n\}\subset D$收敛到$x$,
            \begin{align*}
                \Rightarrow& ||Ax_n-Ax||\leqslant ||A||\cdot ||x_n-x||\rightarrow0{\rm\ as\ }n,m\rightarrow\infty\\
                \Rightarrow& \exists y\in Y{\rm\ s.t.\ }Ax_n\rightarrow y\\
                \Rightarrow& x\in D,Ax=y
            \end{align*}
            则$D$闭。
        \item $A$单射,所以记${\rm Range}(A)={\rm Dom}(A^{-1})=G\subset Y$,
            设$\{y_n=Ax_n\}$是$G$中的收敛列,且$\{x_n\}$是$D$中的收敛列,
            因为$A$是闭算子,可知$x_n\rightarrow x\in D$,$y_n\rightarrow y=Ax\in G$,
            也就是$y_n\rightarrow y\in G$,$x_n\rightarrow A^{-1}y\in D$.所以$A^{-1}$也是闭算子。
        \item $X$完备,$A$单且闭,则${\rm Ran}(A)$在$Y$中稠密,
            $A^{-1}$连续,则${\rm Ran}(A)=Y$,
            $A^{-1}$连续且闭,${\rm Ran}(A)$闭,所以${\rm Ran}(A)=\overline{{\rm Ran}(A)}=Y$.
    \end{enumerate}
\end{solve}

\begin{exer}{\rm (2.3.12)}
    设$X,Y$是Banach空间,$T$是闭线性算子,$D(T)\subset X$,
    $R(T)\subset Y$,$N(T)\defeq \{x\in X|Tx=0\}$.求证:
    \begin{enumerate}[(1).]
        \item $N(T)$是$X$的闭线性子空间。
        \item $N(T)=\{0\},R(T)$在$Y$中闭的充要条件是:
            \begin{equation*}
                \exists \alpha>0{\rm\ s.t.\ }
                ||x||\leqslant \alpha||Tx||,\ \forall x\in D(T)
            \end{equation*}
        \item 点$x\in X$到集合$N(T)$的距离:
            \begin{equation*}
                d(x,N(T))\defeq \fun{inf}{z\in N(T)}||z-x||
            \end{equation*}
            则$R(T)$在$Y$中闭的充要条件是:
            \begin{equation*}
                \exists \alpha>0{\rm\ s.t.\ }
                d(x,N(T))\leqslant \alpha||Tx||,\ \forall x\in D(T)
            \end{equation*}
    \end{enumerate}
\end{exer}
\begin{solve}
    \begin{enumerate}[(1).]
        \item 显然$N(T)$是线性子空间;设$N(T)\ni x_n\rightarrow x_0$,$\{Tx_n=0\}$是收敛列,
        由$T$是闭算子可知$x_0\in D(T)$且$Tx_0=0\Rightarrow x_0\in N(T)\Rightarrow N(T)$闭。
        \item 必要性:设$N(T)=\{0\}$,$R(T)$闭,则$R(T)$是{\rm Banach}空间,
            由{\rm IMT}可得$T^{-1}$存在且有界,于是
            \begin{equation*}
                \exists \alpha>0,||x||\leqslant \alpha||Tx||,\forall x\in D(T)
            \end{equation*}
            充分性:设对于某个$\alpha>0$,$||x||\leqslant \alpha||Tx||,\forall x\in D(T)$,则
            \begin{enumerate}[$1^\circ$]
                \item $N(T)=\{0\}$:$Tx=0\Rightarrow ||x||\leqslant 0\Rightarrow x=0$.
                \item $R(T)$闭:
                    \begin{align*}
                        Tx_n\rightarrow y\Rightarrow & ||x_n-x_m||\leqslant \alpha||Tx_n-Tx_m||\rightarrow 0{\rm\ as\ }n,m\rightarrow\infty\\
                        \Rightarrow& \exists x_0\in X{\rm\ s.t.\ }x_n\rightarrow x_0\\
                        \Rightarrow& x_0\in D(T),Tx_0=y_0
                    \end{align*}
            \end{enumerate}
        \item 注意到$X/N(T)$是{\rm Banach}空间,且$d(x,N(T))=||[x]||,\forall x\in X$.
        定义:
            \begin{equation*}
                \tilde{T}:X/N(T)\rightarrow Y,\ \tilde{T}[x]=Tx
            \end{equation*}
            则
            \begin{equation*}
                D(\tilde{T})=\{ [x]\in X/N(T):x\in D(T) \}
            \end{equation*}
            可见$N(\tilde{T})=[0]$,$R(\tilde{T})=R(T)$,于是只需证明
            $\tilde{T}$是闭算子。设$[x_n-x_0]\rightarrow 0$,$Tx_n-y_0\rightarrow 0$,
            则$\exists x_n^{(n)}\in [x_n],x_0^{(n)}\in [x_0]$使得
            \begin{equation*}
                ||x_n^{(n)}-x_0^{(n)}||\leqslant 2||[x_n-x_0]||\rightarrow 0{\rm\ as\ }n\rightarrow\infty
            \end{equation*}
            设
            \begin{equation*}
                \tilde{x}_n^{(n)}=x_n^{(n)}-(x_0^{(n)}-x_0)
            \end{equation*}
            则:
            \begin{enumerate}[$1^\circ$]
                \item $||\tilde{x}_n^{(n)}-x_0||=||x_n^{(n)}-x_0^{(n)}||\rightarrow 0$,$n\rightarrow\infty$.
                \item $T \tilde{x}_n^{(n)}=Tx_n^{(n)}-Tx_0^{(n)}+Tx_0=Tx_n\rightarrow y_0$,$n\rightarrow\infty$.
            \end{enumerate}
            由$T$是闭算子可知:$x_0\in D(T),Tx_0=y_0$,从而
            $[x_0]\in D(\tilde{T}),\tilde{T}[x_0]=Tx_0=y_0$.
    \end{enumerate}
\end{solve}
%------------------------------------------------------------------------------------------------------------------------------------------------------------------------------------------------------------------------------------------------------------------------------------------------

\subsection{Hahn-Banach定理}
\begin{exer}{\rm (2.4.3)}%done
    设$X$是复线性空间,$p$是$X$上的半范数,
    任取$x_0\in X$满足$p(x_0)\neq 0$,求证:
    存在$X$上的线性泛函$f$满足:
    \begin{enumerate}
        \item $f(x_0)=1$.
        \item $|f(x)|\leqslant p(x)/p(x_0),\forall x\in X$.
    \end{enumerate}
\end{exer}
\begin{solve}
    $\tilde{p}(x)\defeq p(x)/p(x_0)$仍然是一个半范数,
    取$X_0={\rm span}\{x_0\}=\{ \alpha x_0:\alpha\in\C \}$,定义
    $X_0$上的映射
    \begin{equation*}
        f_0: X_0\rightarrow \C, \alpha x_0\mapsto \alpha
    \end{equation*}
    则$f$是线性映射,并且
    \begin{equation*}
        |f_0(\alpha x_0)|=|\alpha|=\frac{p(\alpha x_0)}{p(x_0)}
        =\tilde{p}(x)
    \end{equation*}
    于是由复HBT(定理2.7.2),存在$X$上的线性泛函$f$,使得
    \begin{equation*}
        f|_{X_0}=f_0,\ |f(x)|\leqslant \tilde{p}(x)=\frac{p(x)}{p(x_0)},\ \forall x\in X
    \end{equation*}
    进而$f(x_0)=f_0(x_0)=1$.
\end{solve}

\begin{exer}{\rm (2.4.5)}%done
    $X$是赋范线性空间,$X_0$是其闭子空间,求证:
    \begin{equation*}
        \rho(x,X_0)={\rm sup}\{ |f(x)|:f\in X^*,||f||=1,f(X_0)=0 \}
    \end{equation*}
    其中$\rho(x,X_0)\defeq \fun{inf}{y\in X_0}||x-y||$.
\end{exer}
\begin{solve}
    当$x\in X_0$时等式两边均为零,成立,下设$x\notin X_0$,
    记$F=\{ f\in X^*:
    ||f||=1,f(X_0)=0 \}$.
    
    一方面,
    $\forall f\in F$,$\forall y\in X_0$,
    \begin{equation*}
        |f(x)|=| f(x-y) |\leqslant 
        ||f||\cdot ||x-y||=||x-y||
    \end{equation*}
    所以
    \begin{equation*}
        \fun{sup}{f\in F}|f(x)|\leqslant \fun{inf}{y\in X_0}||x-y||
    \end{equation*}

    另一方面,$X_0$闭$\Rightarrow \rho(x,X_0)>0$,
    由定理2.7.4可得
    \begin{equation*}
        \exists \tilde{f}\in F{\rm\ s.t.\ }
        \tilde{f}(x)=\rho(x,X_0),
        \tilde{f}(X_0)=0.
    \end{equation*}
    于是$|\tilde{f}(x)|=\rho(x,X_0)$,
    因此
    \begin{equation*}
        \rho(x,X_0)\leqslant 
        \fun{sup}{f\in F}|f(x)|
    \end{equation*}
    题目得证。
\end{solve}

\begin{exer}{\rm (2.4.7)}%done
    $X$是赋范线性空间,给定$X$中$n$个线性无关的元素
    $x_1,\cdots,x_n$,求证:存在$f_1,\cdots,f_n\in X^*$使得
    \begin{equation*}
        \ag{f_i,x_j}=\delta_{ij},\ \forall i,j\in\{1,2,\cdots,n\}
    \end{equation*}
\end{exer}
\begin{solve}
    考虑$X_0={\rm span}\{x_j\}_{j=1}^n$及其上的$n$个线性泛函:
    \begin{equation*}
        \tilde{f}_i:X_0\rightarrow \K,\sum_{j=1}^n \alpha_j x_j \mapsto \alpha_i,\ i=1,2,\cdots,n
    \end{equation*}
    则由有限维线性赋范空间中的范数等价,有
    \begin{equation*}
        \left|\tilde{f}_i\left( \sum_{j=1}^n \alpha_j x_j \right) \right|=\alpha_i\leqslant \fun{max}{1\leqslant j\leqslant n}|\alpha_j|\leqslant C\ms{ \sum_{j=1}^n \alpha_j x_j }
    \end{equation*}
    因此$\tilde{f}_i$都是$X_0$上的有界线性泛函,所以由HBT(定理2.7.3),
    \begin{equation*}
        \exists f_i\in X^*{\rm\ s.t.\ }\ag{ f_i,x_j }=\ag{ \tilde{f}_i,x_j }=\delta_{ij},\ i,j=1,2,\cdots,n
    \end{equation*}
\end{solve}

\begin{exer}{\rm (1.5.1)}%done
    $X$是赋范线性空间,$E$是以$0$为内点的真凸子集,
    $P$是$E$产生的{\rm Minkowski}泛函,求证:
    \begin{enumerate}[(1).]
        \item $x\in \mathring{E}\Leftrightarrow P(x)<1$.
        \item $\overline{\mathring{E}}=\overline{E}$.
    \end{enumerate}
\end{exer}
\begin{solve}
\begin{enumerate}[(1).]
        \item 一方面,
        $x\in \mathring{E}\Rightarrow \exists 
        \varepsilon>0{\rm\ s.t.\ }$
        \begin{equation*}
            \frac{x}{1/(1+\varepsilon)}=(1+\varepsilon) x\in E
        \end{equation*}
        于是
        \begin{equation*}
            P(x)\leqslant \frac{1}{1+\varepsilon}<1
        \end{equation*}
    
        另一方面,若$P(x)<1$,设$r\in (1,1/P(x))$,则有$rx\in E$.
        因为$0$是$E$的内点,存在$B(0,\delta)\subset E$,令
        \begin{equation*}
            d=\delta(1-\frac{1}{r})
        \end{equation*}
        对于任一$y\in B(x,d)$,
        \begin{equation*}
            y=\frac{1}{r}\cdot rx+(1-\frac{1}{r})\frac{r(y-x)}{r-1}
        \end{equation*}
        注意到$rx\in E,\frac{r(y-x)}{r-1}\in B(0,\delta)\subset E\Rightarrow y\in E\Rightarrow 
        B(x,d)\subset E\Rightarrow x$是内点。
        \item 即证明$\forall x\in E$,都存在$\mathring{E}$中的点列$x_n$使得$x_n\rightarrow x$,
            取$x_n=(1-\frac{1}{n})x$即可。
\end{enumerate}
\end{solve}

\begin{exer}{\rm (2.4.9)}%done
    $X$是复线性空间,$E$是$X$上的非空均衡集,
    $f$是$X$上的线性泛函,求证:
    \begin{equation*}
        |f(x)|\leqslant \fun{sup}{y\in E}{\rm Re}f(y),\ \forall x\in E
    \end{equation*}
\end{exer}
\begin{solve}
    $\forall x\in E$,
    \begin{equation*}
        |f(x)|=f(x)\e^{-\i\cdot{\rm arg} f(x)}=f(\e^{-\i\cdot{\rm arg} f(x)} x)={\rm Re}f(\e^{-\i\cdot{\rm arg} f(x)} x)
        \leqslant \fun{sup}{y\in E} {\rm Re}f(y)
    \end{equation*}
\end{solve}

\begin{exer}{\rm (2.4.11)}%done
    $E,F$是实赋范线性空间$X$中的两个互不相交的非空凸集,
    且$E$是开的和均衡的,
    求证:$\exists f\in X^*$使得:
    \begin{equation*}
        |f(x)|<\fun{inf}{y\in F} |f(y)|,\forall x\in E
    \end{equation*}
    笔者注:“均衡的”应该改成“对称的”,
        因为前面说了$X$是实的。
\end{exer}
\begin{solve}
    由凸集分离定理(定理2.7.7),存在非零$f\in X^*$,使得
    \begin{equation*}
        \fun{sup}{z\in E}f(z)\leqslant \fun{inf}{y\in F}f(y)\leqslant \fun{inf}{y\in F}|f(y)|
    \end{equation*}
    
    引理:$\forall x\in E$有$f(z)<\fun{sup}{z\in E}f(z)$.因此
    \begin{equation*}
        |f(x)|={\rm sgn}(f(x)) f(x)=f(x\cdot {\rm sgn}(f(x)))
        <\fun{sup}{z\in E}f(z)\leqslant \fun{inf}{y\in F}|f(y)|
    \end{equation*}

    引理的证明:(反证)假设存在$z_0\in E$使得$f(z_0)=\fun{sup}{z\in E}f(z)$,
    $E$开所以存在$B(z_0,\delta)\subset E$,于是
    \begin{equation*}
        \forall z\in B(z_0,\delta),f(z_0)\geqslant f(z)\Rightarrow f(z_0)-f(z)=f(z_0-z)\geqslant 0
    \end{equation*}
    即
    \begin{equation*}
        \forall y\in B(0,\delta),f(y)\geqslant 0
    \end{equation*}
    因此也有$f(-y)=-f(y)\geqslant 0$,
    进而$f(y)\equiv 0 {\rm\ on\ }B(0,\delta)$,
    于是$f$恒为零,这与$f$非零矛盾。
\end{solve}

\begin{exer}{\rm (2.4.13)}%done
    $X$是赋范线性空间,设$M$是
    $X$上的闭凸集,求证:$\forall x\in X\backslash M$,
    一定存在$f_1\in X^*{\rm\ with\ }||f_1||=1$,且
    \begin{equation*}
        \fun{sup}{y\in M}f_1(y)\leqslant f_1(x)-d(x)
    \end{equation*}
    其中$d(x)=\fun{inf}{z\in M}||x-z||$.
\end{exer}
\begin{solve}
    注意到开凸集$B(x,d(x))\cap M=\varnothing$,
    由凸集分离定理可得存在非零$f\in X^*$使得
    \begin{equation*}
        \fun{sup}{y\in M}f(y)\leqslant \fun{inf}{z\in B(x,d(x))}f(z)
        =\fun{inf}{||w||<1}f(x-d(x)w)
        =f(x)-d(x)\cdot \fun{sup}{||w||< 1}f(w)
        =f(x)-d(x)\cdot ||f||
    \end{equation*}
    (这里最后一个等号见习题2.1.2(2).)于是取$f_1=f/||f||$,则
    \begin{equation*}
        \fun{sup}{y\in M}f_1(y)\leqslant f_1(x)-d(x)
    \end{equation*}
\end{solve}

\subsection{对偶空间、自反空间、弱收敛}
\begin{exer}{\rm (2.5.1)}%done
    证明:
    \begin{equation*}
        (\ell^p)^*=\ell^q,\ 1\leqslant p<\infty,\frac{1}{p}+\frac{1}{q}=1
    \end{equation*}
\end{exer}
\begin{solve}
    对于$\forall x=\{ \xi_k \}_{k=1}^\infty\in \ell^p,y=\{ \eta_k \}_{k=1}^\infty\in \ell^q$,
    由Holder不等式:
    \begin{equation*}
        \left| \sum_{k=1}^\infty \xi_k\eta_k \right|\leqslant 
        \left( \sum_{k=1}^\infty |\xi_k|^p \right)^\frac{1}{p}\cdot
        \left( \sum_{k=1}^\infty |\eta_k|^q \right)^\frac{1}{q}
        \leqslant ||x||_p\cdot ||y||_q
    \end{equation*}
    定义
    \begin{equation*}
        T_y:l^p\rightarrow \K,\ x=\{ \xi_k \}_{k=1}^\infty\mapsto \sum_{k=1}^\infty \xi_k\eta_k
    \end{equation*}
    则$T_y$有界且$||T_y||\leqslant ||y||_q$,下面证明映射$\Lambda:y\mapsto T_y$是等距同构。

    对于$T\in (\ell^p)^*$,令$e_k=(0,\cdots,0,\mathop{1}\limits_{k},0,\cdots)$,则
    \begin{equation*}
        T(x)=T\left( \sum_{k=1}^\infty \xi_k e_k \right)
        =\sum_{k=1}^\infty \xi_k T(e_k),\ \forall x=\{ \xi_k \}_{k=1}^\infty\in l^p
    \end{equation*}
    取$y_T\defeq\{ T(e_k) \}_{k=1}^\infty$,
    则$\Lambda(y_T)=T$,下面
    证明$y_T\in \ell^p$且$||y_T||_{q}\leqslant ||T||$,
    从而映射$\Lambda $是满射且保距,进而是等距同构。

    若$p>1$,设$a_k=|T(e_k)|^{q-1}\e^{-\i \cdot{\rm arg}T(e_k)}$,
    $z_n=(a_1,a_2,\cdots,a_n,0,0,\cdots)\in\ell^p$,于是
    \begin{align*}
        \sum_{k=1}^n |T(e_k)|^q=
        \sum_{k=1}^n a_k \cdot T(e_k)
        &=T(z_n)\\
        &\leqslant ||T||\cdot ||z_n||_p\\
        &=||T|| \cdot \left( \sum_{k=1}^n |a_k|^p \right)^{\frac{1}{p}}\\
        &=||T|| \cdot \left( \sum_{k=1}^n |T(e_k)|^{p(q-1)} \right)^{\frac{1}{p}}\\
        &=||T|| \cdot \left( \sum_{k=1}^n |T(e_k)|^{q} \right)^{\frac{1}{p}}
    \end{align*}
    从而
    \begin{equation*}
        \left(\sum_{k=1}^n |T(e_k)|^q\right)^{1-\frac{1}{p}}
        =\left(\sum_{k=1}^n |T(e_k)|^q\right)^{\frac{1}{q}}
        \leqslant ||T||,\ \forall n
    \end{equation*}
    令$n\rightarrow\infty$可得$||y_T||_q\leqslant ||T||$.

    若$p=1,q=\infty$,
    \begin{align*}
        ||y_T||_{\infty}&= \fun{sup}{n\geqslant 1}|T(e_n)|\\
        &=\fun{sup}{n\geqslant 1} T(e_n)\cdot \e^{ -\i\cdot{\rm arg}T(e_n) }\\
        &=\fun{sup}{n\geqslant 1} T\left( \left\{ \e^{-\i\cdot{\rm arg}T(e_n)}\cdot \delta_{kn} \right\}_{k=1}^\infty \right)
        \leqslant ||T||
    \end{align*}
    最后一个不等号来自
    \begin{equation*}
        ||T||=\fun{sup}{||x||=1}||Tx||
    \end{equation*}    
    所以题目得证。
\end{solve}

\begin{exer}{\rm (2.5.2)}%done
    设$C$是收敛数列全体,赋以范数:
    \begin{equation*}
        ||\cdot||:\{\xi_k\}\in C\mapsto \fun{sup}{k\geqslant 1}|\xi_k|
    \end{equation*}
    求证$C^*=\ell^1$.
\end{exer}
\begin{solve}
    记
    \begin{equation*}
        e_0=(1,1,\cdots),\ e_k=(0,\cdots,0,\mathop{1}\limits_{k},0,\cdots)\in C
    \end{equation*}
    对于$x=\{x_n\}\in C$,$x_n\rightarrow x_0$,设$a={a_n}\in \ell^1$,定义映射
    \begin{equation*}
        T_a: C\rightarrow\C,x\mapsto a_1x_0+\sum_{n=1}^\infty a_{n+1}x_n
    \end{equation*}
    右侧级数是收敛的,因为
    \begin{equation*}
        \sum_{n=1}^\infty |a_{n+1}x_n|
        \leqslant \fun{sup}{n\geqslant 1}|x_n|\cdot 
        \sum_{n=1}^\infty |a_{n+1}|\leqslant ||x||\cdot ||a||_1<+\infty
    \end{equation*}
    又因为$x_0\leqslant \fun{sup}{n\geqslant 1}|x_n|$,
    \begin{equation*}
        |T_a(x)|\leqslant 
        |a_1x_0|+\fun{sup}{n\geqslant 1}|x_n|\cdot \sum_{n=2}^\infty |a_{n}|
        \leqslant\fun{sup}{n\geqslant 1}|x_n|\cdot \sum_{n=1}^\infty |a_{n}|
        \leqslant||x||\cdot ||a||_1
    \end{equation*}
    所以$||T_a||\leqslant ||a||_1$,同时,令
    $y_n=(\e^{-\i\cdot{\rm arg}(a_2)},\cdots,\e^{-\i\cdot{\rm arg}(a_n)},\e^{-\i\cdot{\rm arg}(a_1)},\e^{-\i\cdot{\rm arg}(a_1)},\cdots)$
    ,则
    \begin{equation*}
        \left|\sum_{k=1}^{n} |a_{k}|+
        \e^{-\i\cdot{\rm arg}(a_1)}\cdot\sum_{k=n+1}^\infty a_k 
        \right|=
        |T_a(y_n)|\leqslant ||T_a||\cdot ||y_n||
        =||T_a||
    \end{equation*}
    令$n\rightarrow\infty$就得到
    \begin{equation*}
        ||a||_1=\sum_{k=1}^\infty |a_k|\leqslant ||T_a||
    \end{equation*}
    ,从而$||T_a||=||a||_1$.
    那么定义映射$\Lambda:\ell^1\rightarrow C^*,a\mapsto T_a$,
    只需证明$\Lambda$是满射。

    对于任意$T\in C^*$,
    令$a_{k+1}=T(e_k)$,
    $z_n=(\e^{-\i\cdot{\rm arg}(a_2)},\cdots,\e^{-\i\cdot{\rm arg}(a_n)},0,0,\cdots )\in C$,于是
    \begin{equation*}
        \sum_{k=2}^n |a_k|
        =\sum_{k=2}^n a_k\cdot \e^{-\i\cdot{\rm arg}(a_k)}
        =T(z_n)\leqslant ||T||\cdot ||z_n||=||T||<+\infty,\forall n
    \end{equation*}
    这说明级数$\sum_{k=2}^\infty a_k$收敛,取
    \begin{equation*}
        a_1=T( e_0 )-\sum_{k=2}^\infty a_k
    \end{equation*}
    则令$a=\{a_n\}\in\ell^1$,且
    \begin{equation*}
        T_a(x)=a_1x_0
        +\sum_{k=2}^\infty a_{k}x_{k-1}
        =T( (x_0,x_0,\cdots) )+\sum_{k=2}^\infty 
        a_k(x_{k-1}-x_0)
        =T( (x_1,x_2,\cdots) )
        =T(x)
    \end{equation*}
    得证。
\end{solve}

\begin{exer}{\rm (2.5.4)}%done
    证明:有限维赋范空间一定自反。
\end{exer}
\begin{solve}
    设 $\{x_k\}_{k=1}^n$是$X$的一组基,则由习题2.4.7可知
    \begin{equation*}
        \exists\left\{f_{k}\right\}_{k=1}^{n}\subset X^{*}{\rm\ s.t.\ }\langle f_{i},x_{j}\rangle=\delta_{ij}\quad(i,j=1,2,\cdots,n)
    \end{equation*}
    于是对$\forall f\in X^*,\forall x=\sum_{k=1}^n \lambda_k x_k$,
    \begin{equation*}
        f(x)=f\left(\sum_{k=1}^{n}\lambda_{k}x_{k}\right)=\sum_{k=1}^{n}\lambda_{k}f(x_{k})=\sum_{k=1}^{n}f_{k}(x)f(x_{k})
    \end{equation*}
    此即说明,
    \begin{equation*}
        f=\sum_{k=1}^{n}f(x_{k})f_{k}
    \end{equation*}
    现对$\forall y\in X^{**}$,
    \begin{equation*}
        y(f)=y\left(\sum_{k=1}^{n}f(x_k)f_{k}\right)
        =\sum_{k=1}^{n}f(x_{k})y(f_{k})=
        f\left( \sum_{k=1}^n x_ky(f_k) \right)
    \end{equation*}
    于是$x_0\defeq \sum_{k=1}^n x_ky(f_k)\in X$使得$x_0^{**}=y$,
    这意味着从$x$到$x^{**}$的自然映射是满的,所以$X$是自反的.
\end{solve}

\begin{exer}{\rm (2.5.5)}%done
    证明:Banach空间自反当且仅当它的共轭空间(对偶空间)是自反的。
\end{exer}
\begin{solve}
    设$X$是Banach空间,从$X$到$X^{**}$的自然映射是$T$.

    必要性:
    即证明从$X^*$到$X^{***}$的自然映射是满的,
    考虑$y\in X^{***}$,取
    \begin{equation*}
        f:X\rightarrow \K,x\mapsto y(T(x))=y(x^{**})
    \end{equation*}
    于是$\forall x^{**}\in X^{**}$,
    \begin{equation*}
        y(x^{**})=f(x)=x^{**}(f)
    \end{equation*}
    这说明$f^{**}=y$,得证。

    充分性:设$X^*$自反,则由必要性知$X^{**}$自反,
    又因为$X$,
    作为$X^{**}$的子空间(因为自然映射是嵌入)是闭的(因为$X$是Banach空间),
    自反空间的闭子空间也自反(Pettis,定理2.8.11),
    从而$X$自反。
\end{solve}

\begin{exer}{\rm (2.5.6)}%done
    设$X$是赋范线性空间,$T$是从$X$到$X^{**}$的自然映射,求证:
    $R(T)$闭的充要条件是$X$完备。
\end{exer}
\begin{solve}
    $X^{**}$是完备的,
    $R(T)\cong X$作为$X^{**}$的子空间,
    闭$\Leftrightarrow $完备。
\end{solve}

\begin{exer}{\rm (2.5.8)}%done
    在$\ell^2$中定义算子:
    \begin{equation*}
        T:(x_1,x_2,\cdots,x_n,\cdot)\mapsto (x_1,\frac{x_2}{2},\cdots,\frac{x_n}{n},\cdots)
    \end{equation*}
    证明$T\in\mathcal{L}(\ell^2)$,并求$T^*$.
\end{exer}
\begin{solve}
    对$x=(x_1,x_2,\cdots,x_n,\cdots)\in \ell^2$,有
    \begin{equation*}
        \left\|Tx\right\|_{\ell^{2}}^{2}=\sum_{k=1}^{\infty}\left|\frac{x_{k}}{k}\right|^{2}\leqslant\sum_{k=1}^{\infty}\left|x_{k}\right|^{2}=\left\|x\right\|^{2}
    \end{equation*}
    从而$T\in\mathcal{L}(\ell^2)$.
    
    $\ell^2$是Hilbert空间,由Riesz表示定理,
    $\forall f\in (\ell^2)^*=\ell^2$,
    存在$y_f=\{f_n\}_{n=1}^\infty $使得
    \begin{equation*}
        f=\ag{\cdot,y_f}
    \end{equation*}
    $T^*$满足:
    \begin{equation*}
        T^*f(x)=f(T(x))
        =\ag{T(x),y_f}=\sum_{n=1}^\infty \frac{f_nx_n}{n}
        \Rightarrow 
        T^*f=\ag{\cdot,\left\{\frac{f_n}{n}\right\}_{n=1}^\infty}
    \end{equation*}
    所以$T^*:\{f_n\}_{n=1}^\infty\mapsto \{f_n/n\}_{n=1}^\infty$,
    也就是$T^*=T$.
\end{solve}

\begin{exer}{\rm (2.5.9)}%done
    $H$是Hilbert空间,$A\in\mathcal{L}(H)$满足
    \begin{equation*}
        \ag{Ax,y}=\ag{x,Ay},\ \forall x,y\in H
    \end{equation*}
    求证:
    \begin{enumerate}[(1).]
        \item $A^*=A$.
        \item $R(A)$在$H$中稠密$\Rightarrow $方程$Ax=y$对于$\forall y\in R(A)$
            存在唯一解。
    \end{enumerate}
\end{exer}
\begin{solve}
    \begin{enumerate}[(1).]
        \item 对于$f\in H^*$,由Riesz表示定理,存在$y_f$使得$f=\ag{\cdot,y_f}$,
            并且$||f||=||y_f||$,因此$f\mapsto y_f$就是$H^*$到
            $H$的等距同构。因此:$\forall f\in H^*$,$\forall x\in H$,
            \begin{equation*}
                A^*f(x)=f(A(x))\Rightarrow 
                \ag{x,y_{A^*f}}
                =\ag{Ax,y_f}=\ag{x,Ay_f}\Rightarrow 
                Ay_f=y_{A^*f}
            \end{equation*}
            而$A^*:f\mapsto A^*f$对应了$y_f\mapsto y_{A^*f}$,
            所以$A^*y_f=y_{A^*f}=Ay_f\Rightarrow A^*=A$.
        \item 对$\forall y\in R(A)$,若
        \begin{equation*}
            Ax_1=y=Ax_2
        \end{equation*}   
        则$\forall z\in H$,
        \begin{equation*}
            0=\langle A(x_1-x_2),z\rangle=\langle x_1-x_2,Az\rangle
        \end{equation*}        
        由于 $R(A)$ 在 $H$ 中稠密,
        \begin{equation*}
            \exists z_n\in H{\rm\ s.t.\ }Az_n\rightarrow x_1- x_2
        \end{equation*}
        从而
        \begin{equation*}
            ||x_{1}-x_{2}||^{2}=\fun{lim}{n\rightarrow\infty}
            \left\langle x_{1}-x_{2},Az_{n}\right\rangle=0\Rightarrow x_1=x_2
        \end{equation*}
    \end{enumerate}
\end{solve}

\begin{exer}{\rm (2.5.13)}%done
    设$\{x_n\}\subset C[a,b],x\in C[a,b]$,且
    $x_n\wto x$,求证:
    \begin{equation*}
        \fun{lim}{n\rightarrow\infty} x_n(t)=x(t),\ \forall t\in [a,b]
    \end{equation*}
\end{exer}
\begin{solve}
    对于$t\in C[a,b]$,取赋值映射$f_t:x\mapsto x(t)$,
    \begin{equation*}
        \frac{|f_t(x)|}{||x||}=\frac{|x(t)|}{\fun{sup}{t\in [a,b]}
        |x(t)|}\leqslant 1 
    \end{equation*}
    则$f_t\in C[a,b]^*$,
    \begin{equation*}
        x_n\wto x\Rightarrow \fun{lim}{n\rightarrow\infty} f_t(x_n)
        =f_t(x)
    \end{equation*}
    即
    \begin{equation*}
        \fun{lim}{n\rightarrow\infty} x_n(t)=x(t)
    \end{equation*}
\end{solve}

\begin{exer}{\rm (2.5.14)}%done
    已知在赋范线性空间中$x_n\wto x_0$,求证:
    \begin{equation*}
        \fun{liminf}{n\rightarrow\infty}||x_n||\geqslant ||x_0||
    \end{equation*}
\end{exer}
\begin{solve}
    左式非负,所以$x_0=0$时成立。下设$x_0\neq 0$,
    则存在$f\in X^*{\rm\ s.t.\ }||f||=1{\rm\ s.t.\ }f(x_0)=||x_0||$,从而
    \begin{equation*}
        ||x_0||=\fun{lim}{n\rightarrow\infty}
        |f(x_n)|\leqslant 
        \fun{liminf}{n\rightarrow\infty} ||f||\cdot ||x_n||
        =\fun{liminf}{n\rightarrow\infty} ||x_n||
    \end{equation*}
\end{solve}

\begin{exer}{\rm (2.5.15)}%done
    $H$是Hilbert空间,$\{e_n\}$是$H$的正交规范基,
    求证:在$H$中$x_n\wto x_0$的充要条件为
    \begin{enumerate}[$1^\circ$]
        \item $||x_n||$有界。
        \item $\ag{x_n,e_k}\rightarrow \ag{x_0,e_k},\forall k$.
    \end{enumerate}
\end{exer}
\begin{solve}
    充分性:设$f\in H^*$,由Riesz表示定理,
    存在
    \begin{equation*}
        y_k=\sum_{k=1}^\infty y_ke_k
    \end{equation*}
    使得$f=\ag{\cdot,y_f}$,设
    \begin{equation*}
        y_f^n=\sum_{k=1}^n y_ke_k
    \end{equation*}
    则
    \begin{equation*}
        |\ag{x_n-x_0,y_f^n}|
        \leqslant \sum_{k=1}^n |y_k|\cdot | \ag{x_n-x_0,e_k} |
        \rightarrow 0
    \end{equation*}
    因此
    \begin{align*}
        | f(x_n)-f(x_0) |
        =&| \ag{x_n-x_0,y_f} |\\
        \leqslant& | \ag{x_n-x_0,y_f^n} |+| \ag{x_n-x_0,y_f-y_f^n} |\\
        \leqslant& | \ag{x_n-x_0,y_f^n} |+||x_n-x_0||\cdot ||y_f-y_f^n||\\
        \leqslant& | \ag{x_n-x_0,y_f^n} |+2M||y_f-y_f^n||\rightarrow 0
    \end{align*}
    所以$x_n\wto x_0$.

    必要性:令$f=\ag{\cdot,e_k}\in H^*$,则$2^\circ$得证;
    对于$\forall n$,考虑$x_n^{**}\in H^{**}$,则$||x_n^{**}||=||x_n||$,
    收敛列必有界,所以
    \begin{equation*}
        \fun{sup}{n}|x_n^{**}(f)|=\fun{sup}{n}|f(x_n)|<+\infty 
    \end{equation*}
    由{\rm UBP}可得$x_n^{**}$一致有界,进而$1^\circ$得证。
\end{solve}

\begin{exer}{\rm (2.5.17)}%done
    $H$是Hilbert空间,在$H$中$x_n\wto x_0$,且
    $y_n\rightarrow y_0$,求证:
    \begin{equation*}
        \ag{x_n,y_n}\rightarrow \ag{x_0,y_0}
    \end{equation*}
\end{exer}
\begin{solve}
    \begin{align*}
        |\ag{x_n,y_n}-\ag{x_0,y_0}|
        &=| \ag{x_n,y_n}-\ag{x_0,y_n}+\ag{x_0,y_n}-\ag{x_0,y_0} |\\
        &=| \ag{x_n-x_0,y_n}+\ag{x_0,y_n-y_0} |\\
        &\leqslant | \ag{x_n-x_0,y_n}|+|\ag{x_0,y_n-y_0} |
    \end{align*}
    由$y_n\rightarrow y_0$知第二项$\rightarrow 0$;
    $\ag{\cdot,y_n}\in H^*$,所以$x_n\wto x_0\Rightarrow $第一项$\rightarrow 0$.
\end{solve}

\begin{exer}{\rm (2.5.18)}%done
    $H$是Hilbert空间,$\{e_n\}$是$H$的正交规范基,求证:在
    $H$上$e_n\wto 0$,但$e_n\nrightarrow 0$.
\end{exer}
\begin{solve}
    也就是证明$\forall x\in H,\ag{x,e_n}\rightarrow 0$,实际上
    \begin{equation*}
        \sum_{n=1}^\infty |\ag{x,e_n}|^2\leqslant ||x||
        <+\infty
    \end{equation*}
    所以$\ag{x,e_n}\rightarrow 0$.

    $||e_n||=1>0$,所以$e_n\nrightarrow 0$.
\end{solve}

\begin{exer}{\rm (2.5.20)}%done
    求证:在自反的赋范线性空间中,集合的弱列紧性与有界性是等价的。
\end{exer}
\begin{solve}
    弱列紧$\Rightarrow $有界:
    假设$M\subset X$弱列紧且无界,
    则$\forall n\geqslant 1,\exists x_n\in M{\rm\ s.t.\ }||x_n||\geqslant n$,
    取$\{x_n\}$的弱收敛子列$\{x_{n_k}\}$,$X$自反,所以取
    $x_{n_k}^{**}\in X^{**}$有$||x_{n_k}^{**}||=||x_{n_k}||$,而且
    \begin{equation*}
        \fun{sup}{n}|x_{n_k}^{**}(f)|=\fun{sup}{n}|f(x_{n_k})|<+\infty
        \mathop{\Rightarrow}\limits^{\rm UBP}
        \fun{sup}{n}||x_{n_k}||^{**}<+\infty 
        \Rightarrow \{x_{n_k}\}\mbox{有界}
    \end{equation*}
    这与$||x_{n_k}||\geqslant n_k\rightarrow\infty$矛盾。

    有界$\Rightarrow $弱列紧:定理2.8.18{\rm (Eberlein‑Smulian)}.
\end{solve}



\section{第三章}
\begin{exer}{\rm (2.6.1)}
    $X$是Banach空间,求证:$\mathcal{L}(X)$中可逆(存在有界逆)
    算子集是开的。
\end{exer}
\begin{solve}
    设$A\in\mathcal{L}(X)$可逆,
    对于$\forall B\in\mathcal{L}(X)$且$||A-B||\leqslant 1/||A^{-1}||$,
    则$||(A-B)A^{-1}||\leqslant 1$,由引理2.7.1可得
    $BA^{-1}=I-(A-B)A^{-1}$可逆,进而$B$可逆。
    这说明$A$是内点,从而$\mathcal{L}(X)$中可逆算子集是开集。
\end{solve}

\begin{exer}{\rm (2.6.4)}
    在$\ell^2$空间上,考察左推移算子:
    \begin{equation*}
        A:(\xi_1,\xi_2,\cdots)\mapsto (\xi_2,\xi_3,\cdots)
    \end{equation*}
    求证:$\sigma_p(A)=\{ \lambda\in\C:|\lambda|<1 \}$,
    $\sigma_c(A)=\{\lambda\in\C:|\lambda|=1\}$,并且
    \begin{equation*}
        \sigma(A)=\sigma_p(A)\cup\sigma_c(A)
    \end{equation*}
\end{exer}
\begin{solve}
    记$\D=\{ \lambda:|\lambda|<1 \}$,
    \begin{align*}
        \lambda\in \sigma_p(A)\Leftrightarrow& \exists 0\neq x=\{x_n\}_{n=1}^\infty\in \ell^2\\
        \Leftrightarrow&\lambda x_1=x_2,\lambda x_2=x_3,\cdots\\
        \Leftrightarrow&x=\{ \lambda^{n-1} x_1 \}_{n=1}^\infty
    \end{align*}
    $0\neq x\in \ell^2$,所以$|\lambda|<1$,$\sigma_p(A)=\{\lambda:|\lambda|<1\}=\D$.

    $\lambda\in \partial\D$时:
    \begin{enumerate}[$1^\circ$]
        \item ${\rm Ran}(\lambda I-A)\neq \ell^2$:
            \begin{align*}
                y\in {\rm Ran}(\lambda I-A)\Rightarrow& \exists 0\neq x \in\ell^2,y=(\lambda I-A)x=
                ( \lambda x_1-x_2,\lambda x_2-x_3,\cdots )=\{ \lambda x_n-x_{n+1} \}_{n=1}^\infty\\
                \Rightarrow &\sum_{n=1}^k \lambda^{-n}y_n=x_1-\lambda^{-k}x_{k+1}\rightarrow x_1{\rm\ as\ }k\rightarrow\infty
            \end{align*}
            所以
            \begin{equation*}
                \sum_{n=1}^\infty \lambda^{-n}y_n=x_1
            \end{equation*}
            但是取
            \begin{equation*}
                y=\{ \lambda^n\cdot \frac{1}{n} \}_{n=1}^\infty\in\ell^2
            \end{equation*}
            则级数发散,说明满足$y=(\lambda I-A)x$的$x$不存在。因此${\rm Ran}(\lambda I-A)\neq \ell^2$.
        \item $\overline{{\rm Ran}(\lambda I-A)}=\ell^2$:任取$x=\{x_n\}_{n=1}^\infty \in\ell^2$,
            取充分大的$N$使得
            \begin{equation*}
                \sum_{n=N+1}^\infty |x_n|^2<\varepsilon^2
            \end{equation*}
            令$y_j=(x_1,x_2,\cdots,x_j,0,\cdots)$,则
            \begin{equation*}
                ||y_N-x||^2=\sum_{n=N+1}^\infty |x_n|^2<\varepsilon^2\Rightarrow ||y_N-x||<\varepsilon
            \end{equation*}
            因此只需证$y_N\in {\rm Ran}(\lambda I-A)$,令
            \begin{equation*}
                z=\left\{ \begin{array}{ll}
                  \sum_{k\leqslant n\leqslant N} \lambda^{-n+k-1}y_n&,k\leqslant N\\
                  0&,k>N
                \end{array} \right.
            \end{equation*}
            即有$y_N=(\lambda I-A)z$.
    \end{enumerate}
    因此$\partial \D \subset \sigma_c(A)$.

    接下来求$\sigma(A)$,因为$||Ax||^2=||x||^2+|x_1|^2$,所以
    $||Ax||\leqslant ||x||$,等号在$x_1=0$成立,因此
    \begin{equation*}
        ||A||=\fun{sup}{0\neq x\in \ell^2}\frac{||Ax||}{||x||}=1
    \end{equation*}
    所以$\sigma(A)\subset \{ \lambda:|\lambda|\leqslant 1 \}=\overline{\D}$.

    综上可知,$\overline{\D}\supset \sigma(A)\supset \sigma_c(A)\cup \sigma_p(A)\supset \partial\D\cup \D=\overline{\D}$,
    所以都相等。
\end{solve}

\begin{exer}{\rm (3.1.2)}
    $X$是Banach空间,$A\in\mathcal{L}(X)$满足:
    \begin{equation*}
        ||Ax||\geqslant \alpha||x||,\ \forall x\in X
    \end{equation*}
    其中$\alpha>0$为常数,求证:
    $A\in\mathcal{T}(X)$,即$A$是紧算子的充要条件是$X$
    是有穷维的。
\end{exer}
\begin{solve}
    充分性:$X$有限维,所以有界集都列紧,
    $A$将有界集映为有界集,进而是列紧集,所以$A$紧。

    必要性:设$B\subset X$为有界集,$A$紧,
    所以$A(B)$有收敛列$\{ Ax_n \}_{n=1}^\infty ,\{x_n\}\subset B$.
    由于$||x_n||\leqslant \frac{1}{\alpha}||Ax_n||$,
    所以$\{x_n\}_{n=1}^\infty$也是收敛列,因此$B$列紧。
    这说明$X$上有界集都列紧,因此$X$有限维。
\end{solve}

\begin{exer}{\rm (3.1.4)}
    $H$是Hilbert空间,$A:H\rightarrow H$是紧算子,又设
    $x_n\wto x_0,y_n\wto y_0$,求证:
    \begin{equation*}
        \ag{x_n,Ay_n}\rightarrow \ag{x_0,Ay_0}
    \end{equation*}
\end{exer}
\begin{solve}
    注意到
    \begin{equation*}
        |\ag{x_n,Ay_n}-\ag{x_0,Ay_0}|
        \leqslant |\ag{x_n,Ay_n-Ay_0}|+|\ag{x_n,Ay_n-Ay_0}|
    \end{equation*}
    因为$x_n\wto x_0$,所以$\{x_n\}$是有界的,
    $A$紧$\Rightarrow A$全连续$\Rightarrow Ay_n\rightarrow Ay_0$
    $\Rightarrow|(x_n,Ay_n-Ay_0)|\leqslant |x_n|\cdot |Ay_n-Ay_0|\rightarrow 0$;
    $\ag{\cdot,Ay_0}\in H^*$,
    所以由$x_n$弱收敛可知$\ag{x_n-x_0,Ay_0}\rightarrow 0$,
    所以$|\ag{x_n,Ay_n}-\ag{x_0,Ay_0}|\rightarrow 0$.
\end{solve}

\begin{exer}{\rm (3.1.6)}
    设$\omega_n\in\K$,$\omega_n\rightarrow 0$,求证:
    \begin{equation*}
        T:\ell^p\rightarrow \ell^p,\{\xi_n\}\mapsto \{\omega_n\xi_n\}
    \end{equation*}
    是$\ell^p(p\geqslant 1)$上的紧算子。
\end{exer}
\begin{solve}
    $w_n$收敛则有界,从而$T$有界。定义:
    \begin{equation*}
        T_n:(x_1,\cdots,x_n,x_{n+1},\cdots)\mapsto (w_1x_1,\cdots,w_nx_n,0,0,\cdots)
    \end{equation*}
    那么$T_n$都是有界有限秩算子,因此$T_n$紧。
    对于$\forall\varepsilon>0$,存在充分大的$N$使得
    $\forall n>N$有$|w_n|<\varepsilon$,
    从而
    \begin{equation*}
        ||T_nx-Tx||\leqslant \varepsilon||x||
        \Rightarrow ||T_n-T|| \leqslant \varepsilon 
    \end{equation*}
    因此$||T_n-T||\rightarrow 0$,由于
    紧算子全体是闭集,$T$也是紧算子。
\end{solve}

\begin{exer}{\rm (3.1.8)}
    $\Omega\subset\R^n$是一个可测集,$K\in L^2(\Omega\times\Omega)$,
    求证:
    \begin{equation*}
        A:L^2(\Omega)\rightarrow L^2(\Omega),u(x)\mapsto \int_{\Omega} K(x,y)u(y)\d y
    \end{equation*}
    是$L^2(\Omega)$上的紧算子。
\end{exer}
\begin{solve}
    由C-S不等式,$L^2(\Omega)$上范数记为$||\cdot||$,
    \begin{equation*}
        ||Au||^2=\int_{\Omega}
        \left(\int_{\Omega}
        K(x,y)u(y) \d y\right)^2\d x\leqslant 
        \int_{\Omega}\left(
            \int_{\Omega}
            K(x,y)^2 \d y\cdot ||u||^2\right)\d x
    \end{equation*}
    \begin{equation*}
        \Rightarrow 
        \frac{||Au||^2}{||u||^2}
        \leqslant \int_{\Omega\times \Omega}K(x,y)^2\d y\d x<+\infty
    \end{equation*}
    所以$A$有界。

    取$L^2(\Omega)$上的可数正交基$\{u_i\}$,\footnote{这里能取出ONB是因为$L^2(\Omega)$是可分Hilbert空间。}
    设
    \begin{equation*}
        K(x,y)=\sum_{i=1}^\infty K_i(y)u_i(x)
    \end{equation*}
    其中
    \begin{equation*}
        K_i(y)=\int_\Omega K(x,y)u_i(x)
    \end{equation*}
    由Parseval,
    \begin{equation*}
        \int_\Omega |K(x,y)|^2\d x=\sum_{i=1}^\infty |K_i(y)|^2
    \end{equation*}
    因此
    \begin{equation*}
        \int_{\Omega\times\Omega}|K(x,y)|^2\d x\d y
        =\sum_{i=1}^\infty \int_\Omega |K_i(y)|^2\d y
    \end{equation*}
    定义
    \begin{equation*}
        A_n:u\mapsto \int_{\Omega} K_n(\cdot,y)f(y)\d y
    \end{equation*}
    其中
    \begin{equation*}
        K_n(x,y)=\sum_{i=1}^n K_i(y)u_i(x)
    \end{equation*}
    那么$A_N$是有界有限秩算子,因此紧。
    由C-S不等式,
    \begin{align*}
        ||A-A_n||^2\leqslant& \int_{\Omega\times\Omega}|K(x,y)-K_N(x,y)|^2\d x\d y\\
        =&\int_{\Omega\times\Omega} |K(x,y)|^2 \d x\d y
        -2\int_{\Omega\times\Omega} K(x,y)\sum_{i=1}^n K_i(y)u_i(x)\d x\d y
        +\sum_{i=1}^n \int_{\Omega} |K_i(y)|^2\d y\\
        =&\int_{\Omega\times\Omega}|K(x,y)|^2 \d x\d y-
        \int_\Omega |K_i(y)|^2\d y\rightarrow 0{\rm\ as\ }n\rightarrow \infty
    \end{align*}
    因此$A$是紧算子。
\end{solve}

\begin{exer}{\rm (3.2.5)}
    $A$是$X$上的紧算子,$T=I-A$,求证:$\forall k\in\N$,
    $N(T^k)$是有穷维的,$R(T^k)$是闭的。
\end{exer}
\begin{solve}
    注意到$T^k=(T-A)^k=I-A_k$,其中$A_k$是$A$的$k$次多项式,
    因此$A_k$是紧算子,由定理3.2.2(Riesz‑Fredholm)前两条可知得证。
\end{solve}

\begin{exer}{\rm (3.2.6)}
    $X$是Banach空间,
    $M$是其闭线性子空间,若有界线性算子$P:X\rightarrow M$满足
    $P^2=P$,则称为由$X$到$M$上的投影算子,求证:
    \begin{enumerate}[(1).]
        \item 若$M$是$X$的有穷维线性子空间,则必存在由$X$到$M$上的投影算子;
        \item 若$P$是由$X$到$M$上的投影算子,则$I-P$是由$X$到$R(I-P)$上的投影算子;
        \item 若$P$是由$X$到$M$上的投影算子,则$X=M\oplus R(I-P)$.
        \item 若$A$是$X$上的紧算子,$T=I-A$,则在代数与拓扑同构意义下:
            \begin{equation*}
                N(T)\oplus X/N(T)=X=R(T)\oplus X/R(T)
            \end{equation*}
    \end{enumerate}
\end{exer}
\begin{solve}
    \begin{enumerate}[(1).]
        \item 取$\{e_k\}_{k=1}^n$为$M$的正交基,由习题2.4.7可知,存在$f_1,\cdots,f_n\in X^*$使得
            \begin{equation*}
                f_i(e_j)=\delta_{ij}
            \end{equation*}
            于是取
            \begin{equation*}
                P:X\rightarrow M,x\mapsto \sum_{k=1}^n f_k(x)e_k
            \end{equation*}
            就是$X$到$M$的投影算子,因为$P$有界且$P^2=P$.
        \item 注意到$(I-P)^2=I-2P+P^2=I-P$.
        \item 显然有$X=M+R(I-P)$,因为
            \begin{equation*}
                x=Px+(I-P)x
            \end{equation*}
            所以只需证明$M\cap R(I-P)=0$,令$x\in M\cap R(I-P)$,
            $y=(I-P)x$,因此
            \begin{equation*}
                Px=P(I-P)y=(P-P)y=0
            \end{equation*}
            又因为$x\in M$,所以$Px=x=0$.
        \item $N(T)$有限维,由(1)知存在$X$到$N(T)$的投影算子$P$,
            由(3)知$X/N(T)$和$R(I-P)$代数同构。现令
            \begin{equation*}
                F:X/N(T)\rightarrow R(I-P),
                [x]\mapsto (I-P)x
            \end{equation*}
            不难验证$F$良定、线性、双射。对于$\forall [x]\in X/N(T)$,
            存在$x'\in [x]{\rm\ s.t.\ }||x'||\leqslant 2||[x]||$,于是
            \begin{equation*}
                ||F([x])||=||(I-P)x'||\leqslant ||I-P||\cdot ||x'||
                \leqslant 2||I-P||\cdot ||[x]||
            \end{equation*}
            因此$F$有界$\Rightarrow$连续$\Rightarrow F^{-1}$也连续
            $\Rightarrow $拓扑同胚。
    \end{enumerate}
\end{solve}

\begin{exer}{\rm (3.3.1)}
    给定数列$\{a_n\}_{n=1}^\infty$,在空间$\ell^1$
    上定义算子$A$如下:
    \begin{equation*}
        A(x_1,x_2,\cdots)=(a_1x_1,a_2x_2,\cdots)
    \end{equation*}
    求证:
    \begin{enumerate}[(1).]
        \item $A$有界的充要条件是$M=\fun{sup}{n\geqslant 1}|a_n|<\infty$.
        \item $A^{-1}$有界的充要条件是$\fun{inf}{n\geqslant 1}|a_n|>0$.
        \item $A$是紧算子的充要条件是$\fun{lim}{n\rightarrow\infty}a_n=0$.
    \end{enumerate}
\end{exer}
\begin{solve}
    \begin{enumerate}[(1).]
        \item 充分性:$||Ax||\leqslant M||x||\Rightarrow A\in\mathcal{L}(\ell^1)$;
            
            必要性:假设$\{a_n\}$无界,存在$n_1<n_2<\cdots$使得$|a_{n_k}|>k$,
            设$e_n=(0,\cdots,0,\mathop{1}\limits_{m},0,\cdots)\in \ell^1$,
            则
            \begin{equation*}
                \frac{||Ae_{n_k}||}{||e_{n_k}||}=|a_{n_k}|>k\rightarrow\infty
            \end{equation*}
            这与$A$有界矛盾。
        \item 如果$\forall a_n\neq 0$,取
            \begin{equation*}
                T:(x_1,x_2,\cdots)\mapsto (\frac{x_1}{a_1},\frac{x_1}{a_2},\cdots)
            \end{equation*}
            则$TA=AT\Rightarrow T=A^{-1}$,而且
            \begin{equation*}
                \fun{sup}{n\geqslant 1}\left|\frac{1}{a_n}\right|<+\infty
                \Leftrightarrow 
                \fun{inf}{n\geqslant 1}|a_n|>0
            \end{equation*}
            所以$A^{-1}$有界$\Leftrightarrow \fun{inf}{n\geqslant 1}|a_n|>0$.
            如果存在某个$a_n=0$,那么$R(A)\subsetneqq \ell^1$,$A^{-1}$不存在。
        \item 充分性:设
            \begin{equation*}
                A_n:(x_1,x_2,\cdots)\mapsto (a_1x_1,\cdots,a_nx_n,0,\cdots)
            \end{equation*}
            则$A_n$是有界有限秩算子$\Rightarrow A_n$紧,
            同时
            \begin{equation*}
                ||A-A_n||=\sum_{k=n+1}^\infty |a_k||x_k|\rightarrow 0{\rm\ as\ }n\rightarrow\infty
            \end{equation*}
            所以$A$也是紧算子;

            必要性:$A$是紧算子,由定理3.2.3(Riesz‑Schauder)
            知$0\in \sigma(A)$且$0$之外的元素都是特征值,注意到
            $Ae_n=a_n e_n\Rightarrow a_n\in \sigma(A)$,
            \begin{equation*}
                \forall a\in \sigma(A)\backslash 
            \{0,a_1,a_2,\cdots\}\Rightarrow \fun{inf}{n\geqslant 1}|\lambda-\lambda_n|>0
            \end{equation*}
            由(2)知$T=(\lambda I-A)^{-1}$有界$\Rightarrow\lambda\in\rho(A)$,
            所以$\sigma(A)=\{0,a_1,a_2,\cdots\}$,且一定有
            \begin{equation*}
                \fun{lim}{n\rightarrow\infty}a_n=0
            \end{equation*}
    \end{enumerate}
\end{solve}

\begin{exer}{\rm (3.3.2)}
    在$C[0,1]$中,考虑映射
    \begin{equation*}
        T:C[0,1]\rightarrow C[0,1],x(t)\mapsto \int_0^t x(s)\d s
    \end{equation*}
    \begin{enumerate}[(1).]
        \item 证明$T$是紧算子;
        \item 求$\sigma(T)$和$T$的一个非平凡闭不变子空间。
    \end{enumerate}
\end{exer}
\begin{solve}
    \begin{enumerate}[(1).]
        \item 对于任何有界集$B$,我们来证明$T(B)$一致有界且等度连续,从而列紧。
            设$\forall x\in B,||x||<M$,则
            \begin{equation*}
                ||Tx||\leqslant \int_0^1 |x(s)|\d s
                \leqslant \sup{t\in [0,1]}|x(t)|=||x||<M
            \end{equation*}
            所以一致有界;
            \begin{equation*}
                | (Tx)(s')-(Tx)(s'') |
                =| \int_{s'}^{s''} x(s)\d s |\leqslant ||x||\cdot |s''-s'|
            \end{equation*}
            所以等度连续。
        \item 因为$C[0,1]$是无穷维的,由定理3.2.3(Riesz‑Schauder)
            可得$0\in\sigma(T)$,$T$紧$\Rightarrow \sigma(T)$除了$0$之外都是
            特征值。如果$0\neq \lambda\in\sigma(T)$,则
            \begin{equation*}
                Tx=\lambda x\Rightarrow 
                \int_0^t x(s)\d s=\lambda x(t)
            \end{equation*}
            有非零解,进而$x(t)\in C^1[0,1]$,两边求导得
            \begin{equation*}
                x(t)=\lambda x'(t)\Rightarrow x(t)=C\e^{\frac{t}{\lambda}}
            \end{equation*}
            但
            \begin{equation*}
                \int_0^t x(s)\d s=C( \lambda \e^{\frac{t}{\lambda}}-1 )
                =C\lambda\e^{\frac{t}{\lambda}}
                \Rightarrow C=0\Rightarrow x(t)=0
            \end{equation*}
            所以$\sigma(T)=\{0\}$.
            $C^1[0,1]$是$T$的一个闭不变子空间。
    \end{enumerate}
\end{solve}


\section{期末复习相关}
%    chap1:压缩映射原理、列紧$\Leftrightarrow $完全有界,单位球列紧$\Leftrightarrow {\rm dim}X<\infty$.

%    chap2:
%    \begin{enumerate}
%        \item Baire纲定理:Banach空间无可数Hamel基。
%        \item 开映射定理:$\ag{Ax,x}\geqslant A||x||^2$,证明满射的思路:$\overline{{\rm Ran}(T)}={\rm Ran}(T),{\rm Ran}(T)^\perp=0$.
%        \item 闭图像定理:注意算子可逆只需要单射即可(逆映射的定义域取值域),闭算子的逆还是闭的,
%            $A$连续、${\rm Ran}(A)$闭、$A$单射$\Rightarrow A^{-1}$连续,
%            定义域为全空间的共轭算子连续。
%        \item 共鸣定理:证明弱收敛序列有界,命题的应用(习题2.3.7-9)。
%        \item 等价范数定理:题目给了两个范数,证明等价的思路是“取第三个范数为两个范数之和”,然后证明这个范数比其他两个都强,于是都等价。
%        \item Hahn-Banach定理:一般只考代数形式,重要推论有三个:
%            \begin{enumerate}
%                \item $\exists f{\rm\ s.t.\ }f(x_0)=||x_0||,||f||=1$.
%                \item $f(M)=\{0\},f(x_0)=d>0,||f||=1$.
%                \item $f_i(x_j)=\delta_{ij}$.
%            \end{enumerate}
%        \item 对偶空间:自然映射是等距嵌入,所以$X$是$X^{**}$的闭子空间,例如证明“$X$自反$\Leftrightarrow X^*$自反”。
%        \item 共轭算子:一般在Hilbert空间中求共轭算子。
%        \item 弱收敛:很少考弱*列紧,掌握课后习题。
%    \end{enumerate}
%
%    chap3:
%    \begin{enumerate}
%        \item 记住谱半径公式,一般算出来谱半径都是0或者1
%        \item $\sigma(T)$是闭集,比如我们算出来$B(0,1)\subset \sigma(T)$,那$\overline{B(0,1)}\subset \sigma(T)$,不用特意讨论边缘上的点。
%        \item 紧$\Rightarrow $全连续,自反空间时两者等价。
%        \item 用有限秩算子去逼近闭算子!重要思路,见习题
%        \item 二择一律:$A$是紧算子,那么$\lambda I-A$单射$\Leftrightarrow $满射。
%        \item 各种谱的定义。
%    \end{enumerate}

\subsection{部分往年期末题}
    \begin{exer}(19.6)
        定义
        \begin{equation*}
            T:\ell^2\rightarrow \ell^2,
            (x_1,x_2,\cdots)\mapsto (x_2,\frac{x_3}{2},\cdots,\frac{x_{n+1}}{n},\cdots)
        \end{equation*}
        求$T$的谱和特征值:$\sigma(T),\sigma_p(T)$.
    \end{exer}
    \begin{solve}
        首先显然$||T||\leqslant 1$,取$||Te_2||=||e_2||=1$可知$||T||=1$,
        因此$\sigma(T)\leqslant \overline{B(0,1)}$,
        设
        \begin{equation*}
            T_n:(x_1,x_2,\cdots)\mapsto (x_2,\frac{x_2}{x_3},\cdots,\frac{x_{n+1}}{n},0,0,\cdots)
        \end{equation*}
        $T_n\rightarrow T$可知$T$是紧算子,因此$0\in\sigma(T)$.

        对于$\lambda\neq 0$,考虑$\lambda I-T$,若存在$x$
        使得$(\lambda I-T)x=0$,那么
        \begin{equation*}
            \lambda x_n=\frac{x_{n+1}}{n+1}\Rightarrow 
            x_{n+1}=x_1\cdot \frac{(n+1)!}{\lambda^n}
        \end{equation*}
        如果$x_1\neq 0$,$|x_{n+1}|\rightarrow\infty$,所以只能$x=0$,
        所以$\lambda I-T$是单射,由二择一律知是满射,
        所以$\lambda\in\rho(T)$,因此$\sigma(T)=\{0\}$.
        
        因为$Te_1=0$,所以$\sigma_p(T)=\{0\}$.
    \end{solve}

    \begin{exer}(20.7)
        定义:
        \begin{equation*}
            T:\ell^2\rightarrow \ell^2,
            (x_1,x_2,\cdots)\mapsto (0,\lambda_1x_1,\lambda_2x_2,\cdots)
        \end{equation*}
        其中$0\neq \lambda_n\in\C$且$\lambda_n\rightarrow 0$,求$\sigma_p,\sigma_c,\sigma_r$.
    \end{exer}
    \begin{solve}
        类似可证$T$是紧算子,以及$\lambda\neq 0$时$\lambda I-T$是单射:
        \begin{equation*}
            (\lambda I-T)x=0\Rightarrow \lambda x_{n+1}-\lambda_nx_n=0,x_1=0\Rightarrow x=0
        \end{equation*}
        进而是满射,所以只需考虑$0$:
        $Tx=0\Rightarrow x=0$,不是特征值;
        $Tx$的第一个分量始终是$0$,所以${\rm Ran}(T)$的闭包肯定不是全空间,
        所以$0\notin \sigma_c(T)$.

        综上,$\sigma_p(T)=\sigma_c(T)=\varnothing$,
        $\sigma_r(T)=\{0\}$.
    \end{solve}

    \begin{exer}(22.6)
        定义:
        \begin{equation*}
            T:L^2[0,1]\rightarrow L^2[0,1],
            x(t)\mapsto \int_0^t x(s)\d s
        \end{equation*}
        \begin{enumerate}[(1).]
            \item 证明$T$是紧算子。
            \item 证明:
                \begin{equation*}
                    T^n x(t)=\int_0^t \frac{ (t-s)^{n-1} }{(n-1)!}x(s)\d s
                \end{equation*}
            \item 求$T$的谱半径。
            \item 求$T$的各种谱。
            \item 判断$T$是否是对称算子。
        \end{enumerate}
    \end{exer}
    \begin{solve}
        \begin{enumerate}[(1).]
            \item 任取$L^2[0,1]$上的有界集$F$,不妨
                \begin{equation*}
                    \fun{sup}{f\in F} ||f||^2=\fun{sup}{f\in F} \int_0^1 |f(t)|^2\d t<M
                \end{equation*}
                由于积分的绝对连续性,
                $T(F)\subset C[0,1]$,故只需证明$T(F)$一致有界且等度连续,
                对于$\forall f\in L^2[0,1]$,
                \begin{align*}
                    \int_0^1 \left( |f(s)|-\int_0^1 |f(t)|\d t \right)^2\d s
                    &=\int_0^1\left(  |f(s)|^2-2|f(s)|\int_0^1 |f(t)|\d t+\left(\int_0^1 |f(t)|\d t\right)^2 \right)\d s\\
                    &=||f||^2-2\left(\int_0^1 |f(t)|\d t\right)^2+\left(\int_0^1 |f(t)|\d t\right)^2\\
                    &=||f||^2-\left(\int_0^1 |f(t)|\d t\right)^2\geqslant 0
                \end{align*}
                所以
                \begin{equation*}
                    ||T(f)||^2=\int_0^1\left( \int_0^t f(s)\d s \right)^2\d t
                    \leqslant \int_0^1 \left( \int_0^1 |f(s)|\d s \right)^2\d t
                    \leqslant \int_0^1 ||f||^2\d t=||f||^2<M
                \end{equation*}
                故一致有界。对于$\forall \varepsilon>0$,
                取$\delta<\varepsilon$,则$||f_1-f_2||<\delta\Rightarrow$
                \begin{equation*}
                    ||T(f_1)-T(f_2)||=||T(f_1-f_2)||\leqslant ||f_1-f_2||<\varepsilon
                \end{equation*}
                所以等度连续。
            \item 这个用归纳证明,具体过程和泛函没啥关系,就不写了。
            \item 用上一问结论和谱半径公式求出谱半径为$0$,计算过程省略。
            \item 由上一问结论$\sigma(T)=\{0\}$,只需判断$0$是什么谱:
                $T(f)=0\Rightarrow f=0$,所以$0$不是特征值,
                现只需考虑${\rm Ran}(-T)$,显然多项式$P[0,1]\subset {\rm Ran}(-T)$,
                因为$\overline{P[0,1]}=L^2[0,1]$,所以
                $0$是连续谱。
            \item 假如是对称算子,因为$L^2$是{\rm Hilbert}空间,则
                \begin{equation*}
                    \ag{Tx,y}=\ag{x,Ty}
                \end{equation*}
                从而
                \begin{equation*}
                    \int_0^1 y(t)\int_0^t x(s)\d s\d t=
                    \int_0^1 x(t)\int_0^t y(s)\d s\d t
                \end{equation*}
                取$x=1,y=t^2$,则等式不成立,故不是对称算子。
        \end{enumerate}
    \end{solve}

    \begin{exer}(19.7)
        证明一个Hilbert空间是有限维的当且仅当它的任意一组规范正交基都是它的线性基。
    \end{exer}
    \begin{solve}
        回顾一下概念:规范正交基是指:
        \begin{equation*}
            x=\sum_{\alpha\in \Lambda}\ag{x,e_\alpha}e_\alpha
        \end{equation*}
        线性基(Hamel基):任何一个向量可以写成有限个基中向量的线性组合。

        充分性是显然的,必要性:假设是无穷维的,
        取它的一族规范正交基$\{e_\alpha\}$,从中选出可数个$\{e_k\}=A$,
        定义
        \begin{equation*}
            x_n=\frac{1}{n}\sum_{k=1}^n e_k
        \end{equation*}
        那么$\{x_n\}$是柯西列,不妨设$x_n\rightarrow a$,
        $a$是有限个$e_\alpha$的有限线性组合,
        设
        \begin{equation*}
            a=\sum_{i=1}^p a_i e_{k_i}+c
        \end{equation*}
        $c$是不在$A$中的基的有限线性组合,
        设$b=\fun{max}{1\leqslant i\leqslant p} k_i$,则$\forall n>b$,都有
        \begin{equation*}
            ||x_n-a||=\ms{ \frac{1}{n}\sum_{k=b+1}^n e_k+\left(\sum_{k=1}^b \frac{e_k}{n}-\sum_{i=1}^p a_ie_{k_i}\right)-c }>\ms{ \frac{1}{n}\sum_{k=b+1}^n e_k}=1-\frac{b}{n}>\frac{1}{b+1}
        \end{equation*}
        这就与$x_n\rightarrow a$矛盾。
    \end{solve}

    \begin{exer}(18.6)
        $X$是赋范空间,证明:$\forall x\in X$,
        \begin{equation*}
            ||x||=\fun{sup}{}\{ |f(x)|:f\in X^*,||f||=1 \}
        \end{equation*}
    \end{exer}
    \begin{solve}
        $x=0$时等式显然成立,下设$x\neq 0$.
        对于$\forall f\in X^*{\rm\ with\ }||f||=1$,
        \begin{equation*}
            1=||f||\geqslant \frac{|f(x)|}{||x||}
            \Rightarrow ||x||\geqslant |f(x)|
            \Rightarrow ||x||\geqslant \fun{sup}{}\{ |f(x)|:f\in X^*,||f||=1 \}
        \end{equation*}
        另一方面,由HBT可知存在$f\in X^*{\rm\ with} ||f||=1$使得
        $f(x)=||x||$,所以
        \begin{equation*}
            \fun{sup}{}\{ |f(x)|:f\in X^*,||f||=1 \}\geqslant ||x||
        \end{equation*}
        故相等。
    \end{solve}

    \begin{exer}(18.7)
        $X$是Banach空间,$A,B$分别是$X$上的有界算子、紧算子,证明:
        \begin{equation*}
            \sigma(A)\backslash( \sigma_p(A)\cup \sigma_p(A+B) )
            =\sigma(A+B)\backslash(\sigma_p(A)\cup \sigma_p(A+B))
        \end{equation*}
    \end{exer}
    \begin{solve}
        对两边取补集,就是
        \begin{equation*}
            \rho(A)\cup\sigma_p(A)\cup\sigma_p(A+B)=
            \rho(A+B)\cup\sigma_p(A)\cup\sigma_p(A+B)
        \end{equation*}
        对于$\lambda\in\rho(A)$,$\lambda I-A$可逆,考虑
        \begin{equation*}
            \lambda I-(A+B)=(\lambda I-A)(I-(\lambda I-A)^{-1}B)
        \end{equation*}
        注意$C=(\lambda I-A)^{-1}B$是有界算子和紧算子的复合,仍然是紧算子,
        所以$I-C$单当且仅当$I-C$满,注意$(\lambda I-A)$是双射,
        所以$\lambda I-(A+B)$单当且仅当$\lambda I-(A+B)$满:
        \begin{enumerate}
            \item $\lambda I-(A+B)$单$\Rightarrow \lambda\in\rho(A+B)$.
            \item $\lambda I-(A+B)$不单$\Rightarrow \lambda\in\sigma_p(A+B)$.
        \end{enumerate}
        所以
        \begin{equation*}
            \rho(A)\subset \rho(A+B)\cup\sigma_p(A+B)
        \end{equation*}
        所以
        \begin{equation*}
            \rho(A)\cup\sigma_p(A)\cup\sigma_p(A+B)\subset
            \rho(A+B)\cup\sigma_p(A)\cup\sigma_p(A+B)
        \end{equation*}

        另一方面,注意到题设等式是关于$A,A+B$对称的,
        所以另一个方向也成立。(可以理解为两个有界算子$C,D$满足$C-D$紧。)
    \end{solve}

    \begin{exer}(18.9)
        $X$是Banach空间,$T$是$X$上的线性算子,证明:$T$有界当且仅当
        \begin{equation*}
            x_n\wto x\Rightarrow Tx_n\wto Tx
        \end{equation*}
    \end{exer}
    \begin{solve}
        充分性:只需证明$T$是闭算子,若$x_n\rightarrow x,Tx_n\rightarrow y$,则
        由于收敛$\Rightarrow $弱收敛,
        \begin{equation*}
            x_n\wto x\Rightarrow Tx_n\wto Tx,
            Tx_n\wto y
        \end{equation*}
        由弱收敛列极限唯一\footnote{如果$x_n\wto x,x_n\wto y$,则根据$\K$上收敛极限的唯一性,$\forall f$有$f(x)=f(y)$,再由HBT可知$x=y$.},可得$y=Tx$.
        
        必要性:$T$有界,$\forall f\in X^*$,
        $f\circ T\in X^*$,从而
        \begin{equation*}
            x_n\wto x\Rightarrow f(Tx_n)\rightarrow f(Tx),\ \forall f\in X^*
            \Rightarrow Tx_n\wto Tx
        \end{equation*}

        这个题好像不需要Banach的条件。
    \end{solve}

    \begin{exer}
        (18.10)设$X$是Banach空间,$f$是$X$上的非零线性泛函,证明$N(f)$
        要么是$X$的闭子空间,要么是$X$的稠密真子空间。

        这道题其实就是习题2.1.7的思路继续延伸了一点。
    \end{exer}
    \begin{solve}
        $f$有界$\Leftrightarrow N(f)$是$X$的闭子空间;
        若$f$无界,对于任意$n$,存在$x_n$使得
        \begin{equation*}
            |f(x_n)|>n||x_n||
        \end{equation*}
        取
        \begin{equation*}
            a_n=\frac{f(x_n)}{f(x)}\Rightarrow f(x_n-a_nx)=0\Rightarrow x_n-a_nx\in N(f)
        \end{equation*}
        所以
        \begin{equation*}
            x-\frac{x_n}{a_n}=x-\frac{f(x)x_n}{f(x_n)}\in N(f)
        \end{equation*}
        那么令$n\rightarrow\infty$,
        \begin{equation*}
            \ms{\frac{f(x)x_n}{f(x_n)}}
            =\frac{|f(x)|\cdot ||x_n||}{|f(x_n)|}<\frac{|f(x)|}{n}\rightarrow 0
        \end{equation*}
        所以
        \begin{equation*}
            N(f)\ni x-\frac{f(x)x_n}{f(x_n)}\rightarrow x
        \end{equation*}
    \end{solve}

    \begin{exer}
        (19.4)证明弱*收敛序列一定有界。
    \end{exer}
    \begin{solve}
        共鸣定理直接得证。
    \end{solve}

    \begin{exer}(19.5)
        $H$是Hilbert空间,$P$是$H$上的有界算子,且$P^2=P=P^*$,
        证明$P$的值域是闭集,且$P$是正交投影算子。
    \end{exer}
    \begin{solve}
        任取${\rm Ran}(P)$中柯西列$P(x_i)$,不妨设$P(x_i)\rightarrow a$,
        则$P^2(x_i)\rightarrow P(a)=a\Rightarrow 
        a\in {\rm Ran}(P)\Rightarrow {\rm Ran}(P)$闭。

        对$H$作正交分解$H={\rm Ran}(P)+{\rm Ran}(P)^\perp={\rm Ran}(P)+M$,
        对于$\forall x\in H$,可以分解为
        \begin{equation*}
            x=y+z,\ y\in {\rm Ran}(P),z\perp {\rm Ran}(P)
        \end{equation*}
        那么
        \begin{equation*}
            \ag{P(y),P(z)}=\ag{P^2(y),z}=\ag{P(y),z}=0
        \end{equation*}
        同理
        \begin{equation*}
            \ag{P(x),P(z)}=\ag{P(x),z}=0
        \end{equation*}
        所以
        \begin{equation*}
            \ag{P(x),P(z)}-\ag{P(y),P(z)}
            =\ag{P(z),P(z)}=0\Rightarrow P(z)=0
        \end{equation*}
        由于$y\in {\rm Ran}(P)$,所以$P(y)=y$,从而
        $P(x)=P(y)=y$,因此$P$是$X$到${\rm Ran}(P)$的正交投影算子,
    \end{solve}

    \begin{exer}(19.8)
        $X,Y$是Banach空间,$X$有限维,$T$是从$X$到$Y$的线性算子,证明
        存在$x\in X,||x||=1$使得$||Tx||=||T||$.
    \end{exer}
    \begin{solve}
        $T$是有界算子(命题2.1.1),
        则$\forall n$,存在$x_n{\rm\ with\ }||x_n||=1$满足
        \begin{equation*}
            ||T||-\frac{1}{n}\leqslant ||Tx_n|| 
            \leqslant ||T||           
        \end{equation*}
        有限维赋范空间单位球面列紧,所以$\{x_n\}$存在收敛子列
        $\{ x_{n_k} \}$,不妨设$x_{n_k}\rightarrow x$,则
        $||Tx||=||T||$.
    \end{solve}

    \begin{exer}(19.9)
        $X,Y$是赋范空间,且$X\neq 0$,证明$Y$是Banach空间当且仅当
        $L(X,Y)$是Banach空间。
    \end{exer}
    \begin{solve}
        必要性在定理2.1.3,只说明充分性:
        设$\{y_n\}$是$Y$中柯西列,
        取$0\neq x\in X$,由HBT可知存在$f\in X^*{\rm\ with\ }||f||=1,f(x)=||x||$,
        定义算子序列:
        \begin{equation*}
            F_n:X\rightarrow Y,x\mapsto f(x)y_n
        \end{equation*}
        则$F_n$是线性算子,且$||F_n||=||y_n||$,所以
        \begin{equation*}
            ||F_n(x)-F_m(x)||\leqslant ||x||\cdot ||y_n-y_m||
        \end{equation*}
        故$\{F_n\}$是$\mathcal{L}(X,Y)$上的柯西列,
        不妨$F_n\rightarrow F$,取$y=\frac{F(a)}{|a|}$,
        \begin{equation*}
            ||y_n-y||=\ms{ \frac{F_n(x)}{|f(x)|}-\frac{F(x)}{|x|} }
            =\frac{1}{|x|}
            \ms{ F_n(x)-F(x) }\leqslant ||F_n-F||\rightarrow 0
        \end{equation*}
        从而$y_n\rightarrow y$,这就证明了$Y$是{\rm Banach}空间。
    \end{solve}

    \begin{exer}(20.3)
        $X$是赋范空间,$V$是$X$的子空间,证明$V$在$X$中稠密等价于$V^\perp=0$,这里
        \begin{equation*}
            V^\perp=\{ f\in X^*:f(V)=0 \}
        \end{equation*}
    \end{exer}
    \begin{solve}
        必要性由$f\in X^*$的连续性可得;
        充分性由定理2.7.4可得。
    \end{solve}

    \begin{exer}(20.5)
        $X$是Banach空间,$V_n$是一列闭子空间,满足
        \begin{equation*}
            X=\bigcup_{n=1}^\infty V_n
        \end{equation*}
        证明:存在某个$n_0$使得
        \begin{equation*}
            V_{n_0}=X
        \end{equation*}
    \end{exer}
    \begin{solve}
        由BCT1,$X$不是可数个无处稠密集之并,所以存在某个$V_{n_0}$
        不是无处稠密集,即有$\overline{V}=V$内点,但真闭子空间无内点,所以只能$V_{n_0}=X$.
    \end{solve}

    \begin{exer}(20.6)
        $X$是Hilbert空间,$||x_n||\rightarrow ||x||$且$x_n\wto x$,则$x_n\rightarrow x$.
    \end{exer}
    \begin{solve}
        考虑
        \begin{equation*}
            \ag{\cdot,x}\in H^*\Rightarrow \ag{x_n,x}\rightarrow \ag{x,x}=||x||^2
        \end{equation*}
        那么
        \begin{equation*}
            ||x_n-x||^2=\ag{x_n-x,x_n-x}
            =||x_n||^2+||x||^2-2\ag{x_n,x}
            \rightarrow ||x||^2+||x||^2-2||x||^2=0{\rm\ as\ }n\rightarrow\infty
        \end{equation*}
    \end{solve}

    \begin{exer}(20.8)
        $A_n$是Hilbert空间$X$上的一列有界线性算子,且对于$\forall x$都有
        $||A_n x||\rightarrow 0$,证明:对于任意紧算子$K$都有$||A_nK||\rightarrow 0$.
    \end{exer}
    \begin{solve}
        记单位球面为$S$,考虑
        \begin{equation*}
            ||A_nK||=\fun{sup}{x\in S}||A_nK(x)||
            =\fun{sup}{y\in K(S)}||A_ny||
        \end{equation*}
        假设$||A_nK||\nrightarrow 0$,即存在$\delta>0$使得
        \begin{equation*}
            \forall n,\fun{sup}{y\in K(S)}||A_ny||>\delta
        \end{equation*}
        所以
        \begin{equation*}
            \forall n,\exists y_n=K(x_n)\in K(S){\rm\ s.t.\ }
            ||A_ny_n||\geqslant \delta
        \end{equation*}
        $K$是紧算子,$\{x_n\}\in S$有界,所以$K(\{x_n\})=\{y_n\}$列紧,
        取其收敛子列$y_{n_k}\rightarrow y$,则
        \begin{equation*}
            ||A_{n_k}y_{n_k}||\leqslant ||A_{n_k}y||+||A_{n_k}(y_{n_k}-y)||
            \leqslant ||A_{n_k}y||+||A_{n_k}||\cdot ||y_{n_k}-y||\rightarrow 0{\rm\ as\ }k\rightarrow\infty
        \end{equation*}
        这与$||A_{n_k}y_{n_k}||\geqslant \delta$矛盾。
    \end{solve}
    \begin{solve}
        另一种解法:
    \end{solve}

    \begin{exer}(21.5)
        $X$是可分赋范空间,证明:存在$\{f_n\}\in X^*$使得对于任意$x\in X$,都有
        \begin{equation*}
            ||x||=\fun{sup}{n}|f_n(x)|
        \end{equation*}
    \end{exer}
    \begin{solve}
        设$X$的稠密可数子集是${x_n}_{n=1}^\infty$,
        有HBT可得存在$f_n\in X^*{\rm\ with\ }||f_n||=1{\rm\ s.t.\ }f(x_n)=||x_n||$,一方面:
        \begin{equation*}
            \fun{sup}{n}|f_n(x)|\leqslant \fun{sup}{n}||f_n||\cdot ||x||=||x||
        \end{equation*}
        另一方面,设$\{x_{n_k}\}$使得$x_{n_k}\rightarrow x$,那么
        \begin{equation*}
            f_{n_k}(x_{n_k})=||x_{n_k}||\rightarrow ||x||{\rm\ as\ }k\rightarrow\infty
        \end{equation*}
        所以
        \begin{equation*}
            \fun{sup}{n}|f_n(x)|\geqslant ||x||
        \end{equation*}
    \end{solve}

    \begin{exer}(21.7)
        设$X,Y$是实Hilbert空间,$S_x$是$X$中单位球面,$T\in L(X,Y)$,
        且不存在$x\in S_x$使得$||Tx||=||T||$,求证:存在$\{x_n\}\subset S_x$,使得
        $x_n\wto 0$且$||Tx_n||\rightarrow ||T||$
    \end{exer}
    \begin{solve}
        由于
        \begin{equation*}
            ||T||=\fun{sup}{x\in S_x}||Tx||
        \end{equation*}
        所以
        \begin{equation*}
            \forall n,\exists x_n\in S_x{\rm\ s.t.\ }||T||-\frac{1}{n}\leqslant ||Tx_n||<||T||
        \end{equation*}
        则$||Tx_n||\rightarrow ||T||$.
        由{\rm Eberlein Smulian}定理:自反空间上的有界集有弱收敛子列,
        所以不妨假设$x_n\wto x$,只需证明$x=0$.

        假设$x\neq 0$,$||x_n||=1$,所以
        \begin{equation*}
            ||x_n-x||^2=1+||x||^2-2\ag{x_n,x}
            \rightarrow 1-||x||^2
        \end{equation*}
        又因为$T$有界,所以在$Y$上有$Tx_n\wto Tx$,同理有
        \begin{equation*}
            ||Tx_n-Tx||^2\rightarrow ||T||^2-||Tx||^2
        \end{equation*}
        同时
        \begin{equation*}
            ||Tx_n-Tx||^2\leqslant ||T||^2\cdot ||x_n-x||^2
        \end{equation*}
        所以上式左右两边取极限得
        \begin{equation*}
            ||T||^2-||Tx||^2\leqslant ||T||^2(1-||x||^2)\Rightarrow 
            ||T||\cdot ||x||\leqslant ||Tx||
        \end{equation*}
        从而
        \begin{equation*}
            ||T||\leqslant \frac{||Tx||}{||x||}=\ms{ T\left( \frac{x}{||x||} \right) }
            <||T||
        \end{equation*}
        矛盾,所以$x=0$.
    \end{solve}

    \begin{exer}(22.3)
        证明二则一律。
    \end{exer}
    \begin{solve}
        在定理3.2.2,但是为什么会考这个呢...
    \end{solve}
\subsection{其它}

\begin{exer}
    $C[0,1]$上有两个范数$||\cdot||_1$和$||\cdot ||_2$,其中
    \begin{equation*}
        ||f||_1=\fun{sup}{x\in [0,1]}|f(x)|
    \end{equation*}
    $||\cdot||_2$则满足:当$||x_n-x||_2\rightarrow 0$时,
    \begin{equation*}
        \fun{lim}{n\rightarrow\infty}x_n(t)\rightarrow x(t),\ \forall t\in [0,1]
    \end{equation*}
    而且$( C[0,1],||\cdot||_2 )$完备,证明两个范数等价。
\end{exer}
\begin{solve}
    取范数
    \begin{equation*}
        ||\cdot||_3=||\cdot ||_1+||\cdot ||_2
    \end{equation*}
    则$||\cdot||_3$强于$||\cdot||_1$和$||\cdot||_2$,
    只需证明$||\cdot||_3$是$C[0,1]$上的完备范数,则由范数等价定理
    可知三个范数都等价。
\end{solve}

\begin{exer}
    $0<p<1$时,$L^p(\R)$上不存在有界线性泛函。
\end{exer}
\begin{solve}
    设$f:L^p(\R)\rightarrow \R$有界,$||f||=M$,考虑拆分区间$[a,b]$:
    \begin{equation*}
        [a,b]=\bigcup_{k=1}^n [a,a+\frac{k}{n}(b-a)]:=\bigcup_{k=1}^n I_k
    \end{equation*}
    那么
    \begin{equation*}
        ||f( \chi_{[a,b]} )||=\ms{ \sum_{k=1}^n f(\chi_{I_k}) }
        \leqslant M\cdot \sum_{k=1}^n ||\chi_{I_i}||=Mn\left(\frac{b-a}{n}\right)^{\frac{1}{p}}
    \end{equation*}
    所以
    \begin{equation*}
        \frac{||f( \chi_{[a,b]} )||}{||\chi_{[a,b]}||}
        =Mn^{1-\frac{1}{p}}\rightarrow 0
        \Rightarrow f=0
    \end{equation*}
\end{solve}

\begin{exer}
    $X$是Banach空间,$T:X\rightarrow X^*$满足${\rm Dom}(T)=X$,$T(x)(x)\geqslant 0,\forall x$,
    证明$T$有界。
\end{exer}
\begin{solve}
    定义域全空间,求证有界,证明$T$是闭算子即可。
    假设$x_n\rightarrow 0,Tx_n\rightarrow L$,那么对于$\forall \lambda>0$,
    \begin{equation*}
        T(x_n+\lambda y)(x_n+\lambda y)=\lambda^2 T(y)T(y)
        +\lambda T(x_n)(y)+\lambda T(y)(x_n)+T(x_n)(x_n)\geqslant 0
    \end{equation*}
    令$n\rightarrow \infty$得到
    \begin{equation*}
        \lambda^2 T(y)(y)+\lambda L(y)\geqslant 0
        \Rightarrow \lambda T(y)(y)+L(y)\geqslant 0,\forall \lambda>0,y
    \end{equation*}
    则一定有$L(y)\geqslant 0$,否则取一个充分小的$\lambda $就不成立了。
    同理取$T(x_n-\lambda y)(x_n-\lambda y)$就得到$L(-y)\geqslant 0$,
    从而$L=0$.

    然后考虑一般情况:$x_n\rightarrow x,Tx_n\rightarrow L$,那么
    $x_n-x\rightarrow 0\Rightarrow Tx_n-Tx\rightarrow L-Tx=0$,
    从而$L=Tx$,因此$T$是闭算子,由闭图像定理知$T$有界(则连续)。
\end{solve}

\begin{exer}
    $M$是$L^2[0,1]$的闭子空间,且$M\subset C[0,1]$,证明${\rm dim}M<+\infty$.
\end{exer}
\begin{solve}
    $(M,||\cdot||_\infty)$和$(M,||\cdot||_2)$都是完备的,
    考虑
    \begin{equation*}
        \frac{||x||_2}{||x||_\infty}=\frac{ \left(\int_0^1 |x(t)|^2\d t\right)^\frac{1}{2} }{\fun{sup}{t\in [0,1]} |x(t)|}\leqslant 1
    \end{equation*}
    所以两个范数等价,进而存在$C$使得
    \begin{equation*}
        ||x||_\infty\leqslant C||x||_2,\forall x\in M 
    \end{equation*}

    由于$L^2$是Hilbert空间,其闭子空间也是Hilbert空间,对于固定的$t\in [0,1]$:
    \begin{equation*}
        l_t:M\rightarrow \R, x\mapsto x(t)
    \end{equation*}
    那么由Riesz表示定理,存在$q_t\in L^2[0,1]$使得$l_t=\ag{\cdot,q_t},||q_t||_2=||l_t||$,
    \begin{equation*}
        \frac{|l_t(x)}{||x||_2}=\frac{|x(t)|}{||x||_2}
        \leqslant \frac{||x||_\infty}{||x||_2}\leqslant C\Rightarrow ||q_t||_2=||l_t||\leqslant C
    \end{equation*}

    $M$可分\footnote{$L^p$空间可分,而Banach空间可分,则其闭子空间也是可分的。},所以有可数正交基$S=\{h_n\}$,
    \begin{equation*}
        C^2\geqslant ||q_t||_2^2=\sum |\ag{h_n,q_t}|^2
        =\sum |h_n(t)|^2
    \end{equation*}
    于是
    \begin{equation*}
        \# S=\sum 1= \sum ||h_n||_2^2= \sum \int_0^1 |h_n(t)|^2 \d t\leqslant \int_0^1 C^2 \d t=C^2<+\infty
    \end{equation*}
\end{solve}

\begin{exer}
    设
    \begin{equation*}
        \frac{\d}{\d t}:C^1[a,b]\rightarrow C[a,b],f\mapsto \frac{\d f}{\d t}
    \end{equation*}
    证明它是闭算子而非有界算子,说明闭图像定理为何不适用。
\end{exer}
\begin{solve}
    取
    \begin{equation*}
        f_n(t)=\frac{(t-a)^n}{(b-a)^n}
    \end{equation*}
    则
    \begin{equation*}
        ||f_n||=\fun{sup}{t\in [a,b]}|f_n(t)|=1
    \end{equation*}
    但
    \begin{equation*}
        \ms{ \frac{\d }{\d t}f_n }=\fun{sup}{t\in [a,b]}\left| n\frac{(t-a)^{n-1}}{(b-a)^n} \right|=\frac{n}{b-a}\rightarrow\infty{\rm\ as\ }n\rightarrow\infty
    \end{equation*}
    所以是无界算子。
    
    考虑$x_n\in C^1[a,b]$,且$x_n\rightarrow x$,$Tx_n\rightarrow y$,我们转化成数分的语言就是:函数列${x_n(t)}$一致收敛,
    每一项$x_n$都有连续导数,且$x_n'$一致收敛到$y$,那么
    根据一致收敛函数列的性质,$x$也有连续导数,且$x'=y$,即$x\in C^1[a,b]$,$Tx=y$,这就证明了$\frac{\d }{\d t}$是闭算子。

    闭图像定理不适用的原因是$C^1[a,b]$不是闭集,比如一阶可微函数可以一致收敛到折线,后者不是一阶可微的。例如$[a,b]=[-1,1]$,
    \begin{equation*}
        f_n(t)=\sqrt{x^2+\frac{1}{n}}
    \end{equation*}
    $f_n$一致收敛于$f(t)=|t|$,在$t=0$处不可导。具体过程就不详细写了,都是数分的东西。
\end{solve}

\begin{exer}
    证明$\ell^\infty$不可分,$C_0$可分(考虑全体趋于$0$的有理数列即可)。
\end{exer}
\begin{solve}
    假设$\ell^\infty$存在可数稠密子集$D$,那么对于$\forall x\in \ell^\infty$,$\forall \varepsilon>0$,
    存在$y\in D$使得$||x-y||<\varepsilon$.现在考虑
    \begin{equation*}
        S=\{ x=\{x_n\}_{n=1}^\infty:x_n=0{\rm\ or\ }x_n=1 \}\subset \ell^\infty
    \end{equation*}
    即全体由$0,1$构成的数列,集合$S$和$[0,1]$是等势的(全体二进制小数),所以不可数,我们现在希望建立一个
    $S$到$D$的单射,从而$D$不可数,得到矛盾。

    取$\varepsilon=\frac{1}{2}$,任取$s=\{s_n\}\in S$,则存在某个$d=\{d_n\}\in D$使得
    \begin{equation*}
        \forall n,|s_n-d_n|< \frac{1}{2}
    \end{equation*}
    现在改动$s_1$:若$s_1=0$,则取$s_1'=1$,
    \begin{equation*}
        |0-d_1|<\frac{1}{2}\Rightarrow d_1\in (-\frac{1}{2},\frac{1}{2})\Rightarrow |1-d_1|>\frac{1}{2}
    \end{equation*}
    若$s_1=1$,则取$s_1'=0$,
    \begin{equation*}
        |1-d_1|<\frac{1}{2}\Rightarrow d_1\in (\frac{1}{2},\frac{3}{2})\Rightarrow |d_1|>\frac{1}{2}
    \end{equation*}
    这说明$s_1'$一定使得$|s_1'-d_1|>\frac{1}{2}$,
    对于$s'=(s_1',s_2,s_3,\cdots)\in S$,我们需要重新找一个$d'\in D$使得$||s'-d'||<\frac{1}{2}$.
    也就是说,对于某个$d\in D$,最多只有一个$s\in S$和它的距离小于$\frac{1}{2}$.这就得到了从$S\rightarrow D$的单射。
\end{solve}

\begin{solve}
    我们还可以用更“泛函”的方式回答这个问题:
    定理2.8.16告诉我们,如果$\ell^\infty$可分,则$(\ell^\infty)^*$中有界集都弱*列紧,
    类似于习题2.5.2,可以证明$(\ell^\infty)^*=C_0$,
    即每个$(\ell^\infty)^*$中的算子,都能被表示成与某个$f\in C_0$作内积(此处指各分量相乘后求和)。若我们能给出$C_0$上有界但不弱*列紧的例子,
    就能得出矛盾。例如$f_n=(0,0,\cdots,0,\mathop{1}\limits_{n},0,0,\cdots)\in C_0$,
    $e=(1,1,\cdots)\in \ell^\infty$,
    $\{f_n\}$是有界集,但是
    \begin{equation*}
        f_n(e)=1\nrightarrow 0
    \end{equation*}
    这说明$\{f_n\}$不弱*列紧。
\end{solve}

\begin{exer}
    $(X,d)$是度量空间,若$A\subset X$不可数且$(A,d)$是离散空间,则$(X,d)$不可分。
\end{exer}
\begin{solve}
    这道题思路和上一题中证明存在$S\rightarrow D$的单射如出一辙,可以发现上一题构造的集合$S$就是一个离散空间。

    假设$X$存在可数稠密子集$G$,
    我们考虑建立$A\rightarrow G$的单射,从而证明$G$不可数,导出矛盾。
    对于$\forall a\in A$,$\varepsilon=\frac{1}{2}$,存在某个$g\in G$满足
    \begin{equation*}
        d(g,a)<\frac{1}{2}
    \end{equation*}
    而$\forall a'\in A,a\neq a',d(a,a')=1$,所以
    \begin{equation*}
        d(g,a')\geqslant | d(a,a')-d(g,a) |=|1-d(g,a)|>\frac{1}{2}
    \end{equation*}
    这说明我们需要重新找一个$g'\in G$使得$d(g',a')<\frac{1}{2}$,这就构造了$S\rightarrow G$的单射。
\end{solve}